\documentclass{article}
\usepackage{geometry}
\geometry{left=1.2in, right=1.2in, top=1.2in, bottom=1.2in}%change the margins here
\usepackage[utf8]{inputenc}
\usepackage{tikz}
\usetikzlibrary{cd}
\usetikzlibrary{shapes.geometric,arrows,positioning,fit,calc,}
\usepackage[english]{babel}
\usepackage{amsthm} %lets us use \begin{proof}
\usepackage{amssymb} %gives us the character \varnothing
\usepackage{mathtools}
\usepackage{amsmath}
\usepackage{hyperref}
\usepackage[shortlabels]{enumitem}
\usepackage{biblatex}
\addbibresource{references.bib}  % The filename of your .bib file
\usepackage{csquotes}
\usepackage{float}
\usepackage[all]{xy}
\usepackage{mathrsfs}
\usepackage{multirow}
\usepackage{dsfont}
\usepackage{adjustbox}
\usepackage{titlesec}

% Custom chapter format
\titleformat{\section}
  {\normalfont\Large\bfseries}
  {Chapter \thesection}{1em}{}

% Custom section format
\titleformat{\subsection}
  {\normalfont\large\bfseries}
  {Section \thesection.\arabic{subsection}}{1em}{}

% Custom subsection format
\titleformat{\subsubsection}
  {\normalfont\normalsize\bfseries}
  {Exercise \thesubsubsection}{0em}{}

% Make subsections numbered with respect to sections
\renewcommand{\thesubsection}{\arabic{section}.\arabic{subsection}}

% Make subsubsections numbered with respect to subsections
\renewcommand{\thesubsubsection}{\arabic{section}.\arabic{subsection}.}

\newcommand{\abs}[1]{\left| #1 \right|}
\newcommand{\norm}[1]{\left\| #1 \right\|}
\newcommand{\A}{\mathbb{A}}
\newcommand{\R}{\mathbb{R}}
\newcommand{\T}{\mathbb{T}}
\newcommand{\N}{\mathbb{N}}
\newcommand{\Z}{\mathbb{Z}}
\newcommand{\Q}{\mathbb{Q}}
\newcommand{\C}{\mathbb{C}}
\newcommand{\rddots}{\reflectbox{$\ddots$}}
\newcommand{\F}{\mathbb{F}}
\newcommand{\id}{\mathrm{id}}
\newcommand{\ctd}{\Rightarrow \Leftarrow}
\newcommand{\actson}{\circlearrowright}
\newcommand\mapsfrom{\mathrel{\reflectbox{\ensuremath{\mapsto}}}}
\let\Section\S %Here I redefine the normal \S command to be the circle
\renewcommand{\S}{\mathbb{S}}
\newcommand{\RP}{\mathbb{RP}}
\newcommand{\CP}{\mathbb{CP}}
\newcommand{\HP}{\mathbb{HP}}
\newcommand{\B}{\mathbb{B}}
\newcommand{\calC}{\mathcal{C}}
\newcommand{\calO}{\mathcal{O}}
\newcommand{\fA}{\mathscr{A}}
\newcommand{\fB}{\mathscr{B}}
\newcommand{\fC}{\mathscr{C}}
\newcommand{\fD}{\mathscr{D}}
\newcommand{\fE}{\mathscr{E}}
\newcommand{\fF}{\mathscr{F}}
\newcommand{\fG}{\mathscr{G}}
\newcommand{\fH}{\mathscr{H}}
\newcommand{\fI}{\mathscr{I}}
\newcommand{\fJ}{\mathscr{J}}
\newcommand{\fO}{\mathscr{O}}
\newcommand{\fS}{\mathscr{S}}
\newcommand{\fT}{\mathscr{T}}
\newcommand{\frkA}{\mathfrak{A}}
\newcommand{\frkS}{\mathfrak{S}}
\newcommand{\frkm}{\mathfrak{m}}
\newcommand{\frkn}{\mathfrak{n}}
\newcommand{\frkp}{\mathfrak{p}}
\newcommand{\frkq}{\mathfrak{q}}
\newcommand{\frkl}{\mathfrak{l}}
\newcommand{\frkN}{\mathfrak{N}}
\newcommand{\altid}{\mathds{1}}
\newcommand{\nsubset}{\not \subset}
\newcommand\interior[1]{{#1}^{\circ}}
\newcommand{\Hh}{\mathbb{H}}
\newcommand{\D}{\mathbb{D}}
\newcommand{\Ab}{\mathbf{Ab}} %Abelian Groups
\newcommand{\Grp}{\mathbf{Grp}} %Groups
\newcommand{\Ring}{\mathbf{Ring}} %Rings
\newcommand{\CRing}{\mathbf{CRing}} %Commutative Rings
\newcommand{\Rng}{\mathbf{Rng}} %Rings without identity
\newcommand{\Set}{\mathbf{Set}} %Sets
\newcommand{\pSet}{\mathbf{Set}_{\bullet}} %Pointed Spaces
\newcommand{\Top}{\mathbf{Top}} %Topological Spaces
\newcommand{\pTop}{\mathbf{Top}_{\bullet}} %Pointed Topological Spaces
\newcommand{\Op}{\mathbf{Op}} %Open Subsets
\newcommand{\Vect}{\mathbf{Vect}} %Vector Spaces
\newcommand{\Man}{\mathbf{Man}} %Manifolds
\newcommand{\Mod}{\mathbf{Mod}} %Modules
\newcommand{\Mon}{\mathbf{Mon}} %Monoids
\newcommand{\Cat}{\mathbf{Cat}} %Small Categories
\newcommand{\Ssubset}{\mathbf{Subset}} %Subsets
\newcommand{\Com}{\mathbf{Com}} %Complexes
\DeclareMathOperator{\Haus}{\mathbf{Haus}} %Hausdorff Spaces
\DeclareMathOperator{\Comp}{\mathbf{Comp}} %Compact Spaces
\DeclareMathOperator{\Poset}{\mathbf{Poset}} %Partially Ordered Sets
\DeclareMathOperator{\Graph}{\mathbf{Graph}} %Graphs (Not Graph Theory)
\DeclareMathOperator{\Sch}{\mathbf{Sch}} %Schemes
\DeclareMathOperator{\AffSch}{\mathbf{AffSch}} %Affine Schemes
\DeclareMathOperator{\Grph}{\mathbf{Grph}} %Graphs in Graph Theory and Graph Homomorphisms
\DeclareMathOperator{\Rel}{\mathbf{Rel}} %Sets and Relations
\DeclareMathOperator{\CW}{\mathbf{CW}} %CW Complexes and Cellular Maps
\DeclareMathOperator{\PreSh}{\mathbf{PreSh}} %Presheaves
\DeclareMathOperator{\Sh}{\mathbf{Sh}} %Sheaves
\DeclareMathOperator{\catD}{\mathbf{D}} %Derived Category
\DeclareMathOperator{\TopGrp}{\mathbf{TopGrp}} %Topological Groups
\DeclareMathOperator{\Meas}{\mathbf{Meas}} %Measurable Spaces and measurable functions
\DeclareMathOperator{\Cob}{\mathbf{Cob}} %Cobordisms
\DeclareMathOperator{\LieAlg}{\mathbf{LieAlg}} %Lie Algebras
\DeclareMathOperator{\Ban}{\mathbf{Ban}} %Banach Spaces and Bounded Linear Operators
\DeclareMathOperator{\Hilb}{\mathbf{Hilb}} %Hilbert Spaces and Bounded Linear Operators
\DeclareMathOperator{\AlgC}{\mathbf{Alg_C}} %C-Algebras where C isn't necessarily commutative
\DeclareMathOperator{\Rep}{\mathbf{Rep}} %Representations
\DeclareMathOperator{\res}{\mathrm{res}}
\DeclareMathOperator{\pre}{\mathrm{pre}}
\DeclareMathOperator{\ad}{\mathrm{ad}}
\DeclareMathOperator{\Ind}{\mathrm{Ind}}
\DeclareMathOperator{\Res}{\mathrm{Res}}
\DeclareMathOperator{\End}{\mathrm{End}}
\DeclareMathOperator{\PGL}{\mathrm{PGL}}
\DeclareMathOperator{\Aff}{\mathrm{Aff}}
\DeclareMathOperator{\GL}{\mathrm{GL}}
\DeclareMathOperator{\SL}{\mathrm{SL}}
\DeclareMathOperator{\PSL}{\mathrm{PSL}}
\DeclareMathOperator{\U}{\mathrm{U}}
\DeclareMathOperator{\Oo}{\mathrm{O}}
\DeclareMathOperator{\SO}{\mathrm{SO}}
\DeclareMathOperator{\SU}{\mathrm{SU}}
\DeclareMathOperator{\Sp}{\mathrm{Sp}}
\DeclareMathOperator{\Gal}{\mathrm{Gal}}
\DeclareMathOperator{\frkgl}{\mathfrak{gl}}
\DeclareMathOperator{\frksl}{\mathfrak{sl}}
\DeclareMathOperator{\frkso}{\mathfrak{so}}
\DeclareMathOperator{\frksp}{\mathfrak{sp}}
\DeclareMathOperator{\frku}{\mathfrak{u}}
\DeclareMathOperator{\frkg}{\mathfrak{g}}
\DeclareMathOperator{\frkh}{\mathfrak{h}}
\DeclareMathOperator{\Stab}{\mathrm{Stab}}
\DeclareMathOperator{\im}{\mathrm{im}}
\DeclareMathOperator{\coim}{\mathrm{coim}}
\DeclareMathOperator{\cok}{\mathrm{cok}}
\DeclareMathOperator{\colim}{\mathrm{colim}}
\DeclareMathOperator{\spn}{\mathrm{span}}
\DeclareMathOperator{\Sym}{\mathrm{Sym}}
\DeclareMathOperator{\Hom}{\mathrm{Hom}}
\DeclareMathOperator{\Mor}{\mathrm{Mor}}
\DeclareMathOperator{\Nat}{\mathrm{Nat}}
\DeclareMathOperator{\Tr}{\mathrm{Tr}}
\DeclareMathOperator{\Bd}{\mathrm{Bd}}
\DeclareMathOperator{\Ann}{\mathrm{Ann}}
\DeclareMathOperator{\Int}{\mathrm{Int}}
\DeclareMathOperator{\Homeo}{\mathrm{Homeo}}
\DeclareMathOperator{\Char}{\mathrm{char}}
\DeclareMathOperator{\nullity}{\mathrm{nullity}}
\DeclareMathOperator{\Aut}{\mathrm{Aut}}
\DeclareMathOperator{\Frac}{\mathrm{Frac}}
\DeclareMathOperator{\supp}{\mathrm{supp}}
\DeclareMathOperator{\rank}{\mathrm{rank}}
\DeclareMathOperator{\diag}{\mathrm{diag}}
\DeclareMathOperator{\sign}{\mathrm{sign}}
\DeclareMathOperator{\glue}{\mathrm{glue}}
\DeclareMathOperator{\kerpre}{\ker_{\text{pre}}}
\DeclareMathOperator{\cokpre}{\cok_{\text{pre}}}
\DeclareMathOperator{\impre}{\im_{\text{pre}}}
\DeclareMathOperator{\coimpre}{\coim_{\text{pre}}}
\DeclareMathOperator{\sh}{sh}
\DeclareMathOperator{\ev}{ev}
\DeclareMathOperator{\op}{op}
\DeclareMathOperator{\Spec}{\mathrm{Spec}}
\DeclareMathOperator{\Ext}{\mathrm{Ext}}
\DeclareMathOperator{\Tor}{\mathrm{Tor}}
\DeclareMathOperator{\lcm}{\mathrm{lcm}}
\newcommand{\sqdot}{\, \raisebox{0.5ex}{\scalebox{0.2}{$\blacksquare$}} \,}
\makeatletter
\newcommand\xtwoheadrightarrow[2][]{%
  \ext@arrow 0579{\twoheadrightarrowfill@}{#1}{#2}}
\newcommand\twoheadrightarrowfill@{%
  \arrowfill@\relbar\relbar\twoheadrightarrow}
\makeatother
\let\oldemptyset\emptyset
\let\emptyset\varnothing
\newtheorem{definition}{Definition}[section]
\newtheorem{theorem}{Theorem}[section]
\newtheorem{corollary}{Corollary}[theorem]
\newtheorem{lemma}[theorem]{Lemma}
\newtheorem*{remark}{Remark}
\newtheorem*{lemma*}{Lemma}
\newtheorem{setting}{Setting}
\usepackage{lipsum}                     % Dummytext
\usepackage{xargs}                      % Use more than one optional parameter in a new commands
%\usepackage[pdftex,dvipsnames]{xcolor}  % Coloured text etc.
% 
\usepackage[colorinlistoftodos,prependcaption,textsize=tiny]{todonotes}
\newcommandx{\unsure}[2][1=]{\todo[linecolor=red,backgroundcolor=red!25,bordercolor=red,#1]{#2}}
\newcommandx{\change}[2][1=]{\todo[linecolor=blue,backgroundcolor=blue!25,bordercolor=blue,#1]{#2}}
\newcommandx{\info}[2][1=]{\todo[linecolor=OliveGreen,backgroundcolor=OliveGreen!25,bordercolor=OliveGreen,#1]{#2}}
\newcommandx{\improvement}[2][1=]{\todo[linecolor=Plum,backgroundcolor=Plum!25,bordercolor=Plum,#1]{#2}}
\newcommandx{\thiswillnotshow}[2][1=]{\todo[disable,#1]{#2}}
%
\title{Solutions to ``The Rising Sea"}
\author{Jack Westbrook}
\date\today
%This information doesn't actually show up on your document unless you use the maketitle command below

\begin{document}
\maketitle %This command prints the title based on information entered above

%Section and subsection automatically number unless you put the asterisk next to them.
The exercises in this document are taken from the February 21, 2024 draft of Ravi Vakil's ``The Rising Sea".
You can access the draft \href{https://math.stanford.edu/~vakil/216blog/FOAGfeb2124public.pdf}{here}.

\section*{Preliminary Results}
\subsection*{Results in Arbitrary Categories}
\begin{lemma}\label{lem:ker monic}
    If $f:A\to B$, the inclusion map $\iota:\ker f\to A$ is monic.
    \begin{proof}
        If $g_1,g_2:C\to \ker f$ are such that $\iota\circ g_1=\iota\circ g_2$, then the following diagram commutes:
        \begin{center}
            \begin{tikzcd}
                &&C\\
                &\ker f\ar{r}{\iota} \ar{ur}{0}& A \ar{u}[swap]{g\circ f}\\
                D\ar[dashed]{ur}[description]{\exists!} \ar[bend right, shift right]{urr}[swap]{\iota\circ g_1} \ar[bend right, shift left]{urr}{\iota\circ g_2}
            \end{tikzcd}
        \end{center}
        We immediately notice both $g_1$ and $g_2$ satisfy the unique arrow because $\iota\circ g_2=\iota\circ g_1$. By uniqueness, $g_1=g_2$.
    \end{proof}
\end{lemma}
\begin{lemma}\label{lem:cok epic}
    If $f:A\to B$, the projection $\pi:B \to \cok f$ is epic.
    \begin{proof}
        Suppose $g_1,g_2:\cok f\to C$ are such that $g_1\circ \pi=g_2\circ \pi$. Then the following diagram commutes:
        \begin{center}
            \begin{tikzcd}
                &&C\\
                &\cok f \ar[dashed]{ur}[description]{\exists!}&\\
                A\ar{ur}{0} \ar{r}{f}&B\ar{u}{\pi} \ar[bend right, shift right]{uur}[swap]{g_1\circ \pi} \ar[bend right, shift left]{uur}{g_2\circ \pi}&
            \end{tikzcd}
        \end{center}
        We notice immediately that $g_1$ and $g_2$ satisfy the unique arrow because $g_2\circ \pi=g_1\circ \pi$, so by uniqueness $g_1=g_2$.
    \end{proof}
\end{lemma}
\begin{lemma}\label{lem:comp epic then epic}
    If $h=g\circ f$ and $h$ is epic, then $g$ is epic.
\end{lemma}
\begin{proof}
    Suppose $\phi\circ g=\varphi \circ g$. Then it's also true that
    \[
    \phi \circ g\circ f=\varphi\circ g\circ f
    \]
    which by definition implies
    \[
    \phi \circ h=\varphi\circ h
    \]
    Because $h$ is epic, $\phi=\varphi$ as desired.
\end{proof}
\begin{lemma}\label{lem:comp monic then monic}
    If $h=g\circ f$ and $h$ is monic, then $f$ is monic.
\end{lemma}
\begin{proof}
    If $f\circ \phi=f\circ \varphi$, then
    \[
    g\circ f\circ \phi=g\circ f\circ \varphi
    \]
    which means by definition
    \[
    h\circ \phi=h\circ \varphi
    \]
    Because $h$ is monic, then $\phi=\varphi$ as desired.
\end{proof}
\begin{lemma}\label{lem:comp with monic and ker}
    If $A\xrightarrow{f} B\xhookrightarrow{g}C$, then $\ker (g\circ f)=\ker f$.
    \begin{proof}
        If $\ker f\xhookrightarrow{\iota} A$ is the inclusion, it follows that the following diagram commutes:
        \begin{center}
            \begin{tikzcd}
            & C\\
                \ker f \ar[hookrightarrow]{r}{\iota} \ar{ur}{0}& A\ar{u}{g\circ f}
            \end{tikzcd}
        \end{center}
        If there is a morphism $h:D\to A$ such that $g\circ f \circ h=0$, then we notice
        \[
        g\circ f\circ h=0=g\circ 0
        \]
        which implies, by $g$ being monic, that $f\circ h=0$. Then we obtain a unique induced morphism from the following diagram:
        \begin{center}
            \begin{tikzcd}
                &&B\\
                &\ker f\ar[hookrightarrow]{r}{\iota} \ar{ur}{0}& A \ar{u}[swap]{f}\\
                D\ar[dashed]{ur}[description]{\exists!} \ar[bend right]{urr}{h}
            \end{tikzcd}
        \end{center}
        Thus, in particular, the following diagram commutes as well:
        \begin{center}
            \begin{tikzcd}
                &&C\\
                &\ker f\ar[hookrightarrow]{r}{\iota} \ar{ur}{0}& A \ar{u}[swap]{g\circ f}\\
                D\ar[dashed]{ur}[description]{\exists!} \ar[bend right]{urr}{h}
            \end{tikzcd}
        \end{center}
    \end{proof}
\end{lemma}
\begin{corollary}\label{cor:comp with monic and coim}
        If $A\xrightarrow{f}B \xhookrightarrow{g} C$, then $\coim (g\circ f)=\coim f$.
        \begin{proof}
            $\ker (g\circ f)=\ker f$, thus 
            \[
            \coim (g\circ f)=\cok \ker (g\circ f)=\cok \ker f=\coim f
            \]
        \end{proof}
    \end{corollary}
\begin{lemma}\label{lem:comp with epic and cok}
    If $A\overset{f}\twoheadrightarrow B\xrightarrow{g}C$, then $\cok (g\circ f)=\cok g$.
    \begin{proof}
        If $C\overset{\pi}\twoheadrightarrow \cok g$ is the projection, 
        \begin{center}
            \begin{tikzcd}
            &\cok g\\
                A \ar{r}{g\circ f} \ar{ur}{0}&C \ar[two heads]{u}{\pi}
            \end{tikzcd}
        \end{center}
        commutes because $\pi \circ g=0$. If we have some $p:C\to D$ such that $p\circ g\circ f=0$, we notice
        \[
        p\circ g\circ f=0=0 \circ f
        \]
        which implies, by $f$ being epic, that $p\circ g=0$. Thus the following commutes:
        \begin{center}
            \begin{tikzcd}
            &&D\\
            &\cok g\ar[dashed]{ur}[description]{\exists!}&\\
                B \ar{r}{g} \ar{ur}{0}&C \ar[two heads]{u}{\pi}\ar[bend right]{uur}{p}&
            \end{tikzcd}
        \end{center}
        In particular, we have the following unique morphism from the above diagram such that the following diagram commutes:
        \begin{center}
            \begin{tikzcd}
            &&D\\
            &\cok g\ar[dashed]{ur}[description]{\exists!}&\\
                A \ar{r}{g\circ f} \ar{ur}{0}&C \ar[two heads]{u}{\pi}\ar[bend right]{uur}{p}&
            \end{tikzcd}
        \end{center}
    \end{proof}
\end{lemma}
    
    \begin{corollary}\label{cor:comp with epic and im}
    If $A\overset{f}\twoheadrightarrow B \xrightarrow{g} C$, then $\im (g\circ f)=\im g$.
    \begin{proof}
        $\cok (g\circ f)=\cok f$, thus
        \[
        \im (g\circ f)=\ker \cok (g\circ f)=\ker \cok g=\im g
        \]
    \end{proof}
\end{corollary}
\begin{lemma}\label{lem:double comp with monic and ker}
    If $\ker g \xhookrightarrow{\varphi} A\xhookrightarrow{f}B \xrightarrow[]{g}C$ such that $f\circ \varphi$ is the inclusion\\  $\iota:\ker g\hookrightarrow B$, then $\ker (g\circ f)=\ker g$.
    \begin{proof}
        We have the following diagram commutes:
        \begin{center}
            \begin{tikzcd}
                &C\\
                \ker g\ar[hookrightarrow]{r}{\varphi} \ar{ur}{0}& A \ar{u}[swap]{g\circ f}\\
            \end{tikzcd}
        \end{center}
        because
        \[
        g\circ f\circ \varphi=g\circ \iota=0
        \]
        Now if there exists some morphism $h:D\to A$ such that $g\circ f \circ h=0$, then
        \begin{center}
            \begin{tikzcd}
                &&C\\
                &\ker g\ar[hookrightarrow]{r}{\iota} \ar{ur}{0}& B \ar{u}[swap]{g}\\
                D\ar[dashed]{ur}[description]{\exists!} \ar[bend right]{urr}{f\circ h}
            \end{tikzcd}
        \end{center}
        If $\phi$ is the induced morphism, then 
        \[
        f\circ \varphi\circ \phi=\iota\circ \phi=f\circ h
        \]
        Because $f$ is monic, we get that $\varphi \circ \phi=h$ so that the following diagram commutes:
        \begin{center}
            \begin{tikzcd}
                &&C\\
                &\ker g\ar[hookrightarrow]{r}{\varphi} \ar{ur}{0}& A \ar{u}[swap]{g\circ f}\\
                D\ar[dashed]{ur}[description]{\phi} \ar[bend right]{urr}{h}
            \end{tikzcd}
        \end{center}
    \end{proof}
\end{lemma}
\begin{lemma}\label{lem:ker of comp into ker of first}
    If $A\xrightarrow{f}B\xrightarrow{g}C$, then there is a canonical monomorphism $\ker f\hookrightarrow \ker (g\circ f)$.
\end{lemma}
\begin{proof}
    The canonical morphism is the one induced in the following commutative diagram:
    \begin{center}
        \begin{tikzcd}
            &&C\\
            &\ker (g\circ f) \ar[hook]{r}{i} \ar{ur}{0}&A \ar{u}{g\circ f}\\
            \ker f \ar[bend right,hook]{urr}[hook]{j} \ar[dashed]{ur}[description]{\exists!}
        \end{tikzcd}
    \end{center}
    where $g\circ f\circ j=g\circ 0=0$, and the induced morphism is monic by Lemma \ref{lem:comp monic then monic}.
\end{proof}
\begin{lemma}\label{lem:cok of comp onto cok of second}
    If $A\xrightarrow{f}B\xrightarrow{g}C$, then there is a canonical epimorphism $\cok (g\circ f) \twoheadrightarrow \cok g$.
\end{lemma}
\begin{proof}
    The canonical morphism is the one induced in the following commutative diagram:
    \begin{center}
        \begin{tikzcd}
            && \cok g\\
            & \cok (g\circ f) \ar[dashed]{ur}[description]{\exists!}\\
            A \ar{r}{g\circ f} \ar{ur}{0}& C \ar[two heads]{u}{p} \ar[two heads, bend right]{uur}[swap]{q}
        \end{tikzcd}
    \end{center}
    where $q\circ g\circ f=0\circ f=0$, and the induced morphism is epic by Lemma \ref{lem:comp epic then epic}.
\end{proof}
\begin{lemma}\label{lem:im of comp into im of second}
    If If $A\xrightarrow{f}B\xrightarrow{g}C$, then there is a canonical monomorphism $\im (g\circ f) \hookrightarrow \im g$
\end{lemma}
\begin{proof}
    Using the same notation as was used in the Lemmas referenced, by Lemma \ref{lem:cok of comp onto cok of second} we get a morphism $q':\cok(g\circ f)\twoheadrightarrow \cok g$ such that $q=q'\circ p$. Then by Lemma \ref{lem:ker of comp into ker of first}, we get the desired morphism $i':\ker p\hookrightarrow \ker q$ such that $i'=i\circ q'$.
\end{proof}
\subsection*{Results in Abelian Categories}
\begin{lemma}\label{lem:monic iff im is source}
    If $\iota:A\to B$, then $\iota$ is monic if and only if $\im \iota=A$.
\end{lemma}
\begin{proof}
    The forward direction is by definition of an Abelian Category. For the reverse direction, suppose $\im \iota=A$. Then $\iota$ is a kernel of its cokernel, and by Lemma \ref{lem:ker monic} we obtain that $\iota$ is monic.
\end{proof}
\begin{lemma}\label{lem:epic iff coim is target}
    If $\pi: A\to B$, then $\pi$ is epic if and only if $\coim \pi =B$.
\end{lemma}
\begin{proof}
    The forward direction is by definition of an Abelian Category. For the reverse direction, suppose $\coim \pi=B$. Then $\pi$ is a cokernel of its kernel, and by Lemma \ref{lem:cok epic} we obtain that $\pi$ is epic.
\end{proof}
\begin{lemma}\label{lem:double factorization through im}
    If a morphism $f:A\to B$ factorizes as both $\iota \circ q$ and $\iota'\circ q'$ where $\iota'$ is monic and $\iota \circ q$ is the canonical factorization of $f$ through $\im f$, the following diagram commutes:
    \begin{center}
        \begin{tikzcd}
            A\ar{r}{q}\ar{d}{q'}& \ker \pi \ar[hook]{d}{\iota} \ar[dashed]{dl}[description]{\exists!}\\
            \ker \pi' \ar[hook]{r}{\iota'}&B
        \end{tikzcd}
    \end{center}
    \cite{FIT}
    \begin{proof}
        We will let $\pi:B\to \cok \iota$ and $\pi':B\to \cok \iota'$ be the projections. Therefore
        \[
        \pi'\circ f=\pi'\circ \iota'\circ q'=0\circ q'=0
        \]
        Because also $\cok \iota=\cok f$, we get the following commutative diagram:
        \begin{center}
            \begin{tikzcd}
                &&\cok \iota'\\
                &\cok \iota \ar[dashed]{ur}[description]{\exists! \varphi}&\\
                A\ar{ur}{0} \ar{r}{f}& B \ar[two heads]{u}{\pi} \ar[two heads, bend right]{uur}{\pi'}
            \end{tikzcd}
        \end{center}
        Therefore
        \[
        \pi'\circ \iota =\varphi \circ \pi \circ \iota=\varphi \circ 0=0
        \]
        We use the fact that $\pi'\circ \iota=0$ to get the following commutative diagram:
        \begin{center}
            \begin{tikzcd}
                 &&\cok \iota'\\
                &\ker \pi'\ar{ur}{0} \ar[hook]{r}{\iota'}&B\ar[two heads]{u}{\pi'}\\
                \ker \pi \ar[dashed]{ur}[description]{\exists! \chi} \ar[hook, bend right]{urr}{\iota}
            \end{tikzcd}
        \end{center}
         Therefore
        \[
        \iota'\circ q'=\iota \circ q=\iota' \circ \chi \circ q
        \]
        We use the fact that $\iota'$ is monic to obtain
        \[
        q'=\chi \circ q
        \]
        which shows the desired diagram does indeed commute.
    \end{proof}
\end{lemma}
\begin{theorem}\label{thm:map to im is epic}
    For every morphism $f:A\to B$, the following diagram commutes:
    \begin{center}
        \begin{tikzcd}
            &\cok f\\
            \mathrm{im} f\ar[hook]{r}{\iota}\ar{ur}{0}&B\ar[two heads]{u}{\pi}\\
            A\ar[two heads, dashed]{u}[description]{\exists!q} \ar[bend right]{ur}{f}
        \end{tikzcd}
    \end{center}
        \cite{FIT}
\end{theorem}
\begin{proof}
Existence of the morphism $q: A\to \im f$ is simply by definition of $\im f$. The main result is that $q$ is epic. To show this, suppose there are two morphism $g,h:\im f\to C$ such that $g\circ q=h\circ q$. Then 
\[
g\circ q-h\circ q=0\Rightarrow (g-h)\circ q=0 
\]
We will focus our attention on $\ker g-h$, which is the equalizer of $g$ and $h$, from which we obtain the following commutative diagram:
\begin{center}
    \begin{tikzcd}
        &&C\\
        &\ker g-h \ar{ur}{0}\ar[hook]{r}{j}&\im f \ar{u}[swap]{g-h}\\
        A \ar[dashed]{ur}[description]{\exists! p} \ar[bend right]{urr}{q}
    \end{tikzcd}
\end{center}
This implies that
\[
f=\iota \circ q=\iota \circ j\circ p
\]
We notice that $\iota \circ j$ is monic, so by the statement above, we obtain from Lemma \ref{lem:double factorization through im} the following commutative diagram:
\begin{center}
    \begin{tikzcd}
        A\ar{r}{q} \ar{d}{p}& \im f \ar[hook]{d}{\iota}\ar[hook]{dl}[description]{\chi}\\
        \ker g-h \ar[hook]{r}{\iota \circ j}&B
    \end{tikzcd}
\end{center}
Therefore
\[
\iota=\iota \circ j\circ \chi
\]
Using the fact $\iota$ is monic we obtain
\[
\id_{\im f}=j\circ \chi
\]
Using this, we also obtain
\[
j\circ \chi\circ j=\id_{\im f}\circ j=j
\]
Now using the fact $j$ is monic, we get
\[
\chi \circ j=\id_{\ker g-h}
\]
Thus $j$ is an isomorphism and is in particular epic. Then
\[
(g-h)\circ j=0=0\circ j
\]
implies that, by $j$ being epic, that $g-h=0$, or equivalently $g=h$. Thus $q$ is indeed epic.
\end{proof}
\begin{theorem}[The First Isomorphism Theorem or The 1IT]\label{thm:1IT}
    If $f:A\to B$ is a morphism, then $\im f= \coim f$.
\end{theorem}
\begin{proof}
    We have the canonical epimorphism $q:A\to \im f$. Because $f=\iota \circ q$ and $\iota$ is monic, we get by Corollary \ref{cor:comp with monic and coim} that
    \[
    \coim f=\coim (\iota \circ q)=\coim q
    \]
    From Theorem 0.7 we obtain from that $q$ is epic. By definition, in any abelian category the coimage of an epimorphism is the target, so in the case of $q:A\to \im f$
    \[
    \coim q=\im f
    \]
    Hence
    \[
    \coim f=\im f
    \]
\end{proof}
\begin{theorem}[The Third Isomorphism Theorem or the 3IT]\label{thm:3IT}
    If $A\hookrightarrow B\hookrightarrow C$, then $C/B=(C/A)/(B/A)$.
\end{theorem}
\begin{proof}
    To prove this, let $j:A\hookrightarrow B$ and $i: B\hookrightarrow C$ as well as $q=\cok (i\circ j)$ and $p=\cok i$ . We're going to show that $(C/A)/(B/A)$ satisfies the universal property of $C/B$. First, we observe there is a canonical morphism $\iota$ given below:
    \begin{center}
        \begin{tikzcd}
            &&\cok(i\circ j)\\
            &\cok j \ar[dashed]{ur}[description]{\exists!}\\
            A \ar{ur}{0} \ar[hook]{r}{j}& B \ar[two heads]{u}{p} \ar[hook]{r}{i}& C\ar[two heads]{uu}{q}
        \end{tikzcd}
    \end{center}
    We will first show that $\iota$ is monic. We will do this by first proving $\ker(q\circ i)=A$. Suppose that there is some $h:D\to B$ such that $q\circ i\circ h=0$. Then we get the following commutative diagram:
    \begin{center}
        \begin{tikzcd}
            &&\cok(i\circ j)\\
            &\ker q \ar[hook]{r}{i\circ j} \ar{ur}{0}& C \ar[two heads]{u}{q}\\
            D \ar[bend right]{urr}{i\circ h} \ar[dashed]{ur}[description]{\exists!h'}
        \end{tikzcd}
    \end{center}
    Then
    \[
    i\circ j\circ h'=i\circ h
    \]
    implies, by $i$ being monic, that $j\circ h'=h$. Thus $h'$ is the unique morphism satisfying the diagram below, where $\ker q=A$ essentially by definition:
    \begin{center}
        \begin{tikzcd}
            &&\cok(i\circ j)\\
            &A \ar[hook]{r}{j} \ar{ur}{0}& B \ar{u}{q\circ i}\\
            D \ar[bend right]{urr}{h} \ar[dashed]{ur}[description]{\exists!}
        \end{tikzcd}
    \end{center}
    This demonstrates that indeed $A=\ker(q\circ i)$. By Corollary \ref{cor:comp with epic and im}
    \[
    \im \iota=\im(\iota \circ p)
    \]
    which by commutativity is equal to $\im(q\circ i)$. By the 1IT \ref{thm:1IT}, $\im(q\circ i)=\coim(q\circ i)=B/\ker(q\circ i)$. By our work above, $B/\ker(q\circ i)=B/A$. Then we obtain that
    \[
    \im \iota=B/A
    \]
    By Lemma \ref{lem:monic iff im is source}, this shows that $\iota$ is indeed monic.
    We claim that $\cok \iota =\cok i$, where we let $\tau:C\twoheadrightarrow C/B$ be the canonical projection. Suppose that $h\circ \iota=0$ for some morphism $h:C/A\to D$. Therefore
    \begin{align*}
        0=0\circ p=h\circ \iota \circ p=h\circ q \circ i
    \end{align*}
Therefore $h\circ q$ factors uniquely through $\cok i=C/B$ as shown below:
\begin{center}
    \begin{tikzcd}
        &C/B \ar[dashed]{r}[description]{\exists!h'}&D\\
        B \ar[hook]{r}{i} \ar{ur}{0}& C \ar[two heads]{u}{\tau} \ar[two heads]{r}{q}&C/A \ar{u}{h}
    \end{tikzcd}
\end{center}
However, we can also show that $\tau$ factors through $q$, because $\tau \circ i \circ j=0\circ j=0$, so we also have the following commutative diagram:
\begin{center}
    \begin{tikzcd}
        &C/A \ar[two heads]{r}[description]{\exists! \tau'}& C/B\\
        A\ar[hook]{r}{i\circ j} \ar{ur}{0}& C \ar[two heads]{u}{q} \ar[two heads]{ur}{\tau}
    \end{tikzcd}
\end{center}
    Plugging in our result that $\tau=\tau'\circ q$ to the previous result, we obtain that
    \begin{align*}
        h\circ q=h'\circ \tau=h'\circ \tau' \circ q
    \end{align*}
    Now because $q$ is an epimorphism, we obtain that $h=h'\circ \tau'$. The final thing to show is that $\tau'\circ \iota=0$. This is because
    \begin{align*}
        \tau'\circ \iota\circ p=\tau'\circ q \circ i=\tau \circ i=0=0\circ p
    \end{align*}
    and $p$ is epic implies that indeed $\tau'\circ \iota=0$. We have shown that $h$ factors uniquely through $\tau'$ in the below commutative diagram
    \begin{center}
        \begin{tikzcd}
        &&D\\
            &C/B \ar[dashed]{ur}[description]{\exists! h'}\\
            B/A \ar[hook]{r}{\iota}\ar{ur}{0}&C/A \ar[two heads]{u}{\tau'} \ar[bend right]{uur}{h}
        \end{tikzcd}
    \end{center}
    so indeed $C/B=\cok \iota=(C/A)/(B/A)$ because $C/B$ satisfies the universal property of $\cok \iota$.
\end{proof}
\begin{lemma}\label{lem:comp with epic and ker}
    If $A\xtwoheadrightarrow{f}B\xrightarrow{g} C$ and $\ker(g\circ f)=\ker f$, then $g$ is monic.
\end{lemma}
\begin{proof}
    We obtain by taking the cokernel of each side that
    \[
    \coim (g\circ f)=\coim f
    \]
    By the 1IT \ref{thm:1IT}, we obtain that
    \[
    \im (g\circ f)=\im f
    \]
    Thus we have the following commutative diagram:
    \begin{center}
        \begin{tikzcd}
            &&\cok(g\circ f)\\
            \\
            A\ar{uurr}{0}\ar[two heads]{r}{f}&B\ar{r}{g}&C\ar[two heads]{uu}&B=\im (g\circ f) \ar{l}[swap]{g}\ar{uul}[swap]{0}
        \end{tikzcd}
    \end{center}
    By Lemma \ref{lem:ker monic}, $g$ being a kernel is monic.
\end{proof}
\begin{lemma}\label{lem:covariant right exact preserves epic}
    If $F:\fA \to \fB$ is a right exact covariant functor and $f:A\to B$ is epic, then $Ff$ is epic.
\end{lemma}
\begin{proof}
    We have the exact sequence $\ker f \xrightarrow{i} A \xrightarrow{f}B \rightarrow0 $. By right exactness of $F$, then the following is also exact:
    \begin{align*}
        F\ker f \xrightarrow{Fi} FA \xrightarrow{Ff} FB \rightarrow 0
    \end{align*}
    In particular, $\im Ff=FB$ is the target of $Ff$, so by Lemma \ref{lem:epic iff coim is target} $Ff$ is epic.
\end{proof}
\begin{lemma}\label{lem:covariant left exact preserves monic}
    If $F:\fA \to \fB$ is a left exact covariant functor and $f:A\to B$ is monic, then $Ff$ is monic.
\end{lemma}
\begin{proof}
    We have the exact sequence $0\rightarrow A \xrightarrow{f}B\xrightarrow{p}\cok f$. By left exactness of $F$, the following is also exact:
    \begin{align*}
        0\rightarrow FA \xrightarrow{Ff} FB \xrightarrow{Fp} F\cok f
    \end{align*}
    In particular, $\ker Ff=0$ so $Ff$ is monic.
\end{proof}
\begin{lemma}\label{lem:covariant right exact commutes with cok}
    If $F:\fA \to \fB$ is a right exact covariant functor and $f:A\to B$, then $\cok Ff=F\cok f$.
\end{lemma}
\begin{proof}
    We have $A\xrightarrow{f}B \xrightarrow{p}\cok f\rightarrow 0$ is an exact sequence in $\fA$. Then by right exactness,
    \begin{align*}
        &FA \xrightarrow{Ff} FB\xrightarrow{Fp}F\cok f\rightarrow0
    \end{align*}
    is exact. Then we get the following commutative diagram:
    \begin{center}
        \begin{tikzcd}
            &&F\cok f\\
            &\cok Ff \ar[dashed]{ur}[ description]{\exists!}\\
            FA \ar[r, "Ff"] \ar[ur,"0"]&FB \ar[two heads] {u} {\pi}   \ar[bend right, two heads]{uur}{Fp}
            \end{tikzcd}
    \end{center}
    where $Fp$ is epic by Lemma \ref{lem:covariant right exact preserves epic}. Thus by exactness
    \[
    \ker Fp=\im Ff=\ker \pi
    \]
    Therefore
    \begin{align*}
        \ker \pi=\ker Fp\Rightarrow \coim \pi=\coim Fg
    \end{align*}
    Because both $\pi$ and $Fg$ are epic, we also have $\coim \pi=\pi$ and $\coim Fp=Fp$, which proves $\pi=Fp$.
\end{proof}
\begin{corollary}\label{cor:contravariant right exact and ker to cok}
    If $F:\fA \to \fB$ is a right exact contravariant functor and $f:A\to B$, then $\cok Ff=F\ker f$.
\end{corollary}
\begin{proof}
    If $F:\fA \to \fB$ is contravariant, then $F:\fA^{op}\to \fB$ is an equivalent formulation. Because limits in $\fA^{op}$ are colimits in $\fA$ and vice versa,
    \begin{align*}
        \ker f=\cok (f^{op})
    \end{align*}
    implies that by Lemma \ref{lem:covariant right exact commutes with cok}
    \begin{align*}
        F\ker f=F\cok (f^{op})=\cok Ff^{op}=\cok Ff
    \end{align*}
\end{proof}
\begin{lemma}\label{lem:covariant left exact commutes with ker}
    If $F:\fA \to \fB$ is a left exact covariant functor and $f:A\to B$, then $\ker Ff=F\ker f$.
\end{lemma}
\begin{proof}
    We have $0\rightarrow \ker f\xrightarrow{i} A\xrightarrow{f}B$ is exact. By left exactness,
    \begin{align*}
        0\rightarrow F\ker f \xrightarrow{Fi} FA \xrightarrow{Ff} FB
    \end{align*}
    is exact. Therefore we have the following commutative diagram:
    \begin{center}
        \begin{tikzcd}
            &&FB\\
            &\ker Ff \ar[hook]{r}{\iota} \ar{ur}{0}& FA \ar{u}{Ff}\\
            F\ker f \ar[dashed]{ur}[description]{\exists!} \ar[bend right, hook]{urr}{Fi}
        \end{tikzcd}
    \end{center}
    where $Fi$ is monic by Lemma \ref{lem:covariant left exact preserves monic}. By exactness, $\im Fi=\ker Ff$, and because $Fi$ is monic, then $\im Fi=F\ker f$ by Lemma \ref{lem:monic iff im is source}, which proves $\ker Ff=F\ker f$.
\end{proof}
\begin{corollary}\label{cor:contravariant left exact and cok to ker}
    If $F:\fA \to \fB$ is a left exact contravariant functor and $f:A\to B$, then $\ker Ff=F\cok f$.
\end{corollary}
\begin{proof}
    If $F:\fA \to \fB$ is contravariant, then $F:\fA^{op}\to \fB$ is an equivalent formulation. Because limits in $\fA^{op}$ are colimits in $\fA$ and vice versa,
    \begin{align*}
        \cok f=\ker (f^{op})
    \end{align*}
    implies that by Lemma \ref{lem:covariant left exact commutes with ker}
    \begin{align*}
        F\cok f=F\ker (f^{op})=\ker Ff^{op}=\ker Ff
    \end{align*}
\end{proof}
\begin{lemma}\label{lem:covariant exact and commutes with im and coim}
    If $F:\fA \to \fB$ is an exact covariant functor and $f:A\to B$, then $\im Ff=F\im f$ and $\coim Ff=F\coim f$.
\end{lemma}
\begin{proof}
    By Lemmas \ref{lem:covariant left exact commutes with ker} and \ref{lem:covariant right exact commutes with cok}, we have
    \begin{align*}
        F\im f=F\ker \cok f=\ker F\cok f=\ker \cok Ff=\im Ff
    \end{align*}
    as well as
    \begin{align*}
        F\coim f=F\cok \ker f=\cok F\ker f=\cok \ker Ff=\coim Ff
    \end{align*}
\end{proof}
\begin{lemma}\label{lem:contravariant exact and im to coim and coim to im}
     If $F:\fA \to \fB$ is an exact contravariant functor and $f:A\to B$, then $\im Ff=F\coim f$ and $\coim Ff=F\im f$.
\end{lemma}
\begin{proof}
    By Corollaries \ref{cor:contravariant left exact and cok to ker} and \ref{cor:contravariant right exact and ker to cok},
    \begin{align*}
        F\coim f=F\cok \ker f=\ker F\ker f=\ker \cok Ff=\im Ff
    \end{align*}
    and
    \begin{align*}
        F\im f=F\ker \cok f=\cok F\cok f=\cok \ker Ff=\coim Ff
    \end{align*}
\end{proof}
\subsection*{Miscellaneous Results}
\begin{lemma}
        If $D:\N \to \Top$ is a diagram and $D':\N \to \Top$ is another diagram and there exists some embedding $\sigma\in \Nat(D',D)$, then $\colim (D/D') \cong \colim (D)/\colim (D')$.
    \end{lemma}
    \begin{proof}
        Let $D:\N \to \Top$ be a diagram such that $D(i)=X_i$ for every $i\in \N$ with embeddings $\iota_i:X_i\hookrightarrow X_{i+1}$, and $D':\N \to \Top$ is another diagram such that $D(i)=A_i$ for every $i\in \N$  with embeddings $j_i:A_i\hookrightarrow A_{i+1}$, and let $\sigma \in \Nat(D',D)$ be an embedding of diagrams, i.e. $\sigma_i:A_i\hookrightarrow X_i$ is a natural embedding. For each $i\in \N$, let $p_i:X_i \twoheadrightarrow X_i/A_i$ be the quotient map taking $\im \sigma_i$ to a point. For ease of notation, define $X\coloneqq \colim X_i$ and $A\coloneqq \colim A_i$. We observe the following commutative diagram, and in particular, the induced embedding $\kappa:A\hookrightarrow X$:
        \begin{center}
        \begin{tikzcd}
            &&X\\
            \\
            X_i \arrow[hook, bend left=10, crossing over, pos=0.34]{rrrr}[description]{\iota_i} \ar[swap, bend left, hook]{uurr}{f_i}&&A \ar[dashed]{uu}[description]{\exists!}&& X_{i+1} \ar[bend right,hook']{uull}{f_{i+1}}\\
            &A_i\ar[hook]{rr}{j_i}\ar[hook]{ur}{g_i} \ar[hook']{ul}{\sigma_i}&&A_{i+1}\ar[hook']{ul}[swap]{g_{i+1}} \ar[hook]{ur}[swap]{\sigma_{i+1}}
        \end{tikzcd}
    \end{center}
    which commutes because $\sigma$ was natural by assumption. Therefore we let $q:X\twoheadrightarrow X/A$ be the quotient that maps all of $\im \kappa$ to a point. We can create another functor $F:\N \to \Top$ that has objects $X_i/A_i$ and the morphism $\tau_i:X_i/A_i \to X_{i+1}/A_{i+1}$ is defined by the below universal property of $X_i/A_i:$
    \begin{center}
        \begin{tikzcd}
            X_i \ar[hook]{r}{\iota_i} \ar[two heads]{d}{p_i}&X_{i+1} \ar[two heads]{r}{p_{i+1}}&X_{i+1}/A_{i+1}\\
            X_i/A_i \ar[dashed]{urr}[description]{\exists!\tau_i}
        \end{tikzcd}
    \end{center}
    
    
    Now have our two objects of interest in the problem: $X/A$ and $\colim (X_i/A_i)$, defined by the universal properties respectively below:
    \begin{center}
        \begin{tikzcd}
            X \ar[two heads]{d}{q} \ar{r}{\sim}& \bullet\\
            X/A \ar[dashed]{ur}[description]{\exists!}
        \end{tikzcd}
    \end{center}
    \begin{center}
        \begin{tikzcd}
        &\bullet\\
            &\colim (X_i/A_i) \ar[dashed]{u}[description]{\exists!}\\
            X_i/A_i \ar{ur}{h_i} \ar[bend left]{uur} \ar[hook]{rr}{\tau_i}&& X_{i+1}/A_{i+1}\ar{ul}[swap]{h_{i+1}} \ar[bend right]{uul}
        \end{tikzcd}
    \end{center}
    Now we will begin constructing maps via universal properties, and eventually show that the two constructed maps are isomorphisms -- i.e. homeomorphisms. We first notice that 
    \[
    q\circ f_i \circ \sigma_i=q\circ \kappa \circ g_i=c_*\circ g=c_*
    \]
    where for the rest of the homework we let $c_*$ be a constant map. Thus $q\circ f_i$ is constant on $\sigma_i$, hence we obtain a morphism $\varphi_i:X_i/A_i \to X/A$ for each $i\in \N$ given as follows:
    \begin{center}
        \begin{tikzcd}
            X_i \ar{r}{f_i} \ar[two heads]{d}{p_i}& X \ar[two heads]{r}{q}& X/A\\
            X_i/A_i \ar[dashed]{urr}[description]{\exists!\varphi_i}
        \end{tikzcd}
    \end{center}
    Now we observe one more thing:
    \begin{align*}
        \varphi_{i+1}\circ \tau_i\circ p_i\\
        =\varphi_{i+1}\circ p_{i+1}\circ \iota_i\\
        =q\circ f_{i+1}\circ \iota_i\\
        =q\circ f_i\\
        =\varphi_i\circ p_i
    \end{align*}
    which implies, because each $p_i$ is an epimorphism -- i.e. surjective -- that $\varphi_i=\varphi_{i+1}\circ \tau_{i+1}$. Therefore we get the following induced morphism $\Phi:\colim (X_i/A_i)\to X/A$ below:
    \begin{center}
        \begin{tikzcd}
        &X/A\\
            &\colim (X_i/A_i) \ar[dashed]{u}[description]{\exists!\Phi}\\
            X_i/A_i \ar{ur}{h_i} \ar[bend left]{uur}{\varphi_i} \ar[hook]{rr}{\tau_i}&& X_{i+1}/A_{i+1}\ar{ul}[swap]{h_{i+1}} \ar[bend right]{uul}[swap]{\varphi_{i+1}}
        \end{tikzcd}
    \end{center}
    We eventually will show $\Phi$ is an isomorphism. For now though, we turn our attention to the following property:
    \begin{align*}
        h_{i+1}\circ q_{i+1} \circ \iota_i\\
        =h_{i+1}\circ \tau_i \circ p_i\\
        =h_i\circ p_i
    \end{align*}
    by construction of $\tau_i$. Thus we get another morphism $\phi:X\to \colim (X_i/A_i)$ given below:
    \begin{center}
        \begin{tikzcd}
            &&\colim(X_i/A_i)\\
            \\
            X_i \arrow[hook, bend left=10, crossing over, pos=0.34]{rrrr}[description]{\tau_i} \ar[swap, bend left]{uurr}{h_i}&&X \ar[dashed]{uu}[description]{\exists!\phi }&& X_{i+1} \ar[bend right]{uull}{h_{i+1}}\\
            &X_i\ar[hook]{rr}{j_i}\ar[hook]{ur}{f_i} \ar[two heads]{ul}{p_i}&&X_{i+1}\ar[hook']{ul}[swap]{f_{i+1}} \ar[two heads]{ur}[swap]{p_{i+1}}
        \end{tikzcd}
    \end{center}
    We claim that $\phi \circ \kappa=c_*$. To show this, we notice that for $a\in A$, there exists some $a'\in A_i$ for some $i$ such that $a=g_i(a')$, so if we can show that for arbitrary $g_i$ it is constant, we are done. We observe
    \begin{align*}
        \phi \circ \kappa \circ g_i\\
        = \phi \circ f_i\circ \sigma_i\\
        =h_i\circ p_i\circ \sigma_i\\
        =h_i\circ c_*\\
        =c_*
    \end{align*}
    as desired. Therefore $\phi$ descends downstairs to a map $\Psi:X/A\to \colim(X_i/A_i)$ shown below:
    \begin{center}
        \begin{tikzcd}
            X \ar{r}{\phi} \ar[two heads]{d}{q}& \colim (X_i/A_i)\\
            X/A \ar[dashed]{ur}[description]{\exists!\Psi}
        \end{tikzcd}
    \end{center}
    We now claim that $\Phi\circ \Psi=\altid_{X/A}$ and $\Psi \circ \Phi=\altid_{\colim (X_i/A_i)}$. We will use uniqueness of the maps induced by universal properties to prove both. We observe first that
    \begin{align*}
        \Phi \circ \Psi \circ q=\Phi \circ \phi
    \end{align*}
    Now, we realize that every element of $x\in X$ has the property that there exists some $i\in \N$ such that there exists some $x'\in X_i$ where $x=f_i(x')$. Therefore 
    \begin{align*}
        \Phi \circ \phi \circ f_i\\
        =\Phi \circ h_i\circ p_i\\
        =\varphi_i \circ p_i\\
        =q\circ f_i
    \end{align*}
    shows $\Phi\circ \Psi\circ q$ acts the same as $q$ on every $\im f_i$, hence indeed $\Phi \circ \Psi \circ q=q$. Then we observe the following commutative diagram, where the unique arrow is satisfied by both $\Phi \circ \Psi$ and $\altid_{X/A}$, proving, by uniqueness, the two are equal:
    \begin{center}
        \begin{tikzcd}
            X \ar[two heads]{r}{q} \ar[two heads]{d}{q}& X/A\\
            X/A \ar[dashed]{ur}[description]{\exists!}
        \end{tikzcd}
    \end{center}
    For the second claim, we observe
    \begin{align*}
        \Psi \circ \Phi \circ h_i\\
        =\Psi \circ \varphi_i
    \end{align*}
    and
    \begin{align*}
        \Psi \circ \varphi_i \circ p_i\\
        =\Psi \circ q\circ f_i\\
        =\phi \circ f_i\\
        =h_i\circ p_i
    \end{align*}
    The second equation shows, because $p_i$ is an epimorphism, that $h_i=\Psi \circ \varphi_i$, so by the first equation we get $\Psi \circ \Phi\circ h_i=h_i$, thus both $\Psi \circ \Phi$ and $\altid_{\colim(X_i/A_i)}$ satisfy the unique arrow from the universal property below:
    \begin{center}
        \begin{tikzcd}
        &\colim(X_i/A_i)\\
            &\colim (X_i/A_i) \ar[dashed]{u}[description]{\exists!}\\
            X_i/A_i \ar{ur}{h_i} \ar[bend left]{uur}{h_i} \ar[hook]{rr}{\tau_i}&& X_{i+1}/A_{i+1}\ar{ul}[swap]{h_{i+1}} \ar[bend right]{uul}[swap]{h_{i+1}}
        \end{tikzcd}
    \end{center}
    This proves that $\Phi$ (or equivalently $\Psi$) are homeomorphisms, so the claim that $\colim (X_i/A_i)\cong \colim (X_i)/\colim(A_i)$ is true.
    \end{proof}
\section{}
\subsection{}
There are no exercises in this section.
\subsection{}
\subsubsection{A}\label{1.2.A}
\begin{proof}
\begin{enumerate}[(a)]
    \item If we have a groupoid $\mathscr{C}$ with one object $X$, we could define the group of $\mathcal{C}$ to be $\Aut(X)$. On the other hand if we're given a group $G$ by the standard definition, we could define a category with one object, namely the underlying set of $G$, where the morphisms are defined by the action of the elements of $G$, and where composition of morphisms is given by multiplication of the elements.
    \item Consider the following category:
    \begin{center}
        \begin{tikzcd}
        A \arrow[r, leftrightarrow] & B
    \end{tikzcd}
    \end{center}
    This is not a group because it has two objects, or by interpreting the morphisms as the elements of the set, we cannot compose the morphism $A\to B$ with itself, so our operation is not always defined.
\end{enumerate}
\end{proof}
\subsubsection{B}\label{1.2.B}
\begin{proof}
    Since the subcategory of $\fC$ consisting of the single object $A$ and the morphisms $\Aut(A)$ all have inverses, we have a monoid that is also a groupoid, a.k.a. a group.\\
    \newline
    For Example 1.2.2, given any set $S$, $\Aut(S)$ is the set of all bijections from $S$ to itself, a.k.a. the permutation group of $S$.\\
    \newline
    For Example 1.2.3, given any $k$ vector space $V$, $\Aut(V)$ is the set of all bijective linear transformations from $V$ to itself. For $V$ with dimension $n$, these can be interpreted as the group of $n\times n$ matrices with entries in $k$.\\
    \newline
    Suppose $A,B\in \mathscr{C}$ are isomorphic, and let $\varphi\in \Mor(A,B)$ be an isomorphism. For any $f\in \Aut(A)$, we can define a map $\phi:\Aut(A)\to \Aut(B)$ that acts by
    \begin{equation*}
        f\mapsto \varphi \circ f \circ \varphi^{-1}
    \end{equation*}
    To demonstrate $\phi$ is an isomorphism, we need to show it has an inverse. We do this by letting $\Tilde{\phi}:\Aut(B)\to \Aut(A)$ that acts by
    \begin{equation*}
        g\mapsto \varphi^{-1}\circ g\circ \varphi
    \end{equation*}
    Then
    \begin{align*}
        \phi \circ \Tilde{\phi}(g)=\phi(\varphi^{-1}\circ g \circ \varphi)=\varphi \circ (\varphi^{-1}\circ g \circ \varphi)\circ \varphi^{-1}=\id_B \circ g \circ \id_B=g
    \end{align*}
    and 
    \begin{align*}
        \Tilde{\phi}\circ \phi(f)=\Tilde{\phi}(\varphi \circ f \circ \varphi^{-1})=\varphi^{-1}\circ (\varphi \circ f \circ \varphi^{-1}) \circ \varphi=\id_A \circ f \circ \id_A=f
    \end{align*}
    so indeed $\Tilde{\phi}=\phi^{-1}$. Also
    \begin{align*}
        \phi(f\circ g)=\varphi \circ f\circ g \circ \varphi^{-1}=\varphi\circ f \circ \varphi^{-1}\circ \varphi \circ g \circ \varphi^{-1}=\phi(f)\circ \phi(g)
    \end{align*}
    Therefore $\phi$ (and similarly $\phi^{-1}$) preserve compositions of morphisms.
\end{proof}

\subsubsection{C}\label{1.2.C}
\begin{proof}
    We wish to show that the following diagram commutes for all $V,U\in f.d.Vec_k$ and $T\in \Mor(V,U)$:
    \begin{center}
        \begin{tikzcd}
            V \arrow[r, "T"] \arrow [d, "m_V"]& U\arrow[d, "m_U"]\\
            V^{\lor \lor} \arrow[r, "T^{\lor \lor}"]& U^{\lor \lor}
        \end{tikzcd}
    \end{center}
    as well as that $m_V$ is an isomorphism. We first define $m_V:V\to V^{\lor \lor}$ as
    \begin{align*}
        m_V(x)(f)=f(x)
    \end{align*}
    for any $f\in V^\lor$ and any $x\in V$. Then 
    \begin{align*}
        m_V(x+y)(f)=f(x+y)=f(x)+f(y)=(m_V(x)+m_V(y))(f)
    \end{align*}
    and
    \begin{align*}
        m_V(cx)(f)=f(cx)=cf(x)=cm_V(x)(f)
    \end{align*}
    so indeed $m_V\in \Mor(V,V^{\lor \lor})$. 

    
    To construct an inverse to $m_V$, we simultaneously fix bases for all finite dimensional vector spaces so that if $\{e_1,\dots, e_n\}$ be a basis for $V$, we let $\{\epsilon_1, \dots, \epsilon_n\}$ be the corresponding dual basis for $V^{\lor}$, meaning that for each $1\le i,j\le n$,
    \begin{align*}
        \epsilon_i(e_i)=\left\{
    \begin{array}{lr}
        1, & \text{if } i =j\\
        0, & \text{if } i\ne j
    \end{array}
\right\}
    \end{align*}
    %For an arbitrary $\varphi\in V^{\lor \lor}$ we define $c_i\coloneqq \varphi(\epsilon_i)$ for each $1\le i \le n$. Then
    We define $\Tilde{m_V}:V^{\lor \lor}\to V$ as
    \begin{align*}
        \Tilde{m_V}(\varphi)=\sum_{k=1}^n\varphi(\epsilon_i) e_i
    \end{align*}
    Then
    \begin{align*}
        m_V\circ \Tilde{m_V}(\varphi)(\sum_i a_i \epsilon_i)=m_V(\sum_i \varphi(\epsilon_i)e_i)(\sum_i a_i \epsilon_i)=\sum_i a_i \epsilon_i(\sum_j \varphi(\epsilon_j)e_j)\\
        =\sum_i a_i \varphi(\epsilon_i)=\varphi(\sum_i a_i \epsilon_i)
    \end{align*}
    which implies that $m_V\circ \Tilde{m_V}=\id_{V^{\lor \lor}}$. On the other hand for any $\epsilon_j$
    \begin{align*}
        m_V(\sum_i b_ie_i)(\epsilon_j)=\epsilon_j(\sum_i b_ie_i)=\sum_i b_i \epsilon_j(e_i)=b_j
    \end{align*}
    Then we clearly see that 
    \begin{align*}
        \Tilde{m_V}\circ m_V(\sum_i b_ie_i)=\sum_i b_i e_i
    \end{align*}
    so additionally $\Tilde{m_V}\circ m_V=\id_V$ implies that as desired $\Tilde{m_V}=m_V^{-1}$ and that $m_V$ is an isomorphism. Now to prove
    \begin{center}
        \begin{tikzcd}
            V \arrow[r, "T"] \arrow [d, "m_V"]& U\arrow[d, "m_U"]\\
            V^{\lor \lor} \arrow[r, "T^{\lor \lor}"]& U^{\lor \lor}
        \end{tikzcd}
    \end{center}
    commutes, we first observe that for any $\varphi\in V^{\lor \lor}$ and any $g\in U^\lor$, $T^{\lor \lor}(\varphi)(g)=\varphi(g\circ T)$ which makes sense because $g\circ T:V\to k$. Therefore if $\{d_1,\dots, d_m\}$ is a basis for $U$ with dual basis $\{\delta_1,\dots, \delta_m\}$
    \begin{align*}
        m_U\circ T(\sum_i a_i e_i)=m_U(\sum_i a_i Te_i)
    \end{align*}
    and
    \begin{align*}
        m_U(\sum_i a_i Te_i)(\sum_j \alpha_j \delta_j)=\sum_j\alpha_j \delta_j(\sum_i a_i Te_i)\\
        =\sum_j \alpha_j\delta_j(\sum_i a_i \sum_k c_k^i d_k)=\sum_j \alpha_j \sum_i a_i c^i_j
    \end{align*}
    where we have rewritten each
    \[
    Te_i=\sum_{k=1}^m c^i_k d_i
    \]
    On the other hand,
    \begin{align*}
        T^{\lor \lor}(m_V(\sum_i a_i e_i))(\sum_j \alpha_j \delta_j)=m_V(\sum_i a_ie_i)(\sum_j \alpha_j \delta_j\circ T)
        =\sum_j \alpha_j \delta_j \circ T(\sum_i a_ie_i)\\
        =\sum_j \alpha_j \delta_j (\sum_i a_i Te_i)=\sum_j \alpha_j\delta_j (\sum_i a_i \sum_k c^i_k d_k)=\sum_j \alpha_j \sum_i a_i c^i_j
    \end{align*}
    which proves the diagram does indeed commute.
\end{proof}
\subsubsection{D}\label{1.2.D}
\begin{proof}
    First we will simultaneously fix bases for all vector spaces. Then the inverse functor $G:f.d.Vec_k\to \mathscr V$ will map any $V\in f.d.Vec_k$ with dimension $n$ to $k^n$. If $W\in f.d.Vec_k$ has dimension $m$ with fixed basis $\{w_1,\dots,w_m\}$ and $V$ has basis $\{v_1,\dots,v_n\}$, then for any $T:V\to W$ we define $GT\in \Mor(k^n,k^m)$ such that if for each $1\le i \le n$
    \[
    Tv_i=\sum_{j=1}^m c^i_j w_j
    \]
    then
    \[
    GTk_i=\sum_{j=1}^m c^i_j k_j
    \]
    where $\{k_1,\dots, k_n\}$ is a basis for $k^n$ constructed inductively such that for any $k^{n'}\subset k^n$, the basis $\{k_1,\dots, k_{n'}\}$ is a subset of the basis $\{k_1,\dots, k_n\}$.\\
    \newline
    To show $F\circ G$ is naturally isomorphic to $id_{f.d.Vec_k}$, we want to show the following diagram commutes:
    \begin{center}
    \begin{tikzcd}
        V \arrow[r, "T"] \arrow[d,"m_V"]& W \arrow[d,"m_W"]\\
        F\circ G(V) \arrow[r, "F\circ G(T)"] &F\circ G(W)
    \end{tikzcd}
    \end{center}
    where we define $m_V(v_i)=k_i$ for each $1\le i\le n$. $m_V$ is then an isomorphism because its inverse $m_V^{-1}$ is described how you would think: it is the linear map that sends each $k_i$ to $v_i$.  Following the diagram on the bottom,
    \begin{align*}
        F\circ G(T)(m_V(\sum_i a_iv_i))=F\circ G(T)(\sum_i a_ik_i)\\
        =F(\sum_i a_i \sum_j c^i_j k_j)=\sum_i a_i \sum_j c^i_j k_j
    \end{align*}
    On the other hand,
    \begin{align*}
        m_V(T\sum_i a_iv_i)=m_V(\sum_i a_i \sum_j c^i_j w_j)=\sum_i a_i \sum_j c^i_jk_j
    \end{align*}
    so the diagram commutes. To show $G\circ F$ is naturally isomorphic to $id_\mathscr V$, we want to show
    \begin{center}
        \begin{tikzcd}
            k^n \arrow[d, "m_{k^n}"] \arrow [r, "T"]& k^m \arrow[d, "m_{k^m}"]\\
            G\circ F(k^n)\arrow[r, "G\circ F(T)"]& G\circ F(k^m)
        \end{tikzcd}
    \end{center}
    where here $m_{k^n}=\id_{k^n}$. Because $G\circ F(k^n)=k^n$ and preserves bases, the diagram trivially commutes because also $G\circ F(T)=T$.
\end{proof}
\subsection{}
\subsubsection{A}\label{1.3.A}
\begin{proof}
    Suppose both $A$ and $B$ as objects of a category $\fC$ are initial. Then we have
    \begin{center}
        \begin{tikzcd}
            A \arrow[r, "\exists! f", shift left]& B\arrow[l, "\exists! g",shift left]
        \end{tikzcd}
    \end{center}
    because by $A$ being initial $f$ exists and by $B$ being initial $g$ exists. But now we observe
    \begin{center}
        \begin{tikzcd}
            A \arrow[r, "\exists!"]& A
        \end{tikzcd}
    \end{center}
    and because by definition of $\fC$ being a category, $id_A\in \Mor(A,A)$, so the only morphism from $A$ to itself is $\id_A$ by uniqueness. But $f\circ g\in \Mor(A,A)$ implies that $f\circ g=\id_A$. Similarly $g\circ f=\id_B$, so $A\cong B$.\\
    \newline
    If $A,B\in \fC$ are final, then the same diagrams exist but now because $A,B$ are final instead. The same argument holds here.
\end{proof}
\subsubsection{B}\label{1.3.B}
\begin{proof}
    \begin{center}
\begin{tabular}{||c|c|c||} 
 \hline
 Category & Initial Object & Final Object\\ [0.5ex] 
 \hline\hline
 $\Set$ & $\emptyset$ & $\{*\}$ \\ 
 \hline
 $\Ring$ & $\Z$ & $0$ \\
 \hline
 $\Top$& $\emptyset$ & $\{*\}$\\
 \hline
 $\Ssubset(X)$& $\emptyset$ & $X$\\
 \hline
 $\Op(X)$& $\emptyset$& $X$\\
 \hline
\end{tabular}
\end{center}
\end{proof}
\subsubsection{C}\label{1.3.C}
\begin{proof}
    \begin{enumerate}[($\Rightarrow$)]
        \item Assuming $A\hookrightarrow S^{-1}A$, we want to prove $S$ has no zero divisors. Assuming for a contradiction that $s\in S$ is a zero divisor, let $as=0$ for some $a\in A$. Noting that $0\mapsto 0/1$, we also observe that $a\mapsto a/1=0/1$ because $s(1\cdot a-1\cdot 0)=sa=0$. This contradicts the mapping being an injection.
        \item[($\Leftarrow$)]
        Now we assume $S$ has no zero divisors. If $a,b\in A$ are mapped to the same element of $S^{-1}A$, then $a/1=b/1$. This is true if and only if there exists some $s\in S$ such that
        \[
        s(a1-b1)=0\iff s(a-b)=0
        \]
        But $s$ being a non-zero divisor implies that $a-b=0$, hence $a=b$ proving that the canonical map is injective.
    \end{enumerate}
\end{proof}
\subsubsection{D}\label{1.3.D}
\begin{proof}
    Suppose we have an $A$-algebra $B$ such that every element of $A$ is mapped to an invertible element of $B$ via the map $g$. We want to make the following diagram commute:
    \begin{center}
        \begin{tikzcd}
            A \arrow[r] \arrow[rd,"g"]& S^{-1}A \arrow[d, dashed, "\exists!"]\\
            & B
        \end{tikzcd}
    \end{center}
    If we're constructing the unique map $f:S^{-1}A\to B$, by commutativity we have $f(a/1)=g(a)$ for all $a\in A$. Also notice that
    \begin{align*}
        1_B=f(1/1)=f(s/s)=f(s/1)f(1/s)=g(s)f(1/s)
    \end{align*}
    so $f(1/s)=g(s)^{-1}$, which exists since $g$ maps elements of $A$ to invertible elements in $B$. Then
    \[
    f(a/s)=f(a/1)f(1/s)=g(a)g(s)^{-1}
    \]
    means that if $f$ is a morphism, it is uniquely determined by the line above. To show $f$ is linear,
    \begin{align*}
        f(a_1/s_1+a_2/s_2)=f(\frac{s_2a_1+s_1a_2}{s_1s_2})=g(s_2a_1+s_1a_2)g(s_1s_2)^{-1}\\
        =[g(s_2)g(a_1)+g(s_1)g(a_2)]g(s_1)^{-1}g(s_2)^{-1}\\
        =g(a_1)g(s_1)^{-1}+g(a_2)g(s_2)^{-1}=f(a_1/s_1)+f(a_2/s_2)
    \end{align*}
    and
    \begin{align*}
        f(\frac{a_1}{s_1} \frac{a_2}{s_2})=f(\frac{a_1a_2}{s_1s_2})=g(a_1a_2)g(s_1s_2)^{-1}\\
        =g(a_1)g(s_1)^{-1}g(a_2)g(s_2)^{-1}=f(\frac{a_1}{s_1})f(\frac{a_2}{s_2})
    \end{align*}
    which concludes the proof.
\end{proof}
\subsubsection{E}\label{1.3.E}
\begin{proof}
    We will take the construction given in the hint to be $S^{-1}M$ and define the map $\phi:M\to S^{-1}M$ as $m\mapsto \frac{m}{1}$. Clearly this map is an $A$-module map that sends elements of $S$ to invertible elements. We want to show that the following diagram commutes for all $\alpha$ that map elements of $S$ to invertible elements of $N$:
    \begin{center}
        \begin{tikzcd}
            M \arrow[r, "\phi"] \arrow[rd,"\alpha"]& S^{-1}M \arrow[d, dashed, "\exists!"]\\
            & N
        \end{tikzcd}
    \end{center}
    For any such map $\beta:S^{-1}M\to N$, by commutativity we have $\beta(m/1)=\alpha(m)$. We will let $\sigma_s$ to be the isomorphism $s\times \cdot:N\to N$. Then
    \[
    \alpha(m)=\beta(\frac{m}{1})=\beta(s \frac{m}{s})=s\beta(\frac{m}{s})
    \]
    Then applying the isomorphism $\sigma_s^{-1}$ to either side, we get
    \[
    \sigma_s^{-1}\circ \alpha(m)=\beta(\frac{m}{s})
    \]
    which means that if $\beta$ is an $A$-module morphism, then it is uniquely determined by the line above. To show $\beta$ is linear, we see
    \begin{align*}
        \beta(\frac{m_1}{s_1}+\frac{m_2}{s_2})=\beta(\frac{s_2m_1+s_1m_2}{s_1s_2})=\sigma_{s_1s_2}^{-1}\circ \alpha(s_2m_1+s_1m_2)\\
        =\sigma_{s_1s_2}^{-1}(s_2\alpha(m_1)+s_1\alpha(m_2))=\sigma_{s_1}^{-1}\alpha(m_1)+\sigma_{s_2}^{-1}\alpha(m_2)=\beta(\frac{m_1}{s_1})+\beta(\frac{m_2}{s_2})
    \end{align*}
    where we used the fact that $\sigma_{s_1}\circ \sigma_{s_2}=\sigma_{s_1s_2}=\sigma_{s_2s_1}=\sigma_{s_2}\circ \sigma_{s_1}$. Also
    \begin{align*}
        \beta(\frac{a}{s_1} \frac{m}{s_2})=\beta(\frac{am}{s_1s_2})=\sigma_{s_1s_2}^{-1}\alpha(am)=a\sigma_{s_1}^{-1}\circ \sigma_{s_2}^{-1}\alpha(m)=\frac{a}{s_1}\beta(\frac{m}{s})
    \end{align*}
    so $\beta$ is $S^{-1}A$-linear and uniquely satisfies the commutative diagram.
\end{proof}
\subsubsection{F}\label{1.3.F}
\begin{proof}

\begin{enumerate}[(a)]
    \item This is just a special case of the following part of the question.
    \item We define a map $f:S^{-1}\bigoplus M_i\to \bigoplus S^{-1}M_i$ that acts as
    \[
    \frac{(m_i)}{s}\mapsto (\frac{m_i}{s})
    \]
    To show $f$ is linear, we observe
    \begin{align*}
        f(\frac{(m_i)}{s}+\frac{(m_i')}{s'})=f(\frac{(s'm_i+sm_i')}{ss'})=(\frac{s'm_i+sm_i'}{ss'})=(\frac{m_i}{s}+\frac{m_i'}{s'})\\
        =(\frac{m_i}{s})+(\frac{m_i'}{s})=f(\frac{(m_i)}{s})+f(\frac{(m_i')}{s'})
    \end{align*}
    and
    \begin{align*}
        f(\frac{a}{s}\frac{(m_i)}{s'})=f(\frac{(am_i)}{ss'})=(\frac{am_i}{ss'})=\frac{a}{s}(\frac{m_i}{s'})=\frac{a}{s}f(\frac{(m_i)}{s'})
    \end{align*}
    To be completely thorough we should show that $f$ is well defined, but I will not do this for brevity. Now we see that if
    \[
    \frac{(m_i)}{s}\mapsto 0
    \]
    then for each $i$, $\frac{m_i}{s}=0$. This means that for each $m_i$, there exists some $r_i\in S$ such that $r_im_i=0$. But because there are only finitely many $i$, we take $\prod r_i\in S$, and then $(m_i)/s=0$ because
    \[
    \prod r_i(m_i)=(0)=0
    \]
    so $f$ is injective. To show $f$ is surjective, fix any $(\frac{m_i}{s_i})\in \bigoplus S^{-1}M_i$. Then again using the fact that only finitely many $m_i$ are nonzero, we define for each $m_i$ an element $c_i$ of $S$, namely $c_i\coloneqq\prod_{j\ne i} s_j$. Then 
    \[
    f(\frac{(c_i m_i)}{\prod s_i})=(\frac{c_im_i}{\prod s_i})=(\frac{m_i}{s_i})
    \]
    as desired so $f$ is surjective, thus proving $f$ is an isomorphism.
    \item 
    If we let each $M_i=\Z$ and $S=\Z\setminus\{0\}$, then $S^{-1}M_i= \Q$ where we are considering these as $\Z$ modules. Letting $\iota$ be the canonical embedding of $\prod \Z_i \to \prod \Q_i$, then we have
    \begin{center}
        \begin{tikzcd}
            \prod \Z_i \arrow[r, "\phi", hookrightarrow] \arrow[rd, "\iota", hookrightarrow]& S^{-1}\prod \Z_i\arrow[d, dashed, "\exists! \varphi"]\\
            & \prod \Q_i
        \end{tikzcd}
    \end{center}
    However, $\varphi$ does not map to the element $(1,\frac{1}{2},\frac{1}{3},\frac{1}{4},\dots)$. To prove this, we suppose
    \[
    \frac{(n_1,n_2,\dots)}{s}\mapsto (1,\frac{1}{2},\frac{1}{3},\dots)
    \]
    Then
    \[
    s\varphi(\frac{(n_1,n_2,\dots)}{s})=\varphi(s\frac{(n_1,n_2,\dots)}{s})=\varphi(\frac{(n_1,n_2,\dots)}{1})=(n_1,n_2,\dots)
    \]
    But on the other hand, by hypothesis $\varphi(\frac{(n_1,n_2,\dots)}{s})=(1,\frac{1}{2},\dots)$ so also
    \[
    s\varphi(\frac{(n_1,n_2,\dots)}{s})=s(1,\frac{1}{2},\dots)
    \]
    which implies for some nonzero integer $s$, $(s,\frac{s}{2},\frac{s}{3},\dots)=(n_1,n_2,n_3,\dots)$ where each $n_i\in \Z$. This would imply that every prime number $p_i$ divides $s$ because $\frac{s}{p_i}$ would be in the sequence and would have to equal $n_i\in \Z$. But this is a contradiction because there are infinitely many primes, hence no such $s$ can exist. Thus $\varphi$ is not surjective, but $\varphi$ is the unique morphism that preserves the structure of $\prod \Z_i$ which embeds into both $S^{-1}\prod \Z_i$ and $\prod \Q_i$.
\end{enumerate}



\end{proof}
\subsubsection{G}\label{1.3.G}
\begin{proof}
    We have
    \[
    6(1\otimes 2)=1\otimes 12=1\otimes 0=0
    \]
    On the other hand,
    \[
    5(1\otimes 2)=10(1\otimes 1)=10\otimes 1=0\otimes 1=0
    \]
    Therefore
    \[
    1\otimes 1+1\otimes 1=1\otimes 2=(6-5)(1\otimes 2)=6(1\otimes 2)-5(1\otimes 2)=0-0=0
    \]
    We can now see that we only have two elements in $\Z/(10)\otimes_\Z \Z/(12)$, being $0$ and $1\otimes 1$. To show $1\otimes 1\ne 0$, we can show that there is a bilinear map from $\Z/(10)\otimes_\Z \Z/(12)$ to $\Z/(2)$, given by first noticing that any $a\otimes b=ab\otimes 1$, which then we just map $ab\mapsto ab\mod 2$. It's readily verified this is bilinear, and we notice $1\otimes 1\mapsto 1\mod 2$, which is a nonzero element, hence $1\otimes 1\ne 0$ either by the universal property.
\end{proof}
\subsubsection{H}\label{1.3.H}
    For simplicity of the proof we will use the facts that the Hom functor is left exact--proven in Exercise 1.6.F-- and that for all $A$ modules $M,N,P$
    \[
    \Hom_A(M\otimes N,P)\cong \Hom_A(M,\Hom_A(N,P))
    \]
by Exercise 1.5.D. 
\begin{lemma*}
If the sequence $\Hom(C,P)\xrightarrow{g^*}\Hom(B,P)\xrightarrow{f^*}\Hom(A,P)$ is exact for all $P\in \Mod_A$, then $A\xrightarrow{f}B\xrightarrow{g}C$ is exact.
\cite{tensorRexact}
\end{lemma*}
\begin{proof}
    First, we let $P=\cok f=B/ \im f$ and let $\pi$ be the projection from $B$ onto $\cok f$. Then $\pi\in \ker f^*$ because $f^*(\pi)=\pi\circ f=0$, and by exactness $\pi\in \im g^*$. Let $h\in \Hom(C,P)$ such that $g^*(h)=\pi$, or equivalently $h\circ g=\pi$. Then we observe that
    \[
    \ker g\subset \ker \pi=\im f
    \]
    which demonstrates $\ker g\subset \im f$.\\
    To prove the reverse inclusion, we now let $P=C$ and we trace $\id_C$ through the diagram to see
    \[
    0=f^*\circ g^*(\id_C)=f^*(\id_C\circ g)=g\circ f
    \]
    Then clearly $\im f\subset \ker g$. Therefore
    \[
    A\xrightarrow{f}B\xrightarrow{g}C
    \]
    is exact.
\end{proof}
\begin{proof}[Main Result]
    Given $M'\xrightarrow{f} M\xrightarrow{g} M''\to 0$ is exact, we first fix an arbitrary $P\in \Mod_A$ and subsequently apply $\Hom(\cdot,\Hom_A(N,P))$. By left exactness of the Hom functor, the following is exact:
    \[
    0\to \Hom(M'',\Hom_A(N,P))\xrightarrow{g^*} \Hom(M,\Hom_A(N,P))\xrightarrow{f^*} \Hom(M',\Hom_A(N,P))
    \]
    Now we use the fact that $\cdot \otimes N$ is left adjoint to $\Hom(N,\cdot)$ by Exercise 1.5.D so that
    \[
    0\to \Hom(M''\otimes N,P))\xrightarrow{(g\otimes N)^*} \Hom(M\otimes N,P))\xrightarrow{(f\otimes N)^*} \Hom(M'\otimes N,P))
    \]
    is exact for all $P$. The lemma yields that
    \[
    M'\otimes N\xrightarrow{f\otimes N} M\otimes N\xrightarrow{g\otimes N}M''\otimes N
    \]
    is exact. Now to show $g\otimes N$ is surjective given $g$ is, fix any $m''\otimes n\in M''\otimes N$. Because $g$ is surjective, let $g(m)=m''$. Then $g\otimes N(m\otimes n)=g(m)\otimes n=m''\otimes n$ proving that $g\otimes N$ is surjective. This completes the proof.
\end{proof}
\subsubsection{I}\label{1.3.I}
\begin{proof}
    In this category, the objects are pairs $(T,t:M\times N\to T)$ such that $t$ is bilinear, and a morphism $f:T\to T'$ is a morphism of $A$ modules such that $f\circ t=t'$. Defining the tensor product to be the initial objects of this category, by the fact that any initial object in a category is unique up to unique isomorphism, we get the desired result. But for a more concrete proof, suppose we have $(T,t)$ and $(T',t')$ both satisfying the definition of tensor product. Then
    \begin{center}
        \begin{tikzcd}
            M\times N\arrow[rr,"t"] \arrow[dr,"t'"]& & T \arrow[dl, "\exists!f", dashed]\\
            & T'&
        \end{tikzcd}
    \end{center}
    and also
    \begin{center}
        \begin{tikzcd}
            M\times N\arrow[rr,"t'"] \arrow[dr,"t"]& & T' \arrow[dl, "\exists!g", dashed]\\
            & T&
        \end{tikzcd}
    \end{center}
    commute. On the other hand,
    \begin{center}
        \begin{tikzcd}
            M\times N\arrow[rr,"t"] \arrow[dr,"t"]& & T \arrow[dl, "\exists!", dashed]\\
            & T&
        \end{tikzcd}
    \end{center}
    means that $\id_T$ satisfies the definition, as well as $g\circ f$ because $g\circ f\circ t=g\circ t'=t$, so by uniqueness $\id_T=g\circ f$, and a similar argument shows $f\circ g=\id_{T'}$.\\
    \newline
    We could define the product in any category $\fC$ to be the final object in a category whose objects are pairs $(P,p_M,p_N)$ where $P\in \fC$, $p_M\in \Mor_\fC(P,M)$ and $p_N\in \Mor_\fC(P,N)$. The morphisms of the category are morphisms $f\in \Mor_\fC(P',P)$ such that $p_M'=p_M\circ f$ and $p_N'=p_N\circ f$. Again, any final object in a category is unique up to unique isomorphism, so the product is defined up to unique isomorphism.
\end{proof}
\subsubsection{J}\label{1.3.J}
\begin{proof}
    Suppose we have some pair $(T,t)$ as in the previous exercise. To show that
    \begin{center}
        \begin{tikzcd}
            M\times N\arrow[rr,"\phi"] \arrow[dr,"t"]& & M\otimes N \arrow[dl, "\exists!", dashed]\\
            & T&
        \end{tikzcd}
    \end{center}
    If any such $\varphi:M\otimes N\to T$ exists that makes the diagram commute, then by definition
    \[
    \varphi(m\otimes n)=t(m,n)
    \]
    Notice this proves that if $\varphi$ exists, it is unique. To show this $\varphi\in \Hom_A(M\otimes N,T)$ we first need to show that it is well defined. Letting $R$ be the linear subspace of the free module $F(M\times N)$ spanned by all elements of the form
    \begin{align*}
        (m_1+m_2,n)-(m_1,n)-(m_2,n)\\
        (m,n_1+n_2)-(m,n_1)-(m,n_2)\\
        (am,n)-a(m,n)\\
        (m,an)-a(m,n)
    \end{align*}
    we formally have any tensor $m\otimes n=(m,n)+R$ as a coset. But for each basis element $x$ of $R$, we notice that $f(x)=0$ because $f$ is bilinear, so $f\equiv 0$ on $R$. Thus it doesn't matter which representative of the coset $(m,n)+R$ we pick, so $f$ is well defined. 
    To check linearity,
    \[
    \varphi(m_1+m_2\otimes n)=f(m_1+m_2,n)=f(m_1,n)+f(m_2,n)=\varphi(m_1\otimes n)+\varphi(m_2\otimes n)
    \]
    and similarly
    \[
    \varphi(m\otimes n_1+n_2)=f(m,n_1+n_2)=f(m,n_1)+f(m,n_2)=\varphi(m\otimes n_1)+\varphi(m\otimes n_2)
    \]
    and
    \[
    \varphi(am\otimes n)=f(am,n)=af(m,n)=a\varphi(m\otimes n)
    \]
    There are no other linearity relations on $M\otimes N$, so $\varphi$ must be linear on other sums of tensors; indeed $\varphi$ is an $A$-module homomorphism and is the unique one making the diagram commute.
\end{proof}
\subsubsection{K}\label{1.3.K}
\begin{proof}
    \begin{enumerate}[(a)]
        \item 

        We define scalar multiplication by first constructing a bilinear map $\varphi_b:B\times M\to B\otimes_A M$ for each $b\in B$ given by 
    \[
    \varphi_b(b',m)=bb'\otimes m
    \]
    To prove $\varphi_b$ is bilinear,
    \[
    \varphi_b(b_1+b_2,m)=b(b_1+b_2)\otimes m=bb_1+bb_2\otimes m=bb_1\otimes m+bb_2\otimes m=\varphi_b(b_1,m)+\varphi_b(b_2,m)
    \]
    and
    \[
    \varphi_b(b',m_1+m_2)=bb'\otimes m_1+m_2=bb'\otimes m_1+bb'\otimes m_2=\varphi_b(b',m_1)+\varphi_b(b',m_2)
    \]
    as well as
    \[
    \varphi_b(ab',m)=bab'\otimes m=bb'\otimes am=\varphi_b(b',am)=a\varphi_b(b',m)
    \]
    Thus
    \begin{center}
        \begin{tikzcd}
            B\times M \arrow[r] \arrow[dr, "\varphi_b"]& B\otimes_A M \arrow[d, dashed, "\exists!\phi_b"]\\
            & B\otimes_A M
        \end{tikzcd}
    \end{center}
    commutes and we define scalar multiplication by $b$ as the function $\phi_b$. It's easy to verify that
    \[
    \phi_{b_1}\circ \phi_{b_2}(b\otimes m)=\phi_{b_1b_2}(b\otimes m)
    \]
    Thus
    \[
    b_1(b_2 b\otimes m)=(b_1 b_2) b\otimes m
    \]
    By $\phi_b$ being $A$-linear,
    \[
    b(b_1\otimes m_1+b_2\otimes m_2)=b b_1\otimes m_1+b b_2\otimes m_2
    \]
    Also
    \begin{align*}
        \phi_{b_1+b_2}(b\otimes m)=(b_1+b_2)b\otimes m=b_1b+b_2b\otimes m=b_1b\otimes m+b_2b\otimes m\\
        =\phi_{b_1}(b\otimes m)+\phi_{b_2}(b\otimes m)
    \end{align*}
    And finally
    \[
    \phi_1(b\otimes m)=1b\otimes m=b\otimes m
    \]
    so we have indeed defined a $B$-module structure on $B\otimes_A M$. To see that this defines a functor, we want to show that for any $X,Y,Z\in mod_A$, $f\in \Hom_A(X,Y)$ and $g\in \Hom_A(Y,Z)$, that $B\otimes g\circ f=B\otimes g\circ B\otimes f$ where we define $B\otimes f$ as the induced map in the following commutative diagram:
    \begin{center}
        \begin{tikzcd}
            B\times X\arrow[r,"\phi_X"] \arrow[d,"\id_B\times f"]&B\otimes_A X \arrow[d, dashed, "\exists!"]\\
            B\times Y\arrow[r, "\phi_Y"]&B\otimes_A Y
        \end{tikzcd}
    \end{center}
    To be thorough we should prove that $\phi_Y\circ \id_B\times f$ is bilinear, but it is readily verifiable because $f$ is $A$-linear. Now to show that this functor respects compositions, we see
    \[
    B\otimes g\circ B\otimes f(b\otimes x)=B\otimes g(b\otimes f(x))=b\otimes g\circ f(x)=B\otimes g\circ f(b\otimes x)
    \]
    so we do have a functor from $mod_A\to mod_B$.

    \item 

    We have a similar approach for the construction of multiplication: for all $b\in B$ and $c\in C$, we define a map $\varphi_{b,c}:B\times C\to B\otimes_A C$ as
    \[
    \varphi_{b,c}(b',c')=bb'\otimes cc'
    \]
    To show $\varphi_{b,c}$ is $A$-bilinear, we see
    \[
    \varphi_{b,c}(b_1+b_2,c')=b(b_1+b_2)\otimes cc'=bb_1\otimes cc'+bb_2\otimes cc'=\varphi_{b,c}(b_1,c)+\varphi_{b,c}(b_2,c')
    \]
    and
    \[
    \varphi_{b,c}(b',c_1+c_2)=bb'\otimes c(c_1+c_2)=bb'\otimes cc_1+bb'\otimes cc_2=\varphi_{b,c}(b',c_1)+\varphi_{b,c}(b',c_2)
    \]
    as well as
    \[
    \varphi_{b,c}(ab',c')=bab'\otimes cc'=bb'\otimes cac'=\varphi_{b,c}(b',ac')=a\varphi_{b,c}(b',c')
    \]
    \end{enumerate}
    Then we get a commutative diagram induced by the universal property:
    \begin{center}
        \begin{tikzcd}
            B\times C \arrow[r] \arrow[dr, "\varphi_{b,c}" below]& B\otimes_A C \arrow[d, dashed, "\exists!\phi_{b,c}"]\\
            & B\otimes_A C
        \end{tikzcd}
    \end{center}
    Then we use the action of $\phi_{b,c}$ to be multiplication by $b\otimes c$. Therefore
    \begin{align*}
        \phi_{b,c}\circ \phi_{b',c'}(b''\otimes c'')=\phi_{b,c}(b'b''\otimes c'c'')=bb'b''\otimes cc'c''=\phi_{bb',cc'}(b''\otimes c'')
    \end{align*}
    so multiplication is associative. The multiplicative identity is
    \[
    \phi_{1,1}(b\otimes c)=1b\otimes 1c=b\otimes c
    \]
    To show multiplication is distributive,
    \begin{align*}
        \phi_{b,c}(b_1\otimes c_1+b_2\otimes c_2)=\phi_{b,c}(b_1\otimes c_1)+\phi_{b,c}(b_2\otimes c_2)
    \end{align*}
    because $\phi_{b,c}$ is $A$-linear. We notice that because $B,C$ are commutative rings,
    \begin{align*}
        (b_1\otimes c_1)(b_2\otimes c_2)=b_1b_2\otimes c_1c_2=b_2b_1\otimes c2c_1=(b_2\otimes c_2)(b_1\otimes b_1)
    \end{align*}
    implying that multiplication is also right distributive since
    \[
    (b_1\otimes c_1+b_2\otimes c_2)(b\otimes c)=(b\otimes c)(b_1\otimes c_1+b_2\otimes c_2)=(b\otimes c)(b_1\otimes c_1)+(b\otimes c)(b_2\otimes c_2)
    \]
    \[
    =(b_1\otimes c_1)(b\otimes c)+(b_2\otimes c_2)(b\otimes c)
    \]
    This completes the verification of the ring axioms, so indeed $B\otimes_A C$ is a ring.
\end{proof}
\subsubsection{L}\label{1.3.L}
\begin{proof}
    We will use the universal property of tensor products to construct a map $\beta:S^{-1}A\otimes_A M\to S^{-1}M$. We define a map $\alpha:S^{-1}A\times M\to S^{-1}M$ given by $\alpha(\frac{a}{s},m)=\frac{am}{s}$. To show $\alpha$ is $A$ bilinear, we see
    \begin{align*}
        \alpha(\frac{a_1}{s_1}+\frac{a_2}{s_2},m)=\alpha(\frac{s_2a_1+s_1a_2}{s_1s_2},m)=\frac{(s_2a_1+s_1a_2)m}{s_1s_2}=\frac{a_1m}{s_1}+\frac{a_2m}{s_2}=\alpha(\frac{a_1}{s_1},m)+\alpha(\frac{a_2}{s_2},m)
    \end{align*}
    and
    \begin{align*}
        \alpha(\frac{a}{s},m_1+m_2)=\frac{a(m_1+m_2)}{s}=\frac{am_1}{s}+\frac{am_2}{s}=\alpha(\frac{a}{s},m_1)+\alpha(\frac{a}{s},m_2)
    \end{align*}
    as well as
    \begin{align*}
        \alpha(a' \frac{a}{s},m)=\alpha(\frac{a'a}{s},m)=\frac{a'am}{s}=a'\frac{am}{s}=a'\alpha(\frac{a}{s},m)=\frac{aa'm}{s}=\alpha(\frac{a}{s},a'm)
    \end{align*}
    Then $\alpha$ is $A$ bilinear, and hence we get an induced map $\beta:S^{-1}A\otimes_A M\to S^{-1}M$ from the below diagram:
    \begin{center}
        \begin{tikzcd}
            S^{-1}A\times M \arrow[r] \arrow[dr, "\alpha" below]& S^{-1}A\otimes_A M \arrow[d, dashed, "\exists!"]\\
            & S^{-1}M
        \end{tikzcd}
    \end{center}
    We will now construct an inverse to $\beta$. Let $\phi(m)=1\otimes m$. This is clearly $A$ bilinear as well so we obtain a unique $\varphi:S^{-1}M\to S^{-1}A\otimes_A M$
    \begin{center}
        \begin{tikzcd}
            M \arrow[r] \arrow[dr, "\phi" below]& S^{-1}M \arrow[d, dashed, "\exists!"]\\
            & S^{-1}A\otimes_A M
        \end{tikzcd}
    \end{center}
    Then
    \[
    \varphi \circ \beta(\frac{a}{s}\otimes m)=\varphi(\frac{am}{s})=\frac{1}{s}\otimes am=\frac{a}{s}\otimes m
    \]
    and
    \[
    \beta \circ \varphi(\frac{m}{s})=\beta(\frac{1}{s}\otimes m)=\frac{m}{s}
    \]
    so indeed $S^{-1}M\cong S^{-1}A\otimes_A M$ as $A$-modules. We can extend the $A$-module structure into a $S^{-1}A$ module structure by the previous exercise, and in fact the same morphisms we just constructed can be considered to be $S^{-1}A$ linear as well. We can see this by
    \[
    \beta(\frac{a}{s}\frac{a'}{s'}\otimes m)=\frac{a'am}{ss'}=\frac{a'}{s'}\frac{am}{s}=\frac{a'}{s'}\beta(\frac{a}{s}\otimes m)
    \]
    and
    \[
    \varphi(\frac{a}{s}\frac{m}{s'})=\varphi(\frac{am}{ss'})=\frac{1}{ss'}\otimes am=\frac{a}{s}\frac{1}{s'}\otimes m=\frac{a}{s}\varphi(\frac{m}{s'})
    \]
    Therefore they are also isomorphic as $S^{-1}A$ modules as well.
\end{proof}
\subsubsection{M}\label{1.3.M}
\begin{proof}
    We will use the universal property to construct our desired map. We define $\alpha:M\times \bigoplus_{i\in I} N_i\to \bigoplus_{i\in I} M\otimes N_i$ where
    \[
    \alpha(m,\sum_i n_i)=\sum_i m\otimes n_i
    \]
    To verify $\alpha$ is $A$-bilinear,
    \begin{align*}
        \alpha(m_1+m_2,\sum_i n_i)=\sum_i m_1+m_2\otimes n_i=\sum_i m_1\otimes n_i+\sum_i m_2\otimes n_i\\
        =\alpha(m_1,\sum_i n_i)+\alpha(m_2,\sum_i n_i)
    \end{align*}
    and
    \begin{align*}
        \alpha(m,\sum_i n_i+\sum_i n_i')=\alpha(m,\sum_i n_i+n_i')=\sum_i m\otimes n_i+n_i'\\
        =\sum m\otimes n_i+\sum_i m\otimes n_i'=\alpha(m,\sum_i n_i)+\alpha(m,\sum_i n_i')
    \end{align*}
    as well as
    \begin{align*}
        \alpha(am,\sum_i n_i)=\sum_i am\otimes n_i=a\sum_i m\otimes n_i=\sum_i m\otimes an_i=\alpha(m,a\sum_i n_i)
    \end{align*}
Then let $\varphi$ be the unique induced map below:
\begin{center}
        \begin{tikzcd}
            M\times \bigoplus_i N_i \arrow[r] \arrow[dr, "\alpha" below]& M\otimes_A \bigoplus_i N_i \arrow[d, dashed, "\exists!"]\\
            & \bigoplus_i M\otimes_A N_i
        \end{tikzcd}
    \end{center}
    Then $\varphi(m\otimes \sum_i n_i)=\sum_i m\otimes n_i$, and the inverse map $\phi$ is defined as $\phi(\sum_i m\otimes n_i)=m\otimes \sum_i n_i$. The construction of $\phi$ follows from the diagram below:
    \begin{center}
        \begin{tikzcd}
            \bigoplus M\otimes N_i \arrow[r,"\pi_j"] &M\otimes N_j \arrow[r, dashed, "\exists! \varphi_j"] & M\otimes \bigoplus N_i \\
            & M\times N_j\arrow[u]\arrow[ur,"\alpha_j", right]&
        \end{tikzcd}
    \end{center}
    and then defining $\phi=\sum_i \varphi_j\circ \pi_j$ which is well defined because all but finitely many of the $\pi_j$ are nonzero for any given element. These are clearly inverses, hence $M\otimes \bigoplus N_i\cong \bigoplus M\otimes N_i$ as $A$-modules.
\end{proof}
\subsubsection{N}\label{1.3.N}
\begin{proof}
    Letting $S=\{(x,y)\in X\times Y: \alpha(x)=\beta(y)\}$ with the obvious projection maps $\pi_X$ and $\pi_Y$, it is immediate that
    \begin{center}
        \begin{tikzcd}
            S \arrow[r,"\pi_Y"] \arrow[d, "\pi_X"]& Y\arrow[d,"\beta"]\\
            X\arrow[r,"\alpha"]& Z
        \end{tikzcd}
    \end{center}
    commutes by construction of $S$. Now suppose we're given the following commutative diagram:
    \begin{center}
        \begin{tikzcd}
            W \arrow[r,"p_Y"] \arrow[d, "p_X"]& Y\arrow[d,"\beta"]\\
            X\arrow[r,"\alpha"]& Z
        \end{tikzcd}
    \end{center}
    We want to show that
    \begin{center}
        \begin{tikzcd}
            W \arrow[dr, dashed, "\exists!"] \arrow[ddr, bend right=30, "p_X"] \arrow[drr, bend left=30, "p_Y"]&&\\
            &S \arrow[r,"\pi_Y"] \arrow[d, "\pi_X"]& Y\arrow[d,"\beta"]\\
            &X\arrow[r,"\alpha"]& Z
        \end{tikzcd}
    \end{center}
    commutes for some unique map $\varphi:W\to S$. Any such map $\varphi$ that makes the diagram commute has $\pi_X\circ \varphi=p_X$ and $\pi_Y\circ \varphi=p_Y$. It's then clear that if $\varphi(w)=(\varphi_X(w), \varphi_Y(w))$ for all $w\in W$, that then $\varphi_X=p_X$ and $\varphi_Y=p_Y$. Thus uniqueness is proven, and the fact that $\varphi$ makes the diagram commute is trivial so indeed $S=X\times_Z Y$.
\end{proof}
\subsubsection{O}\label{1.3.O}
\begin{proof}
    We claim that if $A,B,C\in Op(X)$ such that $A,B\subset C$, then $A\times_C B=A\cap B$. In $Op(X)$, we observe that there is at most one arrow from any object to any other object, so we needn't prove uniqueness in the universal property argument. It is clear that
    \begin{center}
        \begin{tikzcd}
            A\cap B \arrow[r] \arrow[d]& B\arrow[d]\\
            A\arrow[r]& C
        \end{tikzcd}
    \end{center}
    commutes--notice that commutativity here is just saying that every element of $A\cap B$ is an element of $C$ because the morphisms are inclusions. If we have another open set $W$ such that $W\subset A$ and $W\subset B$, it's clear that every element of $W$ must be an element of $A\cap B$ by definition, which proves that
    \begin{center}
        \begin{tikzcd}
            W \arrow[dr, dashed, "\exists!"] \arrow[ddr, bend right=30] \arrow[drr, bend left=30]&&\\
            &A\cap B \arrow[r] \arrow[d]& B\arrow[d]\\
            &A\arrow[r]& C
        \end{tikzcd}
    \end{center}
    commutes as well, thus proving $A\times_C B=A\cap B$.
\end{proof}
\subsubsection{P}\label{1.3.P}
\begin{proof}
    First of all, given the following data
    \begin{center}
        \begin{tikzcd}
            &W\arrow[d,dashed, "\exists!"]\arrow[bend right]{ddl}[swap]{p_X} \arrow[ddr, bend left=30, "p_Y"]&\\
            &X\times Y\arrow[dl, "\pi_X"] \arrow{dr}[swap]{\pi_Y}&\\
            X& &Y
        \end{tikzcd}
    \end{center}
    we have our unique morphism $\varphi:W\to X\times Y$ that makes the diagram above commute. But by $Z$ being final, there is only one morphism from any object to $Z$, hence the entire diagram below commutes:
    \begin{center}
        \begin{tikzcd}
W\arrow[bend left]{drr}{p_X}
\arrow[bend right]{ddr}[swap]{p_Y}
\arrow[dashed]{dr}[description]{\exists!} & & \\
& X \times Y \arrow{r}[swap]{\pi_Y} \arrow{d}{\pi_X} & Y \arrow{d} \\
& X \arrow{r} & Z
\end{tikzcd}
    \end{center}
    Therefore $X\times Y$ satisfies the definition of $X\times_ZY$, and by the standard universal property argument, are defined up to unique isomorphism.
\end{proof}
\subsubsection{Q}\label{1.3.Q}
\begin{proof}
    The path traced in red below
    \begin{center}
        \begin{tikzcd}
            U\ar[r, red] \ar[d]&V\ar[d,red]\\
            W\ar[r] \ar[d]&X \ar[d,red]\\
            Y\ar[r]&Z
        \end{tikzcd}
    \end{center}
    is equal to
    \begin{center}
        \begin{tikzcd}
            U\ar[r] \ar[d,red]&V\ar[d]\\
            W\ar[r,red] \ar[d]&X \ar[d,red]\\
            Y\ar[r]&Z
        \end{tikzcd}
    \end{center}
    by commutativity of the top square, and by commutativity of the bottom square 
    \begin{center}
        \begin{tikzcd}
            U\ar[r] \ar[d,red]&V\ar[d]\\
            W\ar[r] \ar[d, red]&X \ar[d]\\
            Y\ar[r, red]&Z
        \end{tikzcd}
    \end{center}
    Now we need to show 
    \begin{center}
        \begin{tikzcd}
P\arrow[bend left]{drr}{\chi_2}
\arrow[bend right]{ddr}[swap]{\chi_1}
\arrow[dashed]{dr}[description]{\exists!} & & \\
& U \arrow{r} \arrow{d} & V \arrow{d} \\
& Y \arrow{r} & Z
\end{tikzcd}
    \end{center}
    commutes. We will first use the fact that the lower square is universal to get
    \begin{center}
        \begin{tikzcd}
P\arrow{rr}{\chi_2}
\arrow[bend right]{ddr}[swap]{\chi_1}
\arrow[dashed]{dr}[description]{\exists!\varphi} & & V\ar[d] \\
& W \arrow{r} \arrow{d} & X \arrow{d} \\
& Y \arrow{r} & Z
\end{tikzcd}
    \end{center}
    Now we use this $\varphi:P\to W$ with the universal property of the top square to get
    \begin{center}
        \begin{tikzcd}
P\arrow[bend left]{drr}{\chi_2}
\arrow[bend right]{ddr}[swap]{\varphi}
\arrow[dashed]{dr}[description]{\exists!\phi} & & \\
& U \arrow{r} \arrow{d} & V \arrow{d} \\
& W \arrow{r} & X
\end{tikzcd}
    \end{center}
    It is easily checked that $\phi$ is the desired morphism that makes the original diagram commute, so we have shown that the tower is indeed a Cartesian diagram.
\end{proof}
\subsubsection{R}\label{1.3.R}
\begin{proof}
    We have
    \begin{center}
        \begin{tikzcd}
            X_1\times_Y X_2\ar[r,"\iota_2"] \ar[d,"\iota_1"]& X_2\ar[d,"g"]&\\
            X_1\ar[r,"f"]&Y\ar[dr,"h"]&\\
            && Z
        \end{tikzcd}
    \end{center}
    commuting, then
    \begin{center}
        \begin{tikzcd}
X_1\times_Y X_2\arrow[bend left]{drr}{\iota_2}
\arrow[bend right]{ddr}[swap]{\iota_1}
\arrow[dashed]{dr}[description]{\exists!} & & \\
& X_1\times_ZX_2 \arrow[r] \arrow{d} & X_2 \arrow{d}{h\circ g} \\
& X_1 \arrow{r}{h\circ f} & Z
\end{tikzcd}
    \end{center}
    induces the unique natural morphism demonstrated here.
\end{proof}
\subsubsection{S}\label{1.3.S}
\begin{proof}
    From the previous exercise, the following diagram commutes
    \begin{center}
        \begin{tikzcd}
X_1\times_Y X_2\arrow[bend left]{drr}{\tau_2}
\arrow[bend right]{ddr}[swap]{\tau_1}
\arrow[dashed]{dr}[description]{\exists! \varphi} & & \\
& X_1\times_ZX_2 \arrow[r,"\pi_2"] \arrow{d}{\pi_1} & X_2 \arrow{d}{h\circ g} \\
& X_1 \arrow{r}{h\circ f} & Z
\end{tikzcd}
    \end{center}
    because
    \begin{center}
        \begin{tikzcd}
            X_1\times_Y X_2\arrow{r}{\tau_2} \arrow{d}{\tau_1}& X_2 \arrow{d}{g}\\
            X_1 \arrow{r}{f}& Y
        \end{tikzcd}
    \end{center}
    commutes.
    There is another natural map $\alpha:Y\to Y\times_ZY$ given by
    \begin{center}
        \begin{tikzcd}
Y\arrow[bend left]{drr}{\id_Y}
\arrow[bend right]{ddr}[swap]{\id_Y}
\arrow[dashed]{dr}[description]{\exists! \alpha} & & \\
& Y\times_ZY \arrow{r}{\mu_2} \arrow{d}{\mu_1} & Y \arrow{d}{h} \\
& Y \arrow{r}{h} & Z
\end{tikzcd}
    \end{center}
    Additionally, there is a map $\theta:X_1\times_Z X_2\to Y\times_Z Y$ given by
    \begin{center}
        \begin{tikzcd}
X_1\times_Z X_2\arrow[bend left]{drr}{g\circ \pi_2}
\arrow[bend right]{ddr}[swap]{f\circ \pi_1}
\arrow[dashed]{dr}[description]{\exists!\theta} & & \\
& Y\times_ZY \arrow[r, "\mu_2"] \arrow{d}{\mu_1} & Y \arrow{d}{h} \\
& Y \arrow{r}{h} & Z
\end{tikzcd}
    \end{center}
    Then we would first like to show the following diagram commutes:
    \begin{center}
        \begin{tikzcd}
            X_1\times_YX_2 \arrow{r}{\varphi}\arrow{d}{f\circ \tau_1}& X_1\times_Z X_2 \arrow{d}{\theta}\\
            Y \arrow{r}{\alpha}& Y\times_ZY
        \end{tikzcd}
    \end{center}
    To do this, we will turn to the following commutative diagram:
    \begin{center}
        \begin{tikzcd}
X_1\times_Y X_2\arrow[bend left]{drr}{g\circ \pi_2\circ \varphi}
\arrow[bend right]{ddr}[swap]{f\circ \pi_1\circ \varphi}
\arrow[dashed]{dr}[description]{\exists!} & & \\
& Y\times_ZY \arrow[r, "\mu_2"] \arrow{d}{\mu_1} & Y \arrow{d}{h} \\
& Y \arrow{r}{h} & Z
\end{tikzcd}
    \end{center}
    On one hand, we will show that $\alpha\circ f\circ \tau_1$ makes the diagram commute. We observe
    \[
    \mu_1\circ \alpha\circ f\circ \tau_1=f\circ \tau_1=f\circ \pi_1\circ \varphi
    \]
    as well as
    \[
    \mu_2\circ \alpha \circ f \circ \tau_1=\mu_2\circ \alpha\circ g\circ \tau_2=g\circ \tau_2=g\circ \pi_2\circ \varphi
    \]
    On the other hand, we will show that $\theta\circ \varphi$ makes the diagram commute. We have
    \[
    \mu_1\circ \theta\circ \varphi=f\circ \pi_1\circ \varphi
    \]
    and also
    \[
    \mu_2\circ \theta \circ \varphi=g\circ \pi_2\circ \varphi
    \]
    By uniqueness of the induced map, we obtain that indeed $\theta\circ \varphi=\alpha \circ f\circ \tau_1$. Now we need to show the square is universal. Suppose the following diagram commutes:
    \begin{center}
        \begin{tikzcd}
P\arrow[bend left]{drr}{p_2}
\arrow[bend right]{ddr}[swap]{p_1}
 & & \\
& X_1\times_Y X_2 \arrow[r, "\varphi"] \arrow{d}{f\circ \tau_1} & X_1\times_ZX_2 \arrow{d}{\theta} \\
& Y \arrow{r}{\alpha} & Y\times_Z Y
\end{tikzcd}
    \end{center}
    Then 
    \[
    \alpha\circ p_1=\theta\circ p_2  
    \]
    implies, by applying $\mu_1$ or $\mu_2$ to the left of each equation,
    \[
    p_1=\mu_1\circ \theta \circ p_2=\mu_2\circ \theta \circ p_2
    \]
    which is true by definition if and only if
    \[
    p_1=f\circ \pi_1\circ p_2=g\circ \pi_2\circ p_2
    \]
    Therefore the following diagram commutes:
    \begin{center}
        \begin{tikzcd}
P\arrow[bend left]{drr}{\pi_2\circ p_2}
\arrow[bend right]{ddr}[swap]{\pi_1\circ p_2}
\arrow[dashed]{dr}[description]{\exists!} & & \\
& X_1\times_Y X_2 \arrow[r, "\tau_2"] \arrow{d}{\tau_1} & X_2 \arrow{d}{g} \\
& X_1 \arrow{r}{f} & Y
\end{tikzcd}
    \end{center}
    Let $\chi$ be the induced map, which proves uniqueness. We will show that $\chi$ makes the following diagram commute:
    \begin{center}
        \begin{tikzcd}
P\arrow[bend left]{drr}{p_2}
\arrow[bend right]{ddr}[swap]{p_1}\arrow{dr}{\chi}
 & & \\
& X_1\times_Y X_2 \arrow[r, "\varphi"] \arrow{d}{f\circ \tau_1} & X_1\times_ZX_2 \arrow{d}{\theta} \\
& Y \arrow{r}{\alpha} & Y\times_Z Y
\end{tikzcd}
    \end{center}
    To show $f\circ \tau_1\circ \chi=p_1$, we have
    \[
    f\circ \tau_1\circ \chi=f\circ \pi_1\circ p_2=p_1
    \]
    To show $\varphi \circ \chi=p_2$, we will show that both $\varphi\circ \chi$ and $p_2$ satisfies the following induced map:
    \begin{center}
        \begin{tikzcd}
P\arrow[bend left]{drr}{\pi_2\circ p_2}
\arrow[bend right]{ddr}[swap]{\pi_1\circ p_2} \arrow[dashed]{dr}[description]{\exists!}
 & & \\
& X_1\times_Z X_2 \arrow[r, "\pi_2"] \arrow{d}{\pi_1} & X_2 \arrow{d}{h\circ g} \\
& X_1 \arrow{r}{h\circ f} & Y\times_Z Y
\end{tikzcd}
    \end{center}
    It's obvious that $p_2$ makes the diagram commute. On the other hand,
    \[
    \pi_1\circ \varphi \circ \chi=\tau_1\circ \chi=\pi_1\circ p_2
    \]
    as well as
    \[
    \pi_2\circ \varphi\circ \chi=\tau_2\circ \chi=\pi_2\circ p_2
    \]
    Since both make the diagram commute, by uniqueness, $\varphi\circ \chi=p_2$ which completes the proof.
\end{proof}
\subsubsection{T}\label{1.3.T}
\begin{proof}
    Given an indexed family of sets $A_i$ for $i\in I$, the disjoint union is the set 
    \[
    \coprod_{i\in I} A_i=\bigcup_{i\in I}\{(x,i):x\in A_i\}
    \]
    where each $A_i$ is equipped with a map $\iota_i:A_i\to \coprod_i A_i$ such that
    \[
    \iota_i(x)=(x,i)
    \]
    Now we suppose we have a set $P$ such that for each $i\in I$, there is a map $p_i:A_i\to P$. Then
    
    \begin{center}
        \begin{tikzcd}
            P\\
            \coprod_i A_i\arrow[dashed]{u}{\exists!}\\
            A_i \arrow{u}{\iota_i} \arrow[bend left=50]{uu}{p_i}
        \end{tikzcd}
    \end{center}
    where the unique map $\varphi$ is defined by $\varphi(x,i)=p_i(x)$. This definition is given to us by commutativity, so uniqueness is proven, and the construction is well defined because $\iota_i$ is an injection for each $i\in I$, which proves existence so indeed the disjoint union is the coproduct in $Sets$.
\end{proof}
\subsubsection{U}\label{1.3.U}
\begin{proof}
    Suppose $\beta:A\to B$ and $\gamma:A\to C$ are ring morphisms and \\$\varphi:B\to B\otimes C$ and $\phi:C\to B\otimes C$ are as defined in the exercise. To show $\varphi$ is a ring morphism, we recall that $B$ can be considered an $A$-module where scalar multiplication is defined as $a\cdot b\coloneqq \beta(a) b$. We immediately get $\varphi(1)=1\otimes 1$ which is the identity on $B\otimes C$, so $\varphi$ preserves identities. To show $\varphi$ is linear,
    \begin{align*}
        \varphi(b_1+b_2)=b_1+b_2\otimes 1=b_1\otimes 1+b_2\otimes1=\varphi(b_1)+\varphi(b_2)
    \end{align*}
    and
    \[
    \varphi(b_1b_2)=b_1b_2\otimes 1=(b_1\otimes 1)(b_2\otimes 1)=\varphi(b_1)\varphi(b_2)
    \]
    An almost identical argument shows $\phi$ is a ring morphism as well. Lastly, we suppose we have a ring $P$ with morphisms $f:B\to P$ and $g:C\to P$ such that the following diagram commutes:
    \begin{center}
        \begin{tikzcd}
            P& C\arrow[l, "g"]\\
            B\arrow{u}{f}& A \arrow{l}{\beta} \arrow{u}{\gamma}
        \end{tikzcd}
    \end{center}
    To show
    \begin{center}
        \begin{tikzcd}
            P&&\\
            & B\otimes_A C \arrow[dashed]{ul}[description]{\exists!}&C\arrow{l}{\phi}\arrow[bend right]{ull}[swap]{g}\\
            &B \arrow{u}{\varphi}\arrow[bend left]{uul}{f}&A\arrow{l}{\beta} \arrow{u}{\gamma}
        \end{tikzcd}
    \end{center}
    commutes, we see that if any such map $\chi:B\otimes_A C\to P$ exists that satisfies the diagram, $\chi \circ \varphi=f$ and $\chi\circ \phi=g$. This equivalently says $\chi(b\otimes 1)=f(b)$ and $\chi(1\otimes c)=g(c)$. This actually determines the action of $\chi$ entirely because $\chi$ is a ring morphism and thus
    \[
    \chi(b\otimes c)=\chi((b\otimes 1)(1\otimes c))=\chi(b\otimes 1)\chi(1\otimes c)=f(b)g(c)
    \]
    This proves that $\chi$ is unique. To prove existence, we need to show that $\chi$ is a ring morphism. We can use the universal property of tensor products to do so. Define $\alpha:B\times C\to P$ as $\alpha(b,c)=f(b)g(c)$. To show $\alpha$ is $A$ bilinear, we observe
    \begin{align*}
        \alpha(ab,c)=f(ab)g(c)=f(\beta(a)b)g(c)=f\circ \beta(a)f(b)g(c)=g\circ \gamma(a)f(b)g(c)\\
        =f(b)g(\gamma(a)c)=f(b)g(ac)=\alpha(b,ac)=a\alpha(b,c)
    \end{align*}
    and
    \begin{align*}
    \alpha(b_1+b_2,c)=f(b_1+b_2)g(c)=f(b_1)g(c)+f(b_2)g(c)=\alpha(b_1,c)+\alpha(b_2,c)
    \end{align*}
    and
    \[
    \alpha(b,c_1+c_2)=f(b)g(c_1+c_2)=f(b)g(c_1)+f(b)g(c_2)=\alpha(b,c_1)+\alpha(b,c_2)
    \]
    Then by the universal property of tensor products, we get our induced map $\chi$ defined exactly as we require it to be. This proves existence of $\chi$ and completes the proof.
\end{proof}
\subsubsection{V}\label{1.3.V}
\begin{proof}
    Suppose $\pi_1:X\to Y$ and $\pi_2:X\to Z$ are both monomorphisms and suppose we have two morphisms $f,g:W\to X$. We want to show that $\pi_2\circ \pi_1\circ f=\pi_2\circ \pi_1\circ g\Rightarrow f=g$. Supposing $\pi_2\circ \pi_1\circ f=\pi_2\circ \pi_1\circ g$, by $\pi_2$ being monic we have $\pi_1\circ f=\pi_1\circ g$. Now we use the fact that $\pi_1$ is monic to get $f=g$ as desired.
\end{proof}
\subsubsection{W}\label{1.3.W}
\begin{proof}
    \begin{enumerate}
        \item[($\Rightarrow)$]
        We suppose $\pi:X\to Y$ is monic. To prove $X\times_YX$ exists, we claim that $X$ satisfies the definition of $X\times_Y X$. To show this, we want to show
        \begin{center}
        \begin{tikzcd}
            P\arrow[dashed]{dr}[description]{\exists!} \arrow[bend right]{ddr}[swap]{p_1} \arrow[bend left]{drr}{p_2}&&\\
            &X \arrow{r}{\id_X} \arrow{d}{\id_X}& X \arrow{d}{\pi}\\
            &X\arrow{r}{\pi}& Y
        \end{tikzcd}
    \end{center}
    holds. Using the fact that $\pi$ is monic and the fact that $\pi\circ p_1=\pi\circ p_2$ by commutativity to get that $p_1=p_2$, so we just need to show that
    \begin{center}
        \begin{tikzcd}
            P\arrow[dashed]{dr}[description]{\exists!} \arrow[bend right]{ddr}[swap]{p} \arrow[bend left]{drr}{p}&&\\
            &X \arrow{r}{\id_X} \arrow{d}{\id_X}& X \arrow{d}{\pi}\\
            &X\arrow{r}{\pi}& Y
        \end{tikzcd}
    \end{center}
    commutes. The unique morphism is clearly $p$. Thus $X$ satisfies the definition of $X\times_Y X$, and thus the induced morphism is $\id_X$ which is, in particular, an isomorphism.
    \item[$(\Leftarrow)$]
    Now supposing that there is a unique isomorphism $\varphi:X\to X\times_YX$ and that $X\times_Y X$ exists, we will furthermore suppose that $\pi\circ f=\pi\circ g$ for some $f,g:Z\to X$. Notice that 
    \begin{center}
        \begin{tikzcd}
            X\arrow[dashed]{dr}[description]{\exists! \varphi} \arrow[bend right]{ddr}[swap]{\id_X} \arrow[bend left]{drr}{\id_X}&&\\
            &X\times_YX \arrow{r}{\chi_1} \arrow{d}{\chi_2}& X \arrow{d}{\pi}\\
            &X\arrow{r}{\pi}& Y
        \end{tikzcd}
    \end{center}
    commuting and $\varphi$ being an isomorphism implies that $\chi_2=\varphi^{-1}=\chi_1$.
    Then we obtain a map $\phi$ from the below commutative diagram:
    \begin{center}
        \begin{tikzcd}
            Z\arrow[dashed]{dr}[description]{\exists!} \arrow[bend right]{ddr}[swap]{f} \arrow[bend left]{drr}{g}&&\\
            &X\times_YX \arrow{r}{\varphi^{-1}} \arrow{d}{\varphi^{-1}}& X \arrow{d}{\pi}\\
            &X\arrow{r}{\pi}& Y
        \end{tikzcd}
    \end{center}
    Therefore $f=\varphi^{-1}\circ \phi$ and $g=\varphi^{-1}\circ \phi$. This implies that $f=g$ as desired so $\pi$ is monic.
    \end{enumerate}
\end{proof}
\subsubsection{X}\label{1.3.X}
\begin{proof}
    We will use the same variables in this exercise as in Exercise 1.3.S. Let $\varphi:X_1\times_YX_2\to X_1\times_ZX_2$ be induced in the following diagram, using the fact $\pi$ is monic here so that $\pi\circ f \circ \tau_1=\pi\circ g \circ \tau_2\Rightarrow f\circ \tau_1=g\circ \tau_2$:
    \begin{center}
        \begin{tikzcd}
            X_1\times_Y X_2\arrow[dashed]{dr}[description]{\exists!} \arrow[bend right]{ddr}[swap]{\tau_1} \arrow[bend left]{drr}{\tau_2}&&&\\
            &X_1\times_ZX_2 \arrow{r}{\pi_2} \arrow{d}{\pi_1}& X_2 \arrow{d}{g}&\\
            &X_1\arrow{r}{f}& Y\arrow{dr}{\pi}&\\
            &&&Z
        \end{tikzcd}
    \end{center}
    We also use the fact that $\pi$ is monic implies both $\mu_1,\mu_2$ from Exercise 1.3.S are equal to $\alpha^{-1}$, where $\alpha$ is an isomorphism by Exercise 1.3.X. Then
    \[
    f\circ \pi_1=\mu_1\circ \theta=\alpha^{-1}\circ \theta\Rightarrow \alpha\circ f\circ \pi_1=\theta
    \]
    Now using the magic diagram from Exercise 1.3.S, we have
    \begin{center}
        \begin{tikzcd}
            X_1\times_Z X_2 \arrow[dashed]{dr}[description]{\exists!} \arrow[bend right]{ddr}[swap]{f\circ \pi_1} \arrow[bend left]{drr}{\id_{X_1\times_Z X_2}}&&\\
            &X_1\times_YX_2 \arrow{r}{\varphi} \arrow{d}{f\circ \tau_1}& X_1\times_Z X_2 \arrow{d}{\alpha\circ f\circ \pi_1}\\
            &Y\arrow{r}{\alpha}& Y\times_ZY 
        \end{tikzcd}
    \end{center}
    Let $\phi$ be the map induced in the above diagram, where it is immediate that $\varphi\circ \phi=\id_{X_1\times_Z X_2}$. To show $\phi\circ \varphi=\id_{X_1\times_Y X_2}$, we will show it satisfies the diagram below, which suffices because clearly $\id_{X_1\times_Y X_2}$ also does:
    \begin{center}
        \begin{tikzcd}
            X_1\times_Y X_2 \arrow[dashed]{dr}[description]{\exists!} \arrow[bend right]{ddr}[swap]{\tau_1} \arrow[bend left]{drr}{\tau_2}&&\\
            &X_1\times_YX_2 \arrow{r}{\tau_2} \arrow{d}{\tau_1}& X_2\arrow{d}{g}\\
            &X_1\arrow{r}{f}& Y 
        \end{tikzcd}
    \end{center}
    We need to show that $\tau_1=\tau_1\circ \phi\circ \varphi$ and that $\tau_2=\tau_2\circ\phi\circ \varphi$. Recalling that $\phi$ is a section of $\varphi$ and that $\tau_1=\pi_1\circ \varphi$ and $\tau_2=\pi_2\circ \varphi$, we have
    \[
    \tau_1\circ \phi\circ \varphi=\pi_1\circ \varphi \circ \phi \circ \varphi =\pi_1\circ \varphi=\tau_1
    \]
    as well as
    \[
    \tau_2 \circ \phi \circ \varphi=\pi_2\circ \varphi \circ \phi \circ \varphi=\pi_2\circ \varphi=\tau_2
    \]
    which proves that $\phi=\varphi^{-1}$ and $\varphi$ is an isomorphism.
\end{proof}
\subsubsection{Y}\label{1.3.Y}
\begin{proof}
    \begin{enumerate}[(a)]
        \item We have the following diagram is commutative for all $f:C\to B$:
        \begin{center}
            \begin{tikzcd}
                \Mor(B,A)\arrow{r}{f^*} \arrow{d}{\iota_B}& \Mor(C,A)\arrow{d}{\iota_C}\\
                \Mor(B,A')\arrow{r}{f^*}&\Mor(C,A')
            \end{tikzcd}
        \end{center}
        Now because $B$ was arbitrary, we let $B=A$, and then we have
        \begin{center}
            \begin{tikzcd}
                \Mor(A,A)\arrow{r}{f^*} \arrow{d}{\iota_A}& \Mor(C,A)\arrow{d}{\iota_C}\\
                \Mor(A,A')\arrow{r}{f^*}&\Mor(C,A')
            \end{tikzcd}
        \end{center}
        Now we track $\id_A$ through the bottom portion of the diagram, letting $g=\iota_A(\id_A)$ and $f\in \Mor(C,A)$ arbitrary to get
        \[
        f^*\circ \iota_A(\id_A)=f^*(g)=g\circ f
        \]
        On the top side of the diagram, we get
    \[
    \iota_C\circ f^*(\id_A)=\iota_C(\id_A\circ f)=\iota_C(f)
    \]
    By commutativity, these two are equal, hence $\iota_C(f)=g\circ f$, determining $\iota_C$ entirely.
    \item 
    Now assuming all of the $\iota_C$ are isomorphisms, we get the following diagram where $g\in \Mor(A,A')$ is as defined in the previous part:
    \begin{center}
            \begin{tikzcd}
                \Mor(A',A)\arrow{r}{g^*} \arrow{d}{\iota_{A'}}& \Mor(A,A)\arrow{d}{\iota_A}\\
                \Mor(A',A')\arrow{r}{g^*}&\Mor(A,A')
            \end{tikzcd}
        \end{center}
        By surjectivity of $\iota_A$, for each $f\in \Mor(A,A')$, there exists a unique $f'\in \Mor(A,A)$ such that $\iota_A(f')=f\iff g\circ f'=f$. On the other side of the diagram, for every $\alpha\in \Mor(A',A')$, there exists a unique $\alpha'\in \Mor(A',A)$ such that $\iota_{A'}(\alpha')=\alpha \iff g\circ \alpha'=\alpha$.\\
        \newline
        Thus if $\alpha=\id_{A'}$, we obtain a section $\alpha'$ of $g$. Now by uniqueness of the first statement, there exists a unique $f'\in \Mor(A,A)$ such that $g\circ f'=g$. But $\id_A$ and $\alpha' \circ g$ both satisfy this requirement, which proves that $\alpha'\circ g=\id_A$, proving that $\alpha'=g^{-1}$.
    \end{enumerate}
    
\end{proof}
\subsubsection{Z}\label{1.3.Z}
\begin{proof}
    \begin{enumerate}[(a)]
        \item If we're given some $f\in \Mor(B,A)$, we want to give a natural transformation $m_C:\Mor(A,C)\to \Mor(B,C)$. We define for every $C\in \fC$, we define
        \[
        m_C(\phi)=\phi\circ f
        \]
        To prove $m$ is indeed a natural transformation, we need to show for every $g:C\to C'$, the following diagram commutes:
        \begin{center}
            \begin{tikzcd}
                \Mor(A,C)\arrow{r}{g_*} \arrow{d}{m_C}& \Mor(A,C')\arrow{d}{m_{C'}}\\
                \Mor(B,C)\arrow{r}{g_*}&\Mor(B,C')
            \end{tikzcd}
        \end{center}
        On the bottom side of the diagram, for any $\phi\in \Mor(A,C)$, we observe
        \[
        g_*\circ m_C(\phi)=g_*(\phi\circ f)=g\circ \phi\circ f
        \]
        on the other hand,
        \[
        m_{C'}\circ g_*(\phi)=m_{C'}(g\circ \phi)=g\circ \phi \circ f
        \]
        so $m$ is a natural transformation.\\
        \newline
        Now if we're given a natural transformation $m$, we get the following commutative diagram for arbitrary $C\in \fC$ and $g\in \Mor(A,C)$:
        \begin{center}
            \begin{tikzcd}
                \Mor(A,A)\arrow{r}{g_*} \arrow{d}{m_A}& \Mor(A,C)\arrow{d}{m_{C}}\\
                \Mor(B,A)\arrow{r}{g_*}&\Mor(B,C)
            \end{tikzcd}
        \end{center}
        Tracking $\id_A$ on the bottom and defining $f\coloneqq m_A(\id_A)\in \Mor(B,A)$, we get
        \[
        g_*\circ m_A(\id_A)=g_*(f)=g\circ f
        \]
        On the top, we get
        \[
        m_C\circ g_*(\id_A)=m_C(g\circ \id_A)=m_C(g)
        \]
        By commutativity, $m_C(g)=g\circ f$ for all $g\in \Mor(A,C)$. Because $f$ uniquely defines $m$, we have obtained a unique morphism from every natural transformation.\\
        \newline
        We define a map $\varphi:\Mor(B,A)\to \Nat(h^A,h^B)$ given as $\varphi(f)=\circ f$, and another map $\phi:\Nat(h^A,h^B)\to \Mor(B,A)$ as $\phi(m)=m_A(\id_A)$. To show these are inverse maps, 
        \[
        \phi \circ \varphi(f)=\phi(\circ f)=\id_A\circ f=f
        \]
        and
        \[
        \varphi \circ \phi(m)=\varphi(m_A(\id_A))=\circ m_A(\id_A)=m
        \]
        by our previous work. Thus $\phi=\varphi^{-1}$ and we have given the desired bijection.
        \item 
        Given any $f\in \Mor(A,B)$, define $\varphi(f)=f\circ$ where $\varphi:\Mor(A,B)\to \Nat(h_A,h_B)$. Similarly to part (a), one can readily check that this defines a natural transformation. We can also do a similar process of tracking the identity to realize that any for natural transformation $m$ and any $g\in \Mor(C,A)$, $m_C(g)=m_A(\id_A)\circ g$. We then define $\phi:\Nat(h_A,h_B)\to \Mor(A,B)$ given by $\phi(m)=m_A(\id_A)$. In a very similar manner to part (a), $\phi=\varphi^{-1}$ so we obtain the bijection we want.
        \item 
        If we're given any natural transformation $m$ from $h^A\to F$, we have that for all $f\in \Mor(B,C)$, the following diagram commutes:
        \begin{center}
            \begin{tikzcd}
                \Mor(A,B)\arrow{r}{f_*} \arrow{d}{m_B}& \Mor(A,C)\arrow{d}{m_{C}}\\
                F(B)\arrow{r}{Ff}&F(C)
            \end{tikzcd}
        \end{center}
        Letting $B=A$, we have
        \begin{center}
            \begin{tikzcd}
                \Mor(A,A)\arrow{r}{f_*} \arrow{d}{m_A}& \Mor(A,C)\arrow{d}{m_{C}}\\
                F(A)\arrow{r}{Ff}&F(C)
            \end{tikzcd}
        \end{center}
        Yet again, we track $\id_A$ on the bottom to get $Ff\circ m_A(\id_A)$, and on the top we get $m_C\circ f_*(\id_A)=m_C(f\circ \id_A)=m_C(f)$. Thus by commutativity, for any $f\in \Mor(A,C)$, 
        \[
        m_C(f)=Ff(m_A(\id_A))
        \]
        We now notice then that $m_C$ is determined entirely by $m_A(\id_A)$, so we define $\varphi:\Nat(h^A,F)\to F(A)$ given by $\varphi(m)=m_A(\id_A)$. On the other hand we define $\phi:F(A)\to \Nat(h^A,F)$ to act as $\phi(\chi)_C(f)=Ff(\chi)$ for any $C\in \fC$ and $f\in \Mor(A,C)$.\\
        \newline
        Then
        \[
        \varphi\circ \phi(\chi)=\varphi( \phi(\chi))=\phi(\chi)_A(\id_A)=F(\id_A)(\chi)=\id_{F(A)}(\chi)=\chi
        \]
        and for any $f\in \Mor(A,C)$,
        \[
        \phi\circ \varphi(m)_C(f)=\phi(m_A(\id_A))_C(f)=Ff(m_A(\id_A))=m_C(f)
        \]
        so indeed $\varphi$ is a bijection.
    \end{enumerate}
\end{proof}

\subsection{}
\subsubsection{A}\label{1.4.A}
\begin{proof}
    We claim that if $F:\mathscr{I}\to \fC$ is a functor and $e\in \fI$ is an initial object, then $\varprojlim A_i=A_e$. Because $e$ is initial, there exists a unique morphism into every $i\in \fI$, so there exists a unique morphism $f_i:A_e\to A_i$ for each $i$. If $W$ is another object in $\fC$ with maps $p_i$ for each $i$ that commutes with everything, there exists a morphism $p_e:W\to A_e$ because $e\in \fI$. We also know that by assumption the following diagram must commute:
    \begin{center}
        \begin{tikzcd}
            &W\arrow{dl}[swap]{p_e}\arrow{dr}{p_i}&\\
            A_e \arrow{rr}{\exists! f_i}&&A_i
        \end{tikzcd}
    \end{center}
    so in particular the following diagram commutes for all $f:i\to j$ and all $i,j\in \fI$.
    \begin{center}
        \begin{tikzcd}
            &W\arrow{d}{p_e} \arrow[bend right]{ddl}[swap]{p_i} \arrow[bend left]{ddr}{p_j}&\\
            &A_e \arrow{dl}{f_i} \arrow{dr}[swap]{f_j}&\\
            A_i \arrow{rr}{Ff}&&A_j
        \end{tikzcd}
    \end{center}
    Uniqueness comes from the fact that any morphism $g:W\to A_e$ that makes the diagram commute in particular makes the following subdiagram commute:
    \begin{center}
        \begin{tikzcd}
            W\arrow{d}{g} \arrow[bend right=40]{dd}[swap]{p_e}\\
            A_e \arrow{d}{\id_{A_e}}\\
            A_e
        \end{tikzcd}
    \end{center}
    so that $g=p_e$.
\end{proof}

\subsubsection{B}\label{1.4.B}
\begin{proof}
    To show $X_1\times_Y X_2$ is the limit of the diagram, we need to show
    \begin{center}
        \begin{tikzcd}
            &X_1\arrow{dr}{f}&&\\
            X_1\times_Y X_2\arrow{ur}{\tau_1} \arrow{dr}{\tau_2}&&Y\arrow{r}{h}&Z\\
            &X_2\arrow{ur}{g}&&
        \end{tikzcd}
    \end{center}
    commutes given the Cartesian square below:
    \begin{center}
        \begin{tikzcd}
            X_1\times_YX_2\arrow{r}{\tau_2} \arrow{d}{\tau_1}& X_2\arrow{d}{g}\\
            X_1\arrow{r}{f}&Y
        \end{tikzcd}
        \end{center}
        But the commutativity of the first diagram is trivial then. Now to show the first diagram is universal, suppose we have the following commutative diagram:
        \begin{center}
        \begin{tikzcd}
            &X_1\arrow{dr}{f}&&\\
            P\arrow{ur}{p_1} \arrow{dr}{p_2}&&Y\arrow{r}{h}&Z\\
            &X_2\arrow{ur}{g}&&
        \end{tikzcd}
    \end{center}
Then we get an induced map from the following diagram:
\begin{center}
    \begin{tikzcd}
        P\arrow[bend right]{ddr}{p_1} \arrow[bend left]{drr}{p_2}\arrow[dashed]{dr}[description]{\exists!}&&\\
        & X_1\times_Y X_2 \arrow{d}{\tau_1}\arrow{r}{\tau_2}& X_2\arrow{d}{g}\\
        &X_1 \arrow{r}{f}&Y
    \end{tikzcd}
\end{center}
    This proves uniqueness. This map $\gamma$ makes the following diagram commute trivially
    \begin{center}
        \begin{tikzcd}
            &X_1\arrow{dr}{f}&&\\
            P\arrow{ur}{p_1} \arrow{dr}{p_2}\arrow{r}{\gamma}&X_1\times_Y X_2\arrow{u}{\tau_1}\arrow{d}{\tau_2}&Y\arrow{r}{h}&Z\\
            &X_2\arrow{ur}{g}&&
        \end{tikzcd}
    \end{center}
    which proves existence. Thus $X_1\times_Y X_2$ is the limit of the diagram.\\
    \newline
    To show $Y\times_{(Y\times_ZY)}X_1\times_Z X_2$ is also the limit of the diagram, we first need to show that the following diagram commutes:
    \begin{center}
        \begin{tikzcd}
            &X_1\arrow{dr}{f}&&\\
            Y\times_{(Y\times_ZY)}X_1\times_Z X_2\arrow{ur}{\pi_1\circ \iota_2} \arrow{dr}{\pi_2\circ \iota_2}&&Y\arrow{r}{h}&Z\\
            &X_2\arrow{ur}{g}&&
        \end{tikzcd}
    \end{center}
    where we're given the following Cartesian diagrams:
    \begin{center}
        \begin{tikzcd}
            Y\times_{(Y\times_ZY)}X_1\times_Z X_2\arrow{r}{\iota_2} \arrow{d}{\iota_1}& X_1\times_Z X_2 \arrow{d}{\theta}\\
            Y\arrow{r}{\alpha}& Y\times_Z Y
        \end{tikzcd}
    \end{center}
    and
    \begin{center}
        \begin{tikzcd}
            X_1\times_Z X_2\arrow{r}{\pi_2} \arrow{d}{\pi_1}&X_2 \arrow{d}{h\circ g}\\
            X_1\arrow{r}{h\circ f}&Z
        \end{tikzcd}
    \end{center}
    and the induced map $\alpha$ below:
\begin{center}
    \begin{tikzcd}
        Y\arrow[bend right]{ddr}{\id_Y} \arrow[bend left]{drr}{\id_Y}\arrow[dashed]{dr}[description]{\exists!\alpha }&&\\
        & Y\times_Z Y \arrow{d}{\mu_1}\arrow{r}{\mu_2}& Y\arrow{d}{h}\\
        &Y \arrow{r}{h}&Z
    \end{tikzcd}
\end{center}
    as well as the induced map $\theta$ from the following:
    \begin{center}
    \begin{tikzcd}
        X_1\times_Z X_2\arrow[bend right]{ddr}{f\circ \pi_1} \arrow[bend left]{drr}{g\circ \pi_2}\arrow[dashed]{dr}[description]{\exists!}&&\\
        & Y\times_Z Y \arrow{d}{\mu_1}\arrow{r}{\mu_2}& Y\arrow{d}{h}\\
        &Y \arrow{r}{h}&Z
    \end{tikzcd}
\end{center}
Now to show
    \begin{center}
        \begin{tikzcd}
            &X_1\arrow{dr}{f}&&\\
            Y\times_{(Y\times_ZY)}X_1\times_Z X_2\arrow{ur}{\pi_1\circ \iota_2} \arrow{dr}{\pi_2\circ \iota_2}&&Y\arrow{r}{h}&Z\\
            &X_2\arrow{ur}{g}&&
        \end{tikzcd}
    \end{center}
    does indeed commute, we observe
    \begin{align*}
        \iota_1=\iota_1\\
        \Rightarrow \mu_1\circ \alpha\circ \iota_1=\mu_2\circ \alpha\circ \iota_1\\
        \Rightarrow \mu_1\circ \theta\circ \iota_2=\mu_2\circ \theta\circ \iota_2\\
        \Rightarrow f\circ \pi_1\circ \iota_2=g\circ \pi_2\circ \iota_2
    \end{align*}
    just by recalling the definitions of each. Now that this diagram commutes, we suppose we have the following commutative diagram to prove universality:
    \begin{center}
        \begin{tikzcd}
            &X_1\arrow{dr}{f}&&\\
            P\arrow{ur}{p_1} \arrow{dr}{p_2}&&Y\arrow{r}{h}&Z\\
            &X_2\arrow{ur}{g}&&
        \end{tikzcd}
    \end{center}
    Then we can use the universal property of $X_1\times_Z X_2$ to get a unique map $\beta$ in the following diagram:
    \begin{center}
    \begin{tikzcd}
        P\arrow[bend right]{ddr}{p_1} \arrow[bend left]{drr}{p_2}\arrow[dashed]{dr}[description]{\exists!}&&\\
        & X_1\times_Z X_2 \arrow{d}{\pi_1}\arrow{r}{\pi_2}& X_2\arrow{d}{h\circ g}\\
        &X_1 \arrow{r}{h\circ f}&Z
    \end{tikzcd}
\end{center}
With this map $\beta$, we claim the following diagram commutes:
\begin{center}
    \begin{tikzcd}
        P\arrow{r}{\beta}\arrow{d}{f\circ p_1}& X_1\times_Z X_2 \arrow{d}{\theta}\\
        Y\arrow{r}{\alpha}&Y\times_Z Y
    \end{tikzcd}
\end{center}
To prove this, we turn to the universal property of $Y\times_ZY$ shown below:
\begin{center}
    \begin{tikzcd}
        P\arrow[bend right]{ddr}{f\circ p_1} \arrow[bend left]{drr}{g\circ p_2}\arrow[dashed]{dr}[description]{\exists!}&&\\
        & Y\times_ZY \arrow{d}{\mu_1}\arrow{r}{\mu_2}& Y\arrow{d}{h}\\
        &Y \arrow{r}{h}&Z
    \end{tikzcd}
\end{center}
We will show that both $\alpha \circ f\circ p$ and $\theta \circ \beta$ satisfy the unique arrow.\\
\newline
To show $\alpha \circ f\circ p_1$ satisfies the diagram, we see
\[
\mu_1\circ \alpha\circ f \circ p_1=f\circ p_1
\]
and
\[
\mu_2\circ \alpha \circ f \circ p_1=f\circ p_1=g\circ p_2
\]
Now to show $\theta\circ \beta$ satisfies the diagram,
\[
\mu_1\circ \theta\circ \beta=f\circ \pi_1\circ \beta=f\circ p_1
\]
as well as
\[
\mu_2\circ \theta\circ \beta=g\circ \pi_2\circ \beta=g\circ p_2
\]
This proves that by uniqueness of the arrow, that $\alpha\circ f\circ p_1=\theta\circ \beta$.
Thus we get an induced map $\chi$ in the following commutative diagram:
\begin{center}
    \begin{tikzcd}
        P\arrow[bend left]{drr}{\beta}\arrow[bend right]{ddr}[swap]{f\circ p_1} \arrow[dashed]{dr}[description]{\exists!}&&\\
        & Y\times_{(Y\times_Z Y)} X_1\times_Z X_2\arrow{r}{\iota_2}\arrow{d}{\iota_1}&X_1\times_Z X_2\arrow{d}{\theta}\\
        &Y\arrow{r}{\alpha}&Y\times_ZY
    \end{tikzcd}
\end{center}
Therefore the following diagram commutes as well:
\begin{center}
        \begin{tikzcd}
            &X_1\arrow{dr}{f}&&\\
            P\arrow{ur}{p_1} \arrow{dr}[swap]{p_2}\arrow{r}{\chi}&Y\times_{(Y\times_Z Y)} X_1\times_Z X_2\arrow{u}{\pi_2\circ \iota_2}\arrow{d}[swap]{\pi_1\circ \iota_2}&Y\arrow{r}{h}&Z\\
            &X_2\arrow{ur}{g}&&
        \end{tikzcd}
    \end{center}
    because
    \[
    \pi_2\circ \iota_2\circ \chi=\pi_2\circ \beta=p_1
    \]
    and
    \[
    \pi_1\circ \iota_2\circ \chi=\pi_1\circ \beta=p_2
    \]
    This proves that the limit of the diagram
        \begin{center}
        \begin{tikzcd}
            X_1\arrow{dr}{f}&&\\
            &Y\arrow{r}{h}&Z\\
            X_2\arrow{ur}{g}&&
        \end{tikzcd}
    \end{center}
    is simultaneously $X_1\times_Y X_2$ and $Y\times_{(Y\times_Z Y)} X_1\times_Z X_2$, meaning they are defined up to unique isomorphism, so in particular the following diagram is Cartesian:
    \begin{center}
        \begin{tikzcd}
            X_1\times_Y X_2 \ar{r} \ar{d}& X_1\times_Z X_2 \ar{d}\\
            Y\ar{r}& Y\times_Z Y
        \end{tikzcd}
    \end{center}
\end{proof}

\subsubsection{C}\label{1.4.C}
\begin{proof}
    Let $S$ be the defined set and $\pi_i:S\to A_i$ are the projections. It's clear that for any $i,j\in \fI$ and $m:i\to j$, the following diagram commutes:
    \begin{center}
        \begin{tikzcd}
            &S\ar{dl}[swap]{\pi_i}\ar{dr}{\pi_j}&\\
            A_i\ar{rr}{Fm}&&A_j
        \end{tikzcd}
    \end{center}
    by construction of $S$. Now suppose
    \begin{center}
        \begin{tikzcd}
            &W\ar{dl}[swap]{g_i}\ar{dr}{g_j}&\\
            A_i\ar{rr}{Fm}&&A_j
        \end{tikzcd}
    \end{center}
    commutes under the same hypotheses. If there were a map $\varphi: W\to S$ such that
    \begin{center}
        \begin{tikzcd}
            &W\ar[dashed]{d}{\varphi} \ar[bend right]{ddl}[swap]{g_i} \ar[bend left]{ddr}{g_j}&\\
            &S\ar{dl}[swap]{\pi_i}\ar{dr}{\pi_j}&\\
            A_i\ar{rr}{Fm}&&A_j
        \end{tikzcd}
    \end{center}
    commutes, then for each $w\in W$, $\varphi(w)=s$ where $\pi_i(s)=g_i(w)$. This element $s$ is uniquely defined to be $(g_i(w))_{i\in \fI}$. This demonstrates that $\varphi$ exists and is unique, so indeed $S=\varprojlim_{\fI} A_i$.
\end{proof}
\subsubsection{D}\label{1.4.D}
\begin{proof}
    \begin{enumerate}[(a)]
        \item I'm not entirely sure if the question wants to describe $\Q$ as an object of $\Ring$ or $\Mod_\Z$, but I will assume we want $\Q\in \Ring$. We take the index set to be the set of positive integers with a unique arrow $n\to m$ if and only if there exists some positive integer $k$ such that $m=nk$. If this is the case, we define a ring morphism $\phi_{n,k}:\Z_n \to \Z_{nk}$ defined by $\frac{x}{n^i}\mapsto \frac{k^ix}{(nk)^i}$. Here, is the ring given by localization by the multiplicative subset generated by $n$. We define maps $\iota_n:\Z_n\to \Q$ as $\frac{x}{n^i}\mapsto \frac{x}{n^i}$. Then by the construction, the following diagram commutes:
        \begin{center}
            \begin{tikzcd}
                \Q&\\
                \Z_{nk} \arrow{u}{\iota_{nk}}& \Z_n \ar{ul}[swap]{\iota_n} \ar{l}{\phi_{n,k}}
            \end{tikzcd}
        \end{center}
        because
        \[
        \iota_{nk}\circ \phi_{n,k}(\frac{x}{n^i})=\iota_{nk}(\frac{xk^i}{(nk)^i})=\frac{xk^i}{(nk)^i}=\frac{x}{n^i}=\iota_n(\frac{x}{n^i})
        \]
        If we have another ring $R$ with maps $f_n:\Z_n\to W$ satisfying the commutativity hypotheses, we want to show
        \begin{center}
            \begin{tikzcd}
                R\\
                \Q\ar[dashed]{u}{\exists!}&\\
                \Z_{nk}\ar[bend left=70]{uu}{f_{nk}} \arrow{u}{\iota_{nk}}& \Z_n \ar[bend right]{luu}[swap]{f_n} \ar{ul}[swap]{\iota_n} \ar{l}{\phi_{n,k}}
            \end{tikzcd}
        \end{center}
        By commutativity alone, we would require the unique map $\varphi$ to act as $\frac{x}{n^i}\mapsto f_n(\frac{x}{n^i})$ which shows that the map $\varphi$ is unique. To be precise we should show that $\varphi$ is indeed a ring morphism by showing that it's well defined for different choices of $\frac{x}{n}$, i.e. if $\frac{x}{y}=\frac{p}{q}$ then their images are the same. Notice that $\frac{x}{y}=\frac{p}{q}$ if and only if $\frac{xq}{yq}=\frac{yp}{yq}$. Therefore
        \[
        \varphi(\frac{x}{y})=\varphi\circ \iota_y(\frac{x}{y})=\varphi\circ \iota_{yq}(\frac{xq}{yq})=\varphi\circ \iota_{q}(\frac{p}{q})=\varphi(\frac{p}{q})
        \]
        and the other ring morphism axioms can be easily verified.
        \item 
        For any set $X$, we have the category $\Ssubset(X)$ in which we can define $A_1\cup A_2=\varinjlim_{\fI}A_i$ where $\fI$ is the discrete category
        \begin{center}
            \begin{tikzcd}
                1&2
            \end{tikzcd}
        \end{center}
        Explicitly, we are defining $A_1\cup A_2$ as the coproduct of $A_1$ and $A_2$. If
        \begin{center}
            \begin{tikzcd}
                &B&\\
                A_1\ar[ur]&&A_2\ar[ul]
            \end{tikzcd}
        \end{center}
        commutes, then $A_1\subset B$ and $A_2\subset B$ which directly implies that the standard definition of $A_1\cup A_2\subset B$. Therefore there is a morphism $A_1\cup A_2\to B$, and uniqueness is by uniqueness of arrows in $\Ssubset(X)$.
    \end{enumerate}
\end{proof}
\subsubsection{E}\label{1.4.E}
\begin{proof}
    Let $S$ be the defined set and let $\iota_i(a)=[(a,i)]$ where $\iota_i:A_i\to S$. If $m:i\to j$, then
    \[
    \iota_i(a)=[(a,i)]
    \]
    and
    \[
    \iota_j\circ Fm(a)=[(Fm(a),j)]
    \]
    Also notice that $(a,i)\sim (Fm(a),j)$ because $Fm:A_i\to A_j$ and $\id_{A_j}=F\id_j$ are two maps such that $Fm(a)=\id_{A_j}(f(a))$. This shows that $S$ satisfies the required definition.\\
    \newline
    To show that $S$ is universal, suppose we have another set $W$ equipped with maps $g_i:A_i\to W$ that satisfy the definition. We want to show
    \begin{center}
        \begin{tikzcd}
            &W&\\
            &S\ar[dashed]{u}[description]{\exists!}\\
            A_i \ar[bend left]{uur}{g_i} \ar{rr}{Fm} \ar{ur}{\iota_i}&& A_j \ar{ul}[swap]{\iota_j} \ar[bend right]{uul}[swap]{g_j}
        \end{tikzcd}
    \end{center}
    We define $\varphi:S\to W$ as $\varphi([(a,i)])=g_i(a)$, which proves uniqueness because this condition comes directly from commutativity. To prove existence, we just need to show $\varphi$ is well defined. In other words, we need to show that if $(a_i,i)\sim (a_j,j)$ then $g_i(a_i,i)=g_j(a_j,j)$. If $(a_i,i)\sim (a_j,j)$, then for some $\alpha:i\to k$ and some $\beta:j\to k$,
    \[
    F\alpha(a_i)=F\beta(a_j)
    \]
    Then the following diagram must commute:
    \begin{center}
        \begin{tikzcd}
            &W&\\
            A_i \ar{ur}{g_i} \ar{r}[swap]{F\alpha}& A_k \ar{u}{g_k}& A_j \ar{l}{F\beta}\ar{ul}[swap]{g_j}
        \end{tikzcd}
    \end{center}
    We now observe that
    \[
    g_i(a_i)=g_k\circ F\alpha(a_i)=g_k\circ F\beta(a_j)=g_j(a_j)
    \]
    so $\varphi$ is well defined, which proves existence.
\end{proof}
\subsubsection{F}\label{1.4.F}
\begin{proof}
    For the problem, let $m_i$ denote $(m_i,i)\in \coprod_\fI M_i$ as well as the element in $M_i$ depending on the context for convenience. To prove addition is well defined, suppose $m_i\sim m_{i'}$ and $m_j\sim m_{j'}$ for some $i,i',j,j'\in \fI$. Also pick some $l$ and $l'$ such that we have 
    \begin{center}
        \begin{tikzcd}
            i\ar{r}{u}&l&j\ar{l}[swap]{v}
        \end{tikzcd}
    \end{center}
    and
    \begin{center}
        \begin{tikzcd}
            i'\ar{r}{u'}&l'&j'\ar{l}[swap]{v'}
        \end{tikzcd}
    \end{center}
    Then there exists some $f:i\to n_i$ and some $f':{i'}\to n_i$ such that $Ff(m_i)=Ff'(m_{i'})$ and some $g:j\to n_j$ and $g':j'\to n_j$ such that $Fg(m_j)=Fg'(m_{j'})$. By the first filtered hypothesis, we have the following set of arrows in $\fI$:
    \begin{center}
        \begin{tikzcd}
        i\ar{d}&&j\ar[d]\\
        n_i\ar[r]&n&n_j\ar[l]\\
        i'\ar[u]&& j'\ar[u]
    \end{tikzcd}
    \end{center}
    as well as the other set of arrows
    \begin{center}
        \begin{tikzcd}
        i\ar{r}&l\ar[d]&j\ar[l]\\
        &k&\\
        i'\ar[r]&l'\ar[u]& j'\ar[l]
    \end{tikzcd}
    \end{center}
    We can get another set of arrows
    \begin{center}
        \begin{tikzcd}
            n\ar[r]& m& k\ar[l]
        \end{tikzcd}
    \end{center}
    Therefore we have the paths:
    \begin{center}
        \begin{tikzcd}
            1.&i\ar[r]&n_i\ar[r]&n\ar[r]&m\\
            2.&i\ar[r]&l\ar[r]&k \ar[r]&m\\
            3.&j\ar[r]&n_j\ar[r]&n\ar[r]&m\\
            4.&j\ar[r]&l\ar[r]&k\ar[r]&m\\
            5.&i'\ar[r]&n_i\ar[r]&n\ar[r]&m\\
            6.& i'\ar[r]&l'\ar[r]&k\ar[r]&m\\
            7.& j'\ar[r]&n_j\ar[r]&n\ar[r]&m\\
            8.& j'\ar[r]&l'\ar[r]&k\ar[r]&m
        \end{tikzcd}
    \end{center}
    Then by the second requirement of $\fI$ being filtered, there exists 
    \\$m_1,m_2,m_3,m_4\in \fI$ and arrows such that the following diagram commutes:
    \begin{center}
        \begin{tikzcd}
            &l'\ar[r]&k\ar[r]&m\ar{dr}&\\
            i'\ar{ur}\ar[dr]&&&&m_3\\
            &n_i\ar[r]&n\ar[r]&m\ar{dr}\ar[ur]&\\
            i\ar{ur}\ar[dr]&&&&m_1\\
            &l\ar[r]&k\ar[r]&m\ar[ur]\ar[dr]&\\
            j\ar[ur]\ar[dr]&&&&m_2\\
            &n_j\ar[r]&n\ar[r]&m\ar[ur]\ar[dr]&\\
            j'\ar[ur]\ar[dr]&&&&m_4\\
            &l'\ar[r]&k\ar[r]&m\ar[ur]&
        \end{tikzcd}
    \end{center}
    We will add on to this diagram by obtaining the following commutative diagrams:
    \begin{center}
        \begin{tikzcd}
            &m_3\ar[r]&m_{13}'\ar[dr]&\\
            m\ar[ur]\ar[dr]&&&m_{13}\\
            &m_1\ar[r]&m_{13}'\ar[ur]&\\
            &&&\\
            &m_2\ar[r]&m_{24}'\ar[dr]&\\
            m\ar[ur]\ar[dr]&&&m_{24}\\
            &m_4\ar[r]&m_{24}'\ar[ur]&
        \end{tikzcd}
    \end{center}
    and we will add on to these commutative diagrams one final time to obtain the following commutative diagram:
    \begin{center}
        \begin{tikzcd}
            &m_{13}\ar[r]&m_{0}'\ar[dr]&\\
            m\ar[ur]\ar[dr]&&&m_{0}\\
            &m_{24}\ar[r]&m_{0}'\ar[ur]&
        \end{tikzcd}
    \end{center}
    Adding all of our newest constructions to the large diagram, we get the following commutative diagram:
    \begin{center}
        \begin{tikzcd}
            &l'\ar[r]&k\ar[r]&m\ar{dr}&&&&&\\
            i'\ar{ur}\ar[dr]&&&&m_3\ar[r]&m_{13}'\ar[dr]&&&\\
            &n_i\ar[r]&n\ar[r]&m\ar{dr}\ar[ur]&&&m_{13}\ar[r]&m_0'\ar[ddr]&\\
            i\ar{ur}\ar[dr]&&&&m_1\ar[r]&m_{13}'\ar[ur]&&&\\
            &l\ar[r]&k\ar[r]&m\ar[ur]\ar[dr]&&&&&m_0\\
            j\ar[ur]\ar[dr]&&&&m_2\ar[r]&m_{24}'\ar[dr]&&&\\
            &n_j\ar[r]&n\ar[r]&m\ar[ur]\ar[dr]&&&m_{24}\ar[r]&m_0'\ar[uur]&\\
            j'\ar[ur]\ar[dr]&&&&m_4\ar[r]&m_{24}'\ar[ur]&&&\\
            &l'\ar[r]&k\ar[r]&m\ar[ur]&&&&&
        \end{tikzcd}
    \end{center}
    Now just think of this commutative diagram in $\Mod_A$ with the $A$-modules indexed by the elements in $\fI$ above because it's tedious to relabel the entire diagram. Because all of the morphisms are linear, if we want to show that $Fu(m_i)+Fv(m_j)\sim Fu'(m_{i'})+Fv'(m_{j'})$, it suffices to show that there exists morphisms $\chi_2:l'\to m_0$ and $\chi_1:l\to m_0$ such that $F(\chi_1\circ u)(m_i)= F(\chi_2\circ u')(m_{i'})$ and $F(\chi_1\circ v)(m_j)=F(\chi_2\circ v')(m_{j'})$. We claim that $\chi_1$ is the path from $l\to m_0$ in the above diagram and $\chi_2$ is the path from $l'\to m_0$. Recalling that $Fu(m_i)=Fu'(m_{i'})$, when tracking our elements $m_i,m_{i'},m_j,m_{j'}$, we have that the path $i'\to n_i=i\to n_i$ and $j'\to n_j=j\to n_j$. Therefore we track the path of $m_{i'}$ as
    \begin{center}
        \begin{tikzcd}
            &l'\ar[r,red]&k\ar[r,red]&m\ar[dr,red]&&&&&\\
            i'\ar[ur,red]\ar[dr]&&&&m_3\ar[r,red]&m_{13}'\ar[dr,red]&&&\\
            &n_i\ar[r]&n\ar[r]&m\ar{dr}\ar[ur]&&&m_{13}\ar[r,red]&m_0'\ar[ddr,red]&\\
            i\ar{ur}\ar[dr]&&&&m_1\ar[r]&m_{13}'\ar[ur]&&&\\
            &l\ar[r]&k\ar[r]&m\ar[ur]&&&&&m_0\\
        \end{tikzcd}
    \end{center}
    equals
    \begin{center}
        \begin{tikzcd}
            &l'\ar[r]&k\ar[r]&m\ar[dr]&&&&&\\
            i'\ar[ur]\ar[dr, red]&&&&m_3\ar[r,red]&m_{13}'\ar[dr,red]&&&\\
            &n_i\ar[r,red]&n\ar[r,red]&m\ar[dr]\ar[ur,red]&&&m_{13}\ar[r,red]&m_0'\ar[ddr,red]&\\
            i\ar{ur}\ar[dr]&&&&m_1\ar[r]&m_{13}'\ar[ur]&&&\\
            &l\ar[r]&k\ar[r]&m\ar[ur]&&&&&m_0\\
        \end{tikzcd}
    \end{center}
    equals
    \begin{center}
        \begin{tikzcd}
            &l'\ar[r]&k\ar[r]&m\ar[dr]&&&&&\\
            i'\ar[ur]\ar[dr]&&&&m_3\ar[r,red]&m_{13}'\ar[dr,red]&&&\\
            &n_i\ar[r,red]&n\ar[r,red]&m\ar[dr]\ar[ur,red]&&&m_{13}\ar[r,red]&m_0'\ar[ddr,red]&\\
            i\ar[ur,red]\ar[dr]&&&&m_1\ar[r]&m_{13}'\ar[ur]&&&\\
            &l\ar[r]&k\ar[r]&m\ar[ur]&&&&&m_0\\
        \end{tikzcd}
    \end{center}
    equals
    \begin{center}
        \begin{tikzcd}
            &l'\ar[r]&k\ar[r]&m\ar[dr]&&&&&\\
            i'\ar[ur]\ar[dr]&&&&m_3\ar[r]&m_{13}'\ar[dr]&&&\\
            &n_i\ar[r,red]&n\ar[r,red]&m\ar[dr,red]\ar[ur]&&&m_{13}\ar[r,red]&m_0'\ar[ddr,red]&\\
            i\ar[ur,red]\ar[dr]&&&&m_1\ar[r,red]&m_{13}'\ar[ur,red]&&&\\
            &l\ar[r]&k\ar[r]&m\ar[ur]&&&&&m_0\\
        \end{tikzcd}
    \end{center}
    equals
    \begin{center}
        \begin{tikzcd}
            &l'\ar[r]&k\ar[r]&m\ar[dr]&&&&&\\
            i'\ar[ur]\ar[dr]&&&&m_3\ar[r]&m_{13}'\ar[dr]&&&\\
            &n_i\ar[r]&n\ar[r]&m\ar[dr]\ar[ur]&&&m_{13}\ar[r,red]&m_0'\ar[ddr,red]&\\
            i\ar[ur]\ar[dr,red]&&&&m_1\ar[r,red]&m_{13}'\ar[ur,red]&&&\\
            &l\ar[r,red]&k\ar[r,red]&m\ar[ur,red]&&&&&m_0\\
        \end{tikzcd}
    \end{center}
    which demonstrates that $F(\chi_1\circ u)(m_i)=F(\chi_2\circ u')(m_{i'})$. On the other hand, tracking $m_{j'}$,
    \begin{center}
        \begin{tikzcd}
            &l\ar[r]&k\ar[r]&m\ar[dr]&&&&&m_0\\
            j\ar[ur]\ar[dr]&&&&m_2\ar[r]&m_{24}'\ar[dr]&&&\\
            &n_j\ar[r]&n\ar[r]&m\ar[ur]\ar[dr]&&&m_{24}\ar[r,red]&m_0'\ar[uur,red]&\\
            j'\ar[ur]\ar[dr,red]&&&&m_4\ar[r,red]&m_{24}'\ar[ur,red]&&&\\
            &l'\ar[r,red]&k\ar[r,red]&m\ar[ur,red]&&&&&
        \end{tikzcd}
    \end{center}
    equals
    \begin{center}
        \begin{tikzcd}
            &l\ar[r]&k\ar[r]&m\ar[dr]&&&&&m_0\\
            j\ar[ur]\ar[dr]&&&&m_2\ar[r]&m_{24}'\ar[dr]&&&\\
            &n_j\ar[r,red]&n\ar[r,red]&m\ar[ur]\ar[dr,red]&&&m_{24}\ar[r,red]&m_0'\ar[uur,red]&\\
            j'\ar[ur,red]\ar[dr]&&&&m_4\ar[r,red]&m_{24}'\ar[ur,red]&&&\\
            &l'\ar[r]&k\ar[r]&m\ar[ur]&&&&&
        \end{tikzcd}
    \end{center}
    equals
    \begin{center}
        \begin{tikzcd}
            &l\ar[r]&k\ar[r]&m\ar[dr]&&&&&m_0\\
            j\ar[ur]\ar[dr,red]&&&&m_2\ar[r]&m_{24}'\ar[dr]&&&\\
            &n_j\ar[r,red]&n\ar[r,red]&m\ar[ur]\ar[dr,red]&&&m_{24}\ar[r,red]&m_0'\ar[uur,red]&\\
            j'\ar[ur]\ar[dr]&&&&m_4\ar[r,red]&m_{24}'\ar[ur,red]&&&\\
            &l'\ar[r]&k\ar[r]&m\ar[ur]&&&&&
        \end{tikzcd}
    \end{center}
    equals
    \begin{center}
        \begin{tikzcd}
            &l\ar[r]&k\ar[r]&m\ar[dr]&&&&&m_0\\
            j\ar[ur]\ar[dr,red]&&&&m_2\ar[r,red]&m_{24}'\ar[dr,red]&&&\\
            &n_j\ar[r,red]&n\ar[r,red]&m\ar[ur,red]\ar[dr]&&&m_{24}\ar[r,red]&m_0'\ar[uur,red]&\\
            j'\ar[ur]\ar[dr]&&&&m_4\ar[r]&m_{24}'\ar[ur]&&&\\
            &l'\ar[r]&k\ar[r]&m\ar[ur]&&&&&
        \end{tikzcd}
    \end{center}
    equals
    \begin{center}
        \begin{tikzcd}
            &l\ar[r,red]&k\ar[r,red]&m\ar[dr,red]&&&&&m_0\\
            j\ar[ur,red]\ar[dr]&&&&m_2\ar[r,red]&m_{24}'\ar[dr,red]&&&\\
            &n_j\ar[r]&n\ar[r]&m\ar[ur]\ar[dr]&&&m_{24}\ar[r,red]&m_0'\ar[uur,red]&\\
            j'\ar[ur]\ar[dr]&&&&m_4\ar[r]&m_{24}'\ar[ur]&&&\\
            &l'\ar[r]&k\ar[r]&m\ar[ur]&&&&&
        \end{tikzcd}
    \end{center}
    which shows that $F(\chi_1 \circ v)(m_j)=F(\chi_2\circ v')(m_{j'})$. This proves that
    \[
    F\chi_1(Fu(m_i)+Fv(m_j))=F\chi_2(Fu'(m_{i'})+Fv'(m_{j'}))
    \]
    so that indeed
    \[
    [(Fu(m_i)+Fv(m_j),l)]=[(Fu'(m_{i'})+Fv'(m_{j'}),l')]
    \]
    so addition is independent of choice of $u,v,$ and $l$.\\
    \newline
    We define multiplication as $a[(m_i,i)]=[(am_i,i)]$. To show multiplication is well defined, suppose $(m_i,i)\sim (m_j,j)$. Then for some $f:i\to k$ and some $g:j\to k$, $Ff(m_i)=Fg(m_j)$. Then
    \[
    Ff(am_i)=aFf(m_i)=aFg(m_j)=Fg(am_j)
    \]
    implies 
    \[
    (m_i,i)\sim(m_j,j)\Rightarrow (am_i,i)\sim (am_j,j)\Rightarrow a[(m_i,i)]=a[(m_j,j)]
    \]
    demonstrating multiplication is well defined.\\
    \newline
    The module axioms are readily verifiable. Now suppose we have an $A$-module $W$ that satisfies the commutativity of the diagram indexed by $\fI$ equipped with morphisms $\alpha_i$. We want to show
    \begin{center}
        \begin{tikzcd}
            &W&\\
            &\varinjlim_{\fI} M_i\ar[dashed]{u}[description]{\exists!}\\
            M_i \ar[bend left]{uur}{\alpha_i} \ar{ur} \ar{rr}{Ff}&&M_j\ar{ul} \ar[bend right]{uul}[swap]{\alpha_j}
        \end{tikzcd}
    \end{center}
    We will construct such a unique map. By commutativity, we are required that $\varphi([(m_i,i)])=\alpha_i(m_i)$. This proves $\varphi$ is unique. To prove $\varphi$ is well defined, suppose $(m_i,i)\sim (m_j,j)$. Then there exists some $f:i\to k$ and some $g:j\to k$ such that $Ff(m_i)=Fg(m_j)$. Then the following diagram must commute:
    \begin{center}
        \begin{tikzcd}
            &W&\\
            M_i \ar[bend left]{ur}{\alpha_i} \ar{r}{Ff}&M_k\ar{u}{\alpha_k}&M_j\ar{l}[swap]{Fg} \ar[bend right]{ul}[swap]{\alpha_j}
        \end{tikzcd}
    \end{center}
    Therefore
    \[
    \alpha_i(m_i)=\alpha_k\circ Ff(m_i)=\alpha_k\circ Fg(m_j)=\alpha_j(m_j)
    \]
    so $\varphi$ is well defined. To show $\varphi$ is linear, we have
    \begin{align*}
        \varphi([(m_i,i)]+[(m_j,j)])=\varphi([Ff(m_i)+Fg(m_j),k])=\alpha_k(Ff(m_i)+Fg(m_j))\\
        =\alpha_k\circ Ff(m_i)+\alpha_k\circ Fg(m_j)=\alpha_i(m_i)+\alpha_j(m_j)=\varphi([(m_i,i)])+\varphi([(m_j,j)])
    \end{align*}
    by the below commutative diagram:
    \begin{center}
        \begin{tikzcd}
            &W&\\
            &\varinjlim_{\fI} M_i\ar[dashed]{u}[description]{\exists!}\\
            M_i \ar[bend left]{uur}{\alpha_i} \ar{ur} \ar{r}{Ff}&M_k\ar{u}&M_j\ar{l}[swap]{Fg}\ar{ul} \ar[bend right]{uul}[swap]{\alpha_j}
        \end{tikzcd}
    \end{center}
    Additionally,
    \begin{align*}
        \varphi(a[(m_i,i)])=\varphi([(am_i,i)])=\alpha_i(am_i)=a\alpha_i(m_i)=a\varphi([(m_i,i)])
    \end{align*}
    which proves existence.
\end{proof}
\subsubsection{G}\label{1.4.G}
\begin{proof}
    We take the index category to be the elements of $S$ where there is an arrow $s:s_1\to s_2$ if and only if $s_2=ss_1$ for some $s\in S$. In this case, we define the map $Fs:\frac{1}{s_1}A\to \frac{1}{ss_1}$ as
    \[
    \frac{a}{s_1}\mapsto \frac{sa}{ss_1}
    \]
    Our index category is filtered as for any $s_1,s_2\in S$, there is an arrow $s_1\to s_1s_2$ and an arrow $s_2\to s_1s_2$. By construction, there is at most one arrow from any object to any other object, so the second condition is trivially true. To show that $\varinjlim \frac{1}{s}A$ is isomorphic to $S^{-1}A$, we will first show the existence of a morphism $\varphi:\varinjlim \frac{1}{s}A\to S^{-1}A$. For each $s\in S$, we define a map $\iota_s: \frac{1}{s}A\to S^{-1}A$ given by $\frac{a}{s}\mapsto \frac{a}{s}$. To induce the map $\varphi$, we want to show the following diagram commutes:
    \begin{center}
        \begin{tikzcd}
            &S^{-1}A&\\
            \frac{1}{ss_1}A\ar{ur}{\iota_{ss_1}}&&\frac{1}{s_1}A \ar{ll}[swap]{F(s)} \ar{ul}[swap]{\iota_{s_1}}
        \end{tikzcd}
    \end{center}
    We have
    \[
    \iota_{ss_1}\circ F(s)(\frac{a}{s_1})=\iota_{ss_1}(\frac{as}{s_1s})=\frac{as}{s_1s}=\frac{a}{s}=\iota_s(\frac{a}{s})
    \]
    which proves the diagram commutes, hence we obtain the induced morphism $\varphi:\varinjlim \frac{1}{s}A\to S^{-1}A$. Now to find the inverse morphism, we will use the universal property of $S^{-1}A$. We construct a map $\alpha:A\to \varinjlim \frac{1}{s}A$ given by $\alpha(a)=\frac{a}{1}$. To show that for any $s\in S$, multiplication by $s$ is an automorphism of $\varinjlim \frac{1}{s}A$, if we take any $\frac{a}{s'}\in \varinjlim \frac{1}{s}A$ such that\[
    s\frac{a}{s'}=0\iff
    \frac{sa}{s'}=0
    \]
    if and only if there exists some $F(r)$ such that $F(r)(\frac{sa}{s'})=0$, which by definition means
    \[
    \frac{rsa}{s'r}=0
    \]
    which is true if and only if there exists some $r'\in S$ such that
    \[
    r'rsa=0
    \]
    Assuming that $0\notin S$, using the fact that $A$ is an integral domain and $r'rs\in S$ implies
    \[
    a=0
    \]
   Therefore multiplication is injective. To show multiplication by $s$ is surjective, fix any $\frac{a}{s'}\in \varinjlim \frac{1}{s}A$. Then
   \[
   s \frac{a}{s's}=\frac{as}{ss'}=(\frac{as}{ss'})=F(s)(\frac{a}{s'})=(\frac{a}{s'})=\frac{a}{s'}
   \]
   so indeed multiplication by $s$ is an automorphism. Therefore we use the following universal property to get a map $\phi:S^{-1}A\to \varinjlim \frac{1}{s}A$:
   \begin{center}
       \begin{tikzcd}
           A\ar{r} \ar{dr}{\alpha}& S^{-1}A\ar[dashed]{d}[description]{\exists!}\\
           & \varinjlim \frac{1}{s}A
       \end{tikzcd}
   \end{center}
   These can easily checked to be inverses of each other, which proves the isomorphism. Equivalently, we could have just used the previous exercise to look at the structure of $\varinjlim \frac{1}{s}A$, and observed that the underlying sets are the same, and then proved the structures are isomorphic as well.
\end{proof}
\subsubsection{H}\label{1.4.H}
\begin{proof}
    The commutativity of 
    \begin{center}
        \begin{tikzcd}
            & \bigoplus_{i\in \fI} M_i/\sim\\
            M_i\ar{ur}{\iota_i} \ar{r}{F(n)}&M_j\ar{u}{\iota_j}
        \end{tikzcd}
    \end{center}
    commutes by definition of $\sim$. Now suppose
    \begin{center}
        \begin{tikzcd}
            & W\\
            M_i\ar{ur}{g_i} \ar{r}{F(n)}&M_j\ar{u}{g_j}
        \end{tikzcd}
    \end{center}
    commutes. If 
    \begin{center}
        \begin{tikzcd}
        &W\\
            & \bigoplus M_i/\sim\ar[dashed]{u}[description]{\exists!}\\
            M_i\ar[bend left]{uur}{g_i}\ar{ur}{\iota_i} \ar{r}{F(n)}&M_j\ar{u}{\iota_j}\ar[bend right=80]{uu}[swap]{g_j}
        \end{tikzcd}
    \end{center}
    the induced morphism $\varphi$ would have to satisfy $\varphi\circ \iota_i=g_i$. By linearity, this determines $\varphi$ completely as
    \[
    \varphi(\sum_i \iota_i(m_i))=\sum_i \varphi\circ \iota_i(m_i)=\sum_i g_i(m_i)
    \]
    because also every element of the direct sum is in the image of one of the $\iota$'s.
    This proves uniqueness. To show $\varphi$ is well defined on equivalence classes, suppose $F(n)(m_i)=m_j\iff m_j\sim m_i$. Then
    \[
    \varphi\circ \iota_i(m_i)=g_i(m_i)=g_j\circ F(n)(m_i)=g_j(m_j)=\varphi\circ \iota_j(m_j)
    \]
    so $\varphi$ is well defined. $\varphi$ is $A$-linear as 
    \[
    \varphi(\sum_i\iota_i(m_i)+\sum_j\iota_j(n_j))=\varphi(\sum_i \iota_i(m_i+n_i)=\sum_i \varphi \circ \iota_i(m_i+n_i)=\sum_i\varphi\circ \iota_i(m_i)+\sum_j \varphi\circ \iota_j(n_j)
    \]
    and 
    \[
    \varphi(a\iota_i(m_i))=\varphi(\iota_i(am_i))=g_i(am_i)=ag_i(m_i)=a\varphi(\iota_i(m_i))
    \]
    Then indeed the construction is the colimit.
\end{proof}
\subsection{}
\subsubsection{A}\label{1.5.A}
\begin{proof}
The diagram is below where $g_*=g\circ$ and $Gg_*=Gg\circ $:
    \begin{center}
        \begin{tikzcd}
            \Mor_\fB(F(A),B)\ar{r}{g_*} \ar{d}{\tau_{AB}}& \Mor_\fB(F(A),B')\ar{d}{\tau_{AB'}}\\
            \Mor_\fA(A,G(B))\ar{r}{Gg_*}&\Mor_\fA(A,G(B')) 
        \end{tikzcd}
    \end{center}
    
\end{proof}
\subsubsection{B}\label{1.5.B}
\begin{proof}
    We define $\eta_A$ to be $\tau_{AF(A)}(\id_{F(A)})$ and $\epsilon_B$ as $\tau_{FG(B)B}^{-1}(\id_{G(B)})$. Tracking $\eta_A$ on the bottom of the diagram below: 
    \begin{center}
        \begin{tikzcd}
            \Mor_\fB(F(A),F(A))\ar{r}{g_*} \ar{d}{\tau_{AF(A)}}& \Mor_\fB(F(A),B)\ar{d}{\tau_{AB}}\\
            \Mor_\fA(A,GF(A))\ar{r}{Gg_*}&\Mor_\fA(A,G(B)) 
        \end{tikzcd}
    \end{center}
    we see
    \[
    Gg_*(\eta_A)=Gg\circ \eta_A
    \]
    while on the top we get
    \[
    \tau_{AB}\circ g_*(\id_{F(A)})=\tau_{AB}(g\circ \id_{F(A)})=\tau_{AB}(g)
    \]
    By commutativity, the two must be equal, and $g\in \Mor_\fB(F(A),B)$ was arbitrary.\\
    \newline
    For $\epsilon_B$, we will use the following diagram:
    \begin{center}
        \begin{tikzcd}
            \Mor_\fB(FG(B),B)\ar{r}{Ff^*} \ar{d}{\tau_{G(B)B}}& \Mor_\fB(F(A),B)\ar{d}{\tau_{AB}}\\
            \Mor_\fA(G(B),G(B))\ar{r}{f^*}&\Mor_\fA(A,G(B)) 
        \end{tikzcd}
    \end{center}
    On one hand, we get
    \[
    Ff^*(\epsilon_B)=\epsilon_B\circ Ff
    \]
    while on the other hand we have
    \[
    \tau_{AB}^{-1}\circ f^*\circ \tau_{G(B)B}(\eta_B)=\tau_{AB}^{-1}\circ f^*(\id_{G(B)})=\tau_{AB}^{-1}(\id_{G(B)}\circ f)=\tau_{AB}^{-1}(f)
    \]
    and by commutativity the two are equal, where $f\in \Mor_\fA(A,G(B))$ was arbitrary.
\end{proof}
\subsubsection{C}\label{1.5.C}
\begin{proof}
    We will use the following universal property:
    \begin{center}
        \begin{tikzcd}
            M\times N \ar{r} \ar{dr}{\alpha}& M\otimes N\ar[dashed]{d}[description]{\exists!}\\
            &P
        \end{tikzcd}
    \end{center}
    For an arbitrary $\phi\in \Hom(M, \Hom(N,P))$, we let $\alpha(m,n)=\phi(m)(n)$. Then
    \[
    \alpha(m_1+m_2,n)=\phi(m_1+m_2)(n)=\phi(m_1)(n)+\phi(m_2)(n)=\alpha(m_1,n)+\alpha(m_2,n)
    \]
    and
    \[
    \alpha(m,n_1+n_2)=\phi(m)(n_1+n_2)=\phi(m)(n_1)+\phi(m)(n_2)=\alpha(m,n_1)+\alpha(m,n_2)
    \]
    as well as
    \[
    \alpha(am,n)=\phi(am)(n)=a\phi(m)(n)=\phi(m)(an)=\alpha(m,an)
    \]
    which proves $\alpha$ is bilinear, hence we get our induced map $\beta_\phi \in \Hom(M\otimes N, P)$. Then we can define a map $\varphi:\Hom(M, \Hom(N,P))\to \Hom(M\otimes N, P)$ as $\varphi(\phi)=\beta_\phi$.\\
    \newline
    On the other hand, if we have some $\phi \in \Hom(M\otimes N,P)$, we will define some $\gamma \in \Hom(M, \Hom(N,P))$ where $\gamma(m)(n)=\phi(m\otimes n)$. Then indeed for any arbitrary $m\in M$, $\gamma(m)\in \Hom(N,P)$ because
    \[
    \gamma(m)(n_1+n_2)=\phi(m\otimes n_1+n_2)=\phi(m\otimes n_1)+\phi(m\otimes n_2)=\gamma(m)(n_1)+\gamma(m)(n_2)
    \]
    and
    \[
    \gamma(m)(an)=\phi(m\otimes an)=a\phi(m\otimes n)=a\gamma(m)(n)
    \]
    Also $\gamma \in \Hom(M, \Hom(N,P))$ because
    \[
    \gamma(m_1+m_2)(n)=\phi(m_1+m_2\otimes n)=\phi(m_1\otimes n)+\phi(m_2\otimes n)=\gamma(m_1)(n)+\gamma(m_2)(n)
    \]
    and
    \[
    \gamma(am)(n)=\phi(am\otimes n)=a\phi(m\otimes n)=a\gamma(m)(n)
    \]
    Therefore we can define a map $\tilde \varphi:\Hom(M\otimes N,P)\to \Hom(M, \Hom(N,P))$ given by $\tilde \varphi(\phi)=\gamma_\phi$. To show that $\varphi$ is a bijection, we observe
    \[
    \varphi \circ \tilde \varphi(\phi)(m\otimes n)=\varphi(\gamma_\phi)(m\otimes n)=\gamma_\phi(m)(n)=\phi(m\otimes n)
    \]
    and
    \[
    \tilde \varphi\circ \varphi(\phi)(m)(n)=\tilde \varphi(\beta_\phi)(m)(n)=\beta_\phi(m\otimes n)=\phi(m)(n)
    \]
    so indeed $\tilde \varphi=\varphi^{-1}$ and $\varphi$ is a bijection.
\end{proof}
\subsubsection{D}\label{1.5.D}
\begin{proof}
    We fix arbitrary $f\in \Hom(A',A)$ and $g\in \Hom(B,B')$ and define \\$\tau_{AB}:\Hom(A\otimes N,B)\to \Hom(A, \Hom(N,B))$ as $\varphi^{-1}$ in the previous exercise, which we proved was a bijection. We first want to show that the following diagram commutes:
    \begin{center}
        \begin{tikzcd}
            \Hom(A\otimes N,B)\ar{r}{f\otimes N^*} \ar{d}{\tau_{AB}}& \Hom(A'\otimes N,B)\ar{d}{\tau_{A'B}}\\
            \Hom(A, \Hom(N,B))\ar{r}{f^*}& \Hom(A', \Hom(N,B))
        \end{tikzcd}
    \end{center}
    Fixing any $\phi \in \Hom(A\otimes N,B)$ and any $a'\in A'$ and $n\in N$, we get on one hand that
    \begin{align*}
        \tau_{A'B}\circ f\otimes N^*(\phi)(a')(n)=\tau_{A'B}(\phi \circ f\otimes N)(a')(n)=\phi\circ f\otimes N(a'\otimes n)=\phi(f(a')\otimes n)
    \end{align*}
    On the other side of the diagram, we get
    \begin{align*}
        f^*\circ \tau_{AB}(\phi)(a')(n)=\tau_{AB}(\phi)\circ f(a')(n)=\phi(f(a')\otimes n)
    \end{align*}
    which proves the diagram does commute. Now we want to show the below diagram commutes as well:
    \begin{center}
        \begin{tikzcd}
            \Hom(A\otimes N,B)\ar{r}{g_*} \ar{d}{\tau_{AB}}& \Hom(A\otimes N,B')\ar{d}{\tau_{AB'}}\\
            \Hom(A, \Hom(N,B))\ar{r}{g_{**}}& \Hom(A, \Hom(N,B'))
        \end{tikzcd}
    \end{center}
    where $g_*$ is as usual and $g_{**}=(g_*)_*$. Fixing any $a\in A, n\in N$ and $\phi\in \Hom(A\otimes N,B)$, we get on the top that
    \[
    \tau_{AB'}\circ g_*(\phi)(a)(n)=\tau_{AB'}(g\circ \phi)(a)(n)=g\circ \phi(a\otimes n)
    \]
    On the bottom, we get
    \[
    g_{**}\circ \tau_{AB}(\phi)(a)(n)=g_*\circ \tau_{AB}(\phi)(a)(n)=g\circ \tau_{AB}(\phi)(a)(n)=g\circ \phi(a\otimes n)
    \]
    which shows this diagram commutes as well, proving that $\cdot \otimes N$ and $\Hom(N,\cdot)$ are adjoint functors.
\end{proof}
\subsubsection{E}\label{1.5.E}
\begin{proof}
    We want to first show that the following diagram commutes:
    \begin{center}
        \begin{tikzcd}
            \Hom(N\otimes_B A,M)\ar{r}{f\otimes A^*} \ar{d}{\tau_{NM}}& \Hom(N'\otimes_B A,M)\ar{d}{\tau_{N'M}}\\
            \Hom(N,M_B)\ar{r}{f^*}& \Hom(N',M_B)
        \end{tikzcd}
    \end{center}
    To do this, we first need to define what $\tau_{NM}$ is. Given any $\phi \in \Hom(N\otimes_B A,M)$, let $\varphi(\phi)\in \Hom(N,M_B)$ act as
    \[
    \varphi(\phi)(n)=\phi(n\otimes 1)
    \]
    To show $\varphi(\phi)$ is actually $B$-linear, we observe
    \[
    \varphi(\phi)(n_1+n_2)=\phi(n_1+n_2\otimes 1)=\phi(n_1\otimes 1)+\phi(n_2\otimes 1)=\varphi(\phi)(n_1)+\varphi(\phi)(n_2)
    \]
    as well as
    \[
    \varphi(\phi)(bn)=\phi(bn\otimes 1)=b\phi(n\otimes 1)=b\varphi(\phi)(n)
    \]
    On the other hand if we have some $\phi'\in \Hom(N,M_B)$, let $\tilde \varphi(\phi')\in \Hom(N\otimes_BA,M)$ act as
    \[
    \tilde \varphi(\phi')(n\otimes a)=a \phi'(n)
    \]
    where we can consider elements of $M$ as elements of $M_B$ and vice versa. To show $\tilde \varphi(\phi')$ is well defined, we define for $\phi'$ an $\alpha:N\times A\to M$ as $\alpha(n,a)=a\phi'(n)$. Then
    \begin{align*}
        \alpha(n_1+n_2,a)=a \phi'(n_1+n_2)=a\phi'(n_1)+a \phi'(n_2)=\alpha(n_1,a)+\alpha(n_2,a)
    \end{align*}
    and
    \[
    \alpha(n,a_1+a_2)=(a_1+a_2)\phi'(n)=a_1 \phi'(n)+a_2 \phi'(n)=\alpha(n,a_1)+\alpha(n,a_2)
    \]
    as well as
    \[
    \alpha(bn,a)=a\phi'(bn)=ba\phi'(n)=\alpha(n,ba)
    \]
    which demonstrates $\tilde \varphi$ satisfies the universal property below:
    \begin{center}
        \begin{tikzcd}
            N\times A \ar{r} \ar{dr}{\alpha}& N\otimes_B A\ar[dashed]{d}[description]{\exists!}\\
            &M
        \end{tikzcd}
    \end{center}
    Now to show $\tilde \varphi=\varphi^{-1}$ and $\varphi$ is a bijection, we have
    \[
    \varphi \circ \tilde \varphi(\phi')(n)=\tilde \varphi(\phi')(n\otimes 1)=\phi'(n)
    \]
    as well as
    \[
    \tilde \varphi \circ \varphi(\phi)(n\otimes a)=a\varphi(\phi)(n)=a\phi(n\otimes 1)=\phi(n\otimes a)
    \]
    which proves the two are inverses and are bijective.\\
    \newline
    Therefore we define $\tau_{NM}$ as $\varphi$ was above, so we've already shown $\tau_{NM}$ is a bijection. Now back to the diagram below,
    \begin{center}
        \begin{tikzcd}
            \Hom(N\otimes_B A,M)\ar{r}{f\otimes A^*} \ar{d}{\tau_{NM}}& \Hom(N'\otimes_B A,M)\ar{d}{\tau_{N'M}}\\
            \Hom(N,M_B)\ar{r}{f^*}& \Hom(N',M_B)
        \end{tikzcd}
    \end{center}
    we fix any $\phi\in \Hom(N\otimes_B A,M)$ and any $n'\in N'$, then
    \[
    \tau_{N'M}\circ f\otimes A^*(\phi)(n')=f\otimes A^*(\phi)(n'\otimes 1)=\phi \circ f\otimes A(n'\otimes 1)=\phi(f(n')\otimes 1)
    \]
    On the bottom side, we get
    \[
    f^*\circ \tau_{NM}(\phi)(n')=\tau_{NM}(\phi)\circ f(n')=\tau_{NM}(\phi)(f(n'))=\phi(f(n')\otimes 1)
    \]
    so the diagram commutes. To show
\begin{center}
        \begin{tikzcd}
            \Hom(N\otimes_B A,M)\ar{r}{g_*} \ar{d}{\tau_{NM}}& \Hom(N\otimes_B A,M')\ar{d}{\tau_{NM'}}\\
            \Hom(N,M_B)\ar{r}{g_*}& \Hom(N,M'_B)
        \end{tikzcd}
    \end{center}
    the above diagram commutes where the $g_*$ on the bottom is now considered to be $B$-linear, fix any $\phi\in \Hom(N\otimes_B A,M)$ and any $n\in N$. Then
    \[
    \tau_{NM'}\circ g_*(\phi)(n)=\tau_{NM'}(g\circ \phi)(n)=g\circ \phi(n\otimes 1)
    \]
    On the bottom,
    \[
    g_*\circ \tau_{NM}(\phi)(n)=g\circ \tau_{NM}(\phi)(n)=g\circ \phi(n\otimes 1)
    \]
    This proves that $\cdot_B$ is right adjoint to $\cdot \otimes_B A$.
\end{proof}
\subsubsection{F}\label{1.5.F}
\begin{proof}
    If $G$ is an abelian group, then we claim that the following diagram commutes for every map of abelian semigroups $\varphi$ and every abelian group $H$:
    \begin{center}
        \begin{tikzcd}
            G\ar{r}{\id_G}\ar{dr}{\varphi}&G\ar[dashed]{d}[description]{\exists!}\\
            & H
        \end{tikzcd}
    \end{center}
    If such a unique map $\phi$ were to exist, then it would satisfy
    \[
    \phi \circ \id_G=\varphi \Rightarrow \phi=\varphi
    \]
    This proves uniqueness. Existence is obvious since $\phi=\varphi$ gives existence.
\end{proof}
\subsubsection{G}\label{1.5.G}
\begin{proof}
    We will take the construction given in the problem for our construction, where
    \[
    [(a,b)]+[(c,d)]=[(a+c,b+d)]
    \]
    Suppose that $(a,b)\sim (a',b')$ and $(c,d)\sim (c',d')$. Then there exists $e_1$ and $e_2\in S$ where
    \[
    a+b'+e_1=b+a'+e_1
    \]
    and where
    \[
    c+d'+e_2=d+c'+e_2
    \]
    It follows that $(a+c,b+d)\sim (a'+c',b'+d')$ because 
    \begin{align*}
        a+c+b'+d'+(e_1+e_2)=(a+b'+e_1)+(c+d'+e_2)\\
        =(b+a'+e_1)+(d+c'+e_2)=a'+c'+b+d+(e_1+e_2)
    \end{align*}
    Therefore addition is well defined. Also
    \begin{align*}
        [(a,b)]+[(c,d)]=[(a+c,b+d)]=[(c+a,d+b)]=[(c,d)]+[(a,b)]
    \end{align*}
    so addition is commutative. In a similar fashion, addition is associate because it is in $S$. Because $S$ is nonempty, there exists some $s\in S$, so we claim that $[(s,s)]$ is the identity on $H(S)$. It's clear that this identity is independent of the choice of $s\in S$ because for any $s,r\in S$,
    \[
    s+r+s=r+s+s\Rightarrow (s,s)\sim(r,r)
    \]
    To show $[(s,s)]=0$, for any $a,b\in S$, we have
    \[
    [(c,c)]+[(a,b)]=[(c+a,c+b)]=[(a,b)]
    \]
    because 
    \[
    c+a+b+a=a+c+b+a\Rightarrow (c+a,c+b)\sim (a,b)
    \]
    Also, we claim that $[(a,b)]^{-1}=[(b,a)]$. To show this,
    \[
    [(a,b)]+[(b,a)]=[(a+b,b+a)]=[(a+b,a+b)]=0
    \]
    Therefore the abelian semigroup map is
    \[
    s\mapsto [(s+s,s)]
    \]
    To show this map $\varphi$ is linear, we see
    \[
    \varphi(a+b)=[(a+b+a+b,a+b)]=[(a+a,a)]+[(b+b,b)]=\varphi(a)+\varphi(b)
    \]
    Now to show that $H$ is left-adjoint to the forgetful functor $F$, we first want to show the following diagram commutes:
    \begin{center}
        \begin{tikzcd}
            \Hom(H(A),B) \ar{r}{Hf^*} \ar{d}{\tau_{AB}}& \Hom(H(A'),B)\ar{d}{\tau_{A'B}}\\
            \Mor(A,F(B))\ar{r}{f^*}& \Mor(A', F(B))
        \end{tikzcd}
    \end{center}
     where we define $Hf([(a,b)])=[(f(a),f(b))]$ and where we define $\tau_{AB}(\phi)(a)=\phi([(a+a,a)])$ and $\tau_{AB}^{-1}(\varphi)([(a,b)])=\varphi(a)-\varphi(b)$. To show $\tau_{AB}^{-1}$ is well defined, suppose $(a,b)\sim (c,d)$. Then there exists some $e\in A$ such that
     \[
     a+d+e=c+b+e
     \]
     By linearity of $\varphi$, we get
     \[
     \varphi(a)+\varphi(d)+\varphi(e)=\varphi(c)+\varphi(b)+\varphi(e)
     \]
     Because now these are considered as objects of $B$, we have cancellation so
     \[
     \varphi(a)+\varphi(d)=\varphi(c)+\varphi(b)
     \]
     By subtraction in $B$, we get
     \[
     \varphi(a)-\varphi(b)=\varphi(c)-\varphi(d)\Rightarrow \tau_{AB}^{-1}(\varphi)([(a,b)])=\tau_{AB}^{-1}(\varphi)([(c,d)])
     \]
     Then for any $\varphi \in \Mor(A,F(B))$ and any $a\in A$, we get
     \[
     \tau_{AB}\circ \tau_{AB}^{-1}(\varphi)(a)=\tau_{AB}^{-1}(\varphi)([(a+a,a)])=\varphi(a+a)-\varphi(a)=\varphi(a)
     \]
     and on the other hand
     \[
     \tau_{AB}^{-1}\circ \tau_{AB}(\phi)([(a,b)])=\tau_{AB}(\phi)(a)-\tau_{AB}(\phi)(b)=\phi([(a+a,a)])-\phi([(b+b,b)])
     \]
     \[
     =\phi([(a+a,a)]-[(b+b,b)])=\phi([(a+a,a)]+[(b,b+b)])=\phi([(a+a+b,a+b+b)])=\phi([(a,b)])
     \]
     This proves that our $\tau_{AB}^{-1}$ is actually the inverse of $\tau_{AB}$ and that both are indeed bijections. To prove the diagram commutes, we have for any $a'\in A'$ and any $\phi\in \Hom(H(A),B)$,
     \begin{align*}
         \tau_{A'B}\circ Hf^*(\phi)(a')=Hf^*(\phi)([(a'+a',a')])=\phi\circ Hf([(a'+a',a')])\\
         =\phi([f(a')+f(a'),f(a')])
     \end{align*}
     On the other hand,
     \begin{align*}
         f^*\circ \tau_{AB}(\phi)(a')=\tau_{AB}(\phi)(f(a'))=\phi([f(a')+f(a'),f(a')])
     \end{align*}
     so this diagram does indeed commute. Now we want to show the following diagram commutes:
     \begin{center}
         \begin{tikzcd}
             \Hom(H(A),B)\ar{r}{g_*} \ar{d}{\tau_{AB}}& \Hom(H(A),B')\ar{d}{\tau_{AB'}}\\
             \Mor(A,F(B))\ar{r}{Fg_*}& \Mor(A,F(B'))
         \end{tikzcd}
     \end{center}
     Then for any $\phi \in \Hom(H(A),B)$ and any $a\in A$,
     \begin{align*}
         \tau_{AB'}\circ g_*(\phi)(a)=g_*(\phi)([(a+a,a)])=g\circ \phi([(a+a,a)])
     \end{align*}
     while along the bottom,
     \[
     Fg_*\circ \tau_{AB}(\phi)(a)=Fg\circ \tau_{AB}(\phi)(a)=Fg\circ \phi([(a+a,a)])=g\circ \phi([(a+a,a)])
     \]
     since $Fg(x)=g(x)$ for all $x\in F(B)$. Therefore both diagrams commute, which proves that $H$ is left adjoint to $F$.
\end{proof}
\subsubsection{H}\label{1.5.H}
\begin{proof}
    To show this embedding is fully faithful, it suffices to show that every morphism $f:M\to N$ in $\Mod_A$ defines a unique morphism $f:S^{-1}M\to S^{-1}N$ in $\Mod_{S^{-1}A}$ because it's clear that every $\Mod_{S^{-1}A}$ morphism defines a unique $\Mod_A$ morphism. By the universal property of $M$ and $N$, if $f:M\to N$ then we have the following commutative diagram:
    \begin{center}
        \begin{tikzcd}
            M\ar[hook, two heads]{r} \ar{dr}{f}&S^{-1}M\ar[dashed]{d}[description]{\exists!}\\
            &N
        \end{tikzcd}
    \end{center}
    because $S^{-1}M\cong M$ when $M$ is already an $S^{-1}A$ module. Also $N\cong S^{-1}N$ yields the desired unique $f':S^{-1}M\to S^{-1}N$.
    We could understand the action of the induced map $f':S^{-1}M\to S^{-1}N$ by noticing that
    \[
    1=f'(\frac{s}{s})=sf'(\frac{1}{s})=f'(\frac{1}{s})s
    \]
    so that
    \[
    f'(\frac{1}{s})=\frac{1}{s}
    \]
    which defines the desired $S^{-1}A$ module homomorphism which must act as
    \[
    f'(\frac{m}{s})=\frac{1}{s}\frac{f(m)}{1}=\frac{f(m)}{s}
    \]
    Furthermore, the forgetful functor applied to this induced homomorphism is indeed the original map $f$.\\
    If we let $L:\Mod_A\to \Mod_{S^{-1}A}$ be the localization functor, we claim that $L$ is left adjoint to the forgetful $F:\Mod_{S^{-1}A}\to \Mod_A$. For any objects $X,Y\in \Mod_A$ and $W,Z\in \Mod_{S^{-1}A}$ and any $f:Y\to X$, we have
    \begin{center}
        \begin{tikzcd}
            \Hom_{\Mod_{S^{-1}A}}(L(X),Z)\ar{d}{\tau_{XZ}} \ar{r}{Lf^*}& \Hom_{\Mod_{S^{-1}A}}(L(Y),Z)\ar{d}{\tau_{YZ}}\\
            \Hom_{\Mod_A}(X,F(Z))\ar{r}{f^*}&\Hom_{\Mod_A}(Y, F(Z))
        \end{tikzcd}
    \end{center}
    where $\tau$ is just the forgetful functor acting on homomorphisms, which commutes because
    \[
    f^*\circ \tau_{XZ}(g)(y)=\tau_{XZ}(g)\circ f(y)=g\circ f(y)
    \]
    and
    \[
    \tau_{YZ}\circ Lf^*(g)(y)=\tau_{YZ}(g\circ Lf)(y)=g\circ f(y)
    \]
    where $g\in \Hom_{\Mod_{S^{-1}A}}(L(X),Z)$ and $y\in Y$ were arbitrary. On the other hand,
    \begin{center}
        \begin{tikzcd}
            \Hom_{\Mod_{S^{-1}A}}(L(X),W)\ar{d}{\tau_{XW}} \ar{r}{g_*}& \Hom_{\Mod_{S^{-1}A}}(L(X),Z)\ar{d}{\tau_{XZ}}\\
            \Hom_{\Mod_A}(X,F(W))\ar{r}{Fg_*}&\Hom_{\Mod_A}(X, F(Z))
        \end{tikzcd}
    \end{center}
    which commutes because
    \[
    \tau_{XZ}\circ g_*(f)(x)=\tau_{XZ}(g\circ f)(x)=g\circ f(x)
    \]
    as well as
    \[
    Fg_*\circ \tau_{XY}(f)(x)=Fg\circ \tau_{XY}(f)(x)=g\circ f(x)
    \]
    where $g:W\to Z$, $f:L(X)\to W$ and $x\in X$ are arbitrary.\\
    Then indeed $L$ is left adjoint to $F$.
\end{proof}
\subsection{}
\subsubsection{A}\label{1.6.A}
\begin{proof}
    $\im f^i\xhookrightarrow{\iota^i} A^{i+1}$ by Lemma $\ref{lem:ker monic}$, so $0\rightarrow \im f^i \xrightarrow{\iota^i} A^{i+1}$ being exact is clear. Furthermore, if $\pi^i:A^{i+1}\twoheadrightarrow \cok f^i$ is the projection, $\ker \pi^i=\im \iota^i$ so $\im f^i \xrightarrow{\iota^i} A^{i+1}\xrightarrow{\pi^i} \cok f^i$ is exact. Finally, $\pi^i$ is epic by Lemma \ref{lem:cok epic}, which shows $A^{i+1}\xrightarrow{\pi^i} \cok f^i\rightarrow 0$ is exact as well, thus proving
    \[
    0\rightarrow \im f^i\xrightarrow{\iota^i} A^{i+1}\xrightarrow{\pi^i} \cok f^i \rightarrow 0
    \]
    is exact.\\
    For the second exact sequence, we first want a monomorphism $H^i(A^\bullet)\hookrightarrow \cok f^{i-1}$. For notation, let $j^i:\ker f^i\hookrightarrow A^i$ be the canonical maps for each $i$. First, we obtain the following induced morphism $\varphi^i$ from the below commutative diagram:
    \begin{center}
        \begin{tikzcd}
            &&A^{i+1}\\
            &\cok f^{i-1} \ar[dashed]{ur}[description]{\exists!}\\
            A^{i-1} \ar{ur}{0} \ar{r}{f^{i-1}}& A^i \ar[bend right]{uur}[swap]{f^i} \ar[two heads]{u}[swap]{\pi^{i-1}}
        \end{tikzcd}
    \end{center}
    Using this factorization of $f^i=\varphi^i\circ \pi^{i-1}$, we obtain another induced morphism $\phi^i$ from the following commutative diagram:
    \begin{center}
        \begin{tikzcd}
            &&A^{i+1}\\
            &\ker f^i \ar[hook]{r}{j^i} \ar{ur}{0}& A^i \ar{u}{f^i}\\
            \im d^{i-1} \ar[hook, bend right]{urr}[swap]{\iota^{i-1}} \ar[dashed]{ur}[description]{\exists!}
        \end{tikzcd}
    \end{center}
    where $f^i\circ \iota^{i-1}=0$ because $f^i=\varphi^i\circ \pi^{i-1}$ so
    \[
    f^i\circ \iota^{i-1}=\varphi^i\circ \pi^{i-1}\circ \iota^{i-1}=\varphi^i\circ 0=0
    \]
    Then we define $H^i(A^\bullet)=\ker f^i/\im f^{i-1}$ as $\cok \phi^i$, and let $\sigma^i:\ker f^i \twoheadrightarrow H^i(A^\bullet)$ be the projection. Then we obtain one last induced morphism $\chi^i$ from the following commutative diagram:
    \begin{center}
        \begin{tikzcd}
            &&\cok f^{i-1}\\
            &H^i(A^\bullet) \ar[dashed]{ur}[description]{\exists!}&\\
            \im f^{i-1} \ar{ur}{0} \ar[hook]{r}{\phi^i} \ar[bend right, hook]{rr}{\iota^{i-1}}& \ker{f^i} \ar[hook]{r}{j^i} \ar[two heads]{u}{\sigma^i}&A^i \ar[two heads]{uu}[swap]{\pi^{i-1}}
        \end{tikzcd}
    \end{center}
    where $\pi^{i-1}\circ j^i\circ \phi^i=0$ because $j^i\circ \phi^i=\iota^{i-1}$ and $\pi^{i-1}\circ \iota^{i-1}=0$.
    We claim that $\chi^i$ is the desired monomorphism. By Lemma \ref{lem:comp with monic and ker}, $\ker (\pi^{i-1}\circ j^i)=\ker \pi^{i-1}=\im f^{i-1}$, hence by commutativity of the above diagram $\ker (\chi^i \circ \sigma^i)=\im f^{i-1}$. By Lemma \ref{lem:epic iff coim is target}, since $\sigma^i$ is epic and
    \[
    \ker(\chi^i\circ \sigma^i)=\im f^{i-1}=\ker \sigma^i
    \]
    we obtain that $\chi^i$ is monic as desired. Thus $0\to H^i(A^\bullet)\xrightarrow{\chi^i}\cok f^{i-1}$ is exact.\\
    For the map $\omega^i:\cok f^{i-1}\to \im f^i$, we will let it be the induced map from the following commutative diagram:
    \begin{center}
        \begin{tikzcd}
        &&\im f^i\\
                                            &\cok f^{i-1} \ar[dashed]{ur}[description]{\exists!} \\
            A^{i-1}\ar{r}{f^{i-1}}\ar{ur}{0}&A^i\ar[two heads]{u}[swap]{\pi^{i-1}} \ar[two heads, bend right]{uur}[swap]{\tilde f^{i}}
        \end{tikzcd}
    \end{center}
    It follows from Lemma \ref{lem:comp epic then epic} that $\omega^i$ is epic, so that $\cok f^{i-1}\xrightarrow{\omega^i}\im f^i\rightarrow 0$ is exact.\\
    The last thing to show is that $\ker \omega^i=\im \chi^i$, which we can do by showing that $\cok \chi^i=\omega^i$. Because $\chi^i$ is monic, we have 
    \begin{align*}
        &\cok \chi^i=\cok f^{i-1}/H^i(A^\bullet)=\cok f^{i-1}/(\ker f^i/\im f^{i-1})\\
        &=(A^i/\im f^{i-1})/(\ker f^i/ \im f^{i-1})
    \end{align*}
    
    
    By Theorem \ref{thm:3IT} (or the 3IT) , we get that
    \[
    (A^i/\im f^{i-1})/(\ker f^i/ \im f^{i-1})=A^i/\ker f^i
    \]
    By the 1IT \ref{thm:1IT}, we have that
    \[
    A^i/\ker f^i=\im f^i
    \]
    which shows $\cok \chi^i=\im f^i$ as desired. Therefore $H^i(A^\bullet) \xrightarrow{\chi^i} \cok f^{i-1} \xrightarrow{\omega^i}$ is also exact, proving the following is exact:
    \[
    0\rightarrow H^i(A^\bullet)\xrightarrow{\chi^i}\cok f^{i-1} \xrightarrow{\omega^i} \im f^i \rightarrow 0
    \]
\end{proof}
\subsubsection{B}\label{1.6.B}
\begin{proof}
    Because $H^i(A^\bullet)=\ker d^i/\im d^{i-1}$, we get that $h^i(A^\bullet)=\dim(\ker d^i)-\dim (\im d^{i-1})$ by basic linear algebra. The rank-nullity theorem also gives us that
    \[
    \dim(\im d^i)+\dim(\ker d^i)=\dim A^i
    \]
    Therefore
    \[
    \sum (-1)^i \dim A^i=\sum (-1)^i [\dim(\im d^i)+\dim \ker(d^i)]
    \]
   We claim that the index $i$ is even if and only if $\dim(\ker d^i)$ and $\dim(\im d^i)$ have a positive sign in $\sum (-1)^i h^i(A^\bullet)$. For the $\dim(\ker d^i)$ term, this is immediate. We also notice that the $\dim(\im d^i)$ term actually comes from $h^{i+1}(A^\bullet)$, which has a factor of $(-1)^{i+1}=-1$, so that 
   \[
   (-1)^{i+1} h^{i+1}(A^\bullet)=-(\dim \ker d^{i+1}- \dim \im d^i)=\dim \im d^i-\dim \ker d^{i+1}
   \]
   so indeed the sign of the $\dim \im d^i$ is positive whenever $i$ is even. A very similar proof shows that the index $i$ is odd if and only if $\dim(\ker d^i)$ and $\dim(\im d^i)$ have a negative sign in $\sum (-1)^i h^i(A^\bullet)$. It follows that
   \[
   \sum (-1)^i \dim \im d^i=\sum (-1)^i h^i(A^\bullet)
   \]
   When $A^\bullet$ is exact, then $\ker d^i=\im d^{i-1}$ for every $i$, so in particular $\dim \ker d^i-\dim \im d^{i-1}=0$ for every $i$. By the main result, we get that
   \[
   \sum (-1)^i \dim A^i=\sum (-1)^i h^i(A^\bullet)=0
   \]
\end{proof}
\subsubsection{C}\label{1.6.C}
\begin{proof}
    We can define the addition structure of $\Hom(A^\bullet, B^\bullet)$ as $(\alpha+\beta)^i=\alpha^i+\beta^i$ for each $i$ where $\alpha,\beta\in \Mor(A^\bullet,B^\bullet)$. This gives abelian group structure to $\Hom(A^\bullet,B^\bullet)$ because for each $i$, addition commutes, associativity holds, and inverses and identities exist. This defines a morphism in $\Com_\fC$ because if for each $i$
    \[
    \alpha^{i+1}\circ f^i=g^i\circ \alpha^i
    \]
    and something similar for $\beta$, then
    \[
    (\alpha^{i+1}+\beta^{i+1})\circ f^i=\alpha^{i+1}\circ f^i+\beta^{i+1}\circ f^i=g^i\circ \alpha^i+g^i\circ \beta^i=g^i\circ (\alpha^i +\beta^i)
    \]
    because $\fC$ is an abelian category so composition distributes over addition. This shows that the sum of morphisms in $\Com_\fC$ are indeed commutative diagrams. We also need to show that addition distributes over composition. If $\alpha,\beta :B^\bullet \to C^\bullet$ in $\Com_\fC$ and $f,g:A^\bullet \to B^\bullet$, then
    \[
    [\alpha \circ(f+g)]^i=\alpha^i\circ(f^i+g^i)=\alpha^i\circ f^i+\alpha^i\circ g^i=(\alpha \circ f)^i+(\alpha \circ g)^i
    \]
    as well as
    \[
    [(\alpha+\beta)\circ f]^i=(\alpha^i\circ \beta^i)\circ f^i=\alpha^i\circ f^i+\beta^i\circ f^i=(\alpha \circ f)^i+(\beta\circ f)^i
    \]
    again by Ad1. in $\fC$.
    This shows that Ad1. holds for $\Com_\fC$.\\
    We claim that the zero object $0$ in $\Com_\fC$ is the exact sequence
    \[
    \dots \rightarrow0\rightarrow0\rightarrow0\rightarrow\dots
    \]
    We can prove that $0$ is initial because if we fix any
    \[
    \dots \rightarrow A^{i-1} \rightarrow A^i \rightarrow A^{i+1}\rightarrow \dots \in \Com_\fC
    \]
    then 
    \begin{center}
        \begin{tikzcd}
            \dots \ar{r}& 0 \ar{d} \ar{r}&0 \ar{d} \ar{r}&0 \ar{d} \ar{r}&\dots\\
            \dots \ar{r}& A^{i-1} \ar{r}& A^i \ar{r}& A^{i+1} \ar{r}& \dots
        \end{tikzcd}
    \end{center}
    clearly commutes, and because each arrow $0\to A^i$ is unique, it proves there is a unique morphism in $\Com_\fC$ from $0\to A^\bullet$ so indeed $0$ is the initial object in $\Com_\fC$. A very similar argument shows that $0$ is final in $\Com_\fC$, hence $0$ is the zero object in $\Com_\fC$. This proves that Ad2. holds in $\Com_\fC$.\\
    We define the product $A^\bullet \times B^\bullet$ as the complex where 
    \[
    (A^\bullet \times B^\bullet)^i=A^i\times B^i
    \]
    and the morphism $A^i\times B^i\rightarrow A^{i+1}\times B^{i+1}$ is given by $(A^i\rightarrow A^{i+1})\times(B^i\rightarrow B^{i+1})$, which more precisely is the induces morphism in the following commutative diagram:
    \begin{center}
        \begin{tikzcd}
            &A^i\times B^i \ar[dashed]{d}[description]{\exists!} \ar[bend right]{ddl} \ar[bend left]{ddr}\\
            &A^{i+1}\times B^{i+1} \ar{ddr} \ar{ddl}\\
            A^{i}\ar{d}&&B^i\ar{d}\\
            A^{i+1} &&B^{i+1}
        \end{tikzcd}
    \end{center}
    and the projection $A^\bullet \times B^\bullet\rightarrow A^\bullet$ is the following commutative diagram:
    \begin{center}
        \begin{tikzcd}
            \dots \ar{r}& A^{i-1}\times B^{i-1} \ar{d} \ar{r}&A^i\times B^i \ar{d} \ar{r}&A^{i+1}\times B^{i+1} \ar{d} \ar{r}&\dots\\
            \dots \ar{r}& A^{i-1} \ar{r}& A^i \ar{r}& A^{i+1} \ar{r}& \dots
        \end{tikzcd}
    \end{center}*
    which commutes by definition--the projection to $B^\bullet$ is almost defined identically. It's easy to show that this is indeed the product in $\Com_\fC$. Therefore $\Com_\fC$ satisfies Ad3., so $\Com_\fC$ is additive.\\
    To show $\Com_\fC$ is abelian, we take any $f:A^\bullet \to B^\bullet$ and claim that $\ker f$ is the complex below:
    \begin{center}
        \begin{tikzcd}
            \dots \ar{r}& \ker f^{i-1} \ar{r}&\ker f^i \ar{r}& \ker f^{i+1} \ar{r}& \dots
        \end{tikzcd}
    \end{center}
    where for each $i$ the arrow $\ker f^{i-1}\to \ker f^i$ is the one induced in the following diagram:
    \begin{center}
        \begin{tikzcd}
            &&B^i\\
            &\ker f^i \ar[hook]{r} \ar{ur}{0}& A^i \ar{u}[swap]{f^i}\\
            \ker f^{i-1} \ar[dashed]{ur}[description]{\exists!} \ar[hook]{rr}&& A^{i-1} \ar{u}
        \end{tikzcd}
    \end{center}
    where if for each $i$ we let $g^i:A^i\to A^{i+1}$ be the morphism of the complex $A^\bullet$ and $h^i:B^i\to B^{i+1}$ be the morphisms in the complex $B^\bullet$ and let $\iota^i:\ker f^i\hookrightarrow A^i$ be the inclusion, we have by definition of $f$ being a morphism in $\Com_\fC$ that the following diagram commutes:
    \begin{center}
        \begin{tikzcd}
            A^{i-1} \ar{d}{f^{i-1}} \ar{r}{g^{i-1}} &A^i\ar{d}{f^i}\\
            B^{i-1} \ar{r}{h^{i-1}}& B^i
        \end{tikzcd}
    \end{center}
    Therefore
    \[
    f^i\circ g^{i-1}\circ \iota^{i-1}=h^{i-1}\circ f^{i-1} \circ \iota^{i-1}= h^{i-1}\circ 0=0
    \]
    proving that we do indeed get the desired induced morphisms. By construction, the following diagram also commutes:
    \begin{center}
        \begin{tikzcd}
            \dots \ar{r}&A^{i-1} \ar{r}{g^{i-1}}& A^i \ar{r}{g^i}& A^i \ar{r}& \dots\\
            \dots \ar{r}& \ker f^{i-1} \ar[hook]{u}{\iota^{i-1}} \ar{r}&\ker f^{i} \ar[hook]{u}{\iota^i} \ar{r}&\ker f^{i+1} \ar[hook]{u}{\iota^{i+1}} \ar{r}&\dots
        \end{tikzcd}
    \end{center}
    We can define cokernels dually, and to be precise we should prove that these satisfy the universal property we want them to, but we shall not to save space. It's an easy exercise if you wish.\\
    This shows that kernels and cokernels exist, and $f$ is a monic if and only if each $f^i$ are monic, in which case we get by our constructions and the fact that $\fC$ is an abelian category that $\ker \cok f=f$. Similarly, $\cok \ker f=f$ whenever $f$ is epic. This shows that indeed $\Com_\fC$ is abelian.
\end{proof}
\subsubsection{D}\label{1.6.D}
\begin{proof}
    We will deal with the special case $\Mod_A$ for ease of proof, which suffices because of the Freyd-Mitchell Theorem although in general I try not to invoke this theorem. If $h\in \Hom(A^\bullet, B^\bullet)$, then we define a map $H^i(h):H^i(A^\bullet)\to H^i(B^\bullet)$ given by
    \[
    a+\im f^{i-1}\mapsto h^i(a) +\im g^{i-1}
    \]
    where $a\in \ker f^i$. Notice that if $a\in \ker f^i$, then $h^i(a)\in \ker g^i$ because 
    \[
    g^i\circ h(a)=h^{i+1}\circ f^i(a)=h^{i+1}(0)=0
    \]
    To show $H^i(h)$ is well defined, we need to show it's constant on representatives of $\im f^{i-1}$. To do this, fix any $a\in \im f^{i-1}\subset \ker f^{i}$ and let $f^{i-1}(x)=a$ for some $x\in A^{i-1}$. Then
    \[
    H^i(h)(a)=h^i(a)+\im g^{i-1}=h^i\circ f^{i-1}(x)+\im g^{i-1}=g^{i-1}\circ h^{i-1}(x)+\im g^{i-1}=\im g^{i-1}
    \]
    so indeed $H^i(h)$ is constant on $\im f^{i-1}$ so it is well defined.\\
    Then we can define $H^i:\Com_\fC \to \fC$ to be a functor. We can do this because $H^i(\id_{A^\bullet})$ acts on elements $a+\im f^{i-1}\in H^i(A^\bullet)$ as
    \[
    a+\im f^{i-1}\mapsto \id_{A^i}(a)+\im f^{i-1}=a+\im f^{i-1}
    \]
    which shows $H^i(\id_{A^\bullet})=\id_{H^i(A^\bullet)}$. If we're given $f:A^\bullet\to B^\bullet$ and $g:B^\bullet\to C^\bullet$ as morphisms in $\Com_\fC$, then for any $a\in \ker f^i$
    \begin{align*}
        H^i(g\circ f)([a])=[g\circ f(a)]=H^i(g)([f(a)])=H^i(g)\circ H^i(f)([a])
    \end{align*}
    Then indeed $H^i$ is a covariant functor.
\end{proof}
\subsubsection{E}\label{1.6.E}
\begin{proof}
    Let $f,g:C^\bullet \to D^\bullet$ be homotopic through maps $w:C^i\to D^{i-1}$, i.e. $f-g = dw+wd$. Fixing some index $i$, we have $f^i-g^i =d_D^{i-1} w^i + w^{i+1} d_C^i$. We quickly observe $d^{i-1} w^i$ induces the trivial map on homology since we mod out the image of $d^{i-1}$. In addition, $w^{i+1} d^i$ induces the trivial map on homology since $H^i(C^\bullet) = \ker d^i/\im d^{i-1}$, so applying $d^i$ kills anything in $H^i(C^\bullet)$. Then $f^i-g^i$ induces the trivial map on homology, i.e. $H^i(f) = H^i(g)$.
\end{proof}
\subsubsection{F}\label{1.6.F}
\begin{proof}
    Suppose $A' \xrightarrow{f} A\xrightarrow{g}A''$ is exact. If $F:\fA\to \fB$ is covariant, By Lemmas \ref{lem:covariant exact and commutes with im and coim} and \ref{lem:covariant left exact commutes with ker} we have
    \begin{align*}
        \im Ff=F\im f=F\ker g=\ker Fg
    \end{align*}
    Therefore $F(A')\xrightarrow{Ff} F(A) \xrightarrow{Fg}F(A'')$ is exact as desired. \\
    If $F:\fA\to \fB$ is contravariant, By Lemmas \ref{cor:contravariant left exact and cok to ker} and \ref{lem:contravariant exact and im to coim and coim to im} we get that 
    \begin{align*}
        &\im Fg=F\coim g=F\cok \ker g=F\cok \im f=\ker F\im f=\ker \coim Ff=\ker Ff
    \end{align*}
   
\end{proof}
\subsubsection{G}\label{1.6.G}
\begin{proof}
    \begin{enumerate}[(a)]
        \item To show the localization functor $L:\Mod_A\to \Mod_{S^{-1}A}$ is left exact, suppose $0\rightarrow M' \xrightarrow{f}M \xrightarrow{g}M''$ is exact. Then
        \[
        0\rightarrow S^{-1}M'\xrightarrow{Lf} S^{-1}M
        \]
        is exact because we know $f$ is injective, which we can use to demonstrate $Lf$ is injective as follows:
        \begin{align*}
            Lf(\frac{m'}{s})=0\iff \frac{f(m')}{s}=0
        \end{align*}
        if and only if there exists some $t\in S$ such that $tf(m')=0$. But because $f$ is $A-$linear, we notice
        \[
        tf(m')=f(tm')
        \]
        Therefore $tf(m')=0$ if and only if $f(tm')=0$, and now using the fact that $f$ is injective, we get that $tm'=0$, which proves that indeed
        \begin{align*}
            \frac{m'}{s}=0
        \end{align*}
        Therefore $Lf$ is injective as desired. To show $\ker Lg=\im Lf$, fix any $\frac{m}{s}\in \im Lf$ and let $Lf(\frac{m'}{s'})=\frac{f(m')}{s'}=\frac{m}{s}$. We want to show that $Lg(\frac{m}{s})=0$. To do this, we observe
        \begin{align*}
            Lg(\frac{m}{s})=Lg(\frac{f(m')}{s'})=\frac{g\circ f(m')}{s'}=\frac{0}{s'}=0
        \end{align*}
        because $\im f=\ker g$. This shows $\im Lf \subset \ker Lg$.\\
        For the reverse inclusion, suppose $Lg(\frac{m}{s})=\frac{g(m)}{s}=0$. Then there exists some $r\in S$ such that
        \[
        rg(m)=0
        \]
        If $g(m)\ne 0$, then by $A$-linearity of $g$ we get
        \[
        g(rm)=0\Rightarrow rm\in \ker g
        \]
        Because $\ker g=\im f$, let $f(m')=rm$. Then
        \begin{align*}
            Lf(\frac{m'}{rs})=\frac{f(m')}{rs}=\frac{rm}{rs}=\frac{m}{s}
        \end{align*}
        so indeed $\frac{m}{s}\in \im Lf$. Therefore $\ker Lg\subset \im Lf$, proving the following sequence is left exact:
        \[
        0\to S^{-1}M'\xrightarrow{Lf} S^{-1}M\xrightarrow{Lg} S^{-1}M''
        \]
        To show $L$ is right exact, suppose
        \[
        M'\xrightarrow{f}M \xrightarrow{g}M'' \rightarrow 0
        \]
        is exact. By the second argument in the proof that $L$ is left exact, we get that $M'\rightarrow M\rightarrow M''$ is exact. The last thing to show is that $Lg$ is surjective given $g$ is. To do this, fix any $\frac{m''}{s}\in S^{-1}M''$. Because $m''\in M''$ and $g$ is surjective, there exists some $m\in M$ such that $g(m)=m''$. Therefore
    \[
    Lg(\frac{m}{s})=\frac{g(m)}{s}=\frac{m''}{s}
    \]
    which shows $L$ is right exact.
    \item 
    Check the solution to Exercise 1.3.H
    \item 
    Suppose $0\rightarrow M' \xrightarrow{f}M\xrightarrow{g}M''$ is exact. To show $f_*$ is injective, suppose $f_*(h)=0$ where $h\in \Hom(C,M')$. By definition, then $f\circ h=0$. By Lemma \ref{lem:comp with monic and ker} $\ker(f\circ h)=\ker h$
    \[
    M=\ker 0=\ker (f\circ h)=\ker h
    \]
    which implies that $h=0$. Therefore $f_*$ is indeed injective.\\
    Now suppose $h\in \ker g_*$ or equivalently $g\circ h=0$. Then we get the following induced morphism $h'$:
    \begin{center}
        \begin{tikzcd}
            &&M''\\
            &\ker g \ar[hook]{r}{f} \ar{ur}{0}&M\ar{u}{g}\\
            C\ar[dashed]{ur}[description]{\exists!} \ar[bend right]{urr}{h}
        \end{tikzcd}
    \end{center}
    because $f$ monic implies $\im f=M'$ and we know $\ker g=\im f$.
    Therefore $h=f\circ h'$, or equivalently that $f_*(h')=h$ proving that $\im f_*\subset \ker g_*$.\\
    On the other hand, if $h\in \im f_*$, then let $h=f_*(h')$ or equivalently $h=f\circ h'$. Then clearly $h\in \ker g_*$ because
    \[
    g_*(h)=g\circ h=g\circ f\circ h'=0\circ h'=0
    \]
    since $\im f=\ker g$. This shows $\im f_*=\ker g_*$, which proves the following is exact:
    \begin{center}
        \begin{tikzcd}
            0\ar{r}& \Hom(C,M') \ar{r}{f_*}& \Hom(C,M) \ar{r}{g_*}& \Hom(C,M'')
        \end{tikzcd}
    \end{center}
    If $\fC$ is an abelian category, each hom-set is an abelian group, hence $\Hom(C,\cdot)$ defines a left exact covariant functor into $\Ab$.
    \item 
    Suppose $A\xrightarrow{f}B\xrightarrow{g}C\rightarrow 0$ is exact. To show $0\rightarrow \Hom(C,M) \xrightarrow{g^*} \Hom(B,M)$ is exact, we want to show that $g^*$ has a trivial kernel. If $g^*(h)=0$, then $h\circ g=0$ But because $C=\im g\subset \ker h$, then $C=\ker h$ so $h=0$ so indeed $g^*$ has a trivial kernel.\\
    To show $\Hom(C,M)\xrightarrow{g^*} \Hom(B,M) \xrightarrow{f^*} \Hom(A,M)$ is exact, fix any $h\in \ker f^*$. Then $h\circ f=0$, and $\im f=\ker g$ implies 
    \[
    \cok f=B/\im f=B/\ker g=\coim g=C
    \]
    so we get the following commutative diagram:
    \begin{center}
        \begin{tikzcd}
            &&M\\
            & C \ar[dashed]{ur}[description]{\exists!}\\
            A \ar{ur}{0} \ar{r}{f}& B \ar[two heads]{u}{g} \ar[bend right]{uur}[swap]{h}
        \end{tikzcd}
    \end{center}
    If $h':C\to M$ is the induced morphism, we get $h=h'\circ g=g^*(h')$ so indeed $h\in \im g^*$.\\
    On the other hand, if $h\in \im g^*$, let $h=g^*(h')=h'\circ g$. It's clear then that
    \[
    f^*(h)=h\circ f=h'\circ g\circ f=h'\circ 0=0
    \]
    so then $h\in \ker f^*$, which proves along with the previous result that $\im f^*=\ker g^*$. Then indeed the following sequence is exact:
    \[
    0\rightarrow \Hom(C,M) \xrightarrow{g^*} \Hom(B,M) \xrightarrow{f^*} \Hom(A,M)
    \]
    \end{enumerate}
    
\end{proof}
\subsubsection{H}\label{1.6.H}
\begin{proof}
    By the previous exercise we have that $\Hom(\cdot,N)$ is left exact and that the localization $L:\Mod_A \to \Mod_{S^{-1}}A$ is exact. Therefore on one hand we have
    \[
    A^{\oplus q} \xrightarrow{f} A^{\oplus p}\xrightarrow{g} M\rightarrow 0
    \]
    exact implies
    \[
    0\rightarrow \Hom_A(M,N)\xrightarrow{g^*} \Hom_A(A^{\oplus p},N)\xrightarrow{f^*} \Hom_A(A^{\oplus q},N)
    \]
    is exact and so by left exactness of $L$ we get that
    \[
    0\rightarrow S^{-1}\Hom_A(M,N)\xrightarrow{L(g^*)} S^{-1}\Hom_A(A^{\oplus p},N) \xrightarrow{L(f^*)} S^{-1}\Hom_A(A^{\oplus q},N)
    \]
    is exact. On the other hand by right exactness of $L$, we have
    \[
    S^{-1}A^{\oplus q} \xrightarrow{Lf} S^{-1}A^{\oplus p} \xrightarrow{Lg} S^{-1}M \rightarrow0
    \]
    is exact so by left exactness of $\Hom$ we get
    \begin{align*}
        0\rightarrow \Hom_{S^{-1}A}(S^{-1}M,S^{-1}N) \xrightarrow{Lg^*} \Hom_{S^{-1}A}(S^{-1}A^{\oplus p},S^{-1}N)\xrightarrow{Lf^*} \Hom_{S^{-1}A}(S^{-1}A^{\oplus q},S^{-1}N)
    \end{align*}
    is exact.\\
    If $\frac{h}{s}\in \ker L(f^*)$ where $h\in \Hom_A(A^{\oplus p},N)$, then
    \begin{align*}
        0=L(f^*)(\frac{h}{s})=\frac{f^*(h)}{s}=\frac{h\circ f}{s}
    \end{align*}
    Additionally, we notice that because
    \begin{align*}
        \frac{h}{s}(\frac{x}{s'})=\frac{h(x)}{ss'}
    \end{align*}
    and
    \begin{align*}
        Lf(\frac{y}{s'})=\frac{f(y)}{s'}
    \end{align*}
    it follows that
    \begin{align*}
        Lf^*(\frac{h}{s})=\frac{h}{s}\circ Lf=\frac{h\circ f}{s}
    \end{align*}
    Then indeed $\ker L(f^*)\subset \ker Lf^*$.\\
    If now $Lf^*(h)=0$ where $h\in \Hom_{S^{-1}A}(S^{-1}A^{\oplus p},S^{-1}N)$,
    \begin{align*}
        0=Lf^*(h)=h\circ Lf
    \end{align*}
    and
    \begin{align*}
        L(f^*)(h)=L(f^*)(\frac{h}{1})=\frac{f^*(h)}{1}=\frac{h\circ f}{1}
    \end{align*}
    It's an easy exercise to verify that by $S^{-1}A$ linearity of $h$, indeed 
    \[
    h\circ Lf(\frac{x}{s})=\frac{h\circ f(x)}{s}=\frac{h\circ f}{1}(\frac{x}{s})
    \]
    for arbitrary $\frac{x}{s}\in S^{-1}A^{\oplus q}$. This proves that $\ker Lf^*\subset \ker L(f^*)$, so
    \[
    \ker Lf^*=\ker L(f^*)
    \]
    Therefore by exactness of the two sequence and by Lemma \ref{lem:monic iff im is source} applied to $L(g^*)$ and $Lg^*$,
    \begin{align*}
        \Hom_{S^{-1}A}(S^{-1}M,S^{-1}N)=\im Lg^*=\ker Lf^*=\ker L(f^*)=\im L(g^*)=S^{-1}\Hom_A(M,N)
    \end{align*}
\end{proof}
\subsubsection{I}\label{1.6.I}
\begin{proof}
    For this proof, we will use notation from Exercise 1.6.A for the canonical and induced maps. We will also use the notation that if $f:A\to B$ is a morphism, then $f\vert^{\im} :A\to \im f$ is the induced morphism.
    \begin{enumerate}[(a)]
        \item By Exercise 1.6.A, the following sequence is exact:
    \begin{align*}
        0\rightarrow H^i(C^\bullet) \xrightarrow{\chi^i}\cok d^{i-1} \xrightarrow{\omega^i}\im d^i\rightarrow 0
    \end{align*}
        By right exactness of $F$, the following is exact:
        \begin{align*}
            FH^i(C^\bullet) \xrightarrow{F\chi^i} F\cok d^{i-1} \xrightarrow{F \omega^i} F\im d^i \rightarrow 0
        \end{align*}
        By Lemma \ref{lem:covariant right exact commutes with cok}, we have that $F\cok d^{i-1}=\cok Fd^{i-1}$. By Exercise 1.6.A again, the following sequence is exact as well:
        \begin{align*}
            0\rightarrow H^i(FC^\bullet) \xrightarrow{\chi}\cok Fd^{i-1} \xrightarrow{\omega}\im Fd^i\rightarrow 0
        \end{align*}
        By Exercise 1.6.A again, we have
        \begin{align*}
            0\rightarrow \im d^i \xrightarrow{\iota^i} C^{i+1} \xrightarrow{\pi^i} \cok d^i \rightarrow 0
        \end{align*}
        is exact, so by right exactness of $F$, the following is also exact:
        \begin{align*}
            F\im d^i \xrightarrow{F\iota^{i}} FC^{i+1} \xrightarrow{F\pi^i} F\cok d^i \rightarrow 0
        \end{align*}
    We claim that the following diagram commutes:
    \begin{center}
        \begin{tikzcd}
            F\cok d^{i-1} \ar{r} {F\omega^i} \ar{d}{=}& F\im d^i \ar{d}{F\iota^i\vert^{\im}}\\
            \cok Fd^{i-1} \ar{r}{\omega}& \im Fd^i
        \end{tikzcd}
    \end{center}
   To show this, we just need to recall the definitions of our morphisms. Firstly, we have the following commutative diagram:
   \begin{center}
       \begin{tikzcd}
           &&F\im d^i \ar{r}{F\iota^i}& FC^{i+1}\\
           &F\cok d^{i-1} \ar[two heads]{ur}{F\omega^i}\\
           FC^{i-1} \ar{ur}{0}\ar{r}{Fd^{i-1}}& FC^i \ar[two heads]{u}{F\pi^{i-1}} \ar[bend right, two heads]{uur}[swap]{F(d^i\vert^{\im})} \ar[bend right=45]{uurr}[swap]{Fd^i}
       \end{tikzcd}
   \end{center}
   as well as the commutative diagram below:
   \begin{center}
       \begin{tikzcd}
           &&\im Fd^i \ar[hook]{r}{\iota}& FC^{i+1}\\
           &\cok Fd^{i-1} \ar[two heads]{ur}{\omega}\\
           FC^{i-1} \ar{ur}{0}\ar{r}{Fd^{i-1}}& FC^i \ar[two heads]{u}{F\pi^{i-1}} \ar[bend right, two heads]{uur}[swap]{(Fd^i)\vert^{\im})} \ar[bend right=45]{uurr}[swap]{Fd^i}
       \end{tikzcd}
   \end{center}
   We observe that by commutativity of the two diagrams,
   \begin{align*}
       \iota \circ \omega \circ F\pi^{i-1}=\iota \circ (Fd^i)\vert^{\im}=Fd^i=F\iota^i\circ F(d^i\vert^{\im})=F\iota^i\circ F\omega^i\circ F\pi^{i-1}
   \end{align*}
   Because $F\pi^{i-1}$ is epic, we obtain the following commutative diagram:
   \begin{center}
       \begin{tikzcd}
           F\im d^i \ar{r}{F\iota^i}& FC^{i+1}\\
           \cok Fd^{i-1} \ar[two heads]{u}{F\omega^i} \ar[two heads]{r}{\omega}& \im Fd^i \ar[hook]{u}{\iota}
       \end{tikzcd}
   \end{center}
   Using Lemma \ref{lem:covariant right exact commutes with cok} and the exactness of our sequences, we see
\begin{align*}
    \im F\iota^i=\ker F\pi^i=\ker F\cok d^i=\ker \cok F d^i=\im Fd^i
\end{align*}
   Thus the following diagram commutes:
   \begin{center}
       \begin{tikzcd}
           F\im d^i \ar{r}{F\iota^i} \ar[two heads]{dr}{F\iota^i \vert^{\im}}& FC^{i+1}\\
           \cok Fd^{i-1} \ar[two heads]{u}{F\omega^i} \ar[two heads]{r}{\omega}& \im Fd^i \ar[hook]{u}{\iota}
       \end{tikzcd}
   \end{center}
   Then by Lemma \ref{lem:ker of comp into ker of first}, we get the desired canonical inclusion $\theta:\ker F\omega^i \hookrightarrow \ker \omega$
    which shows
    \begin{align*}
        FH^i(C^\bullet)\xtwoheadrightarrow{F\chi^i\vert^{\im}} \im F\chi^i=\ker F\omega^i \xhookrightarrow{\theta} \ker \omega=\im \chi=H^i(FC^\bullet)
    \end{align*}
    \item 
    By Vakil (1.6.5.3), we have the short exact sequences below:
    \begin{align*}
        0\rightarrow \im d^{i-1}\xrightarrow{j^i} \ker d^i \xrightarrow{\phi^i} H^i(C^\bullet)\rightarrow0\\
        0\rightarrow \im Fd^{i-1}\xrightarrow{j} \ker Fd^i \xrightarrow{\phi} H^i(FC^\bullet)\rightarrow0
    \end{align*}
    By left exactness of $F$, we get from the first sequence the following exact sequence:
    \begin{align*}
        0\rightarrow F\im d^{i-1}\xrightarrow{Fj^i} F\ker d^i \xrightarrow{F\phi^i} FH^i(C^\bullet)
    \end{align*}
    and by Lemma \ref{lem:covariant left exact commutes with ker} $F\ker d^i=\ker Fd^i$. We now observe the following commutative diagram:
    \begin{center}
        \begin{tikzcd}
            && F\cok d^{i-1}\\
            &\cok Fd^{i-1} \ar[dashed]{ur}[description]{\exists!}\\
            FC^{i-1} \ar{ur}{0} \ar{r}{Fd^{i-1}}&FC^i \ar[two heads]{u}{\pi} \ar[bend right]{uur}{F\pi^{i-1}}
        \end{tikzcd}
    \end{center}
    By Lemma \ref{lem:covariant left exact commutes with ker}, we have $\ker F\pi^{i-1}= F\ker \pi^{i-1}=F\im d^{i-1}$. By Lemma \ref{lem:ker of comp into ker of first} using the factorization of $F\pi^{i-1}$ through $\pi$, we get the following commutative diagram:
    \begin{center}
        \begin{tikzcd}
            &&F\cok d^{i-1}\\
            &\ker F\pi^{i-1} \ar{ur}{0} \ar[hook]{r}{F \iota^{i-1}}&FC^i\ar{u}[swap]{F\pi^{i-1}}\\
            \ker \pi \ar[dashed]{ur}[description]{\exists!} \ar[hook,bend right]{urr}{\iota}
        \end{tikzcd}
    \end{center}
    Letting $\alpha:\im Fd^{i-1}\to F\im d^{i-1}$ be the induced monomorphism in the diagram above, we now claim that $j=Fj^i\circ \alpha$. To see this, we need to recall how we obtained $j^i$ (a nearly identical idea is used to define $j$):
    \begin{center}
        \begin{tikzcd}
            &&C^{i+1}\\
            &\cok d^{i-1}\ar[dashed]{ur}[description]{\varphi^i}\\
            C^{i-1} \ar{ur}{0} \ar{r}{d^{i-1}}& C^i \ar[two heads]{u}{\pi^{i-1}} \ar[bend right]{uur}{d^i}
        \end{tikzcd}
    \end{center}
    implies that the following diagram commutes as well
    \begin{center}
        \begin{tikzcd}
            &&C^{i+1}\\
            &\ker d^i \ar[hook]{r}{\kappa^i} \ar{ur}{0}& C^i \ar{u}{d^i}\\
            \im d^{i-1} \ar[dashed]{ur}[description]{\exists! j^i} \ar[hook, bend right]{urr}{\iota^{i-1}}
        \end{tikzcd}
    \end{center}
    So we get that $\kappa^i\circ j^i=\iota^{i-1}$ and similarly $\kappa \circ j=\iota$. Then we have
    \begin{align*}
        \iota=F\iota^{i-1}\circ \alpha =F\kappa^i\circ Fj^i \circ \alpha
    \end{align*}
    as well as
    \begin{align*}
        \iota=\kappa \circ j
    \end{align*}
    Recalling that by Lemma \ref{lem:covariant left exact commutes with ker}, $F\ker d^i=\ker Fd^i$ implies that $\kappa=F\kappa^i$. Therefore
    \begin{align*}
        \kappa \circ j=\kappa \circ Fj^i \circ \alpha
    \end{align*}
    implies, because $\kappa$ is monic, that indeed $j=Fj^i\circ \alpha$. Thus we have the following commutative diagram that is exact across rows:
    \begin{center}
        \begin{tikzcd}
            0\ar{r}& F\im d^{i-1} \ar{r}{Fj^i}& F\ker d^i \ar{r}{F\phi^i}& FH^i(C^\bullet)\\
            0\ar{r}& \im Fd^{i-1} \ar{r}{j} \ar[hook]{u}{\alpha}& \ker Fd^i \ar[<->]{u}{=} \ar{r}{\phi}& H^i(FC^\bullet)\ar{r}& 0
        \end{tikzcd}
    \end{center}
    We have two last morphisms to construct, and composing them will be the desired morphism. The first is the induced morphism $\varphi:\cok Fj^i\to F\cok j^i$ from the following commutative diagram:
        \begin{center}
        \begin{tikzcd}
            &&F\cok j^i\\
            &\cok Fj^i \ar[dashed]{ur}[description]{\exists! \varphi}\\
            F\im d^{i-1} \ar[hook]{r}{Fj^i} \ar{ur}{0}& \ker Fd^i \ar[two heads]{u}{\sigma} \ar[bend right]{uur}{F\phi^i}
        \end{tikzcd}
    \end{center}
    The second comes from Lemma \ref{lem:cok of comp onto cok of second}, where we get an epimorphism $\lambda:\cok j \twoheadrightarrow \cok Fj^i$ such that $\sigma=\lambda \circ \phi$. Then we have
    \begin{align*}
        \cok j=H^i(FC^\bullet)\xtwoheadrightarrow{\lambda} \cok Fj^i \xrightarrow{\varphi} F\cok j^i=FH^i(C^\bullet)
    \end{align*}
    \item 
    If $F$ is exact, then $\omega=F\omega^i$ because by Lemma \ref{lem:covariant exact and commutes with im and coim}. Therefore $\theta$, the canonical inclusion from $\ker F\omega^i\hookrightarrow \ker \omega$ is actually just $\id_{\ker F\omega^i}$. Additionally, $F\chi^i\vert^{\im} =\id_{FH^i(C^\bullet)}$ because by left-exactness of $F$, $F\chi^i=\ker F\omega^i$.\\
    Additionally, by right exactness of $F$, $F\phi^i=\phi$. By our constructions, we have
    \[
    F\phi^i=\varphi\circ \sigma
    \]
    and
    \[
    \sigma=\lambda \circ \phi
    \]
    Therefore
    \[
    \phi=\varphi \circ \sigma=\varphi\circ \lambda \circ \phi
    \]
    which by the fact that $\phi$ is an epimorphism shows $\varphi \circ \lambda=\id_{H^i(FC^\bullet)}$. This proves that indeed $\varphi \circ \lambda$ and $\theta \circ F\chi^i\vert^{\im}$ are inverses of each other because they are both the identity on $H^i(FC^\bullet)=FH^i(C^\bullet)$. Though it may feel strange that all of our maps just turned into the identity but they were originally going from one object to another, it's because each of the objects satisfies the definition of the other when $F$ is exact, so they are the same object.
    \end{enumerate}
\end{proof}
\subsubsection{J}\label{1.6.J}
\begin{proof}
    This exercise is a special case of Exercise 1.6.J below because kernels are limits of the following diagram $\fJ$:
    \begin{center}
        \begin{tikzcd}
            &\bullet \ar{d}\\
            \bullet \ar{r}&\bullet
        \end{tikzcd}
    \end{center}
where $F:\fI\times \fJ\to \fC$ is the product functor of the functors $\fI \to \fC$ and $\fJ\to \fC$ and where $h=F(\id, g)$ where $g$ is the arrow on the bottom of $\fJ$.
\end{proof}
\subsubsection{K}\label{1.6.K}
\begin{proof}
    Let $F:\fI\times \fJ\to \fC$ be a covariant functor so that $\fI\times \fJ$, the product category of $\fI$ and $\fJ$, will be the index category of the desired limits. For the rest of the proof, let $i,i'\in \fI$, $j,j'\in \fJ$, $f:i\to i'$ and $g:j\to j'$ be arbitrary. To begin with, we notice we get a natural transformation $F(f,\id):F(i,\cdot)\to F(i',\cdot)$ demonstrated below:
    \begin{center}
        \begin{tikzcd}
            F(i,j)\ar{r}{F(\id_i, g)}\ar{d}[swap]{F(f,\id_j)}& F(i,j') \ar{d}{F(f,\id_{j'})}\\
            F(i',j) \ar{r}{F(\id_{i'},g)}& F(i',j')
        \end{tikzcd}
    \end{center}
    which commutes essentially by definition of the product category. Then we obtain an induced morphism $\tilde f:\lim_j F(i,j)\to \lim_j F(i',j)$ from the commutative diagram below:
    \begin{center}
        \begin{tikzcd}
            &&\lim_j F(i,j) \ar[swap]{ddll}{p_{ij}} \ar{ddrr}{p_{ij'}} \ar[dashed]{dd}[description]{\exists!}\\
            \\
            F(i,j) \arrow[bend left=10, crossing over, pos=0.34]{rrrr}[description]{F(\id_i,g)} \ar[swap]{dr}{F(f,\id_j)}&&\lim_j F(i',j) \ar[swap]{dl}{p_{i'j}} \ar{dr}{p_{i'j'}}&& F(i,j') \ar{dl}{F(f,\id_{j'})}\\
            &F(i',j)\ar{rr}{F(\id_{i'},g)}&&F(i',j')
        \end{tikzcd}
    \end{center}
    With these induced morphisms $\Tilde{f}$, we can actually index the $\lim_j F(i,j)$'s by $I$. This gives us our definition of $\lim_i \lim_j F(i,j)$ below:
    \begin{center}
        \begin{tikzcd}
            &\lim_i \lim_j F(i,j) \ar{dl}[swap]{\tau_i} \ar{dr}{\tau_{i'}}\\
            \lim_j F(i,j) \ar{rr}{\tilde f}& &\lim_j F(i',j)
        \end{tikzcd}
    \end{center}
    Similar to above, we get an induced $\tilde g:\lim_i F(i,j)\to \lim_i(F(i,j')$ from the following commutative diagram for each $g:j\to j'$:
    \begin{center}
        \begin{tikzcd}
            &&\lim_i F(i,j) \ar[swap]{ddll}{q_{ij}} \ar{ddrr}{q_{i'j}} \ar[dashed]{dd}[description]{\exists!}\\
            \\
            F(i,j) \arrow[bend left=10, crossing over, pos=0.34]{rrrr}[description]{F(f,\id_j)} \ar[swap]{dr}{F(\id_i,g)}&&\lim_i F(i,j') \ar[swap]{dl}{q_{ij'}} \ar{dr}{q_{i'j'}}&& F(i',j) \ar{dl}{F(\id_{i'},g)}\\
            &F(i,j')\ar{rr}{F(f,\id_{j'})}&&F(i',j')
        \end{tikzcd}
    \end{center}
    Similarly, we can index the $\lim_i F(i,j)$'s by $J$ with these induced $\tilde g$'s, so we also obtain the following construction for $\lim_j \lim_i F(i,j)$ below:
    \begin{center}
        \begin{tikzcd}
            &\lim_j \lim_i F(i,j) \ar{dl}[swap]{\sigma_j} \ar{dr}{\sigma_{j'}}\\
            \lim_i F(i,j) \ar{rr}{\tilde g}& &\lim_i F(i,j')
        \end{tikzcd}
    \end{center}
    We also observe the following diagram
   \begin{center}
        \begin{tikzcd}
            &&\lim_i\lim_j F(i,j) \ar[swap]{ddll}{\tau_i} \ar{ddrr}{\tau_{i'}} \ar[dashed]{dd}[description]{\exists!}\\
            \\
            \lim_j F(i,j) \arrow[bend left=10, crossing over, pos=0.34]{rrrr}[description]{\tilde f} \ar[swap]{dr}{p_{ij}}&&\lim_i F(i,j) \ar[swap]{dl}{q_{ij}} \ar{dr}{q_{i'j}}&& \lim_jF(i',j) \ar{dl}{p_{i'j}}\\
            &F(i,j)\ar{rr}{F(f,\id_{j})}&&F(i',j)
        \end{tikzcd}
    \end{center}
    which commutes because
    \begin{align*}
        F(f,\id_j)\circ p_{ij}\circ \tau_i=p_{i'j}\circ \tilde f\circ \tau_i=p_{i'j}\circ \tau_{i'}
    \end{align*}
    Let $\varphi_j:\lim_i\lim_j F(i,j)\to \lim_i F(i,j)$ be the induced morphism above. Now we want to show that $\tilde g\circ \varphi_j=\varphi_{j'}$. To do this, consider the following diagram:
    \begin{center}
        \begin{tikzcd}
            &&\lim_i\lim_j F(i,j) \ar[swap]{ddll}{\tau_i} \ar{ddrr}{\tau_{i'}} \ar{dd}[description]{\tilde g\circ \varphi_j}\\
            \\
            \lim_j F(i,j) \ar[swap]{dr}{p_{ij'}}&&\lim_i F(i,j') \ar[swap]{dl}{q_{ij'}} \ar{dr}{q_{i'j'}}&& \lim_jF(i',j) \ar{dl}{p_{i'j'}}\\
            &F(i,j')\ar{rr}{F(f,\id_{j'})}&&F(i',j')
        \end{tikzcd}
    \end{center}
    which commutes because
    \begin{align*}
        q_{ij'}\circ \tilde g\circ \varphi_j=F(\id_{i},g)\circ q_{ij}\circ \varphi_j=F(\id_i,g)\circ p_{ij}\circ \tau_i=p_{ij'}\circ \tau_i
    \end{align*}
    as well as
    \begin{align*}
        q_{i'j'}\circ \tilde g\circ \varphi_j=F(\id_{i'},g)\circ q_{i'j}\circ \varphi_j=F(\id_{i'},g)\circ p_{i'j}\circ \tau_{i'}=p_{i'j'}\circ \tau_{i'}
    \end{align*}
    Because the diagram commutes, by uniqueness of $\varphi_{j'}$ we get that indeed $\varphi_{j'}=\tilde g\circ \varphi_j$. We claim now that $\lim_i \lim_j F(i,j)$ together with our morphisms $\varphi_j$ are universal with respect to this diagram. To prove this, suppose we have the following commutative diagram:
    \begin{center}
        \begin{tikzcd}
            &W \ar{dl}[swap]{\chi_j} \ar{dr}{\chi_{j'}}\\
            \lim_i F(i,j) \ar{rr}{\tilde g}& &\lim_i F(i,j')
        \end{tikzcd}
    \end{center}
    Then by the below commutative diagram, we get an induced $\mu_i:W\to \lim_j F(i,j)$:
    \begin{center}
        \begin{tikzcd}
            &&W \ar[swap]{ddll}{\chi_j} \ar{ddrr}{\chi_{j'}} \ar[dashed]{dd}[description]{\exists!}\\
            \\
            \lim_i F(i,j) \arrow[bend left=10, crossing over, pos=0.34]{rrrr}[description]{\tilde g} \ar[swap]{dr}{q_{ij}}&&\lim_j F(i,j) \ar[swap]{dl}{p_{ij}} \ar{dr}{p_{ij'}}&& \lim_iF(i,j') \ar{dl}{q_{ij'}}\\
            &F(i,j)\ar{rr}{F(\id_i,g)}&&F(i,j')
        \end{tikzcd}
    \end{center}
    Now we claim that $\mu_{i'}=\tilde f\circ \mu_i$. To show this, observe the following diagram:
\begin{center}
        \begin{tikzcd}
            &&W \ar[swap]{ddll}{\chi_j} \ar{ddrr}{\chi_{j'}} \ar{dd}[description]{\tilde f\circ \mu_i}\\
            \\
            \lim_i F(i,j) \arrow[bend left=10, crossing over, pos=0.34]{rrrr}[description]{\tilde g} \ar[swap]{dr}{q_{i'j}}&&\lim_j F(i',j) \ar[swap]{dl}{p_{i'j}} \ar{dr}{p_{i'j'}}&& \lim_iF(i,j') \ar{dl}{q_{i'j'}}\\
            &F(i',j)\ar{rr}{F(\id_{i'},g)}&&F(i',j')
        \end{tikzcd}
    \end{center}
which commutes because
        \begin{align*}
            p_{i'j}\circ \tilde f\circ \mu_i=F(f,\id_j)\circ p_{ij}\circ \mu_i=F(f,\id_j)\circ q_{ij}\circ \chi_j=q_{i'j}\circ \chi_j
        \end{align*}
        and
        \begin{align*}
            p_{i'j'}\circ \tilde f\circ \mu_i=F(f,\id_{j'})\circ p_{ij'}\circ \mu_i=F(f,\id_{j'})\circ q_{ij'}\circ \chi_{j'}=q_{i'j'}\circ \chi_{j'}
        \end{align*}
        By uniqueness of $\mu_{i'}$, we get that indeed $\mu_{i'}=\tilde f\circ \mu_i$. Thus we an induced $\theta:W\to \lim_i\lim_j F(i,j)$ from the following commutative diagram:
        \begin{center}
            \begin{tikzcd}
                &W \ar[bend right,swap]{ddl}{\mu_i} \ar[bend left]{ddr}{\mu_{i'}} \ar[dashed]{d}[description]{\exists!}\\
                & \lim_i \lim_j F(i,j) \ar[swap]{dl}{\tau_i} \ar{dr}{\tau_{i'}}\\
                \lim_j F(i,j) \ar{rr}{\tilde f}&& \lim_j F(i',j)
            \end{tikzcd}
        \end{center}
        Consider the following commutative diagram:
        \begin{center}
            \begin{tikzcd}
                &W \ar[bend right]{ddl} \ar[bend left]{ddr}\ar[shift right]{d}[swap]{\chi_j} \ar[shift left]{d}{\varphi_j\circ \theta} \\
                &\lim_i F(i,j)\ar{dl}[swap]{q_{ij}} \ar{dr}{q_{i'j}}\\
                F(i,j) \ar{rr}{F(f,\id_j)}&&F(i',j)
            \end{tikzcd}
        \end{center}
        We observe that 
        \begin{align*}
            q_{ij}\circ \chi_j=p_{ij}\circ \mu_i=p_{ij}\circ \tau_i \circ \theta=q_{ij}\circ \varphi_j \circ \theta
        \end{align*}
        as well as
        \begin{align*}
            q_{i'j}\circ \chi_j=p_{i'j}\circ \mu_{i'}=p_{i'j}\circ \tau_{i'}\circ \theta=q_{i'j}\circ \varphi_j\circ \theta
        \end{align*}
        This proves, by uniqueness of the arrow $W\to \lim_i F(i,j)$ that indeed $\chi_j=\varphi_j\circ \theta$. We need to consider one final commutative diagram:
        \begin{center}
            \begin{tikzcd}
                &W \ar[bend right]{ddl} \ar[bend left]{ddr}\ar[shift right]{d}[swap]{\chi_{j'}} \ar[shift left]{d}{\varphi_{j'}\circ \theta} \\
                &\lim_i F(i,j')\ar{dl}[swap]{q_{ij'}} \ar{dr}{q_{i'j'}}\\
                F(i,j') \ar{rr}{F(f,\id_{j'})}&&F(i',j')
            \end{tikzcd}
        \end{center}
        We observe that
            \begin{align*}
                q_{ij'}\circ \chi_{j'}=p_{ij'}\circ \mu_i=p_{ij'}\circ \tau_i\circ \theta=q_{ij'}\circ \varphi_{j'}\circ \theta
            \end{align*}
            as well as
            \begin{align*}
                q_{i'j'}\circ \chi_{j'}=p_{i'j'}\circ \mu_{i'}=p_{i'j'}\circ \tau_{i'}\circ \theta=q_{i'j'}\circ \varphi_{i'}\circ \theta
            \end{align*}
            Again, by uniqueness of the arrow $W\to \lim_i F(i,j')$, we get that $\chi_j=\varphi_{j'}\circ \theta$. Thus indeed $\theta$ is the unique morphism making the following diagram commute:
            \begin{center}
            \begin{tikzcd}
                &W \ar[bend right]{ddl}[swap]{\chi_j} \ar[bend left]{ddr}{\chi_{j'}} \ar[dashed]{d}[description]{\exists!} \\
                &\lim_i\lim_j F(i,j)\ar{dl}[swap]{\varphi_j} \ar{dr}{\varphi_{j'}}\\
                \lim_i F(i,j) \ar{rr}{\tilde g}&&\lim_i F(i,j')
            \end{tikzcd}
        \end{center}
        Therefore $\lim_i\lim_j F(i,j)$ satisfies the universal property of $\lim_j \lim_i F(i,j)$, proving the two are equal.
\end{proof}
\subsubsection{L}\label{1.6.L}
\begin{proof}
By Exercise 1.4.F, we know what colimits look like in $\Mod_A$. Suppose that $F,G,H: \fI\to \Mod_A$ are the index functors and we have $f\in \Nat(F,G)$ and $g\in \Nat( G,H)$ such that the following sequence is exact in the category of functors $\Mod_A^\fI$:
\begin{center}
    \begin{tikzcd}
        0 \ar{r}& F \ar{r}{f}& G \ar{r}{g}& H \ar{r}& 0
    \end{tikzcd}
\end{center}
where in particular, the $f_i$'s are from the natural transformation $f:F\to G$. We want to show that the following sequence is exact:
\begin{center}
    \begin{tikzcd}
        0 \ar{r}& \colim M_i \ar{r}{\colim f}& \colim N_i \ar{r}{\colim g}& \colim P_i \ar{r}& 0
    \end{tikzcd}
\end{center}
where $\colim f$ is induced by the following commutative diagram
\begin{center}
    \begin{tikzcd}
        &&\colim N_i \\
            \\
            N_i \ar{uurr} \arrow[bend left=10, crossing over, pos=0.34]{rrrr}&&\colim M_i \ar[dashed]{uu}[description]{\exists!}&& N_j \ar{uull}\\
            &M_i \ar{rr} \ar{ul}{f_i} \ar{ur}&&M_j \ar{ul} \ar{ur}[swap]{f_j}
    \end{tikzcd}
\end{center}
and $\colim g$ is induced by a similar one. More explicitly, the map $\colim f$ and $\colim g$ acts as
\begin{align*}
    \colim f ([m_i,i])=[f_i(m_i),i]\\
    \colim g([n_i,i])=[g_i(n_i),i]
\end{align*}
Suppose that $\colim f([m_i,i])=0$. By definition, this means $(f_i(m_i),i)\sim 0$, which, by definition of the equivalence relation means there exists some $\kappa:i\to j$ such that $G(\kappa)(f_i(m_i))=0$ for some $j\in \fI$. Then we observe the following commutative diagram, which commutes by naturality of $f$:
\begin{center}
    \begin{tikzcd}
        N_i \ar{r}{G(\kappa)}& N_j\\
        M_i \ar{u}{f_i} \ar{r}{F(\kappa)}& M_j \ar{u}{f_j}
    \end{tikzcd}
\end{center}
Therefore 
\begin{align*}
    0=G(\kappa)(f_i(m_i))=f_j(F(\kappa)(m_i)) 
\end{align*}
implies that, because $f_j$ is injective since $f$ is monic, that $F(\kappa)(m_i)=0$. By definition, this means that $[m_i,i]=0$ so indeed $\colim f$ is monic.\\
To show $\colim g$ is epic --i.e. surjective -- fix any $[p_i,i]\in \colim P_i$. Then because $g_i$ is surjective, we get some $n_i\in N_i$ such that $g_i(n_i)=p_i$. Therefore
\begin{align*}
    \colim g([n_i,i])=[g_i(n_i),i]=[p_i,i]
\end{align*}
so $\colim g$ is surjective as well.\\
The last thing to show is that $\ker \colim g=\im \colim f$. If $[n_i,i]\in \im \colim f$, then $[n_i,i]=[f_i(m_i),i]$ for some $m_i\in M_i$. Then because $\im f_i=\ker g_i$,
\begin{align*}
    \colim g([n_i,i])=[g_i(n_i),i]=[g_i\circ f_i(m_i),i]=[0,i]=0
\end{align*}
shows that $\im \colim f \subset \ker \colim g$. On the other hand, fix any $[n_i,i]\in \ker \colim g$. Because $(n_i,i)\sim 0$, there exists some $\gamma:i\to j$ such that $H(\gamma) g_i(n_i)=0$. We observe the following commutative diagram
\begin{center}
    \begin{tikzcd}
        P_i \ar{r}{H(\gamma)}& P_j\\
        N_i \ar{u}{g_i} \ar{r}{G(\gamma)}& N_j \ar{u}{g_j}
    \end{tikzcd}
\end{center}
which shows then that
\begin{align*}
    0=H(\gamma)g_i(n_i)=g_j G(\gamma)(n_i)
\end{align*}
Therefore $G(\gamma)(n_i)\in \ker g_j$. Because $\ker g_j=\im f_j$, let $f_j(m_j)=G(\gamma)(n_i)$. This shows that $(n_i,i)\sim (f_j(m_j),j)$, proving
\begin{align*}
    [n_i,i]=[f_j(m_j),j]=\colim f([m_j,j])
\end{align*}
so $\ker \colim g\subset \im \colim f$, which proves equality holds and that indeed
\begin{center}
    \begin{tikzcd}
        0 \ar{r}& \colim M_i \ar{r}{\colim f}& \colim N_i \ar{r}{\colim g}& \colim P_i \ar{r}& 0
    \end{tikzcd}
\end{center}
is exact.
\end{proof}
\subsubsection{M}
\begin{proof}
    By Exercise 1.6.K, colimits are exact. Then we can use colimits as a functor from $\Mod_A^\fI$, and obtain by Exercise 1.6.H that
    \begin{align*}
        H \colim C^\bullet =\colim HC^\bullet
    \end{align*}
\end{proof}
\subsubsection{N}\label{1.6.N}
\begin{proof}
    Suppose the following is exact:
    \begin{center}
        \begin{tikzcd}
            0 \ar{r} & A^\bullet \ar{r}{f} & B^\bullet \ar{r}{g}& C^\bullet \ar{r}& 0
        \end{tikzcd}
    \end{center}
    We observe that we get a morphism $\lim f$ from the commutative diagram below:
   \begin{center}
        \begin{tikzcd}
            &&\lim A_n \ar[swap]{ddll} \ar{ddrr} \ar[dashed]{dd}[description]{\exists!}\\
            \\
            A_i \arrow[bend left=10, crossing over, pos=0.34, two heads]{rrrr} \ar{dr}{f_i}&&\lim B_n \ar{dl}{f_{i+1}} \ar{dr}&& A_{i+1} \ar{dl}\\
            &B_i\ar{rr}&&B_{i+1}
        \end{tikzcd}
    \end{center}
    We get a morphism $\lim g:\lim B_n \to \lim C_n$ similar to above. We obtain that these morphisms act as follows:
    \begin{align*}
        \lim f(a_1,a_2,\dots)=(f_1(a_1),f_2(a_2),\dots)\\
        \lim g(b_1,b_2,\dots)=(g_1(b_1),g_2(b_2),\dots)
    \end{align*}
    To show $\lim f$ is injective, suppose $\lim f(a_1,a_2,\dots)=0$. Then for each $i$, $f_i(a_i)=0$, which because each $f_i$ is injective, implies each $a_i=0$ so indeed $\lim f$ has trivial kernel.\\
    To show $\lim g$ is surjective, fix any $(c_1,c_2,\dots)\in \lim C_n$. For each $i$, there exists a $b_i$ such that $g_i(b_i)=c_i$ because each $g_i$ is surjective. Then
    \[
    \lim g(b_1,b_2,\dots)=(g(b_1),g(b_2),\dots)=(c_1,c_2,\dots)
    \]
    as desired. Now to show $\ker \lim g=\im \lim f$, pick any $(f_1(a_1),f_2(a_2),\dots)\in \im \lim f$. Because $\ker g_i=\im f_i$ for each $i$, we have
    \begin{align*}
        \lim g(f_1(a_1),f_2(a_2),\dots)=(g_1f_1(a_1),g_2f_2(a_2),\dots)=(0,0,\dots)=0
    \end{align*}
    which shows $\im \lim f\subset \ker \lim g$. Now suppose $(b_1,b_2,\dots)\in \ker \lim g$, i.e. $g_i(b_i)=0$ for each $i$. Because $\ker g_i=\im f_i$, we get that $b_i=f(a_i)$ for every $i$. Then
    \begin{align*}
        \lim f (a_1,a_2,\dots)=(f_1(a_1),f_2(a_2),\dots)=(b_1,b_2,\dots)
    \end{align*}
    proves $\ker \lim g \subset \im \lim f$. This proves that indeed
    \begin{center}
        \begin{tikzcd}
            0 \ar{r}& \lim A_n \ar{r}{\lim f}& \lim B_n \ar{r}{\lim g}& \lim C_n \ar{r}& 0
        \end{tikzcd}
    \end{center}
    is exact. As a side note, I believe that we used the hypothesis that the transition maps of the left term are surjective because this makes there only be one morphism from $A_i\to A_j$ with $i>j$, though I'm not entirely sure.
\end{proof}
\subsection{}

\section{}
\subsection{}
\subsubsection{A}\label{2.1.A}
\begin{proof}
    Fix any $(f,U)\in \fO_p\setminus \mathfrak{m}_p$. Then $f(p)\ne 0$ because $(f,U)\notin \mathfrak{m}_p$. Therefore $\frac{1}{f}\in \fO(V)$ for a sufficiently small neighborhood $V\subset U$ of $p$ such that $f$ is non vanishing on $W$, and $\frac{1}{f}$ must be smooth because $f$ is and doesn't vanish on $V$ by continuity of $f$. We easily obtain that
    \begin{align*}
        \frac{1}{f}(p)=\frac{1}{f(p)} \ne 0
    \end{align*}
    Therefore $(\frac{1}{f},V)\in \fO_p\setminus \mathfrak{m}_p$ as well. By definition of the equivalence relation, we get that $(f,U)=(f,V)$. Therefore we observe that
    \begin{align*}
        (f,V)(\frac{1}{f},V)=(\frac{f}{f},V)=(1,V)
    \end{align*}
    is the multiplicative identity on $\fO_p$. Because multiplication here is commutative --since it is on $\R^n$ -- we get that indeed $(f,U)$ has an inverse. This shows that if we have some other ideal $\mathfrak{n}\subset \fO_p$, it cannot be maximal because it is either contained in $\mathfrak{m}_p$, or it is the entire ring $\fO_p$.
\end{proof}
\subsubsection{B}\label{2.1.B}
\begin{proof}
    Here, we recall what the definition of the differential $d:C^\infty (M)\to T_p^*M$ given by
    \begin{align*}
        df(v)=v(f)
    \end{align*}
    where $f\in C^\infty (M)$ and $v\in T_p M$, i.e. a linear map $C^\infty(M)\to \R$ that satisfies the product rule. Now we will show that $d$ is constant on $\mathfrak{m}^2$. We recall that $d$ is linear; therefore
    \begin{align*}
        d(\sum_i f_ig_i)(v)=\sum_i d(f_ig_i)(v)=\sum_i v(f_ig_i)=\sum_i f_i(p) v(g)+g(p)v(f)=\sum_i 0+0=0
    \end{align*}
    using the fact that $f_i,g_i\in \frkm_p$ implies the vanish at $p$. Then we get the following unique map $\tilde d:\frkm_p/\frkm_p^2\to T^*_pM$:
    \begin{center}
        \begin{tikzcd}
            \frkm_p \ar{r}{d} \ar[two heads]{d}& T_p^*M\\
            \frkm_p/\frkm_p^2 \ar[dashed]{ur}[description]{\exists!}
        \end{tikzcd}
    \end{center}
    This map is a homomorphism because $d$ is linear. Now suppose $df=0$ for some $f\in C^\infty(M)$. Then by definition,
    \[
    v(f)=0
    \]
    for all $v\in T_pM$, which implies that indeed $f=0$ because if we take the derivation $v=\frac{\partial}{\partial x^i}$ for each $i$ and 
    \[
    \frac{\partial f}{\partial x^i}=0
    \]
    then $f$ is constant, but since $f+\frkm_p^2(p)=0$ it must be that $f=0$, proving $\tilde d$ has a trivial kernel. By \cite{Lee_Manifolds} Page 281, the $dx^i$ form a basis for $T_p^*M$. Thus if we fix any $\sum_i c_i dx^i\in T_p^*M$, we let $f=\sum_i c_i x^i$, which is certainly in $\frkm_p/\frkm_p^2$. Then
    \begin{align*}
        \tilde d(f+\frkm_p^2)=\sum_i\frac{\partial f}{\partial x^i} dx^i=\sum_i c_i dx^i
    \end{align*}
    proves $\tilde d$ is surjective as well, and hence an isomorphism.
\end{proof}
\subsection{}
\subsubsection{A}\label{2.2.A}
\begin{proof}
    We want for each open set $U\subset X$, a set $\fF(U)$. This is given when $\fF:\Op(X)\to \Set$ is a contravariant functor.\\
    We want that for each inclusion $U\hookrightarrow V$ of open sets, a restriction map $\res_{V,U}:\fF(V)\to \fF(U)$. This is equivalent to $\fF(U\hookrightarrow V)$ because $\fF$ is contravariant, and the only maps on $\Op(X)$ are the inclusions. We require that if
    \begin{center}
        \begin{tikzcd}
            W&&V \ar[hook']{ll}\\
            &U \ar[hook']{ur} \ar[hook]{ul}
        \end{tikzcd}
    \end{center}
    commutes in $\Op(X)$, that also the following diagram commutes:
    \begin{center}
        \begin{tikzcd}
            \fF(W) \ar{rr}{\res_{W,V}} \ar{dr}[swap]{\res_{W,U}}&& \fF(V) \ar{dl}{\res_{V,U}}\\
            & \fF(U)
        \end{tikzcd}
    \end{center}
    But this is exactly one of the requirement of $\fF$ being a contravariant functor.\\
    Finally, we require $\res_{U,U}=\altid$, which again, is one of the requirements of $\fF$ being a functor. Also notice that $U\hookrightarrow V\hookrightarrow W$ is the arrow $U\hookrightarrow W$ because every morphism in $\Op(X)$ is a monomorphism, and there is an initial object in the category $\emptyset$. There are no other requirements that $\fF$ is a contravariant functor, so the two definitions coincide.
\end{proof}
\subsubsection{B}\label{2.2.B}
\begin{proof}
Clearly the assignment of any open set in $\C$ to any set of functions of a given type defined on that subset together with the natural restriction of functions satisfies the definitions because $(f\vert_V)\vert_U=f\vert_U$ as well as $f\vert_U=f$ when $f:U\to \C$. Thus we will show that in both cases, the definitions violate the Gluability axiom as well as satisfy restriction being well defined.
    \begin{enumerate}[(a)]
        \item If $f:U\to \C$ is bounded, then by definition for every $x\in U$ $|f(x)|<N$ for some constant $N\in \R$. Then if $V\subset U$ and $y\in V$, then
        \begin{align*}
            |f\vert_V(y)|=|f(y)|<N
        \end{align*}
        so indeed $f\vert_V$ is bounded. Now we show that the bounded functions violate the Gluability axiom. For $n=1,2,\dots$ define $U_n=\D(0,n)$, the open disc of radius $n$ about the origin and define $f_n:U_n\to \C$ as $f_n=\altid$. Then for arbitrary $n$ and $z\in U_n$, we observe
        \begin{align*}
            |f_n(x)|=|x|<n
        \end{align*}
        by definition of $x\in U_n$. Then indeed every $f_n$ is bounded on $U_n$. Now we observe that $\bigcup_{n=1}^\infty U_n=\C$. If there were a global function $f:\C\to \C$ such that $f\vert_{U_n}=f_n$ for every $n$, then it must be that $f=\altid$. But $\altid$ is unbounded on $\C$, which means the Gluability axiom fails.
        \item Restricting holomorphic functions is holomorphic, so again restriction is well defined. Now we define the following open sets $U_1\coloneqq \{re^{i\theta}\in \Z:\theta \in (-\pi,\pi)\}$ and $U_2\coloneqq \{re^{i\theta}\in \C:\theta \in (0,2\pi)\}$, as well as the identity maps on each of them, which are clearly holomorphic. We let $h_1(re^{i\theta})=\sqrt{r}e^{i\theta/2}$ and $h_2(re^{i\theta})=\sqrt{r}e^{i\theta/2}$ as well. We observe primarily that $h_1,h_2$ are holomorphic on $U_1$ and $U_2$ respectively, and also that
        \begin{align*}
            h_1^2=\altid_{U_1}\\
            h_2^2=\altid_{U_2}
        \end{align*}
        Then $\altid_{U_1}$ is a holomorphic function with holomorphic square root on $U_1$ and $\altid_{U_2}$ is a holomorphic function with holomorphic square root on $U_2$. In addition, 
        \begin{align*}
            \altid_{U_1}\vert_{U_1\cap U_2}=\altid_{U_1\cap U_2}=\altid_{U_2}\vert_{U_1\cap U_2}
        \end{align*} However, since $U_1\cup U_2=\C$, the global function $f:\C\to \C$ must be $\altid_{\C}$. However, there is no global square root function of $\altid_{\C}$, so the Gluability axiom fails here as well.
    \end{enumerate}
\end{proof}
\subsubsection{C}\label{2.2.C}
\begin{proof}
    We claim that a presheaf $F$ is a sheaf if and only if $F(\bigcup_{i\in \fI}U_i)=\lim_{i,j\in \fI} F(U_i),F(U_i\cap U_j)$ for every collection of open sets $\{U_i\}_{i\in \fI}$. To be slightly more precise, the system of $F(U_i),F(U_i\cap U_j)$ is where all of the $F(U_i)$'s are not connected by any arrows, and each $F(U_i\cap U_j)$ has the restriction arrows going into it from $F(U_i)$ and $F(U_j)$. Note that this implicitly encodes the restriction arrows going from $F(U_i)$ to $F(U_j)$ because then $F(U_i\cap U_j)=F(U_j)$, and indeed $\res_{U_i,U_i}=\id_{U_i}$.\\
    For the forward direction, suppose $F$ is a sheaf. Letting $U\coloneqq \bigcup_{i\in \fI} U_i$, then for every $i,j\in \fI$, the following diagram commutes by $F$ being a presheaf:
   \begin{center}
       \begin{tikzcd}
           &F(U)\ar{dl} \ar{dr}\ar{dd}\\
        F(U_i) \ar{dr}&&F(U_j)\ar{dl}\\
        &F(U_i\cap U_j)
       \end{tikzcd}
   \end{center}
   where the arrows are the restrictions. We now wish to show $F(U)$ is universal with respect to this property. Notice that by definition of $F$ being a presheaf, the middle arrow is implicit and will be omitted. Now suppose a set $W$ also makes the diagram commute. Notice that the arrows from $W$ into each $F(U_i\cap U_j)$ is determined by the arrows from $W$ to $F(U_i)$ and the arrows from $W$ to $F(U_j)$, because the following diagram must commute:
   \begin{center}
       \begin{tikzcd}
           &W\ar{dl}[swap]{p_i} \ar{dr}{p_j}\ar{dd}{p_{ij}}\\
        F(U_i) \ar{dr}&&F(U_j)\ar{dl}\\
        &F(U_i\cap U_j)
       \end{tikzcd}
   \end{center}
   Therefore we may forget about all of the arrows from $W$ to $F(U_i\cap U_j)$ and only consider those going into each $F(U_i)$. If $W=\emptyset$ then trivially the unique arrow exists, so we may consider $W\ne \emptyset$, and pick any $x\in W$. Define $f_i\coloneqq p_i(x)$ for each $i\in \fI$. Therefore, by definition of $W$ making the system commute, we have that $f_i\vert_{U_i\cap U_j}=f_j\vert_{U_i\cap U_j}$ for each $i,j$. Then by gluablility, there exists some $f\in F(U)$ such that $f\vert_{U_i}=f_i$ for every $i\in \fI$. Then we can define the map $W\to F(U)$ that sends everything to $f$, which proves existence of the arrow going into $F(U)$.\\
   For uniqueness, suppose there exist two maps $\varphi$ and $\phi$ from $W\to F(U)$ that make the following diagram commute:
   \begin{center}
       \begin{tikzcd}
           &W \ar[dashed]{d} \ar[bend right]{ddl}[swap]{p_i} \ar[bend left]{ddr}{p_j}\\
           &F(U) \ar{dl} \ar{dr}\\
           F(U_i)\ar{dr}&&F(U_j)\ar{dl}\\
           &F(U_i\cap U_j)
       \end{tikzcd}
   \end{center}
   If $\varphi \ne \phi$, then there exists some $x\in W$ such that $\varphi(x)\ne \phi(x)$. However,
   \[
   \varphi(x)\vert_{U_i} =p_i(x)=\phi(x)\vert_{U_i}
   \]
   for each $i\in \fI$ by commutativity. By identity, this implies that $\varphi(x)=\phi(x)$, a contradiction. This proves uniqueness, so $F(U)$ is indeed the limit of the system.
   
   \vspace{\baselineskip}
    \noindent Conversely, suppose $F(U)$ is the limit of the system. To show $F$ satisfies gluability, suppose there is a collection of $f_i$'s for each $i$ such that $f_i\vert_{U_i \cap U_j}=f_j\vert_{U_i \cap U_j}$ for each $i,j$. Now, let $W$ be the final set and define maps $p_i:W\to F(U_i)$ that outputs $f_i$ for each $i\in \fI$. Then the following diagram commutes, and induces a unique morphism $\varphi:W\to F(U)$ below:
   \begin{center}
       \begin{tikzcd}
           &W \ar[dashed]{d}[description]{\exists!} \ar[bend right]{ddl}[swap]{p_i} \ar[bend left]{ddr}{p_j}\\
           &F(U) \ar{dl} \ar{dr}\\
           F(U_i)\ar{dr}&&F(U_j)\ar{dl}\\
           &F(U_i\cap U_j)
       \end{tikzcd}
   \end{center}
   Then we can take $\varphi(\ast)$ to be our map in $F(U)$ that restricts to give us each of the maps $f_i$, which shows gluability. To show identity, suppose we have $f_1,f_2\in F(U)$ such that $f_1\vert_{U_i}=f_2\vert_{U_i}$ for every $i\in \fI$. If we define the set $W\coloneqq \{f_1,f_2\}$, then $W\xhookrightarrow{\iota} F(U)$ naturally. Then the following diagram commutes, so we obtain a unique arrow $W\to F(U)$ shown below:
   \begin{center}
       \begin{tikzcd}
           &W \ar[dashed]{d}[description]{\exists!} \ar[bend right]{ddl}[swap]{\res_{U,U_i} \circ \iota} \ar[bend left]{ddr}{\res_{U,U_j}\circ \iota}\\
           &F(U) \ar{dl} \ar{dr}\\
           F(U_i)\ar{dr}&&F(U_j)\ar{dl}\\
           &F(U_i\cap U_j)
       \end{tikzcd}
   \end{center}
   However, we can define two such maps that work, namely $\varphi(W)=\{f_1\}$, as well as $\phi(W)=\{f_2\}$. This implies, by uniqueness, that $\varphi=\phi$, so indeed $f_1=f_2$, proving identity.
\end{proof}
\subsubsection{D}\label{2.2.D}
\begin{proof}
    \begin{enumerate}[(a)]
    \item We will show smooth functions form a sheaf on a smooth manifold $M$, as this is the only example in $\S 2.1$ that I can find. Clearly, this is a presheaf with the obvious restriction maps, so we will just show gluability and identity.

    \vspace{\baselineskip}
    To show gluability, suppose we have $f_i$'s in $C^\infty(U_i)$ with $f_i\vert_{U_i\cap U_j}=f_j\vert_{U_i\cap U_j}$ for every $i,j$ in the index. Define a function $f:U\to \R$ as
    \[
    f(x)=f_i(x) \text{ if } x\in U_i
    \]
    where $U=\bigcup_i U_i$. This is well defined by hypothesis. By the pasting lemma, this function is continuous as well. In addition, because differentiability is a local property, we have that
    \begin{align*}
        \frac{\partial^n f}{(\partial x^i)^n}\vert_{x\in U_i}=\frac{\partial^n f_i}{(\partial x^i)^n}
    \end{align*}
    exists for every $n\in \N$, proving that our function $f\in C^\infty(U)$, so gluability holds.

    \vspace{\baselineskip}
    Identity is rather trivial: if there exists $f_1,f_2\in C^\infty(M)$ such that $f_1\vert_{U_i}=f_2\vert_{U_i}$ for every $i$, then because every point in $U$ is in some $U_i$ and $f_1$ agrees with $f_2$ there, they must be the same at every point, hence $f_1=f_2$.
    \item Let $X\in \Top$, and $F$ be the functor sending an open set $U$ to the set $\Mor(U,\R)$, together with the obvious restriction maps. Again, $F$ is trivially a presheaf. To show gluability, let $f_i\in \Mor(U_i,\R)$ be a family such that $f_i\vert_{U_i\cap U_j}=f_j\vert_{U_i\cap U_j}$ for every $i,j$. Then define a map $f:U\to \R$ as
    \begin{align*}
        f(x)= f_i(x) \text{ if }x\in U_i
    \end{align*}
    which is again well defined by hypothesis. This is also continuous by the pasting lemma. Therefore $f\in \Mor(X,\R)$, so gluability holds.

    \vspace{\baselineskip}
    For exactly the same reasoning as above, identity follows trivially because any two maps that agree pointwise are equal.
\end{enumerate}
\end{proof}
\subsubsection{E}\label{2.2.E}
\begin{proof}
    Like usual, $\fF$ is clearly a presheaf on $X$. To show $\fF$ satisfies gluability, suppose $\{U_i\}$ is an open cover of $U$, and suppose we have a collection of $f_i\in \Mor(U_i,S)$'s for each $i$ such that $f_i\vert_{U_i\cap U_j}=f_j\vert_{U_i\cap U_j}$ for every $i,j$. Then we can define $f:U\to S$ as $f(x)=f_i(x)$ if $x\in U_i$. This is well defined and continuous by the pasting lemma, so gluability holds.\\
    If We have $f_1,f_2:U\to S$ such that $f_1\vert_{U_i}=f_2\vert_{U_i}$ for each $i$, then for every $x\in U$, $f_1(x)=f_1\vert_{U_i}(x)=f_2\vert_{U_i}(x)=f_2(x)$, so $f_1$ agrees with $f_2$ everywhere, hence the two are identical.
\end{proof}
\subsubsection{F}\label{2.2.F}
\begin{proof}
    Like usual, the presheaf axioms are readily verified by manipulation of definitions. To show gluability, if we have a collection of continuous maps $f_i:U_i\to Y$ such that $f_i\vert_{U_i\cap U_j}=f_j\vert_{U_i\cap U_j}$ for each $i,j$, then we can define $f(x)=f_i(x)$ if $x\in U_i$. This is well defined and agrees on the intersections by assumption, so it is continuous by the pasting lemma. Then we have our candidate $f:U\to Y$ that restricts to each $f_i$.\\
    To show identity, if we have $f_1,f_2:U\to Y$ as continuous maps and $f_1\vert_{U_i}=f_2\vert_{U_i}$ for every $i$, then $f_1(x)=f_1\vert_{U_i}(x)=f_2\vert_{U_i}(x)=f_2(x)$. Thus $f_1$ and $f_2$ agree everywhere, so they are identical.
\end{proof}
\subsubsection{G}\label{2.2.G}
\begin{proof}
    \begin{enumerate}[(a)]
        \item This is clearly a presheaf by simply rearranging the definitions, as
    \begin{align}
        \mu\circ (s\vert_V)=(\mu \circ s)\vert V=(\altid_U)\vert_V=\altid_V
    \end{align}
    shows that indeed restricting sections gives more sections. To show gluability, if we have a collection of sections $s_i:U_i\to Y$ such that $s_i\vert_{U_i\cap U_j}=s_j\vert_{U_i\cap U_j}$ for every $i,j$, then we can use the pasting lemma -- using our assumptions -- to obtain a continuous map $s:U\to Y$ where $s(x)=s_i(x)$ if $x\in U_i$. Then indeed, for every $x\in U$, $x\in U_i$ for some $i$, so
    \[
    \mu \circ s(x)=\mu \circ s_i(x)=\altid_{U_i}(x)=x
    \]
    proves $\mu \circ s=\altid_U$ as desired. This shows gluability.\\
    To show identity, if $s_1,s_2:U\to Y$ are sections of $\mu$ such that $s_1\vert U_i=s_2\vert U_i$ for every $i$, then for arbitrary $x\in U$, there exists some $U_i$ containing $x$, hence
    \[
    s_1(x)=s_1\vert_{U_i}(X)=s_2\vert_{U_i}(x)=s_2(x)
    \]
    Thus $s_1$ agrees with $s_2$ everywhere, so they are identical functions. This proves identity.
    \item This is a sheaf of sets by Exercise \ref{2.2.F}F. Thus we want to show each $\fF(U)$ has group structure. Because $Y$ is a topological group, for any $f,g\in \fF(U)$, we may define the product $fg$ to act as
    \[
    fg(x)=f(x)g(x)
    \]
    This is indeed a continuous map from $U$ to $Y$ because multiplication is required to be continuous by $Y$ being a topological group. It follows that the identity element is the map that takes everything to the identity element of $Y$. This operation is associative because $Y$ is a group, so
    \[
    (fg)h(x)=(fg)(x)h(x)=f(x)g(x)h(x)=f(x)gh(x)=f(gh)(x)
    \]
    Finally, every $f\in \fF(U)$ has an inverse $f^{-1}$, where $f^{-1}(x)\coloneqq (f(x))^{-1}$, i.e. a pointwise inverse. This is indeed a continuous map because inversion is required to be continuous since $Y$ is a topological group. Secondly, we easily verify that
    \[
    f f^{-1}(x)=f(x)f^{-1}(x)=f(x)(f(x))^{-1}=1
    \]
    and
    \[
    f^{-1} f(x)=f^{-1}(x)f(x)=(f(x))^{-1}f(x)=1
    \]
    indeed proves each multiplication gives the constant map to the identity, so the notation we gave $f^{-1}$ is appropriate. Because $\fF(U)$ satisfies all of the group axioms, it may be considered to be a topological group with this structure.
    \end{enumerate}
\end{proof}
\subsubsection{H}\label{2.2.H}
\begin{proof}
    To show $\pi_* \fF$ is a presheaf given $\fF$ is a presheaf, take any open set $V\in \Op(Y)$. Then $\pi^{-1}(V)\in \Op(X)$ because $\pi$ is continuous, hence $\pi_* \fF$ is well defined. We can verify that if $W\subset V\subset U$, then indeed $\pi^{-1}(W)\subset \pi^{-1}(V)\subset \pi^{-1}(U)$, and the following diagram commutes with the restrictions given by $\fF$ being a presheaf:
    \begin{center}
        \begin{tikzcd}
            \fF(\pi^{-1}(W)) \ar{rr} \ar{dr}&&\fF(\pi^{-1}(V)) \ar{dl}\\
            &\fF(\pi^{-1}(U))
        \end{tikzcd}
    \end{center}
    This is how we may define our restrictions for $\pi_* \fF$, $\res_{U,V}\coloneqq \res_{\pi^{-1}(U),\pi^{-1}(V)}$, which is well defined because $V\subset U$ implies that $\pi^{-1}(V)\subset \pi^{-1}(U)$. Finally, with this definition of restriction, we observe that
    \[\res_{V,V}=\res_{\pi^{-1}(V),\pi^{-1}(V)}=\id_{\pi^{-1}(V)}=\id_{\pi_\ast\fF(V)}
    \]
    This indeed proves $\pi_*\fF$ is a presheaf as desired.\\
    \indent Now, let's suppose further that $\fF$ is a sheaf. To show identity, suppose we have an open set $U\in \Op(Y)$ and an open cover $\{U_i\}$, as well as $f_1,f_2\in \pi_*\fF(U)$, or equivalently $f_1,f_2\in \fF(\pi^{-1}(U))$ such that $f_1\vert_{\pi^{-1}(U_i)}=f_2\vert_{\pi^{-1}(U_i)}$ for every $i$. By identity of $\fF$, it must be that $f_1=f_2$. This proves identity.\\
    \indent To show gluability, suppose we have a collection of open sets $U_i\in \Op(Y)$ covering an open set $U$, and maps $f_i\in \pi_* \fF(U_i)$ such that $f_i\vert_{\pi^{-1}(U_i)\cap \pi^{-1}(U_j)}=f_j\vert_{\pi^{-1}(U_i)\cap \pi^{-1}(U_j)}$ for every $i,j$. By gluability of $\fF$, we obtain a map $f\in \fF \pi^{-1}(\bigcup \pi^{-1}(U_i))$ that restricts to each $f_i$ on $\pi^{-1}(U_i)$. However, because unions commute with preimages, we obtain that
    \[
    \bigcup \pi^{-1}(U_i)=\pi^{-1}(U)
    \]
    Thus $f\in \fF(\pi^{-1}(U))=\pi_*\fF(U)$, and restricts accordingly, so gluability holds. Thus, $\pi_*\fF$ is a sheaf as well.
\end{proof}
\subsubsection{I}\label{2.2.I}
\begin{proof}
    If we take the definition of a stalk $\fF_p=\colim \fF(U)$ where each $U$ is a neighborhood of $p$, then we get the following commutative diagram
    \begin{center}
        \begin{tikzcd}
            &\fF_p\\
            &(\pi_*\fF)_q \ar[dashed]{u}[description]{\exists!}\\
            \pi_*\fF(U) \ar{rr}{\res} \ar[bend left]{uur} \ar{ur}&&\pi_*\fF(V) \ar[bend right]{uul}\ar{ul}
        \end{tikzcd}
    \end{center}
    because each $\pi_* \fF(U)=\fF(\pi^{-1}(U))$, and each $\pi^{-1}(U)$ is a neighborhood of $p$ since $\pi(p)=q$. Thus we have the maps from $\fF(\pi^{-1}(U))\to \fF_p$ by considering every $\pi_*\fF(U)$ to be a neighborhood of $p$.
\end{proof}
\subsubsection{J}\label{2.2.J}
\begin{proof}
    Because we require the following diagram to commute
    \begin{center}
        \begin{tikzcd}
            \fO_X(V)\times \fF(V) \ar{r}{\text{action}} \ar{d}{\res_{V,U}\times \res_{V,U}}& \fF(V) \ar{d}{\res_{V,U}}\\
            \fO_X(U)\times \fF(U)\ar{r}{\text{action}}&\fF(U)
        \end{tikzcd}
    \end{center}
    we have a well defined action of germs by picking the action of a representative. More explicitly, given a germ $[f,U]\in \fO_{X,p}$ and $[g,V] \in \fF_p$, then we may define $[f,U]\cdot [g,V]=[f\vert_{U\cap V}\cdot g\vert_{U\cap V},U\cap V]$. This is well defined exactly because we require our diagram to commute.
\end{proof}
\subsection{}
\subsubsection{A}\label{2.3.A}
\begin{proof}
    By definition of stalks as colimits of neighborhoods of $p$, if $\phi:\fF\to \fG$ is a morphism of presheaves on $X$, then we get our unique induced morphism $\phi_p$ in the commutative diagram below:
    \begin{center}
        \begin{tikzcd}
            &&\fG_p \\
            \\
            \fG(U) \ar{uurr} \arrow[bend left=10, crossing over, pos=0.4]{rrrr}{\res_{U,V}}&&\fF_p \ar[dashed]{uu}[description]{\exists!\phi_p}&& \fG(V) \ar{uull}\\
            &\fF(U) \ar{rr}{\res_{U,V}} \ar{ul}{\phi(U)} \ar{ur}&&\fF(V) \ar{ul} \ar{ur}[swap]{\phi(V)}
        \end{tikzcd}
    \end{center}
\end{proof}
\subsubsection{B}\label{2.3.B}
\begin{proof}
    By Exercise \ref{2.2.H}H, we've already shown $\pi_*$ takes presheaves on $X$ to presheaves on $Y$. If we're given a morphism $\phi:\fF\to \fG$ of presheaves on $X$, then we obtain a morphism $\pi_* \phi:\pi_* \fF\to \pi_* \fG$ as morphisms of presheaves on $Y$, shown below:
    \begin{center}
        \begin{tikzcd}
            \pi_*\fG(U) \ar{r}{\res_{U,V}} & \pi_*\fG(V)\\
            \pi_*\fF(U) \ar{r}{\res_{U,V}} \ar{u}{\phi(\pi^{-1}(U))}&\pi_*\fF(V) \ar{u}[swap]{\phi(\pi^{-1}(V))}
        \end{tikzcd}
    \end{center}
    In other words, we define $\pi_*\phi(U)$ to be $\phi(\pi^{-1}(U))$, which is a morphism of presheaves on $Y$ because $\phi$ was a morphism of presheaves on $X$. By stacking the commutative diagrams, we can show $\pi_*$ distributes over composition below:
    \begin{center}
        \begin{tikzcd}
            \pi_*\fH(U) \ar{r}{\res_{U,V}} & \pi_*\fH(V)\\
            \pi_*\fG(U) \ar{r}{\res_{U,V}} \ar{u}{\pi_*\varphi(U)} & \pi_*\fG(V)\ar{u}[swap]{\pi_*\varphi(V)}\\
            \pi_*\fF(U) \ar{r}{\res_{U,V}} \ar{u}{\pi_*\phi(U)}&\pi_*\fF(V) \ar{u}[swap]{\pi_*\phi(V)}
        \end{tikzcd}
    \end{center}
    The last thing to show is that $\pi_*$ preserves identity morphisms of presheaves, which it does because $\pi_*\id_{\fF}$ is the natural transformation that acts as $\pi_*\id_{\fF}(U)=\id_{\fF}(\pi^{-1}(U))=\fF(\pi^{-1}(U))=\pi_*\fF(U)$ that is shown below:
    \begin{center}
        \begin{tikzcd}
            \pi_*\fF(U) \ar{r}{\res_{U,V}}& \pi_*\fF(V)\\
            \pi_*\fF(U) \ar{u}{\pi_*\id_{\fF}(U)} \ar{r}{\res_{U,V}}&\pi_*\fF(V) \ar{u}[swap]{\pi_* \id_{\fF}(V)}
        \end{tikzcd}
    \end{center}
    This natural transformation is the identity morphism on $\pi_*\fF$, so as desired $\pi_*$ does preserve identities and is thus a functor.
\end{proof}
\subsubsection{C}\label{2.3.C}
\begin{proof}
    It's clear that $\Hom(\fF,\fG):\Op(X)\to \Set$ is well defined. We may define the restriction maps as follows: Given any $U,V,W\in \Op(X)$ such that $W\subset V\subset U$ and any $\phi\in \Hom(\fF,\fG)(U)$, we define $\phi\vert_{V}$ as the natural transformation that acts as $\phi\vert_{V}(W)\coloneqq \phi(W)$ -- in other words, just forgets its definitions on subsets not contained in $V$. This defines a natural transformation $\fF\vert_{V}\to \fG\vert_{V}$ because $\phi$ is a natural transformation $\fF\vert_{U}\to \fG\vert_{U}$, and $V\subset U$ implies every open subset $W$ of $V$ that $\phi\vert_{V}$ must act on is already taken care of by $\phi$. By this definition, it is clear that $\Hom(\fF,\fG)$ is a presheaf.\\
    \indent To show $\Hom(\fF,\fG)$ satisfies identity, fix any $U\in \Op(X)$ and let $\{U_i\}$ be an open cover of $U$. Furthermore suppose we have two natural transformations $\phi_1,\phi_2\in \Hom(\fF,\fG)(U)$ such that $\phi_1\vert_{U_i}=\phi_2\vert_{U_i}$ for each $i$. For an arbitrary open subset $V\subset U$, we will show $\phi_1(V)=\phi_2(V)$. Because $\{U_i\}$ covers $U$ and $V\subset U$, we obtain the following open cover $\{V_i\}$ of $V$:
    \[
    \{V_i\}\coloneqq \{U_i\cap V\}
    \]
    Notice that each $V_i\subset U_i$ by construction. Thus because $\phi_1\vert_{U_i}=\phi_2\vert_{U_i}$, it's also true that $\phi_1\vert_{V_i}=\phi_2\vert_{V_i}$ for each $i$. Recall that because $\phi_1$ and $\phi_2$ are natural transformations, the following diagram(s) commute for $j=1,2$ and all $i$:
    \begin{center}
        \begin{tikzcd}
            \fF(V) \ar{d}{\res_{V,V_i}} \ar{r}{\phi_j(V)}&\fG(V)\ar{d}{\res_{V,V_i}}\\
            \fF(V_i) \ar{r}{\phi_j(V_i)}& \fG(V_i)
        \end{tikzcd}
    \end{center}
    Fixing an arbitrary $x\in \fF(V)$, we obtain by commutativity that $\res_{V,V_i}\circ \phi_j(V)(x)=\phi_j(V_i)\circ \res_{V,V_i}(x)$. However, since $\phi_1(V_i)=\phi_1\vert_{V_i}(V_i)=\phi_2\vert_{V_i}(V_i)=\phi_2(V_i)$ for every $i$, we obtain that 
    \[
    \res_{V,V_i}( \phi_1(V)(x))=\res_{V,V_i}(\phi_2(V)(x))
    \]
    for each $i$. By identity of $\fG$, we obtain that indeed $\phi_1(V)(x)=\phi_2(V)(x)$. But because $V\subset U$ and $x\in \fF(V)$ were arbitrary, we get that indeed $\phi_1\vert_{U}=\phi_2\vert_{U}$ as desired.\\
    \indent To show gluability, suppose we have an open set $U\in \Op(X)$ and an open cover $\{U_i\}$ of $U$. Suppose further we have a collection $\{\phi_i\}$ where each $\phi_i\in \Hom(\fF,\fG)(U_i)$ are such that $\phi_i\vert_{U_i \cap U_j}=\phi_j\vert_{U_i\cap U_j}$ for each $i,j$, and pick an arbitrary open subset $V\subset U$. We will define a natural transformation $\phi\in \Hom(\fF,\fG)(U)$ pointwise. First, define $V_i\coloneqq U_i \cap V$ for each $i$, and notice that $\{V_i\}$ form an open cover of $V$. For each fixed section $x\in \fF V$, we obtain sections $g_i(x)\coloneqq \phi_i(V_i)\circ \res_{V,V_i}(x)\in \fG(V_i)$ for each $i$. We will now show each $\res_{V_i,V_i\cap V_j}(g_i(x))=\res_{V_j,V_i\cap V_j}(g_j(x))$. By definition of each $\phi_i$ being a natural transformation, the following diagram commutes for every $i,j$:
    \begin{center}
        \begin{tikzcd}
            \fF(V_i) \ar{d}[swap]{\res_{V_i,V_i\cap V_j}} \ar{r}{\phi_i(V_i)}& \fG(V_i) \ar{d}{\res_{V_i,V_i\cap V_j}}\\
            \fF(V_i\cap V_j) \ar{r}{\phi_i(V_i\cap V_j)}& \fG(V_i\cap V_j)
        \end{tikzcd}
    \end{center}
    which makes sense because $V_i\subset U_i$ for each $i$, so we may indeed apply $\phi_i$ to these subsets. By commutativity, we may observe that
    \begin{align*}
        &\res_{V_i,V_i\cap V_j}(g_i(x))\\
        &=\res_{V_i,V_i\cap V_j}\circ \phi_i(V_i)\circ \res_{V,V_i}(x)\\
        &=\phi_i(V_i\cap V_j)\circ \res_{V_i,V_i\cap V_j}\circ \res_{V,V_i}(x)\\
        &=\phi_i(V_i\cap V_j)\circ \res_{V,V_i\cap V_j}(x)\\
        &=\phi_j(V_i\cap V_j)\circ \res_{V,V_i\cap V_j}(x)\\
        &=\phi_j(V_i\cap V_j)\circ \res_{V_j, V_i\cap V_j}\circ \res_{V,V_j}(x)\\
        &=\res_{V_j,V_i\cap V_j}\circ \phi_j(V_j)\circ \res_{V,V_j}(x)\\
        &=\res_{V_j,V_i \cap V_j}(g_j(x))
    \end{align*}
    Thus by gluability of $\fG$, we obtain a section $g_V(x)\in \fG(V)$ such that $\res_{V,V_i}(g_V(x))=g_i(x)=\phi_i(V_i)\circ \res_{V,V_i}(x)$ for every $i$. Then we define the natural transformation $\phi\in \Hom(\fF,\fG)(U)$ that acts as $\phi(V)(x)\coloneqq g_V(x)$ for every $x\in \fF(V)$ and every $V\subset U$. We need to show that this $\phi$ is a natural transformation, and that its restriction to $U_i$ gives $\phi_i$. To show that $\phi$ is a natural transformation, we want to show the following diagram commutes for all $W\subset V\subset U$:
    \begin{center}
        \begin{tikzcd}
            \fF(V) \ar{r}{\phi(V)} \ar{d}{\res_{V,W}}& \fG(V) \ar{d}{\res_{V,W}}\\
            \fF(W) \ar{r}{\phi(W)}&\fG(W)
        \end{tikzcd}
    \end{center}
    We will use identity of $\fG$ to prove this. For arbitrary $x\in \fF(V)$, defining an open cover $\{W_i\}$ of $W$ where $W_i\coloneqq W\cap U_i$, we compute that
    \begin{align*}
        &\res_{W,W_i}\circ \phi(W)\circ \res_{V,W}(x)\\
        &=\res_{W,W_i}\circ g_W(\res_{V,W}(x))\\
        &=\phi_i(W_i)\circ \res_{W,W_i}\circ \res_{V,W}(x)\\
        &=\phi_i(W_i)\circ \res_{V,W_i}(x)
    \end{align*}
    On the other hand, we compute that
    \begin{align*}
        &\res_{W,W_i}\circ \res_{V,W}\circ \phi(V)(x)\\
        &=\res_{W,W_i}\circ \res_{V,W}(g_V(x))\\
        &=\res_{V_i,W_i}\circ\res_{V,V_i}(g_V(x))\\
        &=\res_{V_i,W_i}\circ \phi_i(V_i)\circ \res_{V,V_i}(x)\\
        &=\phi_i(W_i)\circ \res_{V_i,W_i}\circ \res_{V,V_i}(x)\\
        &=\phi_i(W_i)\circ \res_{V,W_i}(x)
    \end{align*}
    Therefore identity of $\fG$ gives us that, because $\phi(W)\circ \res_{V,W}(x)$ agrees with $\res_{V,W}\circ \phi(V)(x)$ on restrictions to every $W_i$, that
    \[
    \res_{V,W}\circ \phi(V)(x)=\phi(W)\circ \res_{V,W}(x)
    \]
    Because $W\subset V\subset U$ were arbitrary with $x\in \fF(V)$, we obtain that indeed $\phi\in \Hom(\fF,\fG)(U)$. The last thing to show is that $\phi\vert_{U_i}=\phi_i$ for each $i$. Fix any $W\subset U_i$ and any $x\in \fF(W)$. Then, like before, we have an open cover $\{W_i\}$ of $W$. We will use identity of $\fG$ one final time to show that $\phi\vert_{U_i}=\phi_i$. We compute that
    \begin{align*}
        &\res_{W,W_i}\circ \phi\vert_{U_i}(W)(x)\\
        &=\res_{W,W_i}\circ \phi(W)(x)\\
        &=\res_{W,W_i}\circ g_W(x)\\
        &= \phi_i(W_i)\circ \res_{W,W_i}(x)
    \end{align*}
    while on the other hand
    \begin{align*}
        \res_{W,W_i}\circ \phi_i(W)(x)=\phi_i(W_i)\circ \res_{W,W_i}(x)
    \end{align*}
    Thus because $\phi\vert_{U_i}(W)(x)$ agrees with $\phi_i(W)(x)$ on all restrictions each $W_i$, the two must be the same by identity of $\fG$. Because $W\subset V$ and $x\in \fF(W)$ were arbitrary, we obtain that indeed $\phi\vert_{U_i}=\phi_i$ as desired, which proves gluability of $\Hom(\fF,\fG)$. Thus $\Hom(\fF,\fG)\in \Set_X$ for all $\fF\in \Set_X^{pre}$ and $\fG\in \Set_X$.
\end{proof}
\subsubsection{D}\label{2.3.D}
\begin{proof}
    \begin{enumerate}[(a)]
        \item Notice that because $\{p\}$ is the terminal object in $\Top$, there exists a unique continuous map from every $U\subset X$ into $\{p\}$, which we will denote as $f_U$. In other words, $\underline{\{p\}}(U)=\{f_U\}$ for each $U$, and $f_U\vert_{V}=f_V$ for every $V\subset U\subset X$. We define $\varphi\in \Nat(\Hom(\underline{\{p\}},\fF)$ that acts on $\phi \in \Hom(\underline{\{p\}},\fF)(U)$ as 
        \[
        \varphi(U)(\phi)=\phi(U)(f_U)
        \]
        For ease of notation, we write $\phi(f_V)$ to denote $\phi(V)(f_V)$ for all $V\subset U\subset X$ and $\phi\in \Hom(\underline{\{p\}},\fF)(U)$. Notice that for each $\phi \in \Hom(\underline{\{p\}},\fF)(U)$, $\phi(f_U)$ determines $\phi$ entirely because each $\underline{\{p\}}(V)=\{f_V\}$, and $\phi$ being a natural transformation implies the following diagram commutes for all $V\subset U$:
        \begin{center}
            \begin{tikzcd}
                \underline{\{p\}}(U) \ar{r}{\phi(U)} \ar{d}{\res_{U,V}}&\fF(U) \ar{d}{\res_{U,V}}\\
                \underline{\{p\}}(V) \ar{r}{\phi(V)}& \fF(V)
            \end{tikzcd}
        \end{center}
        and $\res_{U,V}(f_U)=f_V$, so $\phi(f_V)=\res_{U,V}\circ \phi(f_U)$. We will use this fact to define natural transformations later and show that they are equal. To show $\varphi$ is indeed a natural transformation, we want to show the following diagram commutes for all $V\subset U\subset X$:
        \begin{center}
            \begin{tikzcd}
                \Hom(\underline{\{p\}},\fF)(U) \ar{r}{\varphi(U)} \ar{d}{\res_{U,V}}& \fF(U) \ar{d}{\res_{U,V}}\\
                \Hom(\underline{\{p\}},\fF)(V) \ar{r}{\varphi(V)}& \fF(V)
            \end{tikzcd}
        \end{center}
        This commutes because for any $\phi\in \Hom(\underline{\{p\}},\fF)(U)$, we compute that
        \begin{align*}
            &\varphi(V)\circ \res_{U,V}(\phi)\\
            &= \res_{U,V}(\phi)(f_V)\\
            &=\phi(f_V)\\
            &=\phi\circ \res_{U,V}(f_V)\\
            &=\res_{U,V}(\phi(f_U))\\
            &=\res_{U,V}\circ \varphi(U)(\phi)
        \end{align*}
        by our definition of restriction of natural transformations defined in Exercise \ref{2.3.D}D. Now all that's left to show is that $\varphi$ is an isomorphism, or equivalently, for every $U\subset X$, $\varphi(U)$ is a bijection. To show surjectivity, fix any $s\in \fF(U)$. Then there exists $\phi_s\in \Hom(\underline{\{p\}},\fF)(U)$ such that $\phi_s(f_V)=\res_{U,V}(s)$ for every $V\subset U$. To show $\phi_s$ is a natural transformation, we observe that
        \begin{center}
            \begin{tikzcd}
                \underline{\{p\}}(V) \ar{r}{\phi_s(V)} \ar{d}{\res_{V,W}}&\fF(V) \ar{d}{\res_{V,W}}\\
                \underline{\{p\}}(W) \ar{r}{\phi_s(W)}& \fF(W)
            \end{tikzcd}
        \end{center}
        commutes because
        \begin{align*}
            &\phi_s(W)\circ \res_{V,W}(f_V)\\
            &=\phi_s(W)(f_W)\\
            &=\res_{U,W}(s)\\
            &=\res_{V,W}\circ \res_{U,V}(s)\\
            &=\res_{V,W}\circ \phi_s(f_V)
        \end{align*}
    \end{enumerate}
    Then because $\varphi(U)(\phi_s)=\phi_s(f_U)=\res_{U,U}(s)=s$, we get that indeed $\varphi(U)$ is surjective.\\
    To show $\varphi(U)$ is injective, suppose $\phi_1,\phi_2\in \Hom(\underline{\{p\}},\fF)(U)$ are such that
    \[
    \varphi(U)(\phi_1)=\varphi(U)(\phi_2)
    \]
    By definition, then $\phi_1(f_U)=\phi_2(f_U)$. By our initial observations though, because $\phi(f_U)$ entirely determines $\phi\in \Hom(\underline{\{p\}},\fF)(U)$, then $\phi_1=\phi_2$ so $\varphi(U)$ is injective too. Thus $\varphi$ is an isomorphism.
    \item We may assume, without loss of generality, that $X$ is connected, for if $X=\coprod X_i$ and \\$\Hom(\underline{\Z}, \fF)(X_i)\cong \fF (X_i)$ for every $i$, then because $\fG(X)=\prod \fG(X_i)$ for every sheaf $\fG$ on $X$, we can lift these isomorphisms to obtain $\fF\cong \Hom(\underline{\Z},\fF)$ as desired.\\
    Recall that $\underline{\Z}(U)$ is the set of all continuous maps $U\to \Z$ where $\Z$ is endowed with the discrete topology for each $U\subset X$. Notice we have a particular map $c_U:U\to \Z$ that sends everything to the generator $1\in \Z$. We claim that $\langle c_U \rangle = \underline{\Z}(U)$. To show this, suppose we have some continuous map $f:U\to \Z$. Then because $\Z$ is endowed with the discrete topology, we obtain that $f^{-1}(n)$ is open for every $n\in \Z$. In addition, we directly observe that $f^{-1}(n)=f^{-1}(m)$ if and only if $n=m$. Thus $\{f^{-1}(n)\}$ form a disjoint open cover of $U$. But because $U$ is connected, it must be that exactly one $f^{-1}(n)$ is nonempty. Thus indeed $f(x)=n$ for all $x\in U$ and for some $n\in \Z$. In other words, $f=nc_U$, because we are working with sheaves of abelian groups, so we may multiply sections by values in $\Z$. This proves our claim that $\underline{\Z}(U)=\langle c_U\rangle \cong \Z$.\\
    Therefore for any $\phi \in \Hom(\underline{\Z}, \fF)(U)$, $\phi(U)(c_U)$, or using the same notation as in part (a), $\phi(c_U)$ determines $\phi$ entirely because $c_U$ generates $\underline{\Z}(U)$ and
    \begin{center}
        \begin{tikzcd}
            \underline{\Z}(U) \ar{r}{\phi(U)} \ar{d}{\res_{U,V}}&\fF(U)\ar{d}{\res_{U,V}}\\
            \underline{\Z}(V) \ar{r}{\phi(V)}&\fF(V)
        \end{tikzcd}
    \end{center}
    commutes, along with the fact that $\res_{U,V}(c_U)=c_V$ so that
    \[
    \phi(c_V)=\res_{U,V}(\phi(c_U))
    \]
    Now we may define a map $\varphi \in \Nat(\Hom(\underline{\Z},\fF),\fF)$ that acts as $\varphi(U)(\phi)=\phi(c_U)$ for every $U\subset X$ and $\phi\in \Hom(\underline{\Z},\fF)(U)$. We want to show that the following diagram commutes for all $V\subset U\subset W$ to prove $\varphi$ is a natural transformation:
    \begin{center}
        \begin{tikzcd}
            \Hom(\underline{\Z},\fF)(U) \ar{r}{\varphi(U)} \ar{d}{\res_{U,V}}& \fF(U) \ar{d}{\res_{U,V}}\\
            \Hom(\underline{\Z},\fF)(V) \ar{r}{\varphi(V)}& \fF(V)
        \end{tikzcd}
    \end{center}
    To show this, fix any $\phi\in \Hom(\underline{\Z},\fF)(U)$, then 
    \begin{align*}
        &\res_{U,V}\circ \varphi(U)(\phi)=\res_{U,V}(\phi(c_U))=\phi(\res_{U,V}(c_U))=\phi(c_V)\\
        &=\res_{U,V}(\phi)(c_V)=\varphi(V)\circ \res_{U,V}(\phi)
    \end{align*}
    of course relying on the naturality of $\phi$, and the fact that we define $\phi_{V}$ to act just as $\phi$ does. Now we wish to show that for every $U\subset X$, $\varphi(U)$ is a bijection. To show surjectivity, fix any section $s\in \fF(U)$. We may define $\phi_s\in \Hom(\underline{\Z},\fF)(U)$ that acts as $\phi_s(V)(c_V)=\res_{U,V}(s)$. Then by construction, $\varphi(U)(\phi_s)=\phi_s(c_U)=\res_{U,U}(s)=s$. To show that $\phi_s$ is actually natural, we want to show the following diagram commutes for all $W\subset V\subset U$:
    \begin{center}
        \begin{tikzcd}
            \underline{\Z}(V) \ar{r}{\phi_s(V)} \ar{d}{\res_{V,W}}& \fF(V) \ar{d}{\res_{V,W}}\\
        \underline{\Z}(W) \ar{r}{\phi_s(W)}& \fF(W)
        \end{tikzcd}
    \end{center}
    We may compute that
    \begin{align*}
        &\res_{V,W}\circ \phi_s(c_V)=\res_{V,W}\circ \res_{U,V}(s)=\res_{U,W}(s)\\
        &=\phi_s(W)(c_W)=\phi_s(W)\circ \res_{V,W}(c_V)
    \end{align*}
    which suffices because again $\underline{\Z}(V)=\langle c_V\rangle$ for every $V\subset X$. Thus $\varphi(U)$ is surjective.\\
    To show $\varphi(U)$ is injective, suppose $\varphi(U)(\phi_1)=\varphi(U)(\phi_2)$ for some $\phi_1,\phi_2\in \Hom(\underline{\Z},\fF)(U)$. Then $\phi_1(c_U)=\phi_2(c_U)$. By our previous observations regarding how $\phi(c_U)$ determines $\phi$ entirely, it follows that $\phi_1=\phi_2$ as desired. Thus $\varphi(U)$ is a bijection, hence $\varphi$ is an isomorphism.
    \item Define $\varphi\in \Nat(\Hom(\fO_X,\fF),\fF)$ that acts as $\varphi(U)(\phi)=\phi(1_U)$ for any $\phi \in \Hom(\fO_X,\fF)(U)$, where $1_U\in \fO_X(U)$ is the multiplicative identity. To show $\varphi$ is natural, fix any $V\subset U\subset X$. We claim the following diagram commutes:
    \begin{center}
        \begin{tikzcd}
            \Hom(\fO_X,\fF)(U) \ar{r}{\varphi(U)} \ar{d}{\res_{U,V}}& \fF(U) \ar{d}{\res_{U,V}}\\
            \Hom(\fO_X,\fF)(V) \ar{r}{\varphi(V)}&\fF(V)
        \end{tikzcd}
    \end{center}
    Letting $\phi \in \Hom(\fO_X,\fF)(U)$ be arbitrary, we compute that
    \begin{align*}
        &\res_{U,V}\circ \varphi(U)(\phi)=\res_{U,V}(\phi(1_U))=\phi(\res_{U,V}(1_U))\\
        &=\phi(1_V)=\res_{U,V}(\phi)(1_V)=\varphi(V)\circ \res_{U,V}(\phi)
    \end{align*}
    This comes from the fact that $\phi$ is assumed to be natural, together with the fact that since each $\res_{U,V}:\fO_X(U)\to \fO_X(V)$ is a ring homomorphism, it must preserve multiplicative identities.\\
    Then indeed $\varphi$ is natural. To show $\varphi$ is an isomorphism, it suffices to show each $\varphi(U)$ is a bijection. First, we claim that every natural transformation $\phi\in \Hom(\fO,\fF)(U)$ is uniquely determined by its action on $1_U$. To see this, if we take any $V\subset U$ and any $x\in \fO(U)$, we observe that by definition the following diagram commutes:
    \begin{center}
        \begin{tikzcd}
            \fO_X(U) \ar{r}{\phi(U)} \ar{d}{\res_{U,V}}& \fF(U) \ar{d}{\res_{U,V}}\\
            \fO_X(V) \ar{r}{\phi(V)}& \fF(V)
        \end{tikzcd}
    \end{center}
    For ease of notation, let $\phi(1_U)\coloneqq \phi(U)(1_U)$ for each open $U$. Because $\res_{U,V}(1_U)=1_V$, we obtain that
    \begin{align*}
        \phi(V)(x)=\phi(V)(x\cdot 1_V)=x\cdot \phi(V)(1_V)=x\cdot \res_{U,V}\circ \phi(1_U)
    \end{align*}
    because we have that $\phi$ is a $\Mod_{\fO_X}$ homomorphism. To show $\varphi(U)$ is surjective, fix any section $s\in \fF(U)$. Define $\phi_s\in \Hom(\fO_X,\fF)(U)$ that acts as
    \[
    \phi_s(V)(1_V)=\res_{U,V}(s)
    \]
    for any $V\subset U$. By our previous observation, this defines $\phi_s$ entirely. To show $\phi_s$ is natural, we want to show the following diagram commutes for all $W\subset V\subset U$:
    \begin{center}
        \begin{tikzcd}
            \fO_X(V) \ar{r}{\phi_s(V)} \ar{d}{\res_{V,W}}& \fF(V) \ar{d}{\res_{V,W}}\\
            \fO_X(W) \ar{r}{\phi_s(W)}& \fF(W)
        \end{tikzcd}
    \end{center}
    To see this, by our previous observations it suffices to show both paths action on $1_V$ agrees. We observe
    \begin{align*}
        \phi_s(W)\circ \res_{V,W}(1_V)=\phi_s(1_W)=\res_{U,W}(s)=\res_{V,W}\circ \res_{U,V}(s)=\res_{V,W}\circ \phi_s(1_V)
    \end{align*}
    as desired. We also have that, by construction,
    \begin{align*}
        \varphi(U)(\phi_s)=\phi_s(1_U)=\res_{U,U}(s)=s
    \end{align*}
    so $\varphi(U)$ is surjective.\\
    $\varphi(U)$ is injective because if $\varphi(U)(\phi_1)=\varphi(U)(\phi_2)$ for some $\phi_1,\phi_2\in \Hom(\fO_X,\fF)(U)$, then by definition $\phi_1(1_U)=\phi_2(1_U)$. But by our previous observations, this action determines $\phi_1$ and $\phi_2$ to be the same. Thus $\varphi(U)$ is injective, and hence $\varphi$ is an isomorphism as desired.
\end{proof}
\subsubsection{E}\label{2.3.E}
\begin{proof}
    We use the following diagram to define $\res_{U,V}$ for every $V\subset U\subset X$:
    \begin{center}
        \begin{tikzcd}
            \fF(U) \ar{r}{\res_{U,V}}& \fF(V) \ar{r}{\phi(V)}& \fG(V)\\
            &\ker \phi(V) \ar{ur}{0} \ar[hook]{u}{\iota_V}\\
            \ker \phi(U) \ar[hook]{uu}{\iota_U} \ar[dashed]{ur}[description]{\exists!}
        \end{tikzcd}
    \end{center}
    Indeed,
    \begin{align*}
        \phi(V)\circ \res_{U,V}\circ \iota_U=\res_{U,V}\circ \phi(U)\circ \iota_U=\res_{U,V}\circ 0=0
    \end{align*}
    so we obtain the induced morphism $\res_{U,V}:\ker \phi(U)\to \ker \phi(V)$ that makes the diagram commute. By this construction, it is clear that $\res_{U,U}=\id_{\ker \phi(U)}$ by uniqueness of $\res_{U,U}$ and the fact that $\res_{U,U}:\fF(U)\to \fF(U)=\id_{\fF(U)}$. The last thing to show is that for all $W\subset V\subset U\subset X$, we have that $\res_{U,W}=\res_{V,W}\circ \res_{U,V}$. Notice that by our constructions of the restrictions, the following diagram commutes:
    \begin{center}
        \begin{tikzcd}
            \fF(U) \ar{r}{\res_{U,V}}& \fF(V) \ar{r}{\res_{V,W}}&\fF(W)\\
            \ker \phi(U) \ar{r}{\res_{U,V}} \ar[hook]{u}{\iota_U}& \ker \phi(V) \ar[hook]{u}{\iota_V} \ar{r}{\res_{V,W}}& \ker \phi(W) \ar[hook]{u}{\iota_W}
        \end{tikzcd}
    \end{center}
    By this diagram, it is clear that
    \begin{align*}
        \iota_W\circ \res_{V,W}^{\ker}\circ \res_{U,V}^{\ker}=\res_{V,W}^\fF \circ \iota_V\circ \res_{U,V}^{\ker}=\res_{V,W}^\fF \circ \res_{U,V}^\fF \circ \iota_U=\res_{U,W}^\fF \circ \iota_U=\iota_W\circ \res_{U,W}^{\ker }
    \end{align*}
    where here we use superscripts to denote which presheaf the restriction is occuring in. Now, using the fact that $\iota_W$ is a monomorphism, we obtain that indeed
    \begin{align*}
        \res_{V,W}^{\ker}\circ \res_{U,V}^{\ker}=\res_{U,W}^{\ker}
    \end{align*}
    so $\kerpre \phi$ is a presheaf.
\end{proof}
\subsubsection{F}\label{2.3.F}
\begin{proof}
    Let $\pi:\fG\twoheadrightarrow \cokpre \phi$ be the projection defined on each open set $U\subset X$ as
    \[
    \pi_U=\cok \phi(U)
    \]
    Dually to how we defined the restriction maps in Exercise \ref{2.3.E}E, we obtain natural restriction maps for $\pi$. As shown in the commutative diagram below, we may observe that indeed $\pi \circ \phi$ is the zero morphism in $\Mod_{\fO_X}^{\text{pre}}$ because it is on each open $V\subset U\subset X$:
    \begin{center}
        \begin{tikzcd}
            \cokpre \phi(U) \ar{r}{\res}&\cokpre \phi(V)\\
            \fG(U) \ar[two heads]{u}[swap]{\pi_U} \ar{r}{\res}& \fG(V) \ar[two heads]{u}{\pi_V}\\
            \fF(U) \ar{u}[swap]{\phi(U)} \ar{r}{\res} \ar[bend left=50]{uu}{0}&\fF(V) \ar{u}{\phi(V)} \ar[bend right=50]{uu}[swap]{0}
        \end{tikzcd}
    \end{center}
    Now suppose we have the following commutative diagram in $\Mod_{\fO_X}^{\text{pre}}$:
    \begin{center}
        \begin{tikzcd}
            &\fH\\
            \fF \ar{ur}{0} \ar{r}{\phi}& \fG \ar{u}{\psi}
        \end{tikzcd}
    \end{center}
    Then, in particular, on each open $U\subset X$, we get the following commutative diagram:
    \begin{center}
        \begin{tikzcd}
            &&\fH(U)\\
            &\cokpre \phi (U) \ar[dashed]{ur}[description]{\exists! h_U}\\
            \fF(U) \ar{ur}{0} \ar{r}{\phi(U)}& \fG(U) \ar[two heads]{u}{\pi_U} \ar[bend right]{uur}{\psi(U)}
        \end{tikzcd}
    \end{center}
    Now we define the morphism $h:\cokpre \phi \to \fH$ given on each open set $U$ as $h_U$. We now need to show that $h$ is in fact a natural transformation by showing the following diagram commutes for all open $V\subset U\subset X$:
    \begin{center}
        \begin{tikzcd}
            \fH(U) \ar{r}{\res^H}& \fH(V)\\
            \cokpre \phi(U) \ar{r}{\res^{\cok}} \ar{u}{h_U}&\cokpre \phi(V) \ar{u}{h_V}
        \end{tikzcd}
    \end{center}
    The good news is that $\pi$ is an epimorphism, so we can compute the following equalities:
    \begin{align*}
        &\res^H \circ h_U \circ \pi_U\\
        &=\res^H\circ \psi(U)\\
        &=\psi(V)\circ \res^G\\
        &=h_V\circ \pi_V\circ \res^G\\
        &=h_V\circ \res^{\cok}\circ \pi_U
    \end{align*}
    Because $\pi_U$ is an epimorphism, we get that
    \[
    \res^H\circ h_U=h_V\circ \res^{\cok}
    \]
    as desired. Then indeed $h$ is a natural transformation, and by construction $h\circ \pi=\psi$.
\end{proof}
\subsubsection{G}\label{2.3.G}
\begin{proof}
    We obtain a functor taking $\fF \mapsto \fF(U)$ and taking $\phi:\fF\to \fG$ to $\phi(U)$. This preserves identity morphisms by definition, and if we have $\fF \xrightarrow{\phi} \fG \xrightarrow{\psi} \fH$, then we take $\psi\circ \phi$ to $(\psi\circ \phi)(U)=\psi(U)\circ \phi(U)$ by definition, proving this is a functor.

    Now, to show that this functor is exact, we will show that if $\fF \xrightarrow{\phi} \fG \xrightarrow{\psi} \fH$ is exact, then $\fF(U) \xrightarrow{\phi(U)} \fG(U) \xrightarrow{\psi(U)} \fH(U)$ is also exact. Supposing the first sequence is exact, then $\kerpre \psi=\impre \phi$ by definition. By definition of $\kerpre$ and $\cokpre$ (and hence $\impre$), we obtain that
    \[
    \ker \psi(U)=\kerpre \psi(U)=\impre \phi (U)=\im \psi(U)
    \]
    Thus $\fF(U) \xrightarrow{\phi(U)} \fG(U) \xrightarrow{\psi(U)} \fH(U)$ is exact as desired. To be completely thorough, we would need to show that our functor preserves the additive structures of the hom-sets, but this is simply because the additive structure of hom-sets in $\Ab_X^{\text{pre}}$ is defined by addition on each open set.
\end{proof}
\subsubsection{H}\label{2.3.H}
\begin{proof}
    The forward direction is clear; to convince yourself, look at Exercise \ref{2.3.G}G. For the reverse direction, because $\kerpre$ and $\cokpre$ (and hence $\impre$) are defined "pointwise", meaning on each open set, we immediately obtain that $0 \to \fF_1(U) \to \dots \to \fF_n(U) \to 0$ exact for every open $U$ implies $0\to \fF_1 \to \dots\to \fF_n \to 0$ is also exact.
\end{proof}
\subsubsection{I}\label{2.3.I}
\begin{proof}
    Because the category of sheaves is a full subcategory of the category of presheaves, the universal property is satisfied by a dual argument to Exercise \ref{2.3.F}F. Thus it suffices to show that $\kerpre \phi$  satisfies identity and gluability.

    Suppose $U\subset X$ is open, and $\{U_i\}$ is an open cover of $U$. Now suppose that we have a collection of $f_i:\kerpre \phi(U_i)$ such that
    \[
    f_i\vert_{U_i\cap U_j}=f_j\vert_{U_i\cap U_j}
    \]
    for each $i,j$. If $\iota:\kerpre \phi\to \fF$ is the inclusion, consider $\{\iota_{U_i}(f_i)\}$. Then for each $i,j$, $\iota_{U_i}(f_i)\vert_{U_i\cap U_j}=\iota_{U_j}(f_j)\vert_{U_i\cap U_j}$ because
    \[
    \res^{\fF} \circ \iota =\iota \circ \res^{\kerpre \phi}
    \]
    so 
    \[
    \iota_{U_i}(f_i)\vert_{U_i\cap U_j}=\iota_{U_i\cap U_j}(f_i\vert_{U_i\cap U_j})=\iota_{U_i\cap U_j}(f_j\vert_{U_i\cap U_j})=\iota_{U_j}(f_j)\vert_{U_i\cap U_j}
    \]
    Then by gluability of $\fF$, there exists some $f\in \fF(U)$ such that $f\vert_{U_i}=\iota_{U_i}(f_i)$ for each $i$. To show $f\in \im \iota(U) \cong \kerpre\phi(U)$, we will show $\phi(U)(f)=0$, where here $0$ is the identity element of $\fG(U)$. Notice that
    \[
    0\vert_{U_i}=0=\phi(U_i)(\iota_{U_i}(f_i))=\phi(U_i)(f\vert_{U_i})=\phi(U)(f)\vert_{U_i}
    \]
    for each $i$, where again $0$ here denotes the identity element(s), we obtain by identity of $\fG$ that indeed $0=\phi(U)(f)$. Thus $f\in \im \iota(U)$ as desired, so we take $\iota^{-1}_U(f)$ to be the desired map in $\kerpre\phi(U)$. We compute that
    \[
    \iota^{-1}_U(f)\vert_{U_i}=\iota^{-1}_{U_i}(f\vert_{U_i})=\iota^{-1}_{U_i}(\iota_{U_i}(f_i))=f_i
    \]
    so gluability holds for $\kerpre\phi$.

    To show identity, using the same open set and open cover as before, suppose we have $f_1,f_2\in \kerpre\phi(U)$ such that $f_1\vert_{U_i}=f_2\vert_{U_i}$ for each $i$. Then \[
    \iota_U(f_1)\vert_{U_i}=\iota_{U_i}(f_1\vert_{U_i})=\iota_{U_i}(f_2\vert_{U_i})=\iota_U(f_2)\vert_{U_i}
    \]
    for each $i$. By identity of $\fF$, we get that $\iota_U(f_1)=\iota_U(f_2)$. Then by injectivity of $\iota$, we get $f_1=f_2$ as desired.
\end{proof}
\subsubsection{J}\label{2.3.J}
\begin{proof}
    Recall that $\underline \Z$ takes an open set to the abelian group of continuous maps $U\to \Z$, where $\Z$ is given the discrete topology. There is a natural inclusion of $\underline \Z$ into $\fO_X$, because locally constant functions are holomorphic and $\Z\subset \C$. Thus
    \[
    0\rightarrow \underline \Z \xrightarrow{\iota}\fO_X
    \]
    is exact. For the exactness at $\fF$, any function $f\in \fF(U)$ by definition has some holomorphic $g\in \fO_X(U)$ such that $\exp(g)=f$. Thus the holomorphic function $\frac{g}{2\pi i}$ is sent to $f$, proving
    \[
    \fO_X \xrightarrow{\pi} \fF\rightarrow 0
    \]
    is also exact. To show $\im \iota \subset \ker \pi$, for any $f\in \underline \Z(U)$, we have $\exp(2\pi i f)=c_1$, where $c_1$ is the constant function to $1\in \C$, because all integer multiples of $2\pi i$ are sent to $1$ by $\exp$. This is the identity on $\fF(U)$, as the abelian group structure of $\fF$ is pointwise multiplication.

    To show $\ker \pi \subset \im \iota$, suppose $\exp(2\pi i f)=c_1$. We obtain immediately that for every $z\in U$, $f(z)\in \Z$ because these are the only values of $\C$ for which the exponential evaluates to $1$. In addition, we may pick any small $\epsilon$, and notice that $\bigcup_{n\in \Z} \D(n,\epsilon)$ is a disjoint open cover of $\Z$, hence $\bigcup_{n\in \Z} f^{-1}(\D(n,\epsilon))$ is a disjoint open cover of $U$. Therefore $f$ must be locally constant, so $f\in \underline \Z(U)$ as desired. Thus
    \[
    \underline \Z \rightarrow \fO_X \rightarrow \fF
    \]
    is exact.

    Now, we will show $\fF$ is not a sheaf. Consider the following open cover of $\C^*$: $U\coloneqq \{e^{it}:0<t<2\pi\}$ and $V\coloneqq \{e^{it}:\pi<t<3\pi\}$. Then $\id_U$ and $\id_V$ both have holomorphic logarithms, because $U,V$ by construction have made a branch cut along $\R_{\ge 0}$ and $\R_{\le 0}$ respectively. If $\fF$ satisfied gluability, then $\id_{\C^*}$ would have a logarithm, so there would be a global logarithm on $\C^*$; this is a contradiction because there is no such global logarithm. Thus $\fF$ is not a sheaf.
\end{proof}
\subsection{}
\subsubsection{A}\label{2.4.A}
\begin{proof}
    The natural map sends $f\in \fF(U)$ to $([f,U])_{p\in U}$, the element that projects to the germ $[f,U]$ for each $p\in U$. To show this map is injective, suppose $f,g\in \fF(U)$ have the same image under our map. Then for every $p\in U$, we have that $[f,U]=[g,U]$. By definition, this means that there exists some open neighborhood $V_p\subset U$ of $p$ such that $f\vert_{V_p}=g\vert_{V_p}$. Notice that $\{V_p\}_{p\in U}$ is an open cover of $U$, and $f\vert_{V_p}=g\vert_{V_p}$ for every $p$ implies, by identity of $\fF$, that $f=g$.
\end{proof}
\subsubsection{B}\label{2.4.B}
\begin{proof}
    Let $(s_p)_{p\in U}\in \prod_{p\in U} \fF_p$ be a compatible germ. Then there exists some open cover $\{U_i\}$ of $U$ and sections $f_i\in \fF(U_i)$ such that for every $p\in U$, if $p\in U_i$ then $[f_i,U_i]=s_p$. We claim that for any $i,j$, on $U_i\cap U_j$ it holds that $f_i\vert_{U_i\cap U_j}=f_j\vert_{U_i \cap U_j}$. To show this, for any $p\in U_i\cap U_j$, 
    \[
    [f_i,U_i]=s_p=[f_j,U_j]
    \]
    Then by definition, there exists some open neighborhood $V_p\subset U_i\cap U_j$ of $p$ such that $f_i\vert_{V_p}=f_j\vert_{V_p}$. Letting $p$ range freely over $U_i\cap U_j$, we get an open cover $\{V_p\}$ of $U_i\cap U_j$. Because $f_i$ and $f_j$ restrict to the same thing on each $V_p$, by identity of $\fF$ we get that $f_i\vert_{U_i\cap U_j}=f_j\vert_{U_i\cap U_j}$. With this result, by gluability of $\fF$, there exists some $f\in \fF(U)$ such that $f\vert_{U_i}=f_i$ for each $i$.

    Then $f\mapsto ([f,U])_{p\in P}$. By construction, for every $p\in U$,
    \[
    s_p=[f_i,U_i]=[f,U]
    \]
    because again $f\vert_{U_i}=f_i$. Thus indeed $f$ maps to $(s_p)$, so the set of compatible germs is contained in the image.
\end{proof}
\subsubsection{C}\label{2.4.C}
\begin{proof}
    We want to show that for arbitrary $f\in \fF(U)$, $\phi_1(U)(f)=\phi_2(U)(f)$ given that $\phi_1$ and $\phi_2$ induce the same maps of stalks. Recall that the induced map of stalks by $\phi:\fF\to \fG$ is given by
    \[
    [f,U]\mapsto [\phi(U)(f),U]
    \]
    Fix an arbitrary $p\in U$. Then because the induced maps of $\phi_1,\phi_2$ agree, we get that
    \[
    [\phi_1(U)(f),U]=[\phi_2(U)(f),U]
    \]
    By definition, there exists some open neighborhood $V_p \subset U$ of $p$ such that $\phi_1(U)(f)\vert_{V_p}=\phi_2(U)(f)\vert_{V_p}$. Because $p$ was arbitrary, we get an open cover $\{V_p\}_{p\in U}$ for $U$. But because $\phi_1(U)(f)\vert_{V_p}=\phi_2(U)(f)\vert_{V_p}$ for every $p$, we get by identity of $\fG$ that $\phi_1(U)(f)=\phi_2(U)(f)$ as desired. Thus $\phi_1(U)=\phi_2(U)$ because $f$ was arbitrary, hence $\phi_1=\phi_2$ as $U$ was also arbitrary.
\end{proof}
\subsubsection{D}\label{2.4.D}
\begin{proof}
    For the forward direction, suppose $\phi:\fF \to \fG$ is an isomorphism of sheaves in $\Set_X$. We want to show that the induced map $\phi_p:\fF_p\to \fG_p$ is an isomorphism. We observe
    \[
    \phi_p\circ \phi_p^{-1}([g,U])=\phi_p([\phi^{-1}(U)(g),U])=[\phi(U)\circ \phi^{-1}(U)(g),U]=[g,U]
    \]
    and
    \[
    \phi_p^{-1}\circ \phi_p([f,U])=\phi_p^{-1}([\phi(U)(f),U])=[(\phi^{-1}(U)\circ \phi(U))(f),U]=[f,U]
    \]
    so indeed the induced maps $\phi_p$ and $\phi_p^{-1}$ are inverses, so $\phi_p$ is an isomorphism.

    For the reverse direction, suppose $\phi:\fF\to \fG$ induces isomorphisms (natural bijections) of all stalks. To shown that $\phi$ is injective, suppose $\phi(U)(f_1)=\phi(U)(f_2)$ for any two $f_1,f_2\in \fF(U)$. Then for each $p\in U$,
    \[
    \phi_p([f_1,U])=[\phi(U)(f_1),U]=[\phi(U)(f_2),U]=\phi_p([f_2,U])
    \]
    By injectivity of $\phi_p$, we get $[f_1,U]=[f_2,U]$. Then there exists some neighborhood $V_p\subset U$ of $p$ such that $f_1\vert_{V_p}=f_2\vert_{V_p}$. But because $p\in U$ was arbitrary, we have an open cover $\{V_p\}$ of $U$ such that $f_1\vert_{V_p}=f_2\vert_{V_p}$ for all $p$, so by identity of $\fF$ we get that $f_1=f_2$ as desired; thus $\phi$ is injective.

    To show surjectivity, fix any $g\in \fG(U)$, and we want to show that there exists some $f\in \fF(U)$ such that $\phi(U)(f)=g$. For each $p\in U$, $[g,U]\in \fG_p$; by surjectivity of each $\phi_p$, let
    \[
    \phi_p([f_p,U_p])=[g,U]
    \]
    Then the $\{U_p\}$ forms an open cover of $U$. We now claim that the $f_p$ together with the $\{U_p\}$ is a compatible germ. To show this, we want to show that if $p\in U_q$, that $[f_q,U_q]=[f_p,U_p]$ as stalks at $p$. We notice that
    \begin{align*}
        &\phi_p[f_q,U_q]=[\phi(U_q)(f_q),U_q]=[\phi(U_p\cap U_q)(f_q\vert_{U_p\cap U_q}),U_p\cap U_q]\\
        &=[g\vert_{U_p\cap U_q},U_p\cap U_q]=[g\vert_{U_p},U_p]=[\phi(U_p)(f_p),U_p]=\phi_p[f_p,U_p]
    \end{align*}
    By injectivity of $\phi_p$, we get that $[f_q,U_q]=[f_p,U_p]$ as desired. By Exercise \ref{2.4.B}B, this choice of compatible germs is the image of some section $f$ of $\fF$ over $U$. We claim now that $\phi(U)(f)=g$, which will come from identity on $\fG$. We have that for every $p\in U$,
    \[
    f\vert_{U_p}=f_p
    \]
    and
    \[
    \phi(U)(f)\vert_{U_p}=\phi(U_p)(f\vert_{U_p})=\phi(U_p)(f_p)
    \]
    which agrees with $g$ on some neighborhood $V_p\subset U$. In other words, $\phi(U)(f)$ agrees with $g$ on the open cover $\{V_p\}$, so that, by identity of $\fG$, $\phi(U)(f)=g$. This concludes the proof as we've shown $\phi$ is injective and surjective, hence an isomorphism.
\end{proof}
\subsubsection{E}\label{2.4.E}
\begin{proof}
    \begin{enumerate}[(a)]
        \item As suggested, let $X=\{p,q\}$ be the two point space with the discrete topology. Below is the diagram describing the presheaf -- i.e. contravariant functor from $\Op(X)\to \Set$ -- 
        \begin{center}
            \begin{tikzcd}
                &\fF(X)=\{0,1\} \ar{dl}[swap]{\res_{X,\{p\}}} \ar{dr}{\res_{X,\{q\}}}\\
                \fF(\{p\})=\{0\} \ar{dr}[swap]{\res_{\{p\},\emptyset}}&&\fF(\{q\})=\{1\} \ar{dl}{\res_{\{q\},\emptyset}}\\
                &\fF(\emptyset)=\{*\}
            \end{tikzcd}
        \end{center}
        It's easy to check this is a presheaf by the functor definition. However, $0\vert_{\{p\}}=0=1\vert_{\{p\}}$ and $0\vert_{\{q\}}=1=1\vert_{\{q\}}$, so this is where identity fails. Thus $a$ and $b$ have identical germs at each point. Therefore, under the natural map $\fF(X)\to \prod_{x\in X} \fF_x$, we observe
        \[
        0\mapsto ([0\vert_{\{p\}},\{p\}],[0\vert_{\{q\}},\{q\}])=([0,\{p\}],[1,\{q\}])
        \]
        and
        \[
        1\mapsto ([1\vert_{\{p\}},\{p\}],[1\vert_{\{q\}},\{q\}])=([0,\{p\}],[1,\{q\}])
        \]
        so injectivity fails.
        \item 
        Let $\fF$ be defined as above, let $\phi_1:\fF\to \fF$ be the identity, and $\phi_2:\fF\to \fF$ be defined by $\phi_2(X)$ being the constant function to $0\in \fF(X)$. This defines $\phi_2$ entirely because the other values of $\phi_2$ are uniquely determined since the sheaf $\fF$ evaluates every other set to be the final object in $\Set$. We notice that, as before, there is only one element in $\fF_p$ and one element in $\fF_q$. Therefore $\phi_1$ and $\phi_2$ induce the same maps on each stalk as $\phi_1(0)=\phi_2(0)$, and
        \begin{align*}
            [\phi_1(1),X]_p=[1,X]_p=[0,\{p\}]_p=[0,X]_p=[\phi_2(1),X]_p
        \end{align*}
        where the subscript indicates the stalk we are looking at. Similarly
        \begin{align*}
            [\phi_1(1),X]_q=[1,X]_q=[1,\{q\}]_q=[0,X]_q=[\phi_2(1),X]_q
        \end{align*}
        proves that, because $\phi_1$ agrees with $\phi_2$ on every other open set, that the two endomorphisms of $\fF$ induce the same maps on each stalk, but are not equal.
        \item Let $\fF$ be as above, and let $\fG$ be defined by the commutative diagram below:
        \begin{center}
            \begin{tikzcd}
                &\fG(X)=\{2\} \ar{dl}[swap]{\res_{X,\{p\}}} \ar{dr}{\res_{X,\{q\}}}\\
                \fF(\{p\})=\{0\} \ar{dr}[swap]{\res_{\{p\},\emptyset}}&&\fF(\{q\})=\{1\} \ar{dl}{\res_{\{q\},\emptyset}}\\
                &\fF(\emptyset)=\{*\}
            \end{tikzcd}
        \end{center}
        Now let $\phi:\fF\to \fG$ be the unique morphism of presheaves into $\fG$, because $\fG$ is the final object in $\Set^{\text{pre}}_X$. Similarly to $\fF$, there is only one element in $\fG_p$ as there is in $\fG_q$. Thus, $\phi$ induces bijections (isomorphisms in $\Set$) on each stalk. However, $\phi:\fF\to \fG$ is not an isomorphism because $\fG$ is not the final object, while $\fG$ is, so indeed there cannot be an isomorphism between them.
    \end{enumerate}
\end{proof}
\subsubsection{F}\label{2.4.F}
\begin{proof}
    Suppose $\fF$ is a presheaf, and $\phi:\fF\to \fG$ and $\varphi:\fF\to \fH$ are two sheaves satisfying the universal property of the sheafification $\fF^{\text{sh}}$ of $\fF$. Then, by the universal property of $\fG$, the following diagram commutes:
    \begin{center}
        \begin{tikzcd}
            \fF \ar{r}{\phi} \ar{dr}[swap]{\varphi}&\fG \ar{d}[description]{\exists! \tilde \varphi}\\
            & \fH
        \end{tikzcd}
    \end{center}
    On the other hand, by the universal property of $\fH$, the following diagram commutes:
    \begin{center}
        \begin{tikzcd}
            \fF \ar{r}{\varphi} \ar{dr}[swap]{\phi}&\fH \ar{d}[description]{\exists! \tilde \phi}\\
            & \fG
        \end{tikzcd}
    \end{center}
    Now, consider the following commutative diagram induced by $\fG$:
    \begin{center}
        \begin{tikzcd}
            \fF \ar{r}{\phi} \ar{dr}[swap]{\phi}&\fG \ar{d}[description]{\exists!}\\
            & \fG
        \end{tikzcd}
    \end{center}
    The identity morphism satisfies this unique arrow, as does $\tilde \phi \circ \tilde \varphi$ because
    \[
    \tilde \phi \circ \tilde \varphi \circ \phi=\tilde \phi \circ \varphi=\phi.
    \]
    By uniqueness, the two are equal. Similarly, the unique arrow in the commutative diagram below
    \begin{center}
        \begin{tikzcd}
            \fF \ar{r}{\varphi} \ar{dr}[swap]{\varphi}&\fH \ar{d}[description]{\exists!}\\
            & \fH
        \end{tikzcd}
    \end{center}
    is satisfied by both the identity morphism and $\tilde \varphi \circ \tilde \phi$, so the two are equal. This proves $\tilde \varphi$ and $\tilde \phi$ are inverses, and thus $\fG\cong \fH$ as sheaves. 
    
    Also, if $\fF$ is already a sheaf, then we claim $\fF$ with the identity is the sheafification of $\fF$. Indeed, for every other sheaf $\fG$ and $f:\fF\to \fG$, then the following diagram commutes:
    \begin{center}
        \begin{tikzcd}
            \fF \ar{r}{\id_{\fF}} \ar{dr}[swap]{f}&\fF \ar{d}[description]{\exists!}\\
            & \fG
        \end{tikzcd}
    \end{center}
    because the unique arrow is $f$ itself, and $f$ is a morphism of sheaves because $\Set_X$ is a full subcategory of $\Set^{\text{pre}}_X$.
\end{proof}
\subsubsection{G}\label{2.4.G}
\begin{proof}
    Suppose we have $\phi:\fF\to\fG$ where $\fF$ and $\fG$ are presheaves. Then we have the following commutative diagram by the universal property of $\fF^{\text{sh}}$:
    \begin{center}
        \begin{tikzcd}
            \fF \ar{d}{\phi} \ar{r}{\text{sh}_\fF}&\fF^{\text{sh}} \ar{d}[description]{\exists! \phi^\text{sh}}\\
            \fG \ar{r}{\text{sh}_\fG}& \fG^{\text{sh}}
        \end{tikzcd}
    \end{center}
    To show sheafification is a functor, we need to show that it preserves identity morphisms and respects composition of morphisms. We observe that, by the commutative diagram above defining the induced morphism, that sheafification preserves identities. To show sheafification respects composition, suppose we have
    \[
    \fF \xrightarrow{f} \fG \xrightarrow{g} \fH
    \]
    The following commutative diagram defines $(g\circ f)^{\text{sh}}$:
    \begin{center}
        \begin{tikzcd}
            \fF \ar{d}{g\circ f} \ar{r}{\text{sh}_\fF}&\fF^{\text{sh}} \ar[dashed]{d}[description]{\exists!}\\
            \fH \ar{r}{\text{sh}_\fH}& \fH^{\text{sh}}
        \end{tikzcd}
    \end{center}
    By uniqueness of $(g\circ f)^\text{sh}$, it suffices to show that $g^\text{sh}\circ f^\text{sh}$ satisfies this commutative diagram. We compute that, by definition of $g^\text{sh}$ and $f^\text{sh}$,
    \begin{align*}
        g^\text{sh}\circ f^\text{sh}\circ \text{sh}_\fF=g^\text{sh}\circ \text{sh}_\fG \circ f=\text{sh}_\fH\circ g\circ f
    \end{align*}
    as desired, so sheafification indeed preserves composition of morphisms and identities, and is thus a functor.
\end{proof}
\subsubsection{H}\label{2.4.H}
\begin{proof}
    This construction is easily seen to be a presheaf, so we will just prove it satisfies identity and gluability.

    For identity, suppose we have two sections $(f_p\in \fF_p)_{p\in U}$ and $(g_p\in \fF_p)_{p\in U}$ and and open cover $\{U_i\}$ of $U$ such that $(f_p)_{p\in U}$ and $(g_p)_{p\in U}$ restrict to the same section on each $U_i$, i.e.
    \[
    (f_p)_{p\in U_i}=(g_p)_{p\in U_i}
    \]
    for each $i$. But then indeed, since the $U_i$'s form an open cover for $U$, for each $p\in U$, $p\in U_i$ for some $i$ implies $f_p=g_p$. Then indeed the two sections are equal because they project to the same sections at each point, so identity holds.

    For gluability, suppose we have a set of sections $\{(f^i_p)_{p\in U_i}\}_{i}$ for an open cover $\{U_i\}$ of $U$ such that on each $U_i\cap U_j$, 
    \[
    (f^i_p)_{p\in U_i\cap U_j}=(f^j_p)_{p\in U_i\cap U_j}
    \]
    where I am using the superscript as an index notation for the sections. Let $(f_p)_{p\in U}$ be a choice of sections such that for each $p\in U$, $f_p$ is $f^i_p$ for some neighborhood $U_i$ of $p$. Notice that this is not actually a ``choice" because of the sections agreeing on their intersections, so that
    \[
    f^i_p=f^j_p
    \]
    for every $p\in U_i\cap U_j$ where we would need to make a choice.
    We claim that $(f_p)_{p\in U}$ is indeed a section of $\fF^{\text{sh}}$ over $U$. By the compatibility condition that for all $p\in U_i$, there exists an open neighborhood $V_i\subset U_i$ of $p$, and $s\in \fF(V_i)$ such that $s_q=f^i_q$ for all $q\in V_i$ an all $i$, we obtain that $V_i\cap V_j$ is an open neighborhood of $p$ contained in $U_i\cap U_j$. Fixing $p\in U$ to be arbitrary, we know $p\in U_i$ for some $i$. Then there exists an open neighborhood $V_i\subset U_i$ and $s\in \fF(V_i)$ such that $f^i_q=s_q$ for every $q\in V_i$. By construction of $(f_p)_{p\in U}$, we decided that $f_q$ is $f^i_q$ for every $q\in U_i$, hence $f_q=s_q$ for every $q\in V_i$. Then indeed $(f_p)_{p\in U}$ consists of compatible germs of $U$, so $(f_p)_{p\in U}\in \fF^{\text{sh}}(U)$. Finally, restricting this section to each $U_i$ gives
    \[
    (f_p)_{p\in U_i}=(f^i_p)_{p\in U_i}
    \]
    by construction, so gluability holds.
\end{proof}
\subsubsection{I}\label{2.4.I}
\begin{proof}
    The natural map $\sh:\fF\to \fF^{\sh}$ is defined by 
    \[
    \sh(U)(f)=(f_p)_{p\in U}
    \]
    Clearly the output consists of compatible germs. Furthermore, if $V\subset U\subset X$, we observe that for any $f\in \fF(U)$,
    \[
    \sh(U)(f)\vert_{V}=(f_p)_{p\in V}=(f\vert_{V\ p})_{p\in V}=\sh(V)(f\vert_{V})
    \]
    where the middle equality comes from the fact that germs are local, so the germs of $f\vert_V$ are equal to the germs of $f$ at points in $V$. This shows that $\sh$ is a natural transformation, and is thus a map of presheaves.
\end{proof}
\subsubsection{J}\label{2.4.J}
\begin{proof}
    Suppose we have a sheaf $\fG$ and a map of presheaves $\phi:\fF\to \fG$. Then for any section $(f_p\in \fF_p)_{p\in U}$ consisting of compatible germs, for each $p\in U$, let $V_p\subset U$ denote the open neighborhood of $p$ and $s^p\in \fF(V_p)$ be the section such that $s^p_q=f_q$ for every $q\in V_p$. We define $\phi^{\sh}(U)$ to take $(f_p)_{p\in U}$ to the unique section of $\fG$ over $U$ obtained by gluability applied to the collection of $\{\phi(V_p)(s^p)\in \fG(V_p)\}$. Gluability is applicable here because on any $V_p\cap V_q$, we observe
    \begin{align*}
        \phi(V_p)(s^p)\vert_{V_p\cap V_q}=\phi(V_p\cap V_q)(s^p\vert_{V_p\cap V_q})=\phi(V_p\cap V_q)(s^q\vert_{V_p\cap V_q})=\phi(V_q)(s^q)\vert_{V_p\cap V_q}
    \end{align*}
    The reason $\phi(V_p\cap V_q)(s^p\vert_{V_p\cap V_q})=\phi(V_p\cap V_q)(s^q\vert_{V_p\cap V_q})$ is because by Exercise \ref{2.4.A}A, sections of a sheaf is determined by its germs, and we know that for all  $r\in V_p\cap V_q$,
    \[
    s^p_r=f_r=s^q_r
    \]
    In other words, for every $r\in V_p\cap V_q$, there exists some neighborhood $W_r\subset V_p\cap V_q$ of $r$ such that $s^p\vert_{W_r}=s^q\vert_{W_r}$. Therefore
    \begin{align*}
        \phi(V_p\cap V_q)(s^p\vert_{V_p\cap V_q})\vert_{W_r}=\phi(V_p\cap V_q)(s^p\vert_{W_r})=\phi(V_p\cap V_q)(s^q\vert_{W_r})=\phi(V_p\cap V_q)(s^q\vert_{V_p\cap V_q})\vert_{W_r}
    \end{align*}
    so indeed the two sections have equal germs everywhere, which shows they are equal by Exercise \ref{2.4.A}A.

    Notice that our function $\phi^{\sh}(U)$ is well defined by identity of $\fG$, because our choice of gluability is the unique section with this property. Now we have to show that $\phi^{\sh}$ is natural. If we take $V\subset U$, we want to show that
    \begin{align}
        \phi^{\sh}(U)(f_p)_{p\in U}\vert_{V}=\phi^{\sh}(V)(f_p)_{p\in V}
    \end{align}
    To do this, it suffices by identity of $\fG$ to show that $\phi^{\sh}(U)(f_p)_{p\in U}\vert_V$ agrees with $\phi^{\sh}(V)(f_p)_{p\in V}$ on some open cover of $V$. By definition, we have that
    \[
    \phi^{\sh}(U)(f_p)_{p\in U} \vert_{V_p}=\phi(V_p)(s^p)
    \]
    Now let $W_p\coloneqq V\cap V_p$ for each $p\in V$, so that $\{W_p\}$ forms an open cover of $V$. Then
    \[
    \phi^{\sh}(U)(f_p)_{p\in U}\vert_{W_p}=\phi(V_p)(s^p)\vert_{W_p}=\phi(W_p)(s^p\vert_{W_p})=\phi^{\sh}(V)(f_p)_{p\in V} \vert_{W_p}
    \]
    To show the final equality, we use the fact that the choice of sections whose stalks yield any choice of compatible germs is independent. This follows from Exercise \ref{2.4.A}A, because if we pick some other choice of representing sections , then we use the fact that sections of $\fG$ are determined by their germs. This will be made precise as follows: for each $p\in V$, take $U_p\subset V$ to be a neighborhood of $p$ and $t^p$ to be a section such that for all $q\in U_p$, $t^p_q=f_q$. Then $W_p\cap U_p$ is a neighborhood of $p$ contained in $V$ such that each germ of $t^p$ and $s^p$ are equal. This would enforce that
    \[
    \phi(W_p\cap U_p)(t^p)=\phi(W_p\cap U_p)(s^p)
    \]
    because their germs are the same everywhere by Exercise \ref{2.4.A}A. Then we would replace our open cover $\{W_p\}$ by $\{W_p\cap U_p\}$, and everything would be the same.

    Now we have shown that $\phi^{\sh}(U)(f_p)_{p\in U}$ restricts the same as $\phi^{\sh}(V)(f_p)_{p\in V}$ on the open cover $\{W_p\}$, which by identity shows they are equal. Thus $\phi^{\sh}$ is a map of sheaves.

    To show $\phi^{\sh}$ satisfies the desired universal property, we take any $U\subset X$ and any $f\in \fF(U)$. Then we observe that the following diagram commutes
    \begin{center}
        \begin{tikzcd}
            \fF(U) \ar{r}{\sh(U)}  \ar{dr}[swap]{\phi(U)}& \fF^{\sh}(U) \ar{d}{\phi^{\sh}(U)}\\
            & \fG(U)
        \end{tikzcd}
    \end{center}
        because
        \[
        \phi^{\sh}(U)\circ \sh(U)(f)= \phi^{\sh}(U)(f_p)_{p\in U}=\phi(U)(f).
        \]
        The last equality holds because $(f_p)_{p\in U}$ has the representative section $f$ on the open cover $\{U\}$ of $U$. Then gluability of $\phi(U)(f)$ on the open cover $\{U\}$ of $U$ trivially gives $\phi(U)(f)$ back. Then because the diagram commutes for every open subset and is natural, existence is proven.
        
        The last thing to show is that $\phi^{\sh}$ is unique. Suppose we had another map of sheaves $\varphi:\fF^{\sh}\to \fG$ satisfying the universal property. To show $\varphi=\phi^{\sh}$, it suffices to show for arbitrary $U\subset X$ and $(f_p)_{p\in U}\in \fF^{\sh}(U)$, that $\varphi(U)(f_p)_{p\in U}=\phi^{\sh}(U)(f_p)_{p\in U}$. Let $\{V_p\}_{p\in U}$ be an open cover of $U$ and $s^p\in \fF(V_p)$ be a section such that for each $q\in V_p$, $f_q=s^p_q$. By the universal property that $\varphi$ satisfies, we obtain that
        \begin{align*}
            &\varphi(U)(f_q)_{q\in U}\vert_{V_p}=\varphi(V_p)(f_q)_{q\in V_p}=\varphi(V_p)\circ \sh(V_p)(s^p)\\
            &=\phi(V_p)(s^p)=\phi^{\sh}(V_p)\circ \sh(V_p)(s^p)=\phi^{\sh}(V_p)(f_q)_{q\in V_p}=\phi^{\sh}(U)(f_q)_{q\in U} \vert_{V_p}
        \end{align*}
        Because the $\{V_p\}$ form an open cover of $U$, and we just showed that $\varphi(U)(f_q)_{q\in U}$ restricts the same as $\phi^{\sh}(U)(f_q)_{q\in U}$ on each $V_p$, thus proving by identity of $\fG$ that $\phi^{\sh}(U)(f_q)_{q\in U}=\varphi(U)(f_q)$. This proves that, because $U$ was arbitrary and $(f_q)_{q\in U}$ was as well, that $\varphi=\phi^{\sh}$, so uniqueness holds as well.
\end{proof}
\subsubsection{K}\label{2.4.K}
\begin{proof}
    We want to show that for any presheaves $\fF$ and $\fG$, any sheaves $\fH$ and $\fT$, $\phi:\fH\to \fT$, and $\varphi:\fG \to \fF$ the following diagrams commute:
    \begin{center}
        \begin{tikzcd}
            \Hom(\fF^{\sh},\fH) \ar{d}{\tau_{\fF,\fH}} \ar{r}{\phi_*}&\Hom(\fF^{\sh},\fT) \ar{d}{\tau_{\fF,\fT}}\\
            \Hom(\fF, \fH) \ar{r}{\phi_*}&\Hom(\fF, \fT)
        \end{tikzcd}
        \begin{tikzcd}
            \Hom(\fF^{\sh},\fH) \ar{d}{\tau_{\fF,\fH}} \ar{r}{(\varphi^{\sh})^*}&\Hom(\fG^{\sh},\fH) \ar{d}{\tau_{\fG,\fH}}\\
            \Hom(\fF, \fH) \ar{r}{\varphi^*}&\Hom(\fG, \fH)
        \end{tikzcd}
    \end{center}
    where in addition the $\tau$'s are bijections. By the universal property of sheafification, we define $\hat f\coloneqq \tau_{\fF,\fH}^{-1}(f)$, where $f:\fF\to \fH$; more explicitly, $\hat f$ is the unique morphism induced by the commutative diagram below:
    \begin{center}
        \begin{tikzcd}
            \fF \ar{r}{\sh} \ar{dr}[swap]{f}& \fF^{\sh} \ar[dashed]{d}[description]{\exists!}\\
            & \fH
        \end{tikzcd}
    \end{center}
    On the other hand, given a morphism $\tilde f:\fF^{\sh}\to \fH$, then we define a morphism $\tau_{\fF,\fH}(\tilde f):\fF\to \fH$ given by 
    \[
    \tau_{\fF,\fH}(\tilde f)=\tilde f\circ \sh
    \]
    By uniqueness of the arrow induced by the universal property of $\fF^{\sh}$, we obtain that 
    \[
    \tau_{\fF,\fH}^{-1}\circ \tau_{\fF,\fH}(\tilde f)=\tau_{\fF,\fH}^{-1}(\tilde f\circ \sh)=\tilde f
    \]
    and that also
    \[
    \tau_{\fF,\fH}\circ \tau_{\fF,\fH}^{-1}(f)=\tau_{\fF,\fH}(\hat f)=\hat f\circ \sh=f.
    \]
    Now that the $\tau$'s are bijections, we need to check that the first diagram commutes. We check that
    \begin{align*}
        \tau_{\fF,\fT}\circ \phi_*(f)=\tau_{\fF,\fT}(\phi\circ f)=\phi\circ f\circ \sh
    \end{align*}
    and that
    \begin{align*}
        \phi_*\circ \tau_{\fF,\fH}(f)=\phi_*(f\circ \sh)=\phi\circ f\circ \sh
    \end{align*}
    so the first diagram does commute. To show the second diagram commutes, we see
    \begin{align*}
        \tau_{\fG,\fH}\circ (\varphi^{\sh})^*(f)=\tau_{\fG,\fH}(f\circ \varphi^{\sh})=f\circ \varphi^{\sh}\circ \sh_\fG=f\circ \sh_\fF\circ \varphi
    \end{align*}
    while on the other hand
    \begin{align*}
        \varphi^*\circ \tau_{\fF,\fH}(f)=\varphi^*(f\circ \sh_\fF)=f\circ \sh_\fF\circ \varphi
    \end{align*}
    so both diagrams commute, thus proving that sheafification is left-adjoint to the forgetful functor from sheaves to presheaves.
\end{proof}
\subsubsection{L}\label{2.4.L}
\begin{proof}
    Fix $p\in X$ as an arbitrary point, and consider the induced map $\sh_p:\fF_p\to \fF^{\sh}_p$. To show $\sh_p$ is injective, suppose $\sh_p(x_p)=\sh_p(y_p)$ for $x,y\in \fF_p$. Using the constructive definitions, we have that on some open neighborhood $U$ of $p$, the germs of $x$ agree with the germs of $y$ at every point in $U$. In particular, $x_p=y_p$ so $\sh_p$ is injective. To show $\sh_p$ is surjective, if we fix any $[(f_q)_{q\in U},U]\in \fF_p^{\sh}$, we know that by construction there exists some open neighborhood $V\subset U$ of $p$ such that for every $q\in V$, $f_q=s_q$ for some $s\in \fF(V)$. We claim that $\sh(s)=(f_q)_{q\in V}$. Indeed, $(f_q)_{q\in V}=(s_q)_{q\in V}=\sh(s)$. Therefore
    \[
    \sh_p([s,V])=[\sh(s),V]=[(f_q),V]=[(f_q),U]
    \]
    proves the induced map is surjective as well.
\end{proof}
\subsubsection{M}\label{2.4.M}
\begin{proof}
    \item [$(b\Rightarrow a)$] Suppose we have morphisms of sheaves $\varphi,\psi:\fG\to \fH$ such that $\phi\circ \varphi=\phi\circ \psi$. Our approach will be to show that $\varphi$ and $\psi$ induce the same maps on stalks. Notice that induced maps of stalks distributes over composition, so we get that on each stalk $\fF_p$,
    \[
    \phi_p\circ \varphi_p=\phi_p\circ \psi_p.
    \]
    By injectivity of $\phi_p$, we get that for every $p\in X$
    \[
    \varphi_p=\psi_p.
    \]
    By Exercise \ref{2.4.C}C, morphisms are determined by stalks implies that $\varphi=\psi$, so indeed $\phi$ was a monomorphism.
    \item [$(a\Rightarrow c)$] 
    Let $x,y\in \fF(U)$ be such that $\phi(U)(x)=\phi(U)(y)$, and let $\varphi, \psi:\fI_U\to \fF$ be morphisms from the indicator sheaf $\fI_U$ at $U$ such that $\varphi(U)(*)=x$ while $\psi(U)(*)=y$, thus determining $\varphi,\psi$ entirely by the nature of them being natural transformations from $\fI_U$. By definition, we obtain that
    \[
    \phi\circ \varphi=\phi\circ \psi
    \]
    because $\fI_U(V)=\{*\}$ if $V\subset U$ and $\emptyset$ otherwise, so for $V\subset U$ we have
    \[
    \phi(V)\circ \varphi(V)(*)=\phi(U)\circ \varphi(U)(*)\vert_{V}=\phi(x)\vert_V=\phi(y)\vert_V=\phi(U)\circ \psi(U)(*)\vert_V=\phi(V)\circ \psi(V)(*).
    \]
    By $\phi$ being a monomorphism, we obtain that $\phi=\psi$, which implies \[
    x=\phi(U)(*)=\psi(U)(*)=y
    \]
    so $\phi(U)(x)=\phi(U)(y)$ implies $x=y$, or $\phi(U)$ is injective.
    \item [$(c\Rightarrow b)$]
    Suppose $\phi_p(x_p)=\phi_p(y_p)$ for some $x_p,y_p\in \fF_p$. Then there exists some neighborhood $V$ of $p$ such that $\phi(U_x)(x)\vert_V$ agrees with $\phi(U_y)(y)\vert_V$, where we take $x_p=[x,U_x]$ and $y_p=[y,U_y]$. By our assumptions and the naturality of $\phi$, we get that
    \begin{align*}
        \phi(V)(x\vert_V)=\phi(U_x)(x)\vert_V=\phi(U_y)(y)\vert_V=\phi(V)(y\vert_V)
    \end{align*}
    Because $\phi$ is assumed to be injective on the level of open sets, we get that $x\vert_V=y\vert_V$. Therefore
    \[
    x_p=[x,U_x]=[x\vert_V,V]=[y\vert_V,V]=[y,U_y]=y_p
    \]
    so that indeed $\phi_p$ is injective.
\end{proof}
\subsubsection{N}\label{2.4.N}
\begin{proof}
    \item [$(b\Rightarrow a)$]
    Suppose $\psi \circ \phi=\varphi \circ \phi$ for some maps of sheaves $\psi,\varphi:\fG\to \fH$. Then on each stalk at $p\in X$,
    \[
    \psi_p\circ \phi_p=(\psi \circ \phi)_p=(\varphi\circ \phi)_p=\varphi_p\circ \phi_p
    \]
    Because $\phi_p$ is surjective, also known as an epimorphism in $\Set$, we get that $\psi_p=\phi_p$. By Exercise \ref{2.4.C}C, since $p\in X$ was arbitrary we get that $\psi=\varphi$, so $\phi$ is an epimorphism.
    \item [$(a\Rightarrow b)$]
    We will show that each $\phi_p$ is an epimorphism in $\Set$, also known as a surjective map. Suppose there is some set $S$ and maps of sets $\varphi, \psi:\fG_p\to S$ such that $\varphi \circ \phi_p=\psi\circ \phi_p$. These maps of sets induce maps of sheaves into the skyscraper sheaf $\iota_{p,*}S$ uniquely defined by
    \[
    \Phi(U)(x)=\varphi([x,U])
    \]
    and
    \[
    \Psi(U)(x)=\psi([x,U])
    \]
    for any neighborhood $U$ of $p$, and are otherwise determined since $\iota_{p,*}S(U)=\{*\}$ otherwise. Indeed, these are natural because for any $V\subset U$ both neighborhoods of $p$,
    \begin{align*}
        \Phi(U)(x)\vert_V=\varphi([x,U])\vert_V=\varphi([x,U])=\varphi([x\vert_V,V])=\Phi(V)(x\vert_V)
    \end{align*}
    and similarly for $\Psi$ because the restriction maps on the skyscraper sheaf are just the identity on neighborhoods of $p$. If $V$ is not a neighborhood of $p$, then the restriction is the unique map onto the empty section $*\in \iota_{p,*}(V)$. Notice then that
    \[
    \Phi(U)\circ \phi(U)(x)=\varphi([\phi(U)(x),U])=\varphi \circ \phi_p([x,U])=\psi\circ \phi_p([x,U])=\psi([\phi(U)(x),U])=\Psi(U)\circ \phi(U)(x)
    \]
    by construction of $\Phi$ and $\Psi$. This shows that indeed $\Phi\circ \phi =\Psi \circ \phi$, which, by $\phi$ being an epimorphism, proves that $\Phi=\Psi$. In particular,
    \[
    \varphi([x,U])=\Phi(U)(x)=\Psi(U)(x)=\psi([x,U])
    \]
    so indeed $\varphi=\psi$, proving the result.
\end{proof}
\subsubsection{O}\label{2.4.O}
\begin{proof}
    We will check that $\fO_X^*$ is a quotient sheaf of $\fO_X$ by looking at the level of stalks. At any point $z\in \C$, we claim that for any $[f,U]\in \fO_{X,z}^*$, there is some $g_z\in \fO_{X,z}$ such that
    \[
    \exp(g)_z=f_z
    \]
    In $U$, we may take some small simply connected neighborhood $V$ of $z$ contained in $U$. Then on \( V \), since \( f \) is non-vanishing, there exists a logarithm \( g \) of \( f|_V \) \cite{sarason2007complex}, explicitly given by
\[
g(w)=b+\int_z^w \frac{f'(\xi)}{f(\xi)}d\xi
\]
for any choice of a branch of logarithm \( b \) of \( f(z) \). Then indeed
\[
\exp_p([g,V])=[\exp(g),V]=[f\vert_V,V]=[f,U]
\]
which shows, because $z\in \C$ was arbitrary, that $\exp$ is an epimorphism of sheaves by Exercise \ref{2.4.N}N.

However, $\exp$ is not surjective on the level of open sets. Consider $U\coloneqq \C\setminus \{0\}$, and consider the identity function $\id_U$ on $U$. Then indeed $\id_U$ is nowhere $0$, and is holomorphic, so $\id_U\in \fO_X^*(U)$. However, because there is no branch cut in $U$, $\id_U$ does not admit a logarithm on $U$, so $\id_U$ is not in the image of $\exp(U)$.
\end{proof}
\subsection{}
\subsubsection{A}\label{2.5.A}
\begin{proof}
    Suppose we have a sheaf $\fF$ on $X$, and a basis $\{B_i\}$ for the topology of $X$. To show we can recover $\fF$ entirely from what it does to the basis, let $U\subset X$ be open, and $U=\bigcup_j B_j$. We claim that $\fF(U)=S\coloneqq \{\text{gluability applied to every} \{f_j\in \fF(B_j) : f_j\vert_{B_j\cap B_j}=f_k\vert_{B_j\cap B_k}\}$. We see that $\fF(U)\subset S$ because each section $s\in \fF(U)$ restricts to sections of each $\fF(B_j)$ with the desired property, and by the identity applied to $s$ and gluability of $\{s\vert_{B_j}\}$, then $s\in S$. It is clear by definition that $S\subset \fF(U)$, so we have recovered $\fF(U)$ from the data of $\fF$ on the base of the topology.
    
    \vspace{0.1in}
    First, we need to define what it means for a section to restrict to a basis element from $U$. Let $\{B_i\}$ be an open cover of $U$, and fix any $B_j\in \{B_i\}$. By the previous part, let $s\in \fF(U)$ be gluability of some collection of $\{s_i\in B_i\}$. We then define $s\vert_{B_j}=s_j$. Our choice of open cover doesn't matter by identity.

    \vspace{0.1in}
    For arbitrary restriction maps, suppose we have some $V\subset U$ where $U$ is as before. Let $\{B_j\}$ be an open cover of $V$. We then define $s\vert_V$ to be gluability applied to $\{s\vert_{B_j}\}$. By identity, our construction yields the same result as the original $s\vert_{B_j}$ because our definition of $s\vert_V$ restricts the same as the original $s\vert_V$ to the open cover $\{B_j\}$ of $V$. Thus we can also recover the data of the restriction maps.
\end{proof}
\subsubsection{B}\label{2.5.B}
\begin{proof}
    The natural map $\phi:F(B)\to \fF(B)$ is given by $s\mapsto (s_p)_{p\in B}$. For injectivity, suppose $s,t\in F(B)$ are such that their germs agree everywhere on $B$. Then for each point $p\in B$, there exists a base element $U$ of $p$ contained in $B$ such that $s\vert_U=t\vert_U$. By identity of $F$, we get $s=t$.

    \vspace{0.1in}
    For surjectivity, if $(s_p)_{p\in B}$ is a family of compatible germs with corresponding neighborhoods $B_p$ for each $p$ such that $s_q=f^p_q$ for every $q\in B_p$, we apply gluability to $\{f^p\in F(B_p)\}_{p\in B}$ to get a section $f\in F(B)$ such that $f\vert_{B_p}=f^p$ for each $p\in B$. Then
    \[
    \phi(f)=(f_p)_{p\in B}=(f^p_p)_{p\in B}=(s_p)_{p\in B}.
    \]
\end{proof}
\subsubsection{C}\label{2.5.C}
\begin{proof}
    \begin{enumerate}[(a)]
        \item Suppose $\varphi, \phi:\fF\to \fG$ are morphisms of sheaves such that $\varphi(B_i)=\phi(B_i)$ for every $i$. Now, fixing $U\subset X$ to be an arbitrary open set, let $U=\bigcup_j B_j$, and choose any $s\in \fF(U)$. Letting $\glue$ be the gluability operation of $\fG$, notice we have
        \begin{align*}
            \varphi(U)(s)=\glue\{\varphi(U)(s)\vert_{B_j}\}=\glue \{\varphi(B_j)(s\vert_{B_j})\}=\glue\{\phi(B_j)(s\vert_{B_j})\}=\glue\{\phi(U)(s)\vert_{B_j}\}=\phi(U)(s)
        \end{align*}
        where the first and last equalities come from identity of $\fG$.
        \item If $\phi:F\to G$ is a morphism of sheaves on the base, define $\tilde \phi:\fF\to \fG$ by $(f_p)_{p\in U}\mapsto (\phi_p(f_p))_{p\in U}$. Our image is indeed a choice of compatible germs because for every $p\in U$, there exists a neighborhood $B\subset U$ of $p$ and $\phi(B)(s)\in G(B)$ such that for every $q\in B$, $\phi_q(f_q)=\phi(B)(s)_q$. We used the compatibility of the germs in $F$ to obtain the section $s\in F(B)$ such that for every $q\in B$, $s_q=f_q$, so in other words, for each $q\in B$, there exists some $A\subset B$ containing $q$ such that
        \[
        f_q\vert_A=s\vert_A
        \]
        so that
        \[
        [\phi(B)(s),B]=[\phi(B)(s)\vert_A,A]=[\phi(A)(s\vert_A),A]=[\phi(A)(f_q\vert_A)]=\phi_q(f_q).
        \]
        Our map $\tilde \phi$ is natural because
        \[
        \res_{U,V} \circ \tilde \phi(f_p)_{p\in U}=(\phi_p(f_p))_{p\in V}=\tilde \phi \circ \res_{U,V}(f_p)_{p\in U}.
        \]
    \end{enumerate}
\end{proof}
\subsubsection{D}\label{2.5.D}
\begin{proof}
    By Exercise \ref{2.4.N}N, a morphism of sheaves $\phi:\fF\to \fG$ is an epimorphism if and only if it is surjective on the level of stalks. Let $\varphi:F\to G$ be the morphism of sheaves on the base inducing $\fF$ and $\fG$. If $\tilde \varphi:\fF\to \fG$ is the induced morphism of sheaves, we want to show that every $(g_q)_{q\in U}\in \fG_p$ is in the image of $\tilde \varphi_p$. Because $(g_q)_{q\in U}$ is a choice of compatible germs, let $B\subset U$ be the neighborhood of $p$ and $s\in G(B)$ be such that for every $q\in B$, $s_q=g_q$. By hypothesis, there exists some $t\in F(B)$ such that $\varphi(B)(t)=s$, so in particular, for each $q\in B$, 
    \[
    \varphi_q(t_q)=\varphi_q([t,B])=[\varphi(B)(t),B]=[s,B]=s_q.
    \]
    Then we observe
    \begin{align*}
        \tilde \varphi_p(t_q)_{q\in B}=[\tilde \varphi(t_q)_{q\in B},B]=[(\varphi_q(t_q))_{q\in B},B]=[(s_q)_{q\in B},B]=[(g_q)_{q\in B},B]=[(g_q)_{q\in U},U]
    \end{align*}
    which proves $\tilde \varphi_p$ is surjective, finishing the proof.
\end{proof}
\subsubsection{E}\label{2.5.E}
\begin{proof}
    We will first define a sheaf $F$ on the base of open sets contained in at least one of the $U_i$. This is indeed a base because for any open set $U\subset X$, we have $\{U\cap U_i\}$ is an open cover of $U$, and each $U\cap U_i\subset U_i$ implies that each is some proposed base element. For any open set $U$ that is contained in at least one $U_i$, define $l(U)$ to be the least index such that $U\subset U_i$, which is well defined by the well-ordering theorem (equivalent to the axiom of choice). Then we define
    \[
    F(U)\coloneqq \fF_{l(U)} (U).
    \]
    To define restriction for any $V\subset U$, 
    \[
    \res_{U,V}^F\coloneqq \phi_{l(U)l(V)}(V)\circ \res_{U,V}^{\fF_{l(U)}}
    \]
    which makes sense because $V\subset U\subset U_{l(U)}$ and $V\subset U_{l(V)}$ implies $V\subset U_{l(U)}\cap U_{l(V)}$. To show our construction is indeed a sheaf on the base, we will first show identity.

    \vspace{0.1in}
    Suppose $B$ is a base element covered by $\{B_j\}$ of other base elements, and $f,g\in F(B)$ are such that $f\vert_{B_j}=g\vert_{B_j}$ for each $j$. By definition,
    \begin{align*}
        \phi_{l(B)l(B_j)}(B_j)\circ \res_{B,B_j}^{\fF_{l(B)}}(f)=\phi_{l(B)l(B_j)}(B_j)\circ \res_{B,B_j}^{\fF_{l(B)}}(g).
    \end{align*}
    Because each of the $\phi_{ij}$ are isomorphisms, we get that
    \[
    \res_{B,B_j}^{\fF_{l(B)}}(f)=\res_{B,B_j}^{\fF_{l(B)}}(g)
    \]
    for each $j$. By identity of $\fF_{l(B)}$, we have that indeed $f=g$, proving identity holds for $F$.

    \vspace{0.1in}
    To prove gluability holds for $F$, suppose we have a collection of $f_j\in F(B_j)$ with $B_i\coloneqq \bigcup B_j$ be a basis element as well such that for each $j,k$, we have
    \[
    \phi_{l(B_j)l(B_j\cap B_k)}(B_j\cap B_k)\circ \res_{B_j,B_j\cap B_k}^{\fF_{l(B_j)}}(f_j)=\phi_{l(B_k)l(B_j\cap B_k)}(B_j\cap B_k)\circ \res_{B_k,B_j\cap B_k}^{\fF_{l(B_k)}}(f_k).
    \]
    Notice that we then have isomorphisms
    \[
    \phi_{l(B_j)l(B_i)}:\fF_{l(B_j)}\vert_{U_{l(B_j)}\cap U_{l(B_i)}}\to \fF_{l(B_i)}\vert_{U_{l(B_j)}\cap U_{l(B_i)}}
    \]
    so in particular $\phi_{l(B_j)l(B_i)}(B_j):F(B_j)\to \fF_{l(B_i)}(B_j)$ is an isomorphism because $B_j\subset B_i$ implies that $B_j\subset U_{l(B_j)}$ and $B_j\subset U_{l(B_i)}$ as well. In addition, its inverse is $\phi_{l(B_i) l(B_j)}$ by the cocycle condition. By commutativity of the below diagram
    \begin{center}
        \begin{tikzcd}
            \fF_{l(B_i)}\vert_{B_j\cap B_k} \ar{rr}{\phi_{l(i)l(j)}} \ar{drr}[description]{\phi_{l(B_i) l(B_j\cap B_k)}} \ar{d}[swap]{\phi_{l(B_i)l(B_k)}}&& \fF_{l(B_j)}\vert_{B_j\cap B_k} \ar{d}{\phi_{l(B_j)l(B_j\cap B_k)}}\\
            \fF_{l(B_k)}\vert_{B_j\cap B_k} \ar{rr}[swap]{\phi_{l(B_k)l(B_j\cap B_k)}}&&\fF_{l(B_j\cap B_k)}\vert_{B_j\cap B_k}
        \end{tikzcd}
    \end{center}
        we obtain that
        \begin{align*}
            & \phi_{l(B_i)l(B_j\cap B_i)}(B_j\cap B_k)\circ\phi_{l(B_j)l(B_i)}(B_j\cap B_k)\circ \res_{B_j,B_j\cap B_k}^{\fF_{l(B_j)}}(f_j)\\
            &= \phi_{l(B_i)l(B_j\cap B_i)}(B_j\cap B_k)\circ\phi_{l(B_k)l(B_i)}(B_j\cap B_k)\circ \res_{B_k,B_j\cap B_k}^{\fF_{l(B_k)}}(f_k)
        \end{align*}
         or equivalently by naturality
         \begin{align*}
             & \phi_{l(B_i)l(B_j\cap B_i)}(B_j\cap B_k)\circ \res_{B_j,B_j\cap B_k}^{\fF_{l(B_i)}}\circ\phi_{l(B_j)l(B_i)}(B_j)(f_j)\\
            &= \phi_{l(B_i)l(B_j\cap B_i)}(B_j\cap B_k)\circ \res_{B_k,B_j\cap B_k}^{\fF_{l(B_i)}}\circ \phi_{l(B_k)l(B_i)}(B_k)(f_k).
         \end{align*}
        Now we use the fact that the morphisms on the left of each side of the equation are isomorphisms (so in particular monomorphisms) to get
         \begin{align*}
             & \res_{B_j,B_j\cap B_k}^{\fF_{l(B_i)}}\circ\phi_{l(B_j)l(B_i)}(B_j)(f_j)\\
            &= \res_{B_k,B_j\cap B_k}^{\fF_{l(B_i)}}\circ \phi_{l(B_k)l(B_i)}(B_k)(f_k).
         \end{align*}
    Now we consider the family $\{\phi_{l(B_j)l(B_i)}(B_j)(f_j)\in \fF_{l(B_i)}(B_j)\}$. We can apply gluability of $\fF_{l(B_i)}$ to this family by the previous observation. Let $f\in \fF_{l(B_i)}(B_i)$ be the result of this gluing. Then we observe
    \begin{align*}
        \res_{B_i,B_j}^F(f)=\phi_{l(B_i)l(B_j)}(B_j)\circ \res_{B_i,B_j}^{\fF_{l(B_i)}}(f)=\phi_{l(B_i)l(B_j)}(B_j)(\phi_{l(B_j)l(B_i)}(f_j))=f_j
    \end{align*}
    which proves that gluability holds for $F$ as well.

    \vspace{0.1in}
    Now that $F$ is a sheaf on a base, we get our induced sheaf $\fF$ on $X$. By Theorem 2.5.1, we have that $\fF$ extends $F$ up to isomorphism, so $\fF(U_i)\cong F(U_i)$, and $F(U_i)=\fF_{l(U_i)}(U_i)\cong \fF_i(U_i)$ because even if $l(U_i)\ne i$, then $U_i\subset U_{l(U_i)}$, then by the cocycle condition we obtain $\fF_{l(U_i)}\vert_{U_i}\cong \fF_i$ because
    \[
    \phi_{l(U_i),i}:\fF_{l(U_i)}\vert_{U_{l(U_i)}} \xrightarrow{\sim} \fF_i\vert_{U_i}=\fF_i.
    \]
\end{proof}
\subsection{}
\subsubsection{A}\label{2.6.A}
\begin{proof}
    Suppose $\phi:\fF\to \fG$ is a morphism of sheaves, and $\phi_p:\fF_p\to \fG_p$ is the induced map on stalks. To show $(\ker \phi)_p\cong \ker \phi_p$,
    \begin{align*}
        &\phi_p([f,U])=0\\
        &\iff [\phi(U)(f),U]=0\\
        &\iff \phi(V)(f\vert_V)=0 \text{ for some open $V\subset U$}\\
        &\iff f\vert_V\in \ker \phi(V)\\
        &\iff [f\vert_V,V]\in (\ker \phi)_p.
    \end{align*}
    Then our map $\varphi:\ker \phi_p\to (\ker \phi)_p$ is given by $[f,U]\mapsto [f\vert_V,V]$ where $V$ is some neighborhood of $p$ such that $[f\vert_V,V]\in (\ker \phi)_p$. $\varphi$ is well defined because if $V'$ is another such neighborhood, then $[f\vert_{V},V]=[f\vert_{V'},V']$ because $[f\vert_V,V]=[f\vert_{V\cap V'},V\cap V']=[f\vert_{V'},V']$ so our choice of $V$ doesn't matter, and furthermore if $[f,U]=[f',U']$, then $f\vert_V=f'\vert_V$ for some $V\subset U\cap U'$, then we can take $[f,U]\mapsto [f\vert_{V'}, V']$ where $V'$ is again some neighborhood of $p$ contained in $V$ so that $f\vert_{V'}\in \ker \phi(V')$ ensures that also $\varphi([f,U])=[f\vert_{V'},V']=[f'\vert_{V'},V']=\varphi([f',U'])$.

    \vspace{0.1in}
    $\varphi$ can also be seen to be a homomorphism because
    \begin{align*}
        &\varphi([f,U]+[g,V])\\
        &=\varphi([f\vert_{U\cap V}+g\vert_{U\cap V}])\\
        &= [f\vert_W+g\vert_W,W]\\
        &=[f\vert_W,W]+[g\vert_W,W]\\
        &=\varphi([f,U])+\varphi([g,V]).
    \end{align*}
    To prove $\varphi$ is injective, suppose $\varphi([f,U])=[f\vert_V,V]=0$. Then for some neighborhood $W$ of $p$ contained in $V$, $f\vert_W=0$. This implies that $[f,U]=0$. To prove surjectivity, if $[f,U]\in (\ker \phi)_p$ is arbitrary $f\in \ker \phi(U)$ implies that $[f,U]\in \ker \phi_p$, so $\varphi([f,U])=[f,U]$. Then $\varphi$ is an isomorphism.
\end{proof}
\subsubsection{B}\label{2.6.B}
\begin{proof}
    If $\phi:\fF\to \fG$ is a morphism of sheaves, we will show $\cok \phi_p\cong (\cok \phi)_p$ by showing $(\cok \phi)_p$ satisfies the universal property of $\cok \phi_p$. Suppose the following diagram commutes:
    \begin{center}
        \begin{tikzcd}
            & A\\
            \fF_p \ar{r}{\phi_p} \ar{ur}{0}& \fG_p \ar{u}[swap]{\theta}.
        \end{tikzcd}
    \end{center}
    We consider the constant sheaf $\underline{A}$, and let $\sigma_U:\fG(U)\to \fG_p$ be the map sending a section to its germ and $\tau_U:\fF(U)\to \fF_p$ do the same. Then for $x\in \fG(U)$, we let $f_x:U\to A$ be the constant function to $\theta \circ \sigma_U(x)$. Clearly $f_x$ is continuous (where $A$ is given the discrete topology), so we may define a morphism $\varphi:\fG \to \underline{A}$ given by $\varphi(U)(x)=f_x$. We verify that $\varphi$ is natural because for $x\in \fG(U)$, $\varphi(V)(x\vert_V)=f_{x\vert_V}$, sending everything to $\theta \circ \sigma _V(x\vert_V)=\theta \circ \sigma_U(x)$, which is the same function as $\varphi(U)(x)$ restricted to $V$. We also check that $\varphi \circ \phi = 0$ because if $x\in \fF(U)$, then $\varphi \circ \phi(U)(x)$ sends everything in $U$ to $\theta \circ \sigma_U \circ \phi(U)(x) = \theta \circ \phi_p \circ \tau_U(x)=0\circ \tau_U(x)=0$. By the universal property of the cokernel presheaf, we get the below commutative diagram:
    \begin{center}
        \begin{tikzcd}
            && \underline{A}\\
            & \cokpre \phi \ar[dashed]{ur}[description]{\exists! \alpha}\\
            \fF \ar{r}{\phi} \ar{ur}{0} & \fG \ar[two heads]{u}{\pi} \ar[bend right]{uur}{\varphi}.
        \end{tikzcd}
    \end{center}
    Because $\underline{A}$ is a sheaf, $\alpha=\beta \circ \sh$ for a unique map $\beta:\cok \phi \to \underline{A}$. Notice $\underline{A}_p \cong A$ by taking a germ to its value at $p$, so we have a natural map $\beta_p:(\cok \phi)_p\to A$ which we claim makes the following diagram commute:
    \begin{center}
        \begin{tikzcd}
            && A\\
            & (\cok \phi)_p \ar{ur}[description]{\beta_p}\\
            \fF_p \ar{r}{\phi_p} \ar{ur}{0} & \fG_p \ar{u}{\mu_p} \ar[bend right]{uur}{\theta}
        \end{tikzcd}
    \end{center}
    where $\mu=\sh \circ \pi:\fG\to \cok \phi$ is the map to the cokernel sheaf. Unraveling our definitions, we recall $\beta \circ \mu = \alpha \circ \pi = \varphi$. This is of interest because the below diagram commutes
    \begin{center}
        \begin{tikzcd}
            \underline{A}(U) \ar{r}{\ev_p} & A\\
            \cok \phi(U) \ar{u}{\beta(U)}& (\cok \phi)_p \ar{u}{\beta_p}\\
            \fG(U) \ar{r}{\sigma_U} \ar{u}{\mu(U)}& \fG_p \ar{u}{\mu_p},
        \end{tikzcd}
    \end{center}
    so we take any $\sigma_U(x)\in \fG_p$, and get $\beta_p\circ \mu_p (\sigma_U(x))= \ev_p \circ \varphi(x)=\theta( \sigma_U(x))$ as desired, which proves existence. To show $\beta_p$ is unique, suppose some $\gamma$ has $\gamma \circ \mu_p = \theta$. Because $\mu$ is an epimorphism in the category of sheaves by Proposition 2.6.1, Exercise \ref{2.4.N} tells us that $\mu_p$ is an epimorphism, hence we get $\beta_p \circ \mu_p = \theta = \gamma_p \circ \mu_p $ implies $\gamma = \beta_p$ as desired.
\end{proof}
\subsubsection{C}\label{2.6.C}
\begin{proof}
    We will first show the sheafification satisfies the universal property of the coimage sheaf. Let $\phi:\fF\to \fG$ be a map of sheaves, $i:\ker \phi \hookrightarrow \fF$ be the kernel, and $q:\fF\twoheadrightarrow \coim_{\pre} \phi$ be the cokernel of $i$ in $\Ab_X^{\pre}$. It's clear to see the following diagram commutes in $\Ab_X$:
    \begin{center}
        \begin{tikzcd}
            & (\coim_{\pre} \phi)^{\sh}\\
            \ker \phi \ar[hook]{r}{i} \ar{ur}{0}& \fF \ar[two heads]{u}[swap]{\sh \circ q}.
        \end{tikzcd}
    \end{center}
    Now suppose $\fH$ is a sheaf and $\psi:\fF\to \fH$ is a map such that $\psi \circ i=0$. By the universal property of $\coim_{\pre} \phi$, we get the below commutative diagram:
    \begin{center}
        \begin{tikzcd}
            && \fH\\
            & \coim_{\pre} \phi \ar[dashed]{ur}[description]{\exists! \alpha}\\
            \ker \phi \ar[hook]{r}{i} \ar{ur}{0} & \fF \ar[two heads]{u}{q} \ar[bend right]{uur}[swap]{\psi}.
        \end{tikzcd}
    \end{center}
    Note that we have used the fact that the kernel presheaf is the kernel sheaf. Now, by the universal property of the sheafification, the below diagram commutes:
    \begin{center}
        \begin{tikzcd}
            \coim_{\pre} \phi \ar{r}{\alpha} \ar{dr}[swap]{\sh}& \fH\\
            & (\coim_{\pre} \phi)^{\sh} \ar[dashed]{u}[description]{\exists! \beta}.
        \end{tikzcd}
    \end{center}
    Thus the below diagram commutes:
    \begin{center}
        \begin{tikzcd}
            && \fH\\
            & (\coim_{\pre} \phi)^{\sh} \ar{ur}{\beta}\\
            \ker \phi \ar[hook]{r}{i} \ar{ur}{0} & \fF \ar{u}{\sh \circ q} \ar[bend right]{uur}[swap]{\psi}.
        \end{tikzcd}
    \end{center}
    This shows existence. If there were another map $\gamma:(\coim_{\pre} \phi)^{\sh} \to \fH$ making the diagram commute, since $q$ is an epimorphism we have $\gamma \circ \sh \circ q = \psi = \beta \circ \sh \circ q$ implies that $\gamma \circ \sh =\alpha= \beta \circ \sh$, which implies $\gamma = \beta$ by uniqueness of the arrow in the sheafification diagram, proving $(\coim_{\pre} \phi)^{\sh}$ satisfies the universal property of $\coim \phi$ in $\Ab_X$. However, coimages are the same as images in abelian categories by Theorem \ref{thm:1IT}, and Theorem 2.6.2 and Section 2.3 tell us $\Ab_X$ and $\Ab_X^{\text{pre}}$ are abelian categories.

    In addition, Exercises \ref{2.6.A}A and \ref{2.6.B}B say that stalks commute with kernels and cokernels, hence
    \[
    (\im \phi)_p = (\ker \cok \phi)_p = \ker (\cok \phi)_p = \ker \cok \phi_p= \im \phi_p.
    \]
\end{proof}
\subsubsection{D} \label{2.6.D}
\begin{proof}
    For one direction, suppose $\fF \xrightarrow{\alpha} \fG \xrightarrow{\beta} \fH$ is exact, and let $p\in X$ be arbitrary. Exercise \ref{2.3.A}A tells us that taking stalks at $p$ is functorial so we have a sequence
    \[
    \fF_p \xrightarrow{\alpha_p} \fG_p \xrightarrow{\beta_p} \fH_p
    \]
    and Exercises \ref{2.6.A}A and \ref{2.6.C}C give exactness since kernels and images commute with stalks.
    
    For the other direction, suppose $\fF_p \xrightarrow{\alpha_p} \fG_p \xrightarrow{\beta_p} \fH_p$ is exact for all $p\in X$. Exercise \ref{2.6.C}C tells us that since $\im \alpha$ and $\ker \beta$ induce the same maps on stalks, they are equal by Exercise \ref{2.4.C}C.
\end{proof}
\subsubsection{E}\label{2.6.E}
\begin{proof}
    If 
    \[
    0\to \fF \xrightarrow{\phi} \fG \xrightarrow{\psi} \fH \to 0
    \]
    is exact, Exercise \ref{2.3.A}A tells us that taking stalks at $p$ is functorial so we have a sequence
    \[
    0\to \fF_p \xrightarrow{\phi_p} \fG_p \xrightarrow{\psi_p} \fH_p \to 0
    \]
    and Exercises \ref{2.6.A}A and \ref{2.6.C}C give exactness since kernels and images commute with stalks.
\end{proof}
\subsubsection{F}\label{2.6.F}
\begin{proof}
    To show
    \[
    0\to \underline \Z \xrightarrow{\cdot 2\pi i} \fO_\C \xrightarrow{\exp} \fO_\C^* \to 0
    \]
    is exact, we will first show $\cdot 2\pi i$ is a monomorphism by checking this on the level of open sets. If $U\subset \C$ is an open set with connected components $U_i$, then an arbitrary element of $\underline \Z(U)$ is a choice of $n_i$'s with each $n_i\in \Z$. Then $(n_i)\cdot 2\pi i=(2\pi i n_i)$ is trivial only if each $n_i=0$, proving exactness at $\underline \Z$.

    Exercise \ref{2.4.O}O gives exactness at $\fO_\C^*$. To show exactness at $\fO_\C$, we first claim that $\impre \cdot 2\pi i$ is a sheaf, which will then show $\impre \cdot 2\pi i = \im \cdot 2\pi i$ by Exercise \ref{2.6.C}C. Because we have shown $\cdot 2\pi i$ is a monomorphism, we use the fact that $\Ab_\C^{\pre}$ is an abelian category and apply Corollary \ref{cor:comp with monic and coim} along with the 1IT to get an isomorphism between $\impre \cdot 2\pi i$ and $\underline{\Z}$, proving the required statement. Now that $\impre \cdot 2\pi i = \im \cdot 2\pi i$ and $\kerpre \exp = \ker \exp$, we need to show $\im \cdot 2\pi i = \ker \exp$, which we will do on the level of stalks by Exercise \ref{2.4.D}D. Let $z\in \C$ and $[f,U]\in \ker \exp_z$ be arbitrary, so $[\exp(f),U]=[1,U]$. It's clear that $\im \cdot 2\pi i_z\subset \ker \exp_z$, so we will just show the reverse inclusion. The fact that $[\exp(f),U]=[1,U]$ tells us that there is some open $V\subset U$ containing $z$ such that $\exp(f\vert_V)$ is identically $1$. This implies that $f\vert_V$ is some integer multiple of $2\pi i$, so $[f,U]=[2\pi i n, V]$ for some $n\in \Z$, so $[f,U]\in \im \cdot 2\pi i_z$ as desired.
\end{proof}
\subsubsection{G}\label{2.6.G}
\begin{proof}
    We suppose
    \[
    0\to \fF \xrightarrow{\phi} \fG \xrightarrow{\psi } \fH
    \]
    is an exact sequence of sheaves. Exercise \ref{2.4.M}M gives that since $\phi$ is a monomorphism, $\phi(U):\fF(U) \to \fG(U)$ is also a monomorphism. To show $\im \phi(U)\cong \ker \psi(U)$, we have
    \[
    \ker \psi(U)
    \]
    
    
    we have an isomorphism $\alpha:\im \phi \to \ker \psi$ with inverse $\alpha^{-1}:\ker \psi\to \im \phi$ which, in particular, gives the desired isomorphisms $\alpha(U),\alpha^{-1}(U)$.

    To show the section functor need not be exact, again consider the exponential exact sequence
    \[
    0\to \underline \Z \xrightarrow{\phi } \fO_\C \xrightarrow{ \psi} \fO_\C^* \to 0.
    \]
    However, $\psi(\C)$ is not surjective because $\altid_\C \in \fO_\C^*$, but not in the image of $\psi$ because the $\C$ does not admit a global logarithm.
\end{proof}
\subsubsection{H}\label{2.6.H}
\begin{proof}
    Let
    \[
    0\to \fF \xrightarrow{\phi} \fG \xrightarrow{\psi} \fH
    \]
    be exact. First, we will show that $\pi_*$ commutes with kernels. If $\varphi$ is some map of sheaves, then
    \[
    \pi_* \ker \varphi (U) = \ker \varphi (\pi^{-1}(U)) = \ker \pi_* \varphi(U),
    \]
    which relies on the fact that the kernel sheaf is the kernel presheaf. Indeed the restriction maps are the same, which proves the claim. To show exactness at $\pi_* \fF$, we use our result to see
    \[
    \ker \pi_* \phi = \pi_* \ker \phi = \pi_* 0 = 0.
    \]
    For exactness at $\pi_* \fG$, we first notice that $\pi_* \ker \psi = \ker \pi_* \psi$ by our previous observations. Then by hypothesis and $\Ab_Y$ is an abelian category (we identify the image of a monomorphism with its source by the 1IT and Corollary \ref{cor:comp with monic and coim}
    ),
    \[
    \ker \pi_* \psi = \pi_* \ker \psi = \pi_* \im \phi = \pi_* \fF = \im \pi_* \phi.
    \]
    Alternatively, we could have used Exercise \ref{2.7.B} together with the fact that right adjoint functors are left-exact as stated in 1.6.12.
\end{proof}
\subsubsection{I}\label{2.6.I}
\begin{proof}
    Suppose $\fF\in \Ab_X$, and
    \[
    0\to \fA \xrightarrow{\phi} \fB \xrightarrow{\psi} \fC
    \]
    is exact in $\Ab_X$, so we need to show
    \[
    0 \to \Hom(\fF,\fA) \xrightarrow{\phi_*} \Hom(\fF, \fB) \xrightarrow{\psi_*} \Hom(\fF,\fC)
    \]
    is exact. By Exercise \ref{2.4.M}M, it suffices to show $\phi_*(U)$ is injective for exactness at $\Hom(\fF,\fA)$. Let $\eta:\fF\vert_U \to \fA \vert_U$ be a natural transformation. We note $\phi\vert_U$ is a monomorphism because it is injective on the level of sections, both claims following from Exercise \ref{2.4.M}M. Then $0=\phi_*(U)(\eta)=\phi\vert_U\circ \eta$ implies that, since $\phi\vert_U$ is a monomorphism,  $\eta = 0$ so $\phi_*(U)$ is indeed injective.

    To show $\ker \psi_* = \im \phi_*$, now that we've shown $\phi_*$ is a monomorphism, we get $\im \phi_* \cong \Hom(\fF, \fA)$ which is in particular a sheaf, so $\im \phi_* = \impre \phi_*$. Then we need to show $\kerpre \psi_* = \impre \phi_*$ as subsheaves of $\Hom(\fF, \fB)$, which we can do by checking equality on the level of sections (both are subsheaves of the same sheaf). For arbitrary open $U\subset X$, we have
    \begin{align*}
        &\ker \psi_*(U) = \{\eta \in \Nat(\fF\vert_U, \fB\vert_U) \mid \psi\vert_U \circ \eta = 0\}\\
        &\im \phi_*(U) = \{\eta \in \Nat(\fF \vert_U, \fB\vert_U) \mid \exists \eta' \in \Nat(\fF\vert_U,\fA\vert_U) \text{ where } \eta = \phi\vert_U \circ \eta'\}.
    \end{align*}
    Its clear that the image is contained in the kernel. For the reverse inclusion, we pick any $\eta:\fF\vert_U \to \fB\vert_U$ such that $\psi\vert_U \circ \eta = 0$. Since $A\cong \ker \psi$ and $\ker (\psi \vert_U) = (\ker \psi)\vert_U$ (seen because again the kernel sheaf is the kernel presheaf), we get the below commutative diagram
    \begin{center}
        \begin{tikzcd}
            &&\fC\vert_U\\
            & \fA\vert_U \ar[hook]{r}{\phi \vert_U} \ar{ur}{0}& \fB\vert_U \ar{u}[swap]{\psi\vert_U}\\
            \fF\vert_U \ar[dashed]{ur}[description]{\exists! \eta'} \ar[bend right]{urr}[swap]{\eta}.
        \end{tikzcd}
    \end{center}
    Then $\eta = \phi \vert_U \circ \eta'$, so the kernel is contained in the image.

    \vspace{0.1in}
    Now suppose $$\fA \xrightarrow{\phi} \fB \xrightarrow{\psi} \fC \to 0$$ is exact, so we need to show
    \[
    0 \to \Hom(\fC, \fF) \xrightarrow{\psi^*} \Hom(\fB, \fF) \xrightarrow{\phi^*} \Hom(\fA, \fF)
    \]
    is exact. It suffices to show $\psi^*_p$ is injective for exactness at $\Hom(\fC, \fF)$ by Exercise \ref{2.4.M}M. If $[\eta, U]$ is such that $[\eta \circ \psi \vert_U, U]=0$, then there is some open $V\subset U$ containing $p$ such that $\eta \vert_V \circ \psi \vert_V=0$. Fixing $p\in X$, we use Exercise \ref{2.4.N} to get $\psi_p:\fB_p \twoheadrightarrow \fC_p$, so $\eta_p\circ \psi_p = 0$ implies $\eta_p = 0$. By Exercise \ref{2.4.C}, $\eta$ being a map of sheaves inducing trivial maps on all stalks means $\eta=0$.

    To show $\ker \phi^* = \im \psi^*$, now that we've shown $\psi^*$ is a monomorphism, we get $\im \psi^* \cong \Hom(\fC, \fF)$ which is in particular a sheaf, so $\im \psi^* = \impre \psi^*$. Then we need to show $\kerpre \phi^* = \impre \psi^*$ as subsheaves of $\Hom(\fB, \fF)$, which we can do by checking equality on the level of sections (both are subsheaves of the same sheaf). For arbitrary open $U\subset X$, we have
    \begin{align*}
        &\ker \phi^*(U) = \{\eta \in \Nat(\fB\vert_U, \fF\vert_U) \mid \eta\circ \phi\vert_U = 0\}\\
        &\im \psi^*(U) = \{\eta \in \Nat(\fB\vert_U, \fF\vert_U) \mid \exists \eta' \in \Nat(\fC\vert_U,\fF\vert_U) \text{ where } \eta = \eta' \circ \psi\vert_U \}.
    \end{align*}
    Its clear that the image is contained in the kernel. For the reverse inclusion, we pick any $\eta: \fB\vert_U \to \fF\vert_U$ such that $\eta\circ \phi\vert_U = 0$. We have
    \[
    \fC = \coim \psi = \cok \ker \psi = \cok \im \phi = \cok \phi.
    \]
    We define a sheaf $\tilde \fF$ over $X$ where $\tilde \fF(V) =\fF(U\cap V)$ for open $V\subset X$ with the natural restriction maps $x\vert_W \coloneqq x\vert_{U\cap W}$ (the left side is the definition of restriction in $\tilde \fF$, and the right occurs in $\fF$), and let $\tilde \eta:\fB \to \tilde \fF$ be given as $\tilde \eta(V)(x) = \eta(U\cap V)(x\vert_{U\cap V})$. That $\tilde \fF$ is a sheaf follows from $\fF$ being a sheaf, and $\tilde \eta$ is easily checked to be natural:
    \[
    \tilde \eta(V)(x)\vert_W = \eta(U\cap V)(x\vert_{U\cap V})\vert_{U\cap W} = \eta(U\cap W)(x\vert_{U\cap W}) = \tilde \eta(W)(x\vert_W).
    \]
    We claim that $\tilde \eta \circ \phi =0$ so that we can use our result that $\fC=\cok \phi$ to get a desirable factorization. To see this, we let $V\subset X$ be an open set, and see
    \[
    \tilde \eta(V) \circ \phi(V)(x) = \eta(U\cap V)(\phi(V)(x) \vert_{U\cap V}) = \eta(U\cap V)\circ \phi\vert_U(U\cap V)(x\vert_{U\cap V}) = 0
    \]
    by our assumption that $\eta \circ \phi\vert_U = 0$. Then the below diagram commutes:
    \begin{center}
        \begin{tikzcd}
            && \tilde \fF \\
            & \fC \ar[dashed]{ur}[description]{\exists! \tilde \mu}\\
            \fA \ar{r}{\phi} \ar{ur}{0}& \fB \ar[two heads]{u}{\psi} \ar[bend right]{uur}[swap]{\tilde \eta}.
        \end{tikzcd}
    \end{center}
    We now notice that $\tilde \fF \vert_U = \fF\vert_U$ and that $\tilde \eta \vert_U = \eta$ by our constructions, and thus we have a map $\eta' \coloneqq \tilde \mu \vert_U$ such that $\eta = \eta' \circ \psi \vert_U$ as desired.
\end{proof}
\subsubsection{J}\label{2.6.J}
\begin{proof}
     Let $\fA, \fB, \fC$ be $\fO_X$ modules with $\alpha, \alpha':\fA \to \fB$ and $\beta,\beta': \fB \to \fC$, and $V\subset U\subset X$ be open sets. We will be using the fact that the category of $\fO_X(U)$ modules is an abelian category itself implicitly.
     
     First we will check additivity by showing hom-sets are abelian groups and composition distributes over addition. We define $\alpha+\alpha'$ to be the morphism such that $(\alpha+\alpha')(U)=\alpha(U)+\alpha'(U)$, which is easily checked to be natural. We also observe that $\beta \circ (\alpha+\alpha')=\beta \circ \alpha + \beta \circ \alpha'$ because this equality holds on the level of sections. Similarly $(\beta+\beta')\circ \alpha = \beta \circ \alpha + \beta' \circ \alpha$.

     The zero object is the zero sheaf, which clearly has $\fO_X$-module structure.

     We define $\fA \times \fB$ as the sheaf where $(\fA\times \fB)(U)=\fA(U)\oplus \fB(U)$, and where the restriction maps are the direct sums of the restriction maps. That $\fA \times \fB$ is a sheaf (not just a presheaf) follows from $\fA$ and $\fB$ being sheaves. In addition, $\fA\times \fB$ is canonically an $\fO_X$ module by
     \begin{center}
         \begin{tikzcd}
             &\fO_X(U) \ar[dashed]{d}[description]{\text{action}} \ar[bend right]{ddl}[swap]{\text{action}} \ar[bend left]{ddr}{\text{action}}\\
             &\fA(U)\oplus \fB(U) \ar{dl} \ar{dr}\\
             \fA(U)&& \fB(U).
         \end{tikzcd}
     \end{center}
     This action commutes with restriction to $V$ because the actions on $\fA$ and $\fB$ do.

     We already know that kernels and cokernels exist in the category of sheaves, so we just need to show $\ker \beta$ and $\cok \alpha$ are $\fO_X$ modules as well. Recall that $\cok \alpha$ is the sheafification of the cokernel presheaf, so an arbitrary element of $\cok \alpha(U)$ looks like $(\overline{x_i}_p)_{p\in U}$ for some collection of compatible germs $\overline{x_i}_p\in \fB_p$. Exercise \ref{2.2.J}J, we have a natural action of $\fO_{X,p}$ on $\fB_p$ for each $p\in U$, which induces an action of $\fO_X(U)$ on $(\cokpre \alpha)^{\sh}$ because the sheafification consists of choices of compatible germs. $\fO_X$ gets a canonical action on $\ker \beta$, since this is a subsheaf of $\fB$.

     The last two axioms of every monomorphism being the kernel of its cokernel and every epimorphism being the cokernel of its kernel follow from our previous results, along with the fact that monomorphisms and epimorphisms in $\Ab_X$ already have this property.
\end{proof}
\subsubsection{K}\label{2.6.K}
\begin{proof}
    Categorically, if $\fF, \fG$ are $\fO_X$ modules, $\fF \otimes_{\fO_X} \fG$ should be an $\fO_X$ module equipped with an $\fO_X$-bilinear map from $\fF\times \fG$ where we say $\phi$ is an $\fO_X$-bilinear map if $\phi(U)$ is an $\fO_X(U)$-bilinear map of $\fO_X(U)$ modules for every open $U\subset X$. Moreover, for any $\fO_X$ module $\fH$ with a $\fO_X$-bilinear map $\phi:\fF\times \fG \to \fH$, the below diagram commutes:
    \begin{center}
        \begin{tikzcd}
            \fF\times \fG \ar{r} \ar{dr}[swap]{\phi}& \fF \otimes_{\fO_X} \fG \ar[dashed]{d}[description]{\exists!}\\
            & \fH.
        \end{tikzcd}
    \end{center}
    As usual, this universal property defines our object up to isomorphism. To show existence, we first define the ``presheaf tensor product". If $\fF, \fG$ are $\fO_X$ modules, we let $(\fF \otimes_{\fO_X} \fG)_{\pre}(U)\coloneqq \fF(U) \otimes_{\fO_X(U)} \fG(U)$ be the presheaf with restriction maps given by
    \begin{center}
        \begin{tikzcd}
            \fF(U)\times \fG(U) \ar{r} \ar{dr}[swap]{\res \times \res}& \fF(U) \otimes_{\fO_X(U)} \fG(U) \ar[dashed]{d}[description]{\exists!}\\
            & \fF(V) \times \fG(V) \ar{d}\\
            & \fF(V)\otimes_{\fO_X(V)} \fG(V).
        \end{tikzcd}
    \end{center}
    Then for $\fO_X$ modules $\fF, \fG$, we define the tensor product of $\fF$ and $\fG$ over $\fO_X$ as $(\fF\otimes_{\fO_X} \fG)_{\pre}^{\sh}$. It's clear our construction is a sheaf, so now we must show it is an $\fO_X$ module. We will call $(\fF \otimes_{\fO_X} \fG)_{\pre}^{\sh}$ $\fF\otimes_{\fO_X} \fG$ for ease of notation, keeping in mind that we have not shown this object satisfies the universal property we originally defined. $(\fF \otimes_{\fO_X} \fG)_{\pre}(U)$ is clearly an $\fO_X(U)$-module\iffalse, and we claim that this action commutes with restrictions. We will show the result for pure tensors for simplicity, and the more general result follows by linearity of our maps. For $V\subset U$ open, $f\in \fF(U)$, $g\in \fG(U)$, and $r\in \fO_X(U)$, we compute
    \begin{align*}
        \left( r(f\otimes g) \right) \vert_V = (rf\otimes g)\vert_V = (rf)\vert_V \otimes g \vert_V = r\vert_V f\vert_V \otimes g \vert_V = r\vert_V (f \otimes g)\vert_V.
    \end{align*}
    \fi. Then for $p\in U$, we claim there is an action of $\fO_X(U)$ on $(\fF\otimes_{\fO_X} \fG)_p$ given by \newline $r[x,V] = [r\vert_{U\cap V} x\vert_{U\cap V},U\cap V]$. To show this is well defined, if we picked some other representative where $[x,V]=[y,W]$, so there is some open $S\subset V\cap W$ with $x\vert_S = y\vert_S$, we have
    \[
    r[y,W] = [r\vert_{U\cap W} y\vert_{U\cap W}, U\cap W]=[r\vert_{U\cap S} y\vert_{U\cap S}, U\cap S]=[r\vert_{U\cap S} x\vert_{U\cap S}, U\cap S]=[r\vert_{U\cap V} x\vert_{U\cap V},U\cap V] = r[x,V].
    \]
    Then if $(x_p)\in \prod_{p\in U} (\fF\otimes_{\fO_X} \fG)_{\pre,p}$ is a choice of compatible germs, we define $r(x_p)_{p\in U} = (rx_p)_{p\in U}$ which defines an action of $\fO_X$ on $\fF\otimes_{\fO_X} \fG$ (being the sheafification of $(\fF\otimes_{\fO_X} \fG)_{\pre}$). This action commutes with restrictions because
\begin{align*}
    &\left( r([x_p, U_p])_{p\in U}\right) \vert_{V} = \left( ([r\vert_{U\cap U_p} x_p \vert_{U\cap U_p}, U\cap U_p])_{p\in U} \right) \vert_V=([r\vert_{U\cap U_p} x_p \vert_{U\cap U_p}, U\cap U_p])_{p\in V}\\
    &=([r\vert_{V\cap U_p} x_p \vert_{V\cap U_p}, V\cap U_p])_{p\in V}=  (r\vert_V[x_p ,U_p])_{p\in V} = r\vert_V \left( ([x_p,U_p])_{p\in U}\right)\vert_{V}.
\end{align*}
    Then indeed $\fF \otimes_{\fO_X} \fG$ is an $\fO_X$ module.
    
    We now need to show $\fF \otimes_{\fO_X} \fG$ satisfies the universal property by supposing $\fH$ is an $\fO_X$ module and $\phi:\fF\times \fG \to \fH$ is $\fO_X$-bilinear. Then for each open $U\subset X$, if there were a factor $\alpha:\fF \otimes_{\fO_X} \fG\to \fH$ through which $\phi$ factored, we would see that for $(f,g)\in \fF(U)\times \fG(U)$,
    \[
    \alpha(U)(([f\otimes g,U]_p)_{p\in U})= \phi(U)(f,g)
    \]
    and we extend this linearly. By the universal property of tensor products, for each open $U\subset X$ we get a map $\beta(U):\fF(U) \otimes_{\fO_X(U)} \fG(U)\to \fH(U)$ given by 
    \begin{center}
        \begin{tikzcd}
            \fF(U)\times \fG(U) \ar{r} \ar{dr}[swap]{\phi(U)} & \fF(U) \otimes_{\fO_X(U)} \fG(U) \ar[dashed]{d}[description]{\exists! \beta(U)}\\
            &\fH(U)
        \end{tikzcd}
    \end{center}
    by assumption that $\phi$ is $\fO_X$-bilinear. We claim that $\beta:(\fF \otimes_{\fO_X} \fG)_{\pre}\to \fH$ is a natural transformation. If $V\subset U$ is open, we see 
    \begin{align*}
        \left(\beta(U)(\sum f_i \otimes g_i)\right)\vert_V = \left( \sum \phi(U) (f_i, g_i)\right)\vert_V = \sum \phi(V)( f_i \vert_V, g_i \vert_V)=\sum \beta(V)(f_i \vert_V \otimes g_i \vert_V)=\beta(V)(\left( \sum f_i \otimes g_i\right)\vert_V)
    \end{align*}
    as desired. Because $\fH$ is a sheaf, we get a map $\alpha:\fF \otimes_{\fO_X} \fG \to \fH$ given below:
    \begin{center}
        \begin{tikzcd}
            (\fF \otimes_{\fO_X} \fG)_{\pre} \ar{r}{\sh} \ar{dr}[swap]{\beta}& \fF \otimes_{\fO_X} \fG \ar[dashed]{d}[description]{\exists! \alpha}\\
            &\fH
        \end{tikzcd}
    \end{center}
    By our constructions, this shows existence for our universal property. To show uniqueness, suppose we had another map $\psi:\fF \otimes_{\fO_X} \fG\to \fH$ through which $\phi$ factors. By hypothesis, for any open set $U$, we have
    \[
     \psi(U)([\sum f_i\otimes g_i, U]_p)_{p\in U}=\sum \psi(U)\left(([f_i\otimes g_i, U]_p)_{p\in U}\right)=\sum \phi(U)(f_i,g_i).
    \]
    Our approach in showing that $\alpha = \psi$ will be to show that for every $p\in X$, $\alpha_p=\psi_p$ which suffices by Exercise \ref{2.4.C}C. Fix $z\in U\subset X$, and let $[\left([\sum f_{i,p}\otimes g_{i,p},U_p]\right)_{p\in U},U]$ be an arbitrary element of $(\fF \otimes_{\fO_X} \fG)_z$. Because $\left([\sum_i f_{i,p}\otimes g_{i,p},U_p]_p\right)_{p\in U}$ is a compatible choice of germs, for each $p\in U$ there exists an open set $V_p$ containing $p$ and $\sum_i x_{i,p}\otimes y_{i,p}$ such that
    \[
    [\sum_i x_{i,p}\otimes y_{i,p}, V_p]_q=[\sum_i f_{i,q}\otimes g_{i,q},U_q]_q
    \]
    for all $q\in V_p$. Then
    \begin{align*}
        &\psi_z\left([([\sum f_{i,p}\otimes g_{i,p},U_p]_p)_{p\in U},U]\right)= \psi_z\left([([\sum f_{i,p}\otimes g_{i,p},U_p]_p)_{p\in V_z},V_z]\right)\\
        &=\psi_z\left([\sum x_{i,z}\otimes y_{i,z}, V_z]_p)_{p\in V_z},V_z]\right)=[\psi(V_z)\left(([\sum x_{i,z}\otimes y_{i,z}, V_z]_p)_{p\in V_z}\right),V_z]\\
        &= \sum[ \phi(V_z)(x_{i,z}, y_{i,z}), V_z].
    \end{align*}
    But since
    \[
     \alpha(U)([\sum f_i\otimes g_i, U]_p)_{p\in U}=\sum \alpha(U)\left(([f_i\otimes g_i, U]_p)_{p\in U}\right)=\sum \phi(U)(f_i,g_i)
    \]
    for any open set $U$ as well, we derive that 
    \[
    \alpha_z\left([([\sum f_{i,p}\otimes g_{i,p},U_p]_p)_{p\in U},U]\right)=\sum[ \phi(V_z)(x_{i,z}, y_{i,z}), V_z]
    \]
    as well, so $\alpha_z=\psi_z$ as desired.
    
    Lastly, we want to show that $(\fF \otimes_{\fO_X} \fG)_p \cong \fF_p \otimes_{\fO_{X,p}} \fG_p$. We can reinterpret the universal property defining $\fF \otimes_{\fO_X} \fG$ as the colimit indexed by the final category (the final object in the category $\Cat$) inside a category whose objects are pairs $(\fH, \phi)$ where $\fH$ is an $\fO_X$ module and $\phi:\fF\times \fG \to \fH$ is $\fO_X$-bilinear, and whose morphisms are maps of sheaves making the diagrams commute. Explicitly if $(\fA, \phi)$ and $(\fB, \psi)$ are objects of this category, then $\alpha:\fA \to \fB$ is a morphism of our category whenever $\alpha \circ \phi = \psi$. In the language of category theory, our category is the coslice category of  $\Mod_{\fO_X}$ over $\fF\times \fG$. By the dual of Exercise \ref{1.6.K}K or Vakil (1.6.14), colimits commute with colimits, and as taking the stalk at $p\in X$ is a colimit, we get our result.
\end{proof}
\subsection{}
\subsubsection{A}\label{2.7.A}
\begin{proof}
    If $U\subset U'\subset U''$ are all open, we need to show existence of restriction maps so that
    \begin{center}
        \begin{tikzcd}
            \pi^{-1}_{\pre} \fG(U'') \ar{rr} \ar{dr}&& \pi^{-1}_{\pre} \fG(U') \ar{dl}\\
            & \pi^{-1}_{\pre} \fG(U)
        \end{tikzcd}
    \end{center}
    commutes. First of all, for arbitrary open $V\subset U$, we have $\pi(V)\subset \pi(U)$, hence every $W\in \Op(Y)$ containing $\pi(U)$ also contains $\pi(V)$. In other words, we have the below commutative diagram:
    \begin{center}
        \begin{tikzcd}
            & \colim_{W\supset \pi(V)} \fG(W)\\
            \\
            \fG(W) \ar{uur}&\colim_{W\supset \pi(U)} \fG(W) \ar[dashed]{uu}[description]{\exists! \res_{U,V}}&\fG(W') \ar{uul}\\
            \fG(W) \ar{rr} \ar{ur} \ar[bend left]{u}{\id} && \fG(W') \ar{ul} \ar[bend right]{u}[swap]{\id}.
        \end{tikzcd}
    \end{center}
    The uniqueness of the restriction maps ensure that $\res_{U',U} \circ \res_{U'',U'} = \res_{U'',U}$ as both maps satisfy the unique arrow below:
    \begin{center}
        \begin{tikzcd}
            & \colim_{W\supset \pi(U)} \fG(W)\\
            \\
            \fG(W) \ar{uur}&\colim_{W\supset \pi(U'')} \fG(W) \ar[dashed]{uu}[description]{\exists!}&\fG(W') \ar{uul}\\
            \fG(W) \ar{rr} \ar{ur} \ar[bend left]{u}{\id} && \fG(W') \ar{ul} \ar[bend right]{u}[swap]{\id}.
        \end{tikzcd}
    \end{center}
    Lastly, we need to show that $\pi^{-1}_{\pre}$ preserves identity maps: this is clear by uniqueness of the restriction maps, since the identity makes the diagram commute.

    To see that $\pi^{-1}_{\pre}$ need not be a sheaf, let $X=\C$, and $Y=\{*\}$, $\fG = \underline \Z $ be the sheaf associating $\Z$ to $\{*\}$, and let $\pi:\C \to \{*\}$. Then for any nonempty open $U\subset \C$, the only open $V$ containing $\pi(U)$ is $V=\{*\}$, so $\pi^{-1}_{\pre}(U) = \Z$ (also $\pi^{-1}_{\pre}(\emptyset)=0$). In addition, the restriction maps are all the identity. If we take disjoint open sets $U,V$ in $\C$ and let $0,1$ be sections of $\pi^{-1}_{\pre}$ over $U,V$ respectively, we try to glue $0$ and $1$ together on $U\sqcup V$ (which we should be able to do if $\pi^{-1}_{\pre}$ were a sheaf). If $n\in \Z$ was the glued section, then it would restrict to $0$ and $1$ on $U$ and $V$ respectively. However, as was shown earlier, the restriction maps are the identity, so $n$ would have to simultaneously be equal to $0$ and $1$, impossible.
\end{proof}
\subsubsection{B}\label{2.7.B}
\begin{proof}
    Following the notation in the hint, we will show each hom-set is in bijective correspondence with $\Mor_{YX}(\fG,\fF)$. First, we note that for every open set $U\subset X$ and $V\subset Y$, $\pi(\pi^{-1}(V))\subset V$ and $\pi^{-1}(\pi(U))\supset U$, two facts we will use repeatedly throughout the proof.
    
    Fix $\psi:\fG\to \pi_* \fF$, and for each open $U\subset X$ and $V\supset \pi(U)$, we let $\phi_{VU} = \res_{\pi^{-1}(V),U} \circ \psi(V)$. We claim our defined set of $\phi_{VU}$'s are natural, hence define an element of $\Mor_{YX}(\fG,\fF)$. To show this, fix open $V'\subset V\subset Y$ and $U'\subset U\subset X$ such that $V\supset \pi(U)$ and $V'\supset \pi(U')$. We want to show the below diagram commutes:
    \begin{center}
        \begin{tikzcd}
            \fG(V) \ar{r}{\phi_{VU}} \ar{d}{\res_{V,V'}}& \fF(U) \ar{d}{\res_{U,U'}}\\
            \fG(V') \ar{r}{\phi_{V'U'}}& \fF(U').
        \end{tikzcd}
    \end{center}
    To see this, fix $x\in \fG(V)$. Then 
    \begin{align*}
         \phi_{VU}(x)\vert_{U'} = \res_{\pi^{-1}(V), U'} \circ \psi(V) (x)=\res_{\pi^{-1}(V'),U'}\circ \psi(V')(x\vert_{V'})=\phi_{V'U'}(x\vert_{V'})
    \end{align*}
    as desired. Thus we have a map $\alpha:\Mor(\fG,\pi_* \fF)\to \Mor_{YX}(\fG,\fF)$ given by our above construction. To show $\alpha$ is injective, suppose $\alpha(\psi) = \{\phi_{VU}\} = \alpha(\psi')$. Then for any open $U\subset X$ and $V\subset Y$ with $V\supset \pi(U)$, we have $\psi(V)(x)\vert_U = \psi'(V)(x)\vert_U$ for any $x\in \fG(V)$. Letting $p\in Y$ be arbitrary, $\psi_p=\psi'_p$ because if we take any germ $[x,V]$, we let $U=\pi^{-1}(V)$ so that $V\supset \pi(U)$. We then compute
    \[
    \psi_p[x,V] = [\psi(V)(x), V] = [\psi(V)(x)\vert_U,U]=[\psi'(V)(x)\vert_U,U]=[\psi'(V)(x),V]=\psi'_p[x,V].
    \]
    Then by Exercise \ref{2.4.C}C, we see $\psi=\psi'$ as desired. Now we claim $\alpha$ is surjective. Fix $\{\phi_{VU}\} \in \Mor_{YX}(\fG,\fF)$. For each open $V\subset Y$, we let $U=\pi^{-1}(V)$ so $V\supset \pi(U)$, and then define $\psi(V)=\phi_{VU}:\fG(V)\to \fF(U)=\pi_*\fF(V)$. We now claim $\psi$ is a map of sheaves. For any open $V'\subset V\subset Y$, we let $U=\pi^{-1}(V)$ and $U'=\pi^{-1}(V')$, so the following diagram commutes by naturality of $\{\phi_{VU}\}$:
    \begin{center}
        \begin{tikzcd}
            \fG(V) \ar{r}{\psi(V)} \ar{d}{\res_{V,V'}}& \pi_* \fF(V) \ar{d}{\res_{V,V'}}\\
            \fG(V') \ar{r}{\psi(V')} & \pi_* \fF(V').
        \end{tikzcd}
    \end{center}
    Thus $\alpha$ is indeed a bijection.

    Now fix $\phi =\{\phi_{VU}\} \in \Mor_{YX}(\fG,\fF)$. We will show there exists a unique element of $\Mor(\pi^{-1}_{\pre} \fG,\fF)$ corresponding to $\phi$. Fix open $U\subset X$, so for each open $V\supset V'\supset \pi(U)$ we get the below commutative diagram:
    \begin{center}
        \begin{tikzcd}
            &&\fF(U) \\
            \\
            \pi_*\fF(V) \ar{uurr} \arrow[bend left=10, crossing over, pos=0.4]{rrrr}{\res_{V,V'}}&&\pi^{-1}_{\pre}\fG(U) \ar[dashed]{uu}[description]{\exists!\varphi(U)}&& \pi_*\fF(V') \ar{uull}\\
            &\fG(V) \ar{rr}{\res_{V,V'}} \ar{ul}{\phi_{V,\pi^{-1}(V)}} \ar{ur}{\mu_V}&&\fF(V') \ar{ul}[swap]{\mu_{V'}} \ar{ur}[swap]{\phi_{V',\pi^{-1(V')}}}.
        \end{tikzcd}
    \end{center}
    Indeed, the $\varphi(U)'s$ are natural: if $U'\subset U$, then both $\res_{U,U'}\circ \varphi(U)$ and $\varphi(U')\circ \res_{U,U'}$ satisfy the unique arrow in the below commutative diagram:
    \begin{center}
        \begin{tikzcd}
            &&\fF(U) \\
            \\
            \pi_*\fF(V) \ar{uurr} \arrow[bend left=10, crossing over, pos=0.4]{rrrr}{\res_{V,V'}}&&\pi^{-1}_{\pre}\fG(U) \ar[dashed]{uu}[description]{\exists!}&& \pi_*\fF(V') \ar{uull}\\
            &\fG(V) \ar{rr}{\res_{V,V'}} \ar{ul}{\phi_{V,\pi^{-1}(V)}} \ar{ur}{\mu_V}&&\fF(V') \ar{ul}[swap]{\mu_{V'}} \ar{ur}[swap]{\phi_{V',\pi^{-1(V')}}}.
        \end{tikzcd}
    \end{center}
    Thus we have a unique map of presheaves $\varphi$ (uniqueness is because each map of sections is uniquely determined), which induces a unique $\psi:\pi^{-1}\fG \to \fF$ by the universal property of sheafification. We let $\beta(\phi) = \psi$, so by uniqueness $\beta$ is injective. For surjectivity, fix $\psi:\pi^{-1} \fG\to \fF$. By precomposing with $\sh:\pi^{-1}_{\pre}\fG \to \pi^{-1} \fG$, we get a collection of $\phi_{VU}$'s by $\phi_{VU}=\psi(U)\circ \sh(U) \circ \tau_V$ where $\mu_V:\fG(V)\to \pi^{-1}_{\pre} \fG(U)$. These $\phi_{VU}$'s define an element $\phi \in \Mor_{YX}(\fG,\fF)$ by naturality of $\psi$, and we will now see that $\beta(\phi)=\psi$. This is because on the level of sections, the unique arrow $\varphi(U)$ is satisfied by $\psi(U)\circ \sh(U)$, so $\varphi = \psi \circ \sh$, i.e. $\beta(\phi)=\psi$.

    We now need to show the bijections $\tau = \beta \circ \alpha$ are functorial. First, let $\phi:\fH \to \fG$ be a map of sheaves, and we need to show
    \begin{center}
        \begin{tikzcd}
            \Mor(\fG, \pi_* \fF) \ar{d}{\tau_{\fG \fF}} \ar{r}{\phi^*}& \Mor(\fH,\pi_*\fF) \ar{d}{\tau_{\fH\fF}}\\
            \Mor(\pi^{-1} \fG, \fF) \ar{r}{(\pi^{-1}\phi)^*}& \Mor(\pi^{-1} \fH, \fF)
        \end{tikzcd}
    \end{center}
    commutes. We fix $\psi:\fG \to \pi_*\fF$, and notice that the below four commutative diagrams summarize all of the constructions in play for any open $U\subset X$ with $\pi(U)\subset V'\subset V$:
    \begin{center}
        \begin{tikzcd}
            &&\fF(U) \\
            \\
            \pi_*\fF(V) \ar{uurr}{\res} &&\pi^{-1}_{\pre}\fG(U) \ar{uu}[description]{\tilde \psi(U)}&& \pi_*\fF(V') \ar{uull}[swap]{\res}\\
            &\fG(V) \ar{rr}{\res_{V,V'}} \ar{ul}{\psi(V)} \ar{ur}{\mu_V}&&\fG(V) \ar{ul}[swap]{\mu_{V'}} \ar{ur}[swap]{\psi(V')}.
        \end{tikzcd}
        \begin{tikzcd}
            \pi^{-1}_{\pre} \fG \ar{r}{\sh} \ar{dr}[swap]{\tilde \psi}& \pi^{-1} \fG \ar{d}[description]{\tau_{\fG \fF}(\psi)}\\
            & \fF
        \end{tikzcd}
        \begin{tikzcd}
            &&\pi^{-1}_{\pre} \fG(U) \\
            \\
            \fG(V) \ar{uurr}{\mu_V} \arrow[bend left=10, crossing over, pos=0.4]{rrrr}{\res_{V,V'}}&&\pi^{-1}_{\pre}\fH(U) \ar{uu}[description]{\pi^{-1}_{\pre} \phi}&& \fG(V') \ar{uull}[swap]{\mu_{V'}}\\
            &\fH(V) \ar{rr}{\res_{V,V'}} \ar{ul}{\phi(V)} \ar{ur}{\sigma_V}&&\fH(V) \ar{ul}[swap]{\sigma_{V'}} \ar{ur}[swap]{\phi(V')}.
        \end{tikzcd}
        \begin{tikzcd}
            \pi^{-1}_{\pre} \fH \ar{d}{\pi^{-1}_{\pre} \phi} \ar{r}{\sh}& \pi^{-1} \fH \ar{d}[description]{\pi^{-1}\phi}\\
            \pi^{-1}_{\pre} \fG \ar{r}{\sh}& \pi^{-1} \fG.
        \end{tikzcd}
    \end{center}
    Our strategy in showing commutativity will be checking equality on the level of stalks, which suffices by Exercise \ref{2.4.C}C. For $p\in Y$, we want to show
    \[
    \tau_{\fG \fF}(\psi)_p \circ (\pi^{-1} \phi)_p \circ \sh^{\pi^{-1}_{\pre}\fH}_p = \tau_{\fH \fF}(\psi \circ \phi)_p \circ \sh_p^{\pi^{-1}_{\pre}\fH}
    \]
    because $\sh_p^{\pi^{-1}_{\pre}\fH}$ is an isomorphism by Exercise \ref{2.4.L}L.
    For the left-hand side, we compute for an arbitrary germ $[\sigma_V(x),U]$ that
    \begin{align*}
        &\tau_{\fG \fF}(\psi)_p \circ (\pi^{-1} \phi)_p \circ \sh^{\pi^{-1}_{\pre}\fH}_p ([\sigma_V(x),U])= \tau_{\fG \fF} (\psi)_p \circ \sh_p^{\pi^{-1}_{\pre} \fG} \circ (\pi^{-1}_{\pre} \phi)_p([\sigma_V(x),U])\\
        &=\tilde \psi_p ([\mu_V \circ \phi(V)(x),U]) = [\psi(V)\circ \phi(V)(x)\vert_U,U].
    \end{align*}
    For the right-hand side, we compute
    \begin{align*}
        \tau_{\fH \fF}(\psi \circ \phi)_p \circ \sh_p^{\pi^{-1}_{\pre}\fH} ([\sigma_V(x),U])=\widetilde{\psi \circ \phi}_p([\sigma_V(x),U])=[\psi(V)\circ \phi(V)\vert_U,U].
    \end{align*}

    Now let $\varphi:\fF\to \fH$ be a map of sheaves, so we need to show the below diagram commutes:
    \begin{center}
        \begin{tikzcd}
            \Mor(\fG, \pi_*\fF) \ar{d}{\tau_{\fG \fF}} \ar{r}{(\pi_* \varphi)_*}& \Mor(\fG, \pi_*\fH) \ar{d}{\tau_{\fG \fH}}\\
            \Mor(\pi^{-1} \fG, \fF) \ar{r}{\varphi_*}& \Mor(\pi^{-1} \fG, \fH).
        \end{tikzcd}
    \end{center}
    Fix $\psi:\fG \to \pi_* \fF$. In a similar vein, we want to show $\tau_{\fG \fH} (\pi_* \varphi \circ \psi) = \varphi \circ \tau_{\fG \fF}(\psi)$ by checking stalks. Again, it suffices to show
    \[
    \tau_{\fG \fH} (\pi_* \varphi \circ \psi)_p \circ \sh^{\pi^{-1}_{\pre} \fG}_p = \varphi_p \circ \tau_{\fG \fF}(\psi)_p \circ \sh^{\pi^{-1}_{\pre} \fG}_p.
    \]
    For an arbitrary germ $[\mu_V(x),U]$, we compute that
    \begin{align*}
        &\tau_{\fG \fH} (\pi_* \varphi \circ \psi)_p \circ \sh^{\pi^{-1}_{\pre} \fG}_p([\mu_V(x),U])=\widetilde{\pi_* \varphi \circ \psi}_p([\mu_V(x),U])=[\pi_*\varphi(V)\circ \psi(V)(x)\vert_U,U]\\
        &=[\varphi(\pi^{-1}(V)) \circ \psi(V)(x)\vert_U,U]=\varphi_p\circ \tau_{\fG \fF}(\psi)_p\circ \sh_p^{\pi^{-1}_{\pre}\fG}[\mu_V(x),U]=\varphi_p \circ \tilde \psi_p([\mu_V(x),U])\\
        &= \varphi_p([\varphi(V)(x)\vert_U])=[\varphi(U)\circ \res_{\pi^{-1}(V),U}\circ \psi(V)(x),U]=\varphi_p([\psi(V)\vert_U,U])\\
        &=\varphi_p\circ \tilde \psi_p([\mu_V(x),U])=\varphi_p \circ \tau_{\fG \fF}(\psi)_p \circ \sh^{\pi^{-1}_{\pre} \fG}_p([\mu_V(x),U]).
    \end{align*}
\end{proof}
\begin{lemma} \label{lem:pushforward sheaf distributes}
        If $X\xrightarrow{f}Y\xrightarrow{g}Z$, then $(gf)_* = g_*f_*$ as functors.
    \end{lemma}
    \begin{proof}
        This is clear: if $\fF$ is a sheaf over $X$ and $U\subset Z$ is open, then
        \[
        g_*f_*\fF(U)=f_* \fF(g^{-1}(U))=\fF(f^{-1}(g^{-1}(U)))
        \]
        while simultaneously
        \[
        (gf)_*\fF(U)=\fF((gf)^{-1}(U))=\fF(f^{-1}(g^{-1}(U))).
        \]
    \end{proof}
    \begin{lemma} \label{lem:inverse image sheaf distributes}
        If $X\xrightarrow{f}Y\xrightarrow{g}Z$, then $(gf)^{-1} = f^{-1}g^{-1}$ as functors.
    \end{lemma}
    \begin{proof}
        Recall that being an adjoint defines a functor up to natural isomorphism. In particular, the left adjoint of $(gf)_*$ is defined up to natural isomorphism. By Exercise \ref{2.7.B}B, $(gf)^{-1}$ is a left-adjoint, so to prove the claim it suffices to show $f^{-1}g^{-1}$ is also a left-adjoint of $(gf)_*$. Again by Exercise \ref{2.7.B}B, we have functorial bijections
        \[
        \Mor(f^{-1}g^{-1}\fH, \fF) \xrightarrow{\sim} \Mor(g^{-1} \fH, f_* \fF) \xrightarrow{\sim } \Mor(\fH, g_* f_* \fF)
        \]
        for arbitrary sheaves $\fH$ over $Z$ and $\fF$ over $X$. By Lemma \ref{lem:pushforward sheaf distributes}, $\Mor(\fH,g_*f_*\fF)= \Mor(\fH, (gf)_* \fF)$ which completes the claim.
    \end{proof}
    \begin{lemma}\label{lem:stalk as inverse image sheaf}
        Let $i:\{*\}\hookrightarrow X$ have image $p$, and let $\fF$ be a sheaf over $X$. Then $i^{-1} \fF(\{*\})=\fF_p$.
    \end{lemma}
    \begin{proof}
        By definition, $i^{-1}_{\pre}(\{*\})=\colim_{V\supset \{p\}} \fF(V) = \fF_p$. Then it just remains to show $i^{-1} \fF(\{*\}) = \fF_p$, i.e. the set of all compatible germs over $\{*\}$ is just $\fF_p$. Since $\{*\}$ is a single point, a choice of compatible germs is just a single germ at $*$. But $(i^{-1}_{\pre} \fF)_*=\colim_{*\in V} i^{-1}_{\pre} \fF(V)= i^{-1}_{\pre} \fF(\{*\})= \fF_p$ as desired. 
    \end{proof}
\subsubsection{C}\label{2.7.C}
\begin{proof}
    Fix $p\in X$ and let $q=\pi(p)$, and choose $\fG$ to be a sheaf over $Y$. We then have the chain of continuous maps $$\{p\} \xhookrightarrow{i}X \xrightarrow{\pi}Y.$$ By Lemma \ref{lem:inverse image sheaf distributes}, we get $(\pi  i)^{-1} \fG \cong i^{-1} \pi^{-1} \fG$. In particular, $(\pi i)^{-1} \fG(\{p\}) \cong i^{-1} \pi^{-1} \fG(\{p\})$. We notice $\pi i$ has image $q$ and $i$ has image $p$, so we apply Lemma \ref{lem:stalk as inverse image sheaf} to get that $(\pi i)^{-1} \fG(\{p\})=\fG_q$, whereas $i^{-1} \pi^{-1} \fG(\{p\}) = (\pi^{-1} \fG)_p$ as required.
\end{proof}
\subsubsection{D}\label{2.7.D}
\begin{proof}
\iffalse
    Because the left-adjoint of $i_*$ is defined up to isomorphism, it suffices to show $(\vert_U, i_*)$ is an adjoint pair. Notice that for any open subset $V\subset Y$, $i^{-1}(V)=U\cap V$. On one hand, if we're given some $\phi \in \Mor(\fG\vert_U, \fF)$, we define $\tilde \phi:\fG \to i_* \fF$ be given by $\tilde \phi(V)=\phi(U\cap V)$ for any open $V\subset Y$. \fi
    We will show that $\fG\vert_U = i^{-1}_{\pre} \fG$, which completes the result because the restriction of a sheaf is again a sheaf. We can do this by showing $\fG\vert_U(V) = \colim_{W\supset i(V)} \fG(W) = \colim_{W\supset V} \fG(W)$ for an arbitrary open $V\subset U$, where the index is over all open $V\subset W\subset U$. By definition, $\fG\vert_U(V)=\fG(V)$. But indeed because $\fG(V)$ is in the index and every $\fG(W)$ has a unique restriction map to $\fG(V)$, we can easily see our requirement
    \begin{center}
        \begin{tikzcd}
            &&A\\
            &&\fG(V) \ar[dashed]{u}[description]{\exists!}\\
            \fG(V) \ar{urr}{\id} \ar[bend left]{uurr} &&&& \fG(W) \ar{llll}{\res} \ar{ull}[swap]{\res} \ar[bend right]{uull}
        \end{tikzcd}
    \end{center}
    is satisfied. It's also easy to see the induced restriction maps are the same as those of $\fG\vert_U$.
\end{proof}
\subsubsection{E}\label{2.7.E}
\begin{proof}
    Suppose $0\to \fF \to \fG \to \fH \to 0$ is an exact sequence of sheaves. By Exercise \ref{2.6.E}E, $0\to \fF_q \to \fG_q\to \fH_q\to 0$ is exact. Then by Exercise \ref{2.7.C}C, $0\to (\pi^{-1} \fF)_p \to (\pi^{-1}\fG)_p\to (\pi^{-1} \fH)_p \to 0$ is exact. Then Exercise \ref{2.6.D}D gives that, since $p\in X$ was arbitrary and $q=\pi(p)$, that indeed $0\to \pi^{-1} \fF\to \pi^{-1} \fG \to \pi^{-1} \fH \to 0$ is exact.
\end{proof}
\begin{setting}\label{set:Adjoint functors into 2-category}
    Let $\fA$ be a category and $\fB$ be a 2-category. Assume $R:\fA \to \fB$ and $L:\fA^{\op} \to \fB$ are functors such that for each $f:X\to Y$ in $\fC$, $LX=RX$ and $(L_f, R_f)$ is an adjoint pair, i.e. there are a 2-morphisms $\eta^f:\id_Y \Rightarrow R_fL_f$ and $\epsilon^f:L_fR_f \Rightarrow \id_X$ such that $R_f \epsilon^f \circ \eta^f R_f = \id_{R_f}$ and $\epsilon^f L_f \circ L_f \eta^f = \id_{L_f}$ where a 1-morphism next to a 2-morphism denotes whiskering. We let $\circ$ denote vertical composition and $*$ denote horizontal composition of 2-morphisms. We slightly abuse notation and omit explicit notation denoting composition of 1-morphisms. However, whiskering is still distinguishable from composition of 1-morphisms. Finally, $\altid_M$ denotes the identity 2-morphism of $\id_M$ for an object $M\in \fB$.
\end{setting}
\begin{lemma}\label{lem:unit counit commute}
    As in Setting \ref{set:Adjoint functors into 2-category}, if $X\xrightarrow{f} Y \xrightarrow{g} Z$ in $\fC$, the below diagrams of 2-morphisms commute:
    \begin{center}
        \begin{tikzcd}
            L_g R_g \ar[Rightarrow]{d}{\epsilon^g} \ar[Rightarrow]{r}{\eta^f L_g R_g}& R_f L_f L_g R_g \ar[Rightarrow]{d}{R_f L_f \epsilon^g}\\
            \id_{LY} \ar[Rightarrow]{r}{\eta^f}& R_f L_f
        \end{tikzcd}
    \end{center}
    \begin{center}
        \begin{tikzcd}
            L_g R_g \ar[Rightarrow]{d}{\epsilon^g} \ar[Rightarrow]{r}{L_g R_g\eta^f}& R_f L_f L_g R_g \ar[Rightarrow]{d}{ \epsilon^gR_f L_f}\\
            \id_{LY} \ar[Rightarrow]{r}{\eta^f}& R_f L_f.
        \end{tikzcd}
    \end{center}
    \end{lemma}
    \begin{proof}
        We have
        \begin{align*}
            &\eta^f \circ \epsilon^g =(\eta^f *\altid_{LY})\circ (\altid_{LY}*\epsilon^g)=(\eta^f \circ \altid_{LY})*(\altid_{LY}\circ \epsilon^g)=(\id_{R_f L_f}\circ \eta^f)*(\epsilon^g \circ \id_{L_g R_g})\\
            &= R_fL_f \epsilon^g \circ \eta^f L_g R_g
        \end{align*}
        because horizontal composition by $\altid$ does nothing, as does vertical composition by identities by \cite{maclane1998categories}. Commutativity of the second diagram is analogous. To see diagrammatically what we are doing for commutativity of the second diagram, we observe
        \begin{center}
    \begin{tikzcd}
        LY \ar[rr, ""{name=U1, below}]{}{\id}&& LY \ar{r}{R_g}&LZ \ar{r}{L_g} \ar[Rightarrow]{d}{\id}&LY\\
        LY \ar{r}{L_f}&LX \ar[Rightarrow]{d}{\id} \ar{r}{R_f} \arrow[Rightarrow, from=U1, "\eta^f"]&LY \ar{r}{R_g}\ar[phantom, ""{name=D1, below}]{rr}{}& LZ  \ar{r}{L_g} & LY \\
        LY \ar{r}{L_f}&LX \ar{r}{R_f} &LY \ar[rr, ""{name=D, above = 1.5mm}]{}{\id}&&LY \arrow[Rightarrow, from=D1, to=D, "\epsilon^g"]
    \end{tikzcd}
\end{center}
is the same as
\begin{center}
    \begin{tikzcd}
        LY \ar[rr, ""{name=U1, below}]{}{\id}&& LY && LY\ar{r}{R_g}&LZ \ar{r}{L_g} \ar[Rightarrow]{d}{\id}&LY\\
        LY \ar{r}{L_f}&LX \ar[Rightarrow]{d}{\id} \ar{r}{R_f} \arrow[Rightarrow, from=U1, "\eta^f"]&LY &*& LY\ar{r}{R_g}\ar[phantom, ""{name=D1, below}]{rr}{}& LZ  \ar{r}{L_g} & LY \\
        LY \ar{r}{L_f}&LX \ar{r}{R_f} &LY && LY \ar[rr, ""{name=D, above = 1.5mm}]{}{\id}&&LY \arrow[Rightarrow, from=D1, to=D, "\epsilon^g"]
    \end{tikzcd}
\end{center}
is the same as
\begin{center}
    \begin{tikzcd}
        LY \ar[rr, ""{name=U2, below}]{}{\id}&& LY && LY\ar{r}{R_g}\ar[phantom, ""{name=D1, below}]{rr}{}& LZ  \ar{r}{L_g} & LY \\
        LY \ar[phantom, rr, ""{name=U3, above=1.5mm}]{}\ar[rr, ""{name=U1, below}]{}{\id}&& LY&*& LY \ar[rr, ""{name=D, above = 1.5mm}]{}{\id} \ar[phantom, rr, ""{name=D3, below }]{}&&LY \\
        LY \ar{r}{L_f}&LX  \ar{r}{R_f} \arrow[Rightarrow, from=U1, "\eta^f"]&LY  && LY \ar[rr, ""{name=D2, above = 1.5mm}]{}{\id}&& LY \arrow[Rightarrow, from=D1, to=D, "\epsilon^g"] \ar[Rightarrow, from = U2, to=U3, "\altid_{LY}"] \ar[Rightarrow, from = D3, to= D2, "\altid_{LY}"]
    \end{tikzcd}
\end{center}
is the same as
\begin{center}
    \begin{tikzcd}
        LY \ar[rr, ""{name=U2, below}]{}{\id}&&  LY\ar{r}{R_g}\ar[phantom, ""{name=D1, below}]{rr}{}& LZ  \ar{r}{L_g} & LY \\
        LY \ar[phantom, rr, ""{name=U3, above=1.5mm}]{}\ar[rr, ""{name=U1, below}]{}{\id}&&  LY \ar[rr, ""{name=D, above = 1.5mm}]{}{\id} \ar[phantom, rr, ""{name=D3, below }]{}&&LY \\
        LY \ar{r}{L_f}&LX  \ar{r}{R_f} \arrow[Rightarrow, from=U1, "\eta^f"]&LY  \ar[rr, ""{name=D2, above = 1.5mm}]{}{\id}&& LY \arrow[Rightarrow, from=D1, to=D, "\epsilon^g"] \ar[Rightarrow, from = U2, to=U3, "\altid_{LY}"] \ar[Rightarrow, from = D3, to= D2, "\altid_{LY}"]
    \end{tikzcd}
\end{center}
is the same as
\begin{center}
    \begin{tikzcd}
        LY \ar[rr, ""{name=U2, below}]{}{\id}&&  LY\ar{r}{R_g}\ar[phantom, ""{name=D4, below}]{rr}{}& LZ  \ar{r}{L_g} & LY \\
        LY \ar[phantom, rr, ""{name=U3, above=1.5mm}]{}\ar[rr, ""{name=U1, below}]{}{\id}&&  LY \ar[rr, ""{name=D5, above = 1.5mm}]{}{\id} \ar[phantom, rr, ""{name=D3, below }]{}&&LY \\
        &&\circ \ar[Rightarrow, from = U2, to=U3, "\altid_{LY}"]  \\
        LY \ar[phantom, rr, ""{name=U3, above=1.5mm}]{}\ar[rr, ""{name=U1, below}]{}{\id}&&  LY \ar[rr, ""{name=D, above = 1.5mm}]{}{\id} \ar[phantom, rr, ""{name=D3, below }]{}&&LY \\
        LY \ar{r}{L_f}&LX  \ar{r}{R_f} \arrow[Rightarrow, from=U1, "\eta^f"]&LY  \ar[rr, ""{name=D2, above = 1.5mm}]{}{\id}&& LY \arrow[Rightarrow, from=D4, to=D5, "\epsilon^g"]  \ar[Rightarrow, from = D3, to= D2, "\altid_{LY}"]
    \end{tikzcd}
\end{center}
is the same as
\begin{center}
    \begin{tikzcd}
        LY\ar{r}{R_g}\ar[phantom, ""{name=D4, below}]{rr}{}& LZ  \ar{r}{L_g} & LY \\
        LY \ar[rr, ""{name=D5, above = 1.5mm}]{}{\id} \ar[phantom, rr, ""{name=D3, below }]{}&&LY \\
        &\circ   \\
        LY \ar[phantom, rr, ""{name=U3, above=1.5mm}]{}\ar[rr, ""{name=U1, below}]{}{\id}&&  LY  \\
        LY \ar{r}{L_f}&LX  \ar{r}{R_f} \arrow[Rightarrow, from=U1, "\eta^f"]&LY  \arrow[Rightarrow, from=D4, to=D5, "\epsilon^g"] 
    \end{tikzcd}
\end{center}
which is simply $\eta^f \circ \epsilon^g$.
    \end{proof}
    \begin{lemma}\label{lem:unit counit distribute}
        As in Setting \ref{set:Adjoint functors into 2-category}, if $X\xrightarrow{f} Y \xrightarrow{g} Z$ in $\fC$, then $\epsilon^{gf} = \epsilon^f\circ L_f \epsilon^g R_f$ and $\eta^{gf}=R_g \eta^f L_g \circ \eta^g$.
    \end{lemma}
    \begin{proof}
        By uniqueness of the unit and counit, it suffices to show that $\epsilon^f\circ L_f \epsilon^g R_f$ (as a counit $\epsilon$)  and $R_g \eta^f L_g \circ \eta^g$ (as a unit $\eta$) satisfy the triangle identities, namely $\epsilon L_{gf} \circ L_{gf} \eta = \id_{L_{gf}}$ and $R_{gf} \epsilon \circ \eta R_{gf} = \id_{R_{gf}}$. Using Lemma \ref{lem:unit counit commute}, we see
        \begin{align*}
            &L_f \epsilon^g R_f L_{gf}\circ L_{gf}R_g \eta^f L_g= L_f \epsilon^g R_f L_f L_g \circ L_f L_g R_g \eta^f L_g\\
            &=L_f(\epsilon^g R_f L_f \circ L_g R_g \eta^f)L_g = L_f(\eta^f \circ \epsilon^g) L_g=L_f \eta^f L_g \circ L_f \epsilon^g L_g.
        \end{align*}
        Using this observation, we compute
        \begin{align*}
            &(\epsilon^f\circ L_f \epsilon^g R_f) L_{gf} \circ L_{gf} (R_g \eta^f L_g \circ \eta^g)=\epsilon^fL_{gf}\circ L_f \epsilon^g R_f L_{gf} \circ L_{gf} R_g \eta^f L_g \circ L_{gf}\eta^g\\
            &=\epsilon^fL_{f}L_g\circ L_f \eta^f L_g \circ L_f \epsilon^g L_g \circ L_{f}L_g\eta^g=(\epsilon^f L_f \circ L_f \eta^f)L_g \circ L_f(\epsilon^gL_g \circ L_g \eta^g)\\
            &=(\id_{L_f})L_g\circ L_f(\id_{L_g})=\id_{L_{gf}}.
        \end{align*}
        Again using Lemma \ref{lem:unit counit commute}, we observe
        \[
         R_gR_f L_f \epsilon^g R_f \circ R_g \eta^fL_g R_g R_f = R_g(R_fL_f \epsilon^g\circ \eta^fL_g R_g)R_f=R_g(\eta^f \circ \epsilon^g)R_f=R_g \eta^f R_f \circ R_g \epsilon^g R_f
        \]
       Using this, we compute that on the other hand,
        \begin{align*}
            &R_{gf} (\epsilon^f \circ L_f \epsilon^g R_f)\circ (R_g \eta^f L_g \circ \eta^g)R_{gf}=R_gR_f \epsilon^f \circ R_gR_f L_f \epsilon^g R_f \circ R_g \eta^fL_g R_g R_f \circ \eta^g R_g R_f\\
            &=R_gR_f \epsilon^f \circ R_g \eta^f R_f \circ R_g \epsilon^g R_f \circ \eta^g R_g R_f=R_g(R_f \epsilon^f \circ \eta^f R_f) \circ (R_g \epsilon^g \circ \eta^g R_g)R_f\\
            &= R_g(\id_{R_f}) \circ (\id_{R_g})R_f = \id_{R_{gf}}.
        \end{align*}
    \end{proof}
    \begin{corollary}\label{cor:swap counits}
        As in Setting \ref{set:Adjoint functors into 2-category}, if $\beta \alpha' = \alpha \beta':W\to Z$ in $\fC$, the below diagram of 2-morphisms commutes in $\fB$:
        \begin{center}
            \begin{tikzcd}
                L_{\beta \alpha'} R_{\beta \alpha'} \ar[Rightarrow]{r}{L_{\alpha'} \epsilon^\beta R_{\alpha'}} \ar[Rightarrow]{d}[swap]{L_{\beta'} \epsilon^\alpha R_{\beta'}}& L_{\alpha'}R_{\alpha'} \ar[Rightarrow]{d}{\epsilon^{\alpha'}}\\
                L_{\beta'} R_{\beta'} \ar[Rightarrow]{r}{\epsilon^{\beta'}}& \id_{LW}.
            \end{tikzcd}
        \end{center}
\end{corollary}
\begin{proof}
Immediate from Lemma \ref{lem:unit counit distribute}.
\end{proof}
\begin{lemma}\label{lem:unit counit distribute with identity in middle}
    As in Setting \ref{set:Adjoint functors into 2-category}, if $\beta \alpha' = \alpha \beta':W\to Z$ in $\fC$, the below diagram of 2-morphisms commutes in $\fB$:
        \begin{center}
            \begin{tikzcd}
                L_{\beta } R_{\alpha} \ar[Rightarrow]{r}{L_{\beta} R_\alpha \eta^{\beta'}} \ar[Rightarrow]{d}[swap]{\eta^{\alpha'}L_{\beta} R_\alpha  }& L_\beta R_{\alpha \beta'} L_{\beta'} \ar[Rightarrow]{d}{\eta^{\alpha'}L_\beta R_{\beta \alpha'}L_{\beta'}}\\
                R_{\alpha'}L_{\beta \alpha'} R_\alpha \ar[Rightarrow]{r}[swap]{R_{\alpha'}L_{\beta \alpha'}R_{\alpha}\eta^{\beta'}}& R_{\alpha'} L_{\beta \alpha'} R_{\beta \alpha'} L_{\beta'}.
            \end{tikzcd}
        \end{center}
\end{lemma}
\begin{proof}
    In a similar vein as the proof of Lemma \ref{lem:unit counit commute}, we compute
    \begin{align*}
        &\eta^{\alpha'}L_\beta R_\alpha R_{\beta'}L_{\beta'}\circ L_{\beta} R_\alpha  \eta^{\beta'}=(\eta^{\alpha'}\circ \altid_{LY})*\id_{L_\beta R_\alpha}*(\id_{R_{\alpha'}L_{\beta'}})\\
        &=(\id_{R_{\alpha'}L_{\alpha'}}\circ \eta^{\alpha'})*\id_{L_\beta R_\alpha}*(\eta^{\beta'}\circ \altid_{LX})=R_{\alpha'}L_{\alpha'}L_\beta R_\alpha \eta^{\beta'}\circ \eta^{\alpha'}L_\beta R_\alpha.
    \end{align*}
\end{proof}
\begin{lemma}\label{lem:2.7.F generalization}
    Suppose \begin{tikzcd}
        W \ar{r}{\beta'} \ar{d}{\alpha'}& X \ar{d}{\alpha}\\
        Y \ar{r}{\beta}& Z
    \end{tikzcd}
    commutes in $\fC$. As in Setting \ref{set:Adjoint functors into 2-category}, the below diagram of 2-morphisms commutes in $\fB$:
    \begin{center}
        \begin{tikzcd}
            L_\beta R_\alpha \ar[Rightarrow]{r}{\eta^{\alpha'}L_\beta R_\alpha} \ar[Rightarrow]{d}[swap]{L_\beta R_\alpha \eta^{\beta'}}&R_{\alpha'} L_{\beta \alpha'} R_\alpha \ar[Rightarrow]{d}{R_{\alpha'}L_{\beta'} \epsilon^\alpha}\\
            L_\beta R_{\alpha \beta'} L_{\beta'} \ar[Rightarrow]{r}{\epsilon^\beta R_{\alpha'} L_{\beta'}}& R_{\alpha'}L_{\beta'} .
        \end{tikzcd}
    \end{center}
\end{lemma}
\begin{proof}
    By the above Lemmas and Corollary, we get the below commutative diagram of 2-morphisms in $\fB$:
    \begin{center}
        \begin{tikzcd}
            L_\beta R_\alpha \ar[Rightarrow]{r}{\eta^{\alpha'}L_\beta R_\alpha} \ar[Rightarrow]{d}[swap]{L_\beta R_\alpha \eta^{\beta'}} & R_{\alpha'}L_{\beta \alpha'}R_{\alpha} \ar[Rightarrow]{r}{R_{\alpha'}L_{\beta'} \epsilon^\alpha} \ar[Rightarrow]{d}{R_{\alpha'}L_{\beta \alpha'}R_\alpha \eta^{\beta'}}& R_{\alpha'}L_{\beta'} \ar[Rightarrow]{d}{R_{\alpha'}L_{\beta'} \eta^{\beta'}}\\
            
            L_\beta R_{\alpha \beta'} L_{\beta'} \ar[Rightarrow]{d}[swap]{\epsilon^\beta R_{\alpha'}L_{\beta'}} \ar[Rightarrow]{r}{\eta^{\alpha'}L_\beta R_{\beta \alpha'} L_{\beta'}}&R_{\alpha'}L_{\beta \alpha'}R_{\beta \alpha'}L_{\beta'} \ar[Rightarrow]{d}{R_{\alpha'} L_{\beta'} \epsilon^\beta R_{\alpha'}L_{\beta'}} \ar[Rightarrow]{r}{R_{\alpha'}L_{\beta'}\epsilon^\alpha R_{\beta'}L_{\beta'}}& R_{\alpha'}L_{\beta'} R_{\beta'}L_{\beta'} \ar[Rightarrow]{d}{R_{\alpha'}\epsilon^{\beta'}L_{\beta'}} \\
            
            R_{\alpha'}L_{\beta'}\ar[Rightarrow]{r}[swap]{\eta^{\alpha'}R_{\alpha'}L_{\beta'}} & R_{\alpha'}L_{\alpha'} R_{\alpha'}L_{\beta'} \ar[Rightarrow]{r}[swap]{R_{\alpha'}\epsilon^{\alpha'}L_{\beta'}}&R_{\alpha'}L_{\beta'}
        \end{tikzcd}
    \end{center}
    where the bottom left and top right boxes commute by Lemma \ref{lem:unit counit commute}, the bottom right box commutes by Corollary \ref{cor:swap counits}, and the top right box commutes by Lemma \ref{lem:unit counit distribute with identity in middle}. By the triangle identities, we realize the bottom row and the rightmost column are both the identity 2-morphisms by the triangle identities, which concludes the result.
\end{proof}
\subsubsection{F}\label{2.7.F}
\begin{proof}
    The result is now immediate by Lemma \ref{lem:2.7.F generalization}, since Lemmas \ref{lem:pushforward sheaf distributes} and \ref{lem:inverse image sheaf distributes} tell us that we have functors $L:\Top^{\op} \to \Cat$ and $R:\Top \to \Cat$ each assigning a topological space to the category of sheaves over it, and where $L_\pi = \pi^{-1}$ and $R_\pi = \pi_*$, which satisfy Setting \ref{set:Adjoint functors into 2-category} by Exercise \ref{2.7.B}B. To be explicit, the composition running across the bottom and left of the commutative diagram is Vakil's construction, whereas the composition running across the top and right is the dual construction mentioned in the exercise.
\end{proof}
\subsubsection{G}\label{2.7.G}
\begin{proof}
    The claim is equivalent to showing $\supp s$ contains all of its limit points, so suppose $q\in X$ is a limit point of $\supp s$, i.e. for every neighborhood $U$ of $q$, there is a point $p\in U$ such that $s_p\ne 0$. Towards a contradiction, suppose $s_q =0$. Then there is some neighborhood $U$ of $q$ such that $s\vert_U = 0$. By hypothesis, there is some $p\in U$ with $s_p\ne 0$, so in particular $s\vert_U \ne 0$. This is a contradiction .
\end{proof}
\subsubsection{H}\label{2.7.H}
\begin{proof}
    \begin{enumerate}[(a)]
        \item First, we show that if $q\notin Z$, then $(i_* \fF)_q = 1$. Because $Z$ is closed and $q\notin Z$, then $q$ is not a limit point of $Z$, hence there is some neighborhood $V$ of $q$ such that $V\cap Z = \emptyset$. Then $i_*\fF(V) = \fF(V\cap Z)=\fF(\emptyset)=1$ because $1$ is the terminal object in $\Grp$ and $\fF$ is a sheaf. Let $U\subset Y$ be a neighborhood of $q$ and let $s\in i_*\fF(U)$ be an arbitrary section. We observe $\res_{U,U\cap V}:i_* \fF(U) \to i_*\fF(U\cap V)= \fF(U\cap V \cap Z)=\fF(\emptyset)=1$ gives that $s\vert_{U\cap V}=1$, so $s_q=1$. As $s$ was an arbitrary section, we conclude $(i_* \fF)_q =1$.
        
        Now suppose $q\in Z$. Then the neighborhoods $U\subset Y$ of $q$ are in bijective correspondence with the neighborhoods $V\subset Z$ of $q$ given by $V\leftrightsquigarrow U\cap Z$, hence
        \[
        (i_* \fF)_q = \colim_{Y\supset U\ni q} i_* \fF(U) = \colim_{Z\supset V \ni q} \fF(V) =\fF_q.
        \]
         \item By Exercise \ref{2.4.D}, it suffices to show the natural map induces isomorphisms on the level of stalks. Fix $q\in Y$. If $q\notin Z$, then $q\notin \supp \fG$, so $\fG_q=1$, and $(i_*i^{-1} \fG)_q =1$ by (a). Then any morphism induces an isomorphism of stalks outside of $Z$.

         Now suppose $q\in Z$, and let $[s,U]_q \in \fG_q$. For each open $V$ containing $U\cap Z$, let $\mu_V:\fG(U)\to i^{-1}_{\pre}\fG (U\cap Z)$ be the map sending a section to its equivalence class, and similarly define $\nu_V:\fG(U)\to i^{-1}_{\pre} \fG(V\cap Z)$ for any open $V$ containing $V\cap Z$. Then if our natural map sends $[s,U]$ to $1$, by definition of the map we have $[\mu_U(s),U\cap Z]=1$, i.e. there exists some neighborhoods $V$ of $q$ such that $\mu_U(s)\vert_V =1$, i.e. $\nu_U(s)=1$. By this definition, there exists an open $W$ containing $V\cap Z$ such that $s\vert_W=1$. This demonstrates our germ was trivial to begin with, showing injectivity.

         For surjectivity, an arbitrary element of $i_* i^{-1} \fG(U)$ is a choice of compatible germs of $i^{-1}_{\pre} \fG$. Thus picking an arbitrary element of $(i_* i^{-1} \fG)_q$, we can take the compatible germ at $q$ and restrict to its open neighborhood, so an arbitrary germ can be taken to be $[\mu_U(s),U\cap Z]$, which is exactly the image of $s$ under our natural map.

         Since $\fG$ is naturally isomorphic to $i_* i^{-1} \fG$, we don't lose any data by just considering $i^{-1} \fG$, because we can always push-forward this sheaf over $Z$ and recover $\fG$.
    \end{enumerate}
   
\end{proof}

\section{}
\subsection{}
\subsubsection{A}\label{3.1.A}
\begin{proof}
    By \cite{Lee_Manifolds}, we need to show that for every $p\in X$, there are smooth charts $(U,\varphi)$ containing $p$ and $(V,\psi)$ containing $q=\pi(p)$ where $\pi(U)\subset V$ and $\psi\circ \pi \circ \varphi^{-1}:\varphi(U)\to \psi(V)$ is smooth.

    Fix $p\in X$, and choose a smooth chart $(V,\psi)$ containing $q$. By assumption, $\psi \circ \pi:\pi^{-1}(V)\to \psi(V)$ is smooth, hence for every point in $\pi^{-1}(V)$ there is a smooth chart $(U,\varphi)$ containing the point such that $\psi \circ \pi \circ \varphi:\varphi(U) \to \psi(V)$ is smooth. This gives the desired result, taking the point to be $p$.
\end{proof}
\subsubsection{B}\label{3.1.B}
\begin{proof}
    Define $\pi^\#(f_q)=(f\circ \pi)_p$. To show this is well defined, if $g_q=f_q$, there is some neighborhood $W$ of $q$ with $f\vert_W = g\vert W$. Noticing $g\circ \pi = f\circ \pi$ on $\pi^{-1}(W)$, we see the map is well defined. Let $*$ be either multiplication or addition. We compute
    \[
    \pi^\#(f*g)_q = ((f*g)\circ \pi)_p = (f\circ \pi * g\circ \pi)_p=(f\circ \pi)_p * (g\circ \pi)_p=\pi^\#(f_q)*\pi^\#(g_q).
    \]
    It's clear $\pi^\#(0)=0$ and $\pi^\#(1)=1$, which proves $\pi^\#$ is a morphism of stalks.

    In addition, $f_q\in \frkm_{Y,q}$ if and only if $f(q)=0$, i.e. $f\circ \pi(p)=0$, so $\pi^\#(f_q)=(f\circ \pi)_p$ is in $\frkm_{X,p}$ as well. Thus $\pi^\#$ is a local ring homomorphism as well.
\end{proof}
\subsection{}
\subsubsection{A}\label{3.2.A}
\begin{proof}
    \begin{enumerate}[(a)]
        \item Prime ideals of $k[\epsilon]/\epsilon^2$ are the same as prime ideals in $k[\epsilon]$ containing $(\epsilon^2)$. Such a prime $\frkp$ containing $\epsilon^2$ then contains $\epsilon$. Thus if $f\in \frkp$, we do the division algorithm and write $f=g\epsilon+m$ for some $m\in k$, so we see $m\in \frkp$ implies $m=0$. Then $\epsilon \mid f$, and as $f\in \frkp$ was arbitrary, we get $\frkp = (\epsilon)$. Thus $\Spec k[\epsilon]/\epsilon^2 = \{ (\epsilon)\}$.
        \item Prime ideals of a localized ring are the same as prime ideals not intersecting the multiplicative subset by Exercise \ref{3.2.K}K. Thus the elements of $\Spec k[x]_{(x)}$ are the same as prime ideals contained in $(x)$. Because $k[x]$ is a PID, let $(f)$ be an arbitrary prime contained in $(x)$. If $f\ne 0$, then $x\mid f$ means we can write $f=g\cdot x$ for some $g\in k[x]$. As $\deg g < \deg f$, we see $g\notin (f)$, so $x\in (f)$ by assumption of being prime. Thus $(f) = (x)$, hence $\Spec k[x]_{(x)} = \{ 0, (x)\}$.
    \end{enumerate}
\end{proof}
\subsubsection{B}\label{3.2.B}
\begin{proof}
    Using the fact that $\C = \bar \R$, we get a tower of extensions 
    \begin{center}
        \begin{tikzcd}
            \C \ar[dash]{d}\\
            k \ar[dash]{d}{2}\\
            \R \ar[dash, bend right = 40]{uu}[swap]{2}
        \end{tikzcd}
    \end{center}
    where the numbers indicate the degree of the field extensions, and where $k=\R[x]/(x^2+ax+b)$. Because extension degrees are multiplicative, we see $[\C: k] =1$, i.e. $\C \cong k$.

    An explicit isomorphism $k\to \C$ could be given by $x\mapsto -\frac{a}{2}+i\sqrt{b-\frac{a^2}{4}}$, but will not be checked in this proof.
\end{proof}
\subsubsection{C}\label{3.2.C}
\begin{proof}
    $\Q[x]$ is a PID, so primes of $\Q[x]$ are the same as irreducible polynomials over $\Q$. Because irreducible polynomials in $\Q$ are uniquely determined by their roots in $\overline{\Q}$ (an irreducible polynomial splits in $\overline \Q$), we get a bijective correspondence between orbits of Galois conjugates and prime ideals of $\Q[x]$. Thus we may view $\Spec \Q[x]$ as $\bar \Q$ modulo the orbits of the $\Gal \bar \Q/ \Q$.
\end{proof}
\subsubsection{D}\label{3.2.D}
\begin{proof}
    Suppose, aiming for a contradiction, that $f_1,\dots,f_n$ is a complete list of all of the nonzero primes in $k[x]$, i.e. irreducible polynomials since $k[x]$ is a PID. Then set $g= 1+\prod_i f_i$, and notice $g\equiv 1 \mod \frkp$ for each $\frkp\in \Spec k[x]$, so $g$ is indivisible by each $f_i$. However, we then see that $g$ cannot be written as a product of irreducibles as we have a complete list $f_1,\dots, f_n$, which is a contradiction because PID implies UFD.
\end{proof}
\subsubsection{E}\label{3.2.E}
\begin{proof}
    We claim that every $\frkp\in \Spec \C[x,y]$ is principally generated by an irreducible polynomial or of the form $(x-a, y-b)$ for some $a,b\in \C$. It's clear that if a prime ideal is principally generated, its generator must be irreducible, so we fix a nonprincipally generated prime $\frkp$ and first suppose for a contradiction that for every $f,g\in \frkp$, there is a nonconstant common factor in $\C[x,y]$. We will make some inductive constructions here. $\frkp$ must contain two elements $f_1,g_1$ such that $(f_1,g_1)$ is not principal because $\C[x,y]$ is Noetherian by the Hilbert-basis theorem. We then set $I_1 = (f_1, g_1)$ and $I_0=0$.

    Now we inductively have some $I_n= (f_n, g_n)$ that is not principally generated, is contained in $\frkp$, and properly contains $I_{n-1}$. By hypothesis, we can find some nonconstant factor $h$ of $f_n$ and $g_n$, so we can write $f_n = p h $ and $g_n = q h$ for some polynomials $p, q$. Since $\frkp$ is prime, either $h\in \frkp$ or $p,q\in \frkp$. If $h\in \frkp$, we see $(h)\subsetneq \frkp$, so there exists some $h'\in \frkp \setminus (h)$. Then we set $I_{n+1}=(h,h')$, which satisfies our hypothesis. If $h\notin \frkp$, then both $p$ and $q$ are in $\frkp$. If $(p,q)=(h)$ for some $h\in \frkp$, we are able to find some $h'\in \frkp\setminus (h)$, and let $I_{n+1}=(h,h')$, which again satisfies our hypothesis. If $(p,q)$ is not principal, we set $I_{n+1}=(p,q)$. The only non-immediate condition to check is that $(f_n,g_n)\subsetneq (p,q)$. If the containment is not proper, we replace $(f_n,g_n)$ with $(p,q)$, and do the same case division. After a finite number of case divisions, the containment either becomes proper or we move into one of the other outlined cases. This is because each common factor is nonconstant, and if we always were able to write $(f_n, g_n)=(p,q)h=(p,q)$, we observe that $(p,q)$ have either smaller $x$ or $y$-degree than $f_n$ and $g_n$, so we cannot do this procedure for infinity. Then the induction hypothesis holds.

    We have now constructed an infinite ascending chain of proper ideals in $\C[x,y]$, which is impossible by the Hilbert basis theorem.

    Now we can find some $f,g\in \frkp$ which have no non-constant common factor in $\C[x,y]$. Considering these polynomials as elements of $\C(x)[y]$, which is a Euclidean domain, there is a greatest common factor $h'\in \C(x)[y]$ and may write $h' = a' f+b'g$ for some $a',b'\in \C(x)[y]$. Since $h'$ is defined up to unit, we may take $h'\in \C[x,y]$. We will now show that $h'\in \C[x]$. Since $\C[x,y]$ is a UFD, we write $f=\prod_{i=1}^m f_i$, $g=\prod_{i=1}^n g_i$, and $h'=\prod_{i=1}^l h_i$ where each $f_i, g_i, h_i$ is irreducible in $\C[x,y]$. We rearrange the indices to be such that $f_i= h_i$ for $1\le i \le m'$, and $g_i = h_i$ for $m'+1\le i \le n'$. It then follows that
    \[
    a= \frac{f_{m'+1}f_{m'+2}\dots f_{m}}{h_{m'+1}h_{m'+2}\dots h_l}
    \]
    and 
    \[
    b= \frac{g_1 g_2 \dots g_{m'} g_{n'+1} g_{n'+2} \dots g_n}{h_1 h_2 \dots h_{m'} h_{n'+1} h_{n'+2} \dots h_l}.
    \]
    But because the denominators of $b$ can only be in the variable $x$, we see each $h_i$ must be in the variable $x$ only. Thus $h'\in \C[x]$ as desired. Now that
    \[
    h'=a'f+b'g
    \]
    we may clear the denominators of both $a'$ and $b'$ (remember, the denominators are in $\C[x]$) to get some expression
    \[
    h=\alpha f +\beta g
    \]
    for some $\alpha, \beta \in \C[x,y]$ and $h\in \C[x]$. Thus $h \in (f,g)\subset \frkp$, and as $\C$ is algebraically closed, $h$ splits into a product of linear factors, one of which, say $x-a$, must be in $\frkp$ because $\frkp$ is prime.

    An identical proof, swapping the roles of $x$ and $y$, shows that some $y-b$ is in $\frkp$ as well. However, as $(x-a,y-b)$ is maximal ($\C[x,y]/(x-a,y-b) \cong \C$ is a field), we get $\frkp = (x-a,y-b)$.
    \vspace{0.1in}

    A very short proof can also be given assuming two powerful results, being the weak Nullstellensatz and that the dimension of $k[x_1,\dots,x_n]$ is $n$ for every field $k$. If we take a nonprincipal prime ideal $\frkp\in \C[x,y]$, we can find some irreducible element $f\in \frkp$. Then we get the ascending chain
    \[
    0\subsetneq (f) \subsetneq \frkp.
    \]
    We see $\frkp$ must be maximal since $\dim \C[x,y]=2$, and by the weak Nullstellensatz, since $\C$ is algebraically closed, $\frkp=(x-a,y-b)$ for some $a,b\in \C$.
\end{proof}
\subsubsection{F}\label{3.2.F}
\begin{proof}
    Suppose Hilbert's Nullstellensatz, stating that for any field $k$, every maximal ideal of $k[x_1,\dots,x_n]$ has residue field a finite extension of $k$. To prove the weak Nullstellensatz, let $k$ be an algebraically closed field. It's clear that each ideal of the form $(x_1-a_1,x_2-a_2,\dots,x_n-a_n)$ is a maximal ideal because its residue field is isomorphic to $k$, a field. Conversely, we fix an arbitrary maximal ideal $\frkm$ of $k[x_1,\dots,x_n]$. By the Nullstellensatz, we have $k[x_1,\dots,x_n]/\frkm $ is a finite extension of $k$, and thus an algebraic extension of $k$. However, since $k$ is algebraically closed, the inclusion $k\hookrightarrow k[x_1,\dots,x_n]/\frkm$ must then be an isomorphism. Thus for each index $i$, there is some $a_i \in k$ such that $x_i\equiv a_i \mod \frkm$. Then $x_i-a_i \equiv 0 \mod \frkm$, i.e. $x_i-a_i\in \frkm$. Then $\frkm$ contains the ideal $(x_1-a_1,\dots,x_n-a_n)$, which is also a maximal ideal, hence $\frkm=(x_1-a_1,\dots,x_n-a_n)$.
\end{proof}
\subsubsection{G}\label{3.2.G}
\begin{proof}
    It's a general fact in dimension theory that if $A$ is a finitely generated $k$-algebra that is also a domain, then $\dim A = \mathrm{tr}.\deg_k(\Frac A)$. In our case, $A$ is finite dimensional over $k$ means $A$ is algebraic over $k$, thus $\dim A = 0$. Then $A$ being Noetherian and dimension $0$ is the same as $A$ being Artinian, and $A$ a domain implies that $A$ is reduced. Because reduced Artinian rings are the same thing as fields, we see $A$ is a field as well.
\end{proof}
\subsubsection{H}\label{3.2.H}
\begin{proof}
    The maximal ideal of $\Q[x,y]$ corresponding to $(\sqrt{2},\sqrt{2})$ is the ideal $(x^2-2, x-y)$ because in modding out by this ideal, we get that $x=y$ and that $x=\sqrt{2}$.

    The maximal ideal corresponding to $(\sqrt{2},-\sqrt{2})$ is $(x^2-2,x+y)$ so that $x=-y$ and $x=\sqrt{2}$.
    
    It's easy to see both residue fields are isomorphic to $ \Q(\sqrt{2})$.
\end{proof}
\subsubsection{I}\label{3.2.I}
\begin{proof}
    With a slight generalization to the proof of Exercise \ref{3.2.E}E (replacing $\C$ by an arbitrary field $k$), we see every non-principal prime ideal $\frkp$ in $\Spec k[x,y]$ contains some irreducible $f(x)$ and $g(y)$. However, $k[x,y]/(f,g)\cong k$ shows that $(f,g)$ is maximal, and as $\frkp\supset (f,g)$, equality holds. To summarize, every non-principal $\frkp \in \Spec k[x,y]$ can be written as $(f(x),g(y))$ with $f,g$ both irreducible.
    \begin{enumerate}[(a)]
        \item We claim $\phi(\pi, \pi^2)=(x^2-y)$, with one containment clear. Suppose for a contradiction that  $\phi(\pi, \pi^2)$ were non-principally generated. By our lemma, we would then be able to find some $f(x)\in \phi(\pi,\pi^2)$, which implies $\pi$ is algebraic over $\Q$, impossible. Then $\phi(\pi,\pi^2)$ contains the prime $x^2-y$ and is principal, so indeed $\phi(\pi,\pi^2)=(x^2-y)$.
        \item First, we show that $0\in \Spec \Q[x,y]$ is equal to $\phi(\pi,0)$. Similarly to (a), if there were some nontrivial $f\in \phi(\pi,0)$, then $\pi$ would be algebraic over $\Q$, contradiction, so $\phi(\pi,0)=0$.

        Now we take $\frkp=(f)$ for some irreducible $f\in \Q[x,y]$. We consider $f \mod x-\pi$, i.e. substituting $\pi$ for $x$ in $f$ which gives us a polynomial in $\C[y]$. Because $\C$ is algebraically closed, there is some root $\alpha$ of this polynomial in $\C$. We then claim $\phi(\pi, \alpha)=(f)$, where it's clear by construction that $(f)\subset \phi(\pi, \alpha)$. If $\phi(\pi, \alpha)$ were non-principal, we would get some $g(x)\in \phi(\pi, \alpha)$, again contradicting that $\pi$ is transcendental over $\Q$. Thus $\phi(\pi, \alpha)$ is principal and contains the prime $(f)$, hence must equal $(f)$.

        For the last case, we take $\frkp$ to be non-principal. Our lemma then tells us that $\frkp=(f(x),g(y))$ for some irreducible $f,g$. Let $\alpha\in \C$ be a root of $f(x)$ and $\beta\in \C$ be a root of $g(y)$. We then claim $\phi(\alpha,\beta)=(f,g)=\frkp$. This is easy to see as $(f,g)\subset \phi(\alpha,\beta)$, and $(f,g)$ is maximal since $\Q[x,y]/(f,g)\cong \Q(\alpha,\beta)$ is a field.
    \end{enumerate}
\end{proof}

\subsubsection{J}\label{3.2.J}
\begin{proof}
    Fix $\frkp \in \Spec A/I$, and first we show $\phi^{-1}(\frkp)$ is an ideal of $A$. If $x,y\in \phi^{-1}(\frkp)$, then $\phi(x-y)=\phi(x)-\phi(y)\in \frkp$ by hypothesis, so $x-y\in \phi^{-1}(\frkp)$. In addition, if $r\in A$, we have $\phi(rx)=\phi(r)\phi(x)\in \frkp$ so $rx\in \phi^{-1}(\frkp)$.

    Next, we show that $\phi^{-1}(\frkp)$ contains $I$. This is simply because preimages are inclusion preserving, and $\phi^{-1}(0)=I$.

    Now we show $\phi^{-1}(\frkp)$ is prime. Suppose $xy\in \phi^{-1}(\frkp)$. Then $\phi(x)\phi(y) \in \frkp$ implies that either $\phi(x)\in \frkp$ or $\phi(y)\in \frkp$, i.e. one of $x$ or $y$ is in $\phi^{-1}(\frkp)$.

    It remains to show $\phi^{-1}$ is a bijection. Suppose $\frkp, \frkq$ are two prime ideals of $A/I$ such that $\phi^{-1}(\frkp)=\phi^{-1}(\frkq)$. Fixing $x+I \in \frkp$, we have $x\in \phi^{-1}(\frkp)=\phi^{-1}(\frkq)$. Then $x+I\in \frkq$ by definition, so $\frkp\subset \frkq$. The reverse inclusion is completely analogous, so indeed $\frkp=\frkq$. For surjectivity, fix a prime ideal $\frkq\in \Spec A$ containing $I$. We claim that $\phi(\frkq)$ is prime. In general, images of ideals under ring homomorphisms are not ideals, so we have to show $\phi(\frkq)$ is an ideal of $A/I$. For $x,y\in \frkq$ (so that $x+I$ and $y+I$ are arbitrary elements of $\phi(\frkq)$), we have $x-y\in \frkq$, so $\phi(x)-\phi(y)=\phi(x-y)\in \phi(\frkq)$ as well. For $r+I\in A/I$, $(r+I)(x+I)=rx+I=\phi(rx)$ is in $\phi(\frkq)$ because $rx\in \frkq$. Thus $\phi(\frkq)$ is an ideal of $A/I$. Suppose $xy+I \in \phi(\frkq)$, so there is some element $z\in I$ such that $xy+z\in \frkq$. Because $z\in I\subset \frkq$, we also get $xy\in \frkq$. By $\frkq$ being prime, one of $x$ or $y$ is in $\frkq$, so one of $x+I$ or $y+I$ is in $\phi(\frkq)$. Now we claim $\phi^{-1}(\phi(\frkq))= \frkq$. By general set theory, $\phi^{-1}(\phi(\frkq))$ contains $\frkq$. If $x\in \phi^{-1}(\phi(\frkq))$, i.e. $x+I \in \phi(\frkq)$, again there exists some $z\in I$ such that $x+z\in \frkq$. Since $z\in \frkq$, $x\in \frkq$, which shows equality holds.
\end{proof}
\subsubsection{K}\label{3.2.K}
\begin{proof}
    As usual, the map $\phi:A\to S^{-1}A$ induces a map $\phi^{-1}:\Spec S^{-1}A \to \Spec A$ by Exercise \ref{3.2.M}M. In addition, if $\frkq \in \Spec S^{-1}A$, $\phi^{-1}(\frkq)$ cannot intersect $S$. To see this, if some $x$ were in the intersection, by definition $\phi(x)=\frac{x}{1}\in \frkq$, and as $x\in S$, we have $\frac{1}{x}\in S^{-1}A$, so $\frac{1}{x}\cdot \frac{x}{1}=1 \in \frkq$, implying $\frkq$ is not prime. By general set theory, $\phi^{-1}$ is also inclusion preserving.

    Next, we will show $\phi^{-1}$ is injective by supposing $\phi^{-1}(\frkp)=\phi^{-1}(\frkq)$ for some $\frkp,\frkq \in \Spec S^{-1}A$. Fix $\frac{a}{s}\in \frkp$. Then by multiplying by $\frac{s}{1}\in S^{-1}A$, we get $\frac{a}{1}\in \frkp$ as well. Then $a\in \phi^{-1}(\frkp)=\phi^{-1}(\frkq)$, so $\frac{a}{1}\in \frkq$. Then upon multiplication by $\frac{1}{s}\in S^{-1}A$, we get $\frac{a}{s}\in \frkq$, so $\frkp\subset \frkq$. The reverse inclusion is entirely analogous, so $\frkp=\frkq$ and thus $\phi^{-1}$ is injective.

    For surjectivity, fix $\frkp\in \Spec A$ with $\frkp \cap S=\emptyset$. We define $\frkq = \{ \frac{a}{s}\in S^{-1}A \mid a \in \frkp \}$. Indeed, we can make this definition, i.e. if $\frac{a}{s}=\frac{b}{t}$, then $\frac{a}{s}$ having numerator in $\frkp$ is equivalent to $\frac{b}{t}$ having numerator in $\frkp$. This is because by assumption, there is some $r\in S$ such that $r(at-bs)=0$, i.e. $art=brs$. By assuming $a\in \frkp$, the left hand side is in $\frkp$, so $brt\in \frkp$. By $\frkp$ being prime, $b\in \frkp$ or $rt\in \frkp$. But because $rt\notin \frkp$ ($S\cap \frkp=\emptyset$), by primeness $b\in \frkp$. That $b\in \frkp$ implies $a\in \frkp$ is completely analogous, so our definition makes sense. Next we will show $\frkq \in \Spec S^{-1}A$.

    If $\frac{a}{s}, \frac{b}{t}\in \frkq$, then $\frac{a}{s}-\frac{b}{t}=\frac{at-bs}{st}\in \frkq$ because $at-bs\in \frkp$ by assumption that $a,b\in \frkp$. If $\frac{r}{t}\in S^{-1}A$, then $\frac{r}{t}\cdot \frac{a}{s}=\frac{ra}{st}\in \frkq$ because $ra\in \frkp$ since $a\in \frkp$. To show $\frkq$ is prime, suppose $\frac{a}{s}\cdot \frac{b}{t}\in \frkq$. Then $ab\in \frkp$ by definition, and by primeness of $\frkp$, we get $a\in \frkp$ or $b\in \frkp$, so $\frac{a}{s}\in \frkq$ or $\frac{b}{t}\in \frkq$.

    Now, we claim that $\phi^{-1}(\frkq)=\frkp$, which would show $\phi^{-1}$ is surjective onto $\{\frkp\in \Spec A \mid \frkp \cap S = \emptyset\}$. It's clear $\phi^{-1}(\frkq)\subset \frkp$ by construction (an element $x\in A$ sent to $\frac{x}{1}$ in $\frkq$ implies $x\in \frkp$). The reverse inclusion is also easy (fix $x\in \frkp$, and then $\phi(x)=\frac{x}{1}\in \frkq$, i.e. $x\in \phi^{-1}(\frkq)$).
\end{proof}
\subsubsection{L}\label{3.2.L}
\begin{proof}
    To show $(\C[x,y]/(xy))_x\cong \C[x]_x$, we first notice every element of $\C[x,y]/(xy)$ has representative $\sum a_i x^i+ y \sum b_j y^j$ since a $\C$-basis for the $i$-th graded piece of $\C[x,y]/(xy)$ is just $x^i,y^i$. Then an arbitrary element of the localization by $x$ is of the form $\frac{\sum a_i x^i+ y \sum b_j y^j}{x^k}$. We define a map $\phi:(\C[x,y]/(xy))_x\to \C[x]_x$ given by $\frac{\sum a_i x^i+ y \sum b_j y^j}{x^k}\mapsto \frac{\sum a_i x^i}{x^k}$, and claim this is a ring homomorphism, where it is immediate that $\phi(0)=0$ and $\phi(1)=1$. We compute that
    \begin{align*}
        &\phi(\frac{\sum_{i=0}^m a_i x^i+ y \sum_{j=0}^n b_j y^j}{x^k}+\frac{\sum_{i=0}^{m'} a_i' x^i+ y \sum_{j=0}^{n'} b_j' y^j}{x^{k'}})=\frac{\sum_{i=0}^m a_i x^i+\sum_{i=0}^{m'}a_i'x^{i+d}}{x^k}\\
        &=\frac{\sum_{i=0}^m a_i x^i}{x^k}+\frac{\sum_{i=0}^{m'}a_i'x^{i+d}}{x^k}=\frac{\sum_{i=0}^m a_i x^i}{x^k}+\frac{\sum_{i=0}^{m'}a_i'x^{i}}{x^{k'}}\\
        &=\phi(\frac{\sum_{i=0}^m a_i x^i+ y \sum_{j=0}^n b_j y^j}{x^k})+\phi(\frac{\sum_{i=0}^{m'} a_i' x^i+ y \sum_{j=0}^{n'} b_j' y^j}{x^{k'}})
    \end{align*}
    where we have assumed without loss of generality that $k'\le k$ and where we set $d=k-k'$. In addition,
    \begin{align*}
        &\phi(\frac{\sum_{i=0}^m a_i x^i+ y \sum_{j=0}^n b_j y^j}{x^k} \cdot \frac{\sum_{i=0}^{m'} a_i' x^i+ y \sum_{j=0}^{n'} b_j' y^j}{x^{k'}})=\frac{\sum_{\alpha=0}^{m+m'} (\sum_{i+j=\alpha} a_ia_j')x^\alpha}{x^{k+k'}}\\
        &=\frac{\sum_{i=0}^m a_i x^i}{x^k} \cdot \frac{\sum_{i=0}^{m'} a_i' x^i}{x^{k'}}=\phi(\frac{\sum_{i=0}^m a_i x^i+ y \sum_{j=0}^n b_j y^j}{x^k})\phi( \frac{\sum_{i=0}^{m'} a_i' x^i+ y \sum_{j=0}^{n'} b_j' y^j}{x^{k'}}).
    \end{align*}
\end{proof}
Now suppose $\frac{\sum_{i=0}^m a_i x^i+ y \sum_{j=0}^n b_j y^j}{x^k}$ is in the kernel of $\phi$, hence
\[
\frac{\sum_{i=0}^m a_i x^i+ y \sum_{j=0}^n b_j y^j}{x^k}=\frac{y \sum_{j=0}^n b_j y^j}{x^k} = \frac{xy \sum_{j=0}^n b_j y^j}{x^{k+1}}=\frac{0}{x^{k+1}}=0
\]
so the map is injective. It's immediate that $\phi$ is surjective, and thus $\phi$ is an isomorphism.
\subsubsection{M}\label{3.2.M}
\begin{proof}
    Fix $\frkp \in \Spec A$, and first we show $\phi^{-1}(\frkp)$ is an ideal of $B$. If $x,y\in \phi^{-1}(\frkp)$, then $\phi(x-y)=\phi(x)-\phi(y)\in \frkp$ by hypothesis, so $x-y\in \phi^{-1}(\frkp)$. In addition, if $r\in A$, we have $\phi(rx)=\phi(r)\phi(x)\in \frkp$ so $rx\in \phi^{-1}(\frkp)$.

    Now we show $\phi^{-1}(\frkp)$ is prime. Suppose $xy\in \phi^{-1}(\frkp)$. Then $\phi(x)\phi(y) =\phi(xy)\in \frkp$ implies that either $\phi(x)\in \frkp$ or $\phi(y)\in \frkp$, i.e. one of $x$ or $y$ is in $\phi^{-1}(\frkp)$, so $\phi^{-1}(\frkp)\in \Spec B$.

    Next, we show that $\phi^{-1}$ is inclusion preserving by supposing $\frkq \subset \frkp$. Then $\phi^{-1}(\frkp)$ contains $\phi^{-1}(\frkq)$ simply because preimages are inclusion preserving by general set theory.
\end{proof}
\subsubsection{N}\label{3.2.N}
\begin{proof}
    \begin{enumerate}[(a)]
        \item By the proof of Exercise \ref{3.2.J}J.
        \item By the proof of Exercise \ref{3.2.K}K.
    \end{enumerate}
\end{proof}
\subsubsection{O}\label{3.2.O}
\begin{proof}
    Let $\phi:\C[y]\to \C[x]$ be given by $y\mapsto x^2$. By Exercise \ref{3.2.M}M, we get a map $\phi^{-1}:\Spec \C[x]\to \Spec \C[y]$. Our goal is to show the preimage of $(y-a)$ under $\phi^{-1}$ is the set containing $(x-\sqrt{a})$ and $(x+\sqrt{a})$. First, we will show that $\phi^{-1}(x-\sqrt a)=(y-a)$. Indeed, $y-a\in \phi^{-1}(x-\sqrt a)$ because $\phi(y-a)=x^2-a=(x-\sqrt a)(x+\sqrt a) \in (x-\sqrt a)$. Thus $\phi^{-1}(x-\sqrt a)\supset (y-a)$, but as $y-a$ is maximal, equality holds. An analogous argument shows that $\phi^{-1}(x+\sqrt a)=(y-a)$.

    Now suppose $\frkp \in \Spec \C[x]$ is in the preimage of $(y-a)$ under $\phi^{-1}$, i.e. $\phi^{-1}(\frkp)=(y-a)$. By general set theory, we get
    \[
    \frkp \supset \phi(\phi^{-1}(\frkp)) = \phi(y-a)=(x^2-a).
    \]
    Then $(x-\sqrt a)(x+\sqrt a) \in \frkp$ and $\frkp$ prime implies that one of $x-\sqrt{a}$ or $x+\sqrt a$ is in $\frkp$. Because these elements generate maximal ideals, we get that either $\frkp = (x-\sqrt a)$ or $\frkp = (x+\sqrt a)$ as desired.
\end{proof}
\subsubsection{P}\label{3.2.P}
\begin{proof}
    \begin{enumerate}[(a)]
        \item Suppose $\phi:B\to A$ is a ring homomorphism, and $J\subset B$ and $I\subset A$ are ideals such that $\phi(J)\subset I$. We claim that $\phi$ induces a map $\Spec A/I\to \Spec B/J$.

        By Exercise \ref{3.2.M}M, it suffices to show $\phi$ induces a ring homomorphism $B/J\to A/I$ given by $x+J\mapsto \phi(x)+I$. This map is clearly additive and multiplicative because $\phi$ is, and is well defined because if we instead pick a representative $x+j$ with $j\in J$, then
        \[
        x+j+J\mapsto \phi(x+j)+I = \phi(x)+\phi(j)+I=\phi(x)+I
        \]
        since $\phi(j)\in I$ by hypothesis.
        \item Suppose $\phi:k[y_1,\dots,y_n]\to k[x_1,\dots,x_m]$ is a morphism of $k$-algebras with $f_i\coloneqq \phi(y_i)$ for each $1\le i \le n$. We need to show $\phi^{-1}:\Spec k[x_1,\dots,x_m]\to \Spec k[y_1,\dots,y_n]$ sends $(x_1-a_1,\dots,x_m-a_m)$ to $(y_1-f_1(a_1,\dots,a_m),\dots,y_n-f_n(a_1,\dots,a_m))$. Because the latter ideal is maximal, it suffices to show $y_i-f_i(a_1,\dots,a_m)\in \phi^{-1}(x_1-a_1,\dots,x_m-a_m)$ for each $i$. This is because
        \[
        \phi(y_i-f_i(a_1,\dots,a_m))=f_i-f_i(a_1,\dots,a_m) \in (x_1-a_1,\dots,x_m-a_m)
        \]
        because $(x_1-a_1,\dots,x_m-a_m)$ is the kernel of the evaluation at the tuple $(a_1,\dots,a_m)\in k^m$, and $f_i-f_i(a_1,\dots,a_m)$ is clearly in this kernel. Note we used that $\phi$ is a morphism of $k$-algebras so that $\phi$ is $k$-linear, and in particular fixes elements of $k$ like $f_i(a_1,\dots,a_m)$.
    \end{enumerate}
\end{proof}
\subsubsection{Q}\label{3.2.Q}
\begin{proof}
    Notice that $\pi^{-1}(p)=\{\frkq \in \A^n_\Z \mid \frkq \cap \Z = (p)\}=\{\frkq \in \A^n_\Z \mid p\in \frkq\}$, with the last equality holding because $p\in \frkq$ implies $\frkq\cap \Z$ is an ideal containing the maximal $(p)$. By Exercise \ref{3.2.J}J, we have a bijection between $\Spec \Z[x_1,\dots,x_n]/(p)=\A^n_{\F_p}$ and $\{ \frkq \in \A^n_\Z \mid (p)\subset \frkq\}=\{ \frkq \in \A^n_\Z \mid p\in \frkq\}$, which is equal to $\pi^{-1}(p)$.

    We claim $\pi^{-1}(0)$ corresponds to $\A^n_\Q$, and notice that $\pi^{-1}(0)=\{\frkq \in \A^n_\Z \mid \frkq \cap \Z = (0)\}$. We view $\Q[x_1,\dots,x_n]$ as $S^{-1}\Z[x_1,\dots,x_n]$ where $S=\Z\setminus 0$. By Exercise \ref{3.2.K}K, $\A^n_\Q = \Spec \Q[x_1,\dots,x_n]$ corresponds with $\{\frkq \in \Spec \Z[x_1,\dots,x_n]\mid \frkq \cap S = \emptyset \} = \{\frkq \in \A^n_\Z \mid \frkq \cap \Z = 0 \} = \pi^{-1}(0)$.
\end{proof}
\subsubsection{R}\label{3.2.R}
\begin{proof}
    \begin{enumerate}[(a)]
        \item Suppose $I$ is an ideal of nilpotents. By Exercise \ref{3.2.J}J, $\Spec B/I \cong \{\frkp \in \Spec B\mid \frkp \supset I \}$. Let $\frkp \in \Spec B$ be arbitrary. Then for each $x\in I$, there is some $n\in \N$ with $x^n=0 \in \frkp$, hence by primeness, $x\in \frkp$. Thus $I\subset \bigcap_{\frkp \in \Spec B} \frkp$, and in particular, $\{p\in \Spec B \mid p\supset I\} = \Spec B$.
        \item To show $\frkN(B)$ is an ideal, suppose $x^m=0=y^n$, and let $a\in B$ be arbitrary. To show $x-y\in \frkN(B)$, we compute
        \[
        (x-y)^{m+n}=\sum_{i=0}^{m+n} \binom{m+n}{i}(-1)^{m+n-i} x^i y^{m+n-i}
        \]
        and notice that if $i\geq m$, then $x^i=0$ and if $i\le m$, then $m+n-i\ge n$, so $y^{m+n-i}=0$. In other words, every term of our sum vanishes, so indeed $x-y\in \frkN(B)$. To show $ax\in \frkN(B)$, we easily see
        \[
        (ax)^m=a^m x^m =0.
        \]
    \end{enumerate}
\end{proof}
\subsubsection{S}\label{3.2.S}
\begin{proof}
    By the proof of Exercise \ref{3.2.R}R, $\frkN(A)\subset \bigcap_{\frkp\in \Spec A} \frkp$, so it remains to show the reverse inclusion by fixing $x\notin \frkN(A)$, and showing $x\notin \bigcap_{\frkp\in \Spec A} \frkp$. What we want is equivalent to showing there exists a prime not containing $x$, and to do this, it suffices to show $A_x \ne 0$, for then there is a maximal ideal of $A_x$, which corresponds to a prime ideal of $\Spec A$ not intersecting $\{1,x,x^2,\dots \}$ by Exercise \ref{3.2.K}K, i.e. a prime not containing $x$. Showing $A_x\ne 0$ is equivalent to showing $0\ne 1$ in $A_x$, so we will show the latter. Supposing for a contradiction that $0=1$, then by definition of localization, there is some $x^n$ such that $x^n(1-0)=0$, i.e. $x^n=0$. This is impossible by assumption that $x\notin \frkN(A)$.
\end{proof}
\subsubsection{T}\label{3.2.T}
\begin{proof}
    Fix $f=\sum_{i=0}^n a_i x^i \in k[x]$. Then
    \[
    f(x+\epsilon)=\sum_{i=0}^n a_i (x+\epsilon)^i = \sum_{i=0}^n a_i \sum_{j=0}^i \binom{i}{j}  \epsilon^j x^{i-j}.
    \]
    Because $\epsilon^2=0$, every $\epsilon^j=0$ for $j\ge 2$, so we have
    \[
    \sum_{i=0}^n a_i \sum_{j=0}^i \binom{i}{j}  \epsilon^j x^{i-j}=\sum_{i=0}^n a_i \left[ x^i + i\epsilon x^{i-1} \right]=\sum_{i=0}^n a_i x^i +\epsilon \sum_{i=1}^n a_i i x^{i-1}=f+\epsilon f'
    \]
    where $f'$ is the formal derivative of $f$.
\end{proof}
\subsection{}
\subsection{}
\subsubsection{A}\label{3.4.A}
\begin{proof}
    The x-axis is the ideal $(y,z)$, which is clearly prime. In addition, $(y,z)\supset \{xy, yz\}$ because $y$ divides each of these elements, and $y\in (y,z)$. By definition, $(y,z)\in V(xy,yz)$.
\end{proof}
\subsubsection{B}\label{3.4.B}
\begin{proof}
    Suppose $\frkp \in V(S)$, i.e. $\frkp \supset S$. Then for an arbitrary element $\sum_{i=1}^n a_i s_i$ with each $s_i\in S$ and $a_i\in A$, each $s_i \in \frkp$ by hypothesis, hence $\sum_{i=1}^n a_i s_i\in \frkp$ as well. This shows $\frkp \supset (S)$, so $\frkp \in V((S))$.

    On the other hand, suppose $\frkp \in V((S))$, i.e. $\frkp \supset (S)$. Because $(S)\supset S$, we get $\frkp \supset S$, so $\frkp \in V(S)$.
\end{proof}
\subsubsection{C}\label{3.4.C}
\begin{proof}
    \begin{enumerate}[(a)]
        \item $\Spec A$ is closed because $\Spec A = V(\emptyset)$ as every prime contains $\emptyset$. Thus $\emptyset$ is open. $\emptyset$ is closed because $\emptyset = V(A)$, since every prime is proper. Thus $\Spec A$ is open.
        \item Fix $\frkp \in \Spec A$. It's easy to show that $\frkp \supset I_i$ for each $i$ if and only if $\frkp \supset \sum_i I_i$. For the forward direction, we let $\sum_{k=0}^n x_{i_k}\in \sum_i I_i$ be an arbitrary element with $x_{i_k}\in I_{i_k}$ for each $k$. As each $x_i\in \frkp$, indeed the sum is in $\frkp$, showing $\frkp \supset \sum_i I_i$. For the reverse direction, as $\sum_i I_i\supset I_i$ for each index $i$, we see $\frkp \supset I_i$ for each $i$ as well. By definition, $\frkp\in \bigcap_i V(I_i)$ is equivalent to $\frkp\supset I_i$ for each $i$, and $\frkp \in V(\sum_i I_i)$ means $\frkp \supset \sum_i I_i$. Thus arbitrary intersections of closed sets is closed, which is equivalent to arbitrary unions of open sets being open.
        \item To show $V(I_1)\cup V(I_2)=V(I_1I_2)$, first fix $\frkp \in V(I_1)\cup V(I_2)$. If $\frkp \in V(I_1)$, i.e. $\frkp \supset I_1$, then as $I_1\supset I_1I_2$, we get $\frkp \supset I_1I_2$, so $\frkp \in V(I_1I_2)$. The case where $\frkp \in V(I_2)$ is analogous, so $V(I_1)\cup V(I_2) \subset V(I_1I_2)$. For the reverse inclusion, suppose $\frkp \not \supset I_1$ and $\frkp \not \supset I_2$. Then we let $x\in I_1$ and $y\in I_2$ be such that $x,y\notin \frkp$. Then by primeness of $\frkp$, $xy\notin \frkp$, and as $xy\in I_1I_2$, we see $\frkp \not \supset I_1I_2$, so $\frkp \notin V(I_1I_2)$. Thus finite unions of closed sets are closed, or equivalently, finite intersections of open sets are open.
    \end{enumerate}
\end{proof}
\subsubsection{D}\label{3.4.D}
\begin{proof}
    To show $\sqrt I$ is an ideal, fix $x,y \in \sqrt I$, and $a\in A$, and assume $x^m \in I$ and $y^n\in I$. To show $x-y\in \sqrt I$, we compute
        \[
        (x-y)^{m+n}=\sum_{i=0}^{m+n} \binom{m+n}{i}(-1)^{m+n-i} x^i y^{m+n-i}
        \]
        and notice that if $i\geq m$, then $x^i\in I$ and if $i\le m$, then $m+n-i\ge n$, so $y^{m+n-i} \in I$. In other words, every term of our sum is an element of $I$, so $(x-y)^{m+n}\in I$, proving $x-y\in \sqrt I$. To show $ax\in \sqrt I$, we easily see
        \[
        (ax)^m=a^m x^m \in I
        \]
        because $x^m\in I$.

        It's clear that $I\subset \sqrt I$, so easily $V(\sqrt I)\subset V(I)$. To show the reverse inclusion, suppose $\frkp \in V(I)$, so $\frkp \supset I$. Then for any element $x\in \sqrt I$, we have $x^n \in I \subset \frkp$, which implies by primeness of $\frkp$ that $x\in \frkp$. Therefore $\frkp \supset \sqrt I$, and thus $\frkp \in V(\sqrt I)$.

        Since an ideal is always contained in its radical, we have immediately that $\sqrt I \subset \sqrt{\sqrt I}.$ For the reverse inclusion, suppose $x\in \sqrt{\sqrt I}$, so there exists some $m>0$ such that $x^m \in \sqrt I$. By definition of $\sqrt I$, there exists some $n>0$ such that $(x^m)^n = x^{mn} \in I$. This implies that $x\in \sqrt{I}$, proving $\sqrt{\sqrt{I}}\subset \sqrt{I}$.

        To show prime ideals are radical, it suffices to show $\sqrt{\frkp}\subset \frkp$ for $\frkp \in \Spec A$. If $x\in \sqrt{\frkp}$, then let $x^n \in \frkp$. By primeness of $\frkp$, we get $x\in \frkp$, so $\sqrt{\frkp}\subset \frkp$ as desired.
\end{proof}
\subsubsection{E}\label{3.4.E}
\begin{proof}
    For $\sqrt{\bigcap_{i=1}^n I_i} \subset  \bigcap_{i=1}^n \sqrt{I_i}$, suppose $x\in A$ is such that $x^m \in I_i$ for each $i$. Then $x \in \sqrt{I_i}$ for each $i$, proving this inclusion.

    For the reverse inclusion, suppose $x\in A$ is such that for each $i$, $x\in \sqrt{I_i}$. Then for each $i$, there is some $m_i>0$ such that $x^{m_i} \in I_i$. Letting $m=\max \{m_1,m_2,\dots,m_n\}$, we then observe $x^m \in I_i$ for each $i$. Then $x\in \sqrt{\bigcap_{i=1}^n I_i}$ as desired.
\end{proof}
\subsubsection{F}\label{3.4.F}
\begin{proof}
    By Exercise \ref{3.2.S}S, we have $\frkN(A/I)=\bigcap_{\frkq \in \Spec A/I} \frkq$. We have $x+I \in \frkN(A/I)$ if and only if there is some $n>0$ with $x^n +I=I$ if and only if $x^n \in I$. Thus $x\in \sqrt I$ if and only if $x+I \in \bigcap_{\frkq \in \Spec A/I} \frkq$. By Exercise \ref{3.2.J}J, $\Spec A/I \cong \{ \frkp \in \Spec A \mid \frkp \supset I\}$. Moreover, by the proof of this result, the bijections are taking images and preimages under the quotient map. Thus $x+I \in \bigcap_{\frkq \in \Spec A/I}$ if and only if $x\in \bigcap_{\frkp \supset I \in \Spec A} \frkp$. To see this, for the forward direction, if there is some prime $\frkp \supset I$ such that $x\notin \frkp$, we then get $x+I\notin \frkp/I\in \Spec A/I$. For the reverse direction, if $x\in \frkp \supset I$, then $x+I \in \frkp/I$, and as every $\frkq \in \Spec A/I$ is realized as the quotient of a prime containing $I$, the result follows.
\end{proof}
\subsubsection{G}\label{3.4.G}
\begin{proof}
    Recall that $\A^1_k$ is just the set of irreducible polynomials of $k[x]$ (the maximal ideals), along with $0$. As Exercise \ref{3.2.D}D, points out, there are infinitely many points in $\A^1_k$. Because $V(S)=V((S))$ by Exercise \ref{3.4.B}B for an arbitrary subset $S\subset A$, an arbitrary closed set is of the form $V(I)$ for some ideal $I\subset A$.
    We inspect an arbitrary closed set $V(I)$, where we have:

    If $I=0$, $V(I)=\A^1_k$.

    If $I=A$, $V(I)=\emptyset$.

    If $0\subsetneq I \subsetneq A$,  $I=(f)$ for some $f\in k[x]$ since $k[x]$ is a PID, and as $f$ has finitely many irreducible factors, we see $I$ is contained in finitely many maximal ideals. Thus $V(I)=\{\frkm_1,\frkm_2,\dots,\frkm_n\}$, a finite set of maximal ideals, i.e. a finite set of points of $\A^1_k\setminus [0]$. Then we know the only possible closed sets of $\A^1_k$ are the empty set, $\A^1_k$ itself, and a finite set of points of $\A^1_k \setminus [0]$. It thus remains to show every set of the above form is closed. $\emptyset$ and $\A^1_k$ are closed by Exercise \ref{3.4.C}C, and as $\{\frkm_1,\dots,\frkm_n\}=\{\frkm_1\}\cup \dots \cup \{\frkm_n\}$ and each $\{\frkm_i\} = V(\frkm_i)$, we get from Exercise \ref{3.4.C}C that since finite unions of closed sets are closed, indeed $\{\frkm_1,\dots,\frkm_n\}$ is closed. We remark that since the only closed set that contains the generic point $[0]$ is $\A^1_k$, $[0]$ is in every nonempty open set.
\end{proof}
\subsubsection{H}\label{3.4.H}
\begin{proof}
    We take $V(I)$ to be an arbitrary closed set (allowable by Exercise \ref{3.4.C}C as $V(S)=V((S))$), hence it suffices to show $\pi^{-1}(V(I))=\{\frkp \in \Spec A \mid \pi(\frkp)\supset I\}$ is closed. We claim $V(\phi(I))= \pi^{-1}(V(I))$, which would conclude our proof, and remark that $\phi(I)$ may not be an ideal, so we just consider $\phi(I)$ as a set. If $\frkp \in \Spec A$ is such that $\phi^{-1}(\frkp)=\pi(\frkp)\supset I$, by general set theory we get $\frkp \supset \phi(\phi^{-1}(\frkp))\supset \phi(I)$, thus showing $\frkp \in V(\phi(I))$. On the other hand, if $\frkp \supset \phi(I)$, then $\phi^{-1}(\frkp)\supset \phi^{-1}(\phi(I))\supset I$, so $\frkp \in \pi^{-1}(V(I))$.

    Then $\Spec:\Ring \to \Top$ assigns rings to topological spaces and ring homomorphisms to continuous maps in a contravariant fashion. It's clear that the induced map on spectrum of the identity is again the identity, and if $C\xrightarrow{\psi} B \xrightarrow{\phi} A$, then we get $\Spec A \xrightarrow{\pi} \Spec B \xrightarrow{\tau} \Spec C$ and also a map $\sigma:\Spec A \to \Spec C$ induced by $\phi \circ \psi$. Moreover, for $\frkp \in \Spec A$, $\tau \circ \pi(\frkp)=\tau(\phi^{-1}(\frkp))=\psi^{-1}(\phi^{-1}(\frkp))$, and $\sigma(\frkp)=(\phi \circ \psi)^{-1}(\frkp)=\psi^{-1}(\phi^{-1}(\frkp))$ so $\sigma = \tau \circ \pi$, thus showing $\Spec$ is functorial.
\end{proof}
\subsubsection{I}\label{3.4.I}
\begin{proof}
    \begin{enumerate}[(a)]
        \item By Exercise \ref{3.2.N}N, $\Spec B/I$ is in bijection with $\{ \frkp \in \Spec B \mid \frkp \supset I\}$. By definition, the latter subset is $V(I)$, which is closed in $\Spec B$.
        
        We take $S=\{1,f,f^2,\dots\}$, and in addition, by Exercise \ref{3.2.N}N, $\Spec S^{-1} B$ is in bijection with $\{ \frkp \in \Spec B \mid \frkp \cap S = \emptyset \} =\{ \frkp \in \Spec B \mid f\notin \frkp \}$, where $\frkp \cap S = \emptyset$ if $f \notin \frkp$ by primeness of $\frkp$ ($f^n \in \frkp$ implies $f\in \frkp$). To show the latter set is open, we will show its complement $\{\frkp \in \Spec B \mid f \in \frkp \}$ is closed. The subset is $V(\{f\})$, hence closed.

        To show for arbitrary $S$, $\Spec S^{-1}B$ need not be open nor closed in $\Spec B$, we take $B=\Z$ and $S=\Z\setminus \{0\}$ so $S^{-1}B=\Q$. We notice $\Spec \Q = \{ 0\}$ since $\Q$ is a field, so we must show $\{0\}\subset \Spec \Z$ is neither open nor closed. As is mentioned on page 116 in Vakil, the open sets of $\Spec \Z$ are the empty set, and $\Spec \Z$ minus a finite number of ``ordinary" ($(p)$ where $p$ is prime) primes. Indeed $\{0\}$ is not of the form above (since $\Spec \Z$ has infinitely many ``ordinary" primes), so $\{0\}$ is not open. Equivalent to the statement in Vakil is that the closed sets of $\Spec \Z$ are $\Spec \Z$ itself, and a finite number of ``ordinary" primes. Also $\{0\}$ is not of this form, so $\{0\}$ is not closed.
        \item We first consider $\Spec B/I$, and want to show $\Spec B/I$ is homeomorphic to $\{ \frkp \in \Spec B \mid \frkp \supset I\}$ as a subspace of $\Spec B$. By Exercise \ref{3.2.N}N, if we let $\phi:B\twoheadrightarrow B/I$ be the quotient, taking $\phi$ and $\phi^{-1}$ give an inclusion-preserving bijection. Thus we need to show each map is continuous. That $\phi^{-1}:\Spec B/I \to \Spec B$ is continuous is by Exercise \ref{3.4.H}H. Then it remains to show $\phi:\{ \frkp \in \Spec B \mid \frkp \supset I\}\to \Spec B/I$ is continuous. By Exercise \ref{3.4.B}B, it suffices to show $\phi^{-1}(V(J))$ is closed for an ideal $J$ of $B/I$. By definition, $\phi^{-1}(V(J))=\{\frkp \in \Spec B \mid \phi(\frkp)\supset J\}$. In addition, $\phi(\frkp)\supset J$ if and only if $\frkp \supset \phi^{-1}(J)$ because by the proof of Exercise \ref{3.2.J}J, we have $\phi^{-1}(\phi(\frkp))=\frkp$ and we can similarly show $\phi(\phi^{-1}(J))=J$ ($x+I\in J$ implies $x\in \phi^{-1}(J)$ so $x+I \in \phi(\phi^{-1}(J))$, and it's always true that $\phi(\phi^{-1}(J))\subset J$). Thus $\phi^{-1}(V(J))=\{\frkp \in \Spec B \mid \frkp \supset \phi^{-1}(J) \text{ and } \frkp \supset I\}=V(\phi^{-1}(J))\cap \{ \frkp \in \Spec B \mid \frkp \supset I\}$ is closed.

        Now we consider $\Spec S^{-1} B$, and want to show $\Spec S^{-1}B$ is homeomorphic to $\{\frkp \in \Spec B \mid \frkp \cap S = \emptyset\}$ as a subspace of $\Spec B$. By Exercise \ref{3.2.N}N, $\phi:B\to S^{-1}B$ induces a bijection between $\Spec S^{-1}B$ and $\{\frkp \in \Spec B \mid \frkp \cap S = \emptyset\}$, and is continuous by Exercise \ref{3.4.H}H. Then it remains to show the inverse map $\phi:\{\frkp \in \Spec B \mid \frkp \cap S = \emptyset\} \to \Spec S^{-1}B$ sending such a $\frkp$ to $\phi(\frkp)$ is continuous. By Exercise \ref{3.4.B}B, it suffices to show $\phi^{-1}(V(J))=\{ \frkp \in \Spec B \mid \phi(\frkp) \supset J \text{ and } \frkp \cap S = \emptyset\}$ is closed for an arbitrary ideal $J$ of $S^{-1}B$. We claim $\{ \frac{b}{s} \mid b \in \frkp \} = \phi(\frkp)\supset J$ (the first equality by the proof of Exercise \ref{3.2.K}K) if and only if $\frkp \supset \phi^{-1}(J)$, assuming $\frkp \cap S = \emptyset$. By the proof of Exercise \ref{3.2.K}K, we have $\phi^{-1}(\phi(\frkp))=\frkp$, so the forward direction is immediate, and if $\frkp$ contains $\phi^{-1}(J)$, if we fix $\frac{b}{s}\in J$, we get $b\in \frkp$, so indeed $\frac{b}{s}\in \{ \frac{b}{s} \mid b \in \frkp \} = \phi(\frkp)$. Thus $\{ \frkp \in \Spec B \mid \phi(\frkp) \supset J \text{ and } \frkp \cap S = \emptyset\}=\{\frkp \in \Spec B \mid \frkp \supset \phi^{-1}(J) \text{ and } \frkp \cap S = \emptyset\} =V(\phi^{-1}(J)) \cap \{\frkp \in \Spec B \mid  \frkp \cap S = \emptyset\}$ is closed.
    \end{enumerate}
\end{proof}
\subsubsection{J}\label{3.4.J}
\begin{proof}
    $f$ vanishes on $V(I)$ by definition if and only if $f\equiv 0 \mod \frkp$ for every $\frkp \in \Spec B$ containing $I$, i.e. $f\in \frkp $ for every $\frkp \in \Spec B$ containing $I$, i.e. $f\in \bigcap_{\frkp \supset I \in \Spec B} \frkp = \sqrt I$ by Exercise \ref{3.4.F}F.
\end{proof}
\subsubsection{K}\label{3.4.K}
\begin{proof}
    Exercise \ref{3.2.A}A tells us that $\Spec k[x]_{(x)} = \{ 0, (x)\}$: let's classify the closed subsets of $\Spec k[x]_{(x)}$. Let $V((f))$ be an arbitrary closed subset by Exercise \ref{3.4.B}B. If $f=0$, then $V(0)=\Spec k[x]_{(x)}$. If $f\in (x)\setminus 0$ , then $V(f)=\{(x)\}$, and if $f\notin (x)$, $(f)= k[x]_{(x)}$, hence $V((f))=\emptyset$. Then the only possible closed subsets are $\emptyset, \{(x)\}$, and $\Spec k[x]_{(x)}.$ Indeed, each of these are realized as the vanishing set of $1$, $x$, and $0$ respectively, so these are the three closed subsets.
\end{proof}
\subsection{}
\subsubsection{A}\label{3.5.A}
\begin{proof}
    That the distinguished open sets form a base for the Zariski topology is equivalent to showing that every closed set can be written as an intersection of complements of distinguished open sets. Let $V(S)$ be an arbitrary closed set. Then
    \[
    V(S)=\{\frkp \in \Spec A \mid \frkp \supset S\} = \bigcap_{f\in S} \{\frkp \in \Spec A \mid \frkp \ni f\}=\bigcap_{f\in S} \Spec A \setminus D(f).
    \]
\end{proof}
\subsubsection{B}\label{3.5.B}
\begin{proof}
    For $\bigcup_{i\in J} D(f_i) = \Spec A$ implies $(\{f_i\}_{i \in J})=A$, suppose $\bigcup_{i\in J} D(f_i)= \Spec A$. Then for each $\frkp \in \Spec A$, there is some $i\in J$ such that $\frkp \in D(f_i) = \{ \frkq \in \Spec A \mid f_i \notin \frkq\}$, i.e. $f_i \notin \frkp$ for some $i\in J$. Then if $(\{f_i\}_{i \in J})$ was proper, we would have $(\{f_i\}_{i \in J})\subset \frkm$ for some maximal $\frkm \in \Spec A$, which contradicts our assumption that there is some $f_i \notin \frkm$ because each $f_i \in (\{f_i\}_{i \in J})\subset \frkm$. Then indeed $(\{f_i\}_{i \in J})= A$.

    Conversely, suppose $\bigcup_{i\in J} D(f_i) \ne \Spec A$, so there is some $\frkp \in \Spec A$ such that $\frkp \notin D(f_i)$ for each $i$, or equivalently $f_i \in \frkp$ for each $i$. Then $A\supsetneq \frkp \supset (\{f_i\}_{i\in J})$, implying $(\{f_i\}_{i\in J}) \ne A$.

    That $(\{f_i\}_{i \in J})=A$ is equivalent to the existence of some $a_i$ ($i\in J$), all but finitely many $0$, such that $\sum_{i\in J} a_i f_i = 1$ is by definition of $(\{f_i\}_{i \in J})$.
\end{proof}
\subsubsection{C}\label{3.5.C}
\begin{proof}
    Suppose $\Spec A = \bigcup_{j\in J} D(f_j)$. Equivalently by Exercise \ref{3.5.B}B, there are some $a_j$ ($j\in J$), all but finitely many $0$, such that $\sum_{j\in J} a_j f_j = 1$. By reordering $J$, suppose $f_1,\dots,f_n$ are such that $\sum_{j=1}^n a_j f_j=1$. Then no proper ideal can contain every $f_j$ for $j=1,\dots, n$, and for arbitrary $\frkp \in \Spec A$ (being proper), we see there must be some $j=1,\dots, n$ such that $f_j\notin \frkp$, i.e. $\frkp \in D(f_j)$. Since $\frkp \in \Spec A$ was arbitrary, we get $\Spec A = \bigcup_{j=1}^n D(f_j)$. 
\end{proof}
\subsubsection{D}\label{3.5.D}
\begin{proof}
    $\frkp \in D(f)\cap D(g)$ if and only if $f\notin \frkp$ and $g\notin \frkp$, if and only if $fg\notin \frkp$ by primeness, if and only if $\frkp \in D(fg)$.
\end{proof}
\subsubsection{E}\label{3.5.E}
\begin{proof}
    For $D(f)\subset D(g)$ equivalent to $f^n \in (g)$ for some $n\ge 1$ is the same as the statement ``For every $\frkp \in \Spec A$, $f\notin \frkp$ implies $g\notin \frkp$ if and only if $f\in \sqrt{(g)}$". For every $\frkp \in \Spec A$, $f\notin \frkp$ implies $g\notin \frkp$ is equivalent to the statement ``For every $\frkp \in \Spec A$, $g\in \frkp$ implies $f\in \frkp$." By Exercise \ref{3.4.F}F, $\sqrt {(g)} = \bigcap_{\frkp \ni g} \frkp$, hence $f\in \sqrt{(g)}$ if and only if $f$ is in every prime containing $g$ if and only if for every $\frkp \in \Spec A$, $g\in \frkp $ implies $f\in \frkp$.

     $g$ is invertible in $A_f$ if and only if there is some $a\in A$ and $n\ge 0$ such that $1 = \frac{ag}{f^n}$ if and only if there is some $m\ge 0$ such that $f^m(f^n-ag)=0$ if and only if there is some $n\ge 0$ and $a\in A$ with $f^n = ag$ if and only if there is some $n\ge 0$ with $f^n \in (g)$. If $f^0=1 \in (g)$, then $(g)=A$ implies that also $f^1 \in (g)$. Thus there is some $n\ge 0$ with $f^n\in (g)$ if and only if there is some $n\ge 1$ with $f^n \in (g)$.
\end{proof}
\subsubsection{F}\label{3.5.F}
\begin{proof}
    Notice $D(0)=\emptyset$ since every prime contains $0$. Then $D(f)=\emptyset$ if and only if $D(f)\subset D(0)$ if and only if $f\in \sqrt{0}=\frkN(A)$ by Exercise \ref{3.5.E}E.
\end{proof}
\subsubsection{G}\label{3.5.G}
\begin{proof}
    Suppose $B \subset A$. We want to show that the induced map $\pi:\Spec A \to \Spec B$ has dense image. By Exercise \ref{3.5.A}A, the distinguished open sets form a base for the Zariski topology, so our claim is equivalent to showing that for every $\frkp \in \Spec B$ and every $f\in B$ such that $\frkp \in D(f)$, $D(f)\cap \pi(\Spec A) \ne \emptyset$. Suppose this is false, so there is some $\frkp \in \Spec B$ and some $f\notin \frkp$ such that for every $\frkq \in \Spec A$, $\frkq \cap B$ contains $f$, i.e. $f\in \bigcap_{\frkq \in \Spec A} \frkq \cap B = B\cap \frkN(A)$ by Exercise \ref{3.2.S}S. But then $f^n=0 \in \frkp$ for some $n>0$ implies by primeness of $\frkp$ that $f\in \frkp$, a contradiction.
\end{proof}
\subsection{}
\subsubsection{A}\label{3.6.A}
\begin{proof}
    Let $A=A_1\times \dots \times A_n$, and let $p_i:A\twoheadrightarrow A_i$ be the projection. Then for each $i$, we get maps $\phi_i:\Spec A_i \to \Spec A$. By Exercise \ref{3.4.I}I, we have that each $\phi_i$ is a homeomorphism onto the subspace $V( \ker p_i) = \{\frkq \in \Spec A \mid \frkq \supset \prod_{j\ne i} A_j \}=\{\frkq \in \Spec A \mid \frkq \supset \{f_j, j\ne i\} \}$ where $f_j=(\delta_{ij})_{i=1}^n$ and $\delta$ is the Kronecker delta. We claim that $\frkq \in \Spec A$ contains each $f_j$ for $j\ne i$ if and only if $f_i \notin \frkq$. For the forward direction, if a $\frkq$ containing each $f_j$ for $j\ne i$ in addition contained $f_i$, then $\frkq \ni f_1+\dots+f_n =1$, contradicting that $\frkq$ is proper. For the backwards direction, we suppose $\frkq \in \Spec A$ does not contain $f_i$. Then for any $j\ne i$, we have $\frkq \ni 0 = f_i f_j$ implies by primeness that $f_j \in \frkq$. Then we have homeomorphisms $\phi_i:\Spec A_i\to D(f_i)$ as required.

    We now want to show $\Spec A = \coprod_{i=1}^n D(f_i)$. Because the distinguished open sets are open, all that remains is for $i\ne j$, $D(f_i)\cap D(f_j)=\emptyset$ and that $\Spec A = \bigcup_{i=1}^n D(f_i)$. By Exercise \ref{3.5.D}D, we have $D(f_i)\cap D(f_j)=D(f_if_j)=D(0)=\emptyset$. Suppose for a contradiction that some $\frkq \in \Spec A$ contains each $f_i$. Then $\frkq \ni f_1+\dots+f_n = 1$, a contradiction to the assumption that $\frkq$ is proper. Thus $\frkq \in D(f_i)$ for some $i$.
\end{proof}
\subsubsection{B}\label{3.6.B}
\begin{proof}
    \begin{enumerate}[(a)]
        \item Let $U\subset X$ be nonempty and open. If $U$ is not dense in $X$, then there is some $p\in X$ and neighborhood $V$ of $p$ such that $V\cap U= \emptyset$. Then $X$ is reducible.
        \item We first claim that for any open $U\subset X$, $U\cap \bar Z \ne \emptyset$ if and only if $U\cap Z \ne \emptyset$. Let $Z'$ be the set of limit points of $Z$ in $X$, i.e. the set of all points $x\in X$ such that every neighborhood $U$ of $x$ intersects $Z$ at some point other than itself. Because $\bar Z = Z\cup Z'$, so $Z\subset \bar Z$, the backward direction is immediate. For the forward direction, suppose $x\in U\cap \bar Z$. If $x\in Z$, the claim follows. If $x\in Z'$, then as $U$ is a neighborhood of $x$, there is some $y\in U\cap Z$, proving the forwards direction.

        By definition of the subspace topology, $Z$ is irreducible if and only if for every open $U,V\subset X$ with $U\cap Z\ne \emptyset$ and $V\cap Z \ne \emptyset$, $(U\cap V)\cap Z =(U\cap Z) \cap (V\cap Z)  \ne \emptyset$. Similarly, $\bar Z$ is irreducible if and only if for every open $U,V\subset X$ with $U\cap \bar Z\ne \emptyset$ and $V\cap \bar Z \ne \emptyset$, $(U\cap V)\cap \bar Z =(U\cap \bar Z) \cap (V\cap \bar Z)  \ne \emptyset$. The result now follows easily from our claim.
    \end{enumerate}
\end{proof}
\subsubsection{C}\label{3.6.C}
\begin{proof}
    Suppose $\Spec A = V(I)\sqcup V(J)$ and $A$ is a domain, i.e. $0\in \Spec A$. If $0\in V(I)$, then $0\supset I$ implies $I=0$ implies $V(I)=\Spec A$. Similarly if $0\in V(J)$ then $V(J)=\Spec A$. Thus $\Spec A$ cannot be written as the disjoint union of two proper closed subsets.
\end{proof}
\subsubsection{D}\label{3.6.D}
\begin{proof}
    Suppose $X$ is not connected, so $X=U\sqcup V$ with $U,V$ open an nonempty. Then $U\cap V=\emptyset$ by assumption, so $X$ is not irreducible as two nonempty open subsets do not intersect.
\end{proof}
\subsubsection{E}\label{3.6.E}
\begin{proof}
    Let $A=\C[x,y]/(y^2-x^2)$. By Exercise \ref{3.2.E}E, we have $\Spec \C[x,y]$ consists of principally generated ideals and the ideals of the form $(x-a,y-b)$. In addition, by Exercise \ref{3.2.I}I, we have $\Spec A \cong V(y^2-x^2)=V(y-x)\cup V(y+x)$ as a subspace of $\Spec \C[x,y]$, with the equality by Exercise \ref{3.4.C}C. Thus $\Spec A = \{(y-x),(y+x), (x-a,y-a), (x-a, y+a)\}$ where $a\in \C$ ranges over all possible values.
    
    To show $\Spec A$ is connected, suppose $\Spec A = V(I)\cup V(J)$ and $V(J)$ and $V(J)$ are proper subsets of $\Spec A$. Notice that if $(y+x)$ and $(y-x)$ are both in $V(I)$, then $V(I)=\Spec A$ contrary to assumption, so by symmetry we assume $(y+x)\notin V(I)$, so $(y-x)\in V(I)$ and $(y+x)\in V(J)$. But $(x,y)= (y+x) + (y-x)\supset I+J$ means $(x,y)\in V(I+J) = V(I)\cap V(J)$ by Exercise \ref{3.4.C}C. We have shown $\Spec A$ cannot be written as the disjoint union of two proper closed subsets, i.e. $\Spec A$ is connected.

    To show $\Spec A$ is reducible, we will give two nonempty open sets with empty intersection. Let $f=y-x$ and $g=y+x$, so $(y+x)\in D(f)$ and $(y-x)\in D(g)$. But $\emptyset = D(0)=D(y^2-x^2)=D(fg)=D(f)\cap D(g)$ by Exercise \ref{3.5.D}D.
\end{proof}
\subsubsection{F}\label{3.6.F}
\begin{proof}
    \begin{enumerate}[(a)]
        \item We show $I=(wz-xy,wy-x^2, xz-y^2)$ is prime by showing $K[w,x,y,z]/I\cong K[a^3, a^2b, ab^2, b^3]$, a subring of the integral domain $K[a,b]$. We have a map $\phi:K[w,x,y,z]\twoheadrightarrow K[a^3, a^2b, ab^2, b^3]$ taking the tuple $(w,x,y,z)$ to the tuple $(a^3, a^2b, ab^2, b^3)$. Indeed each generator of $I$ is in the kernel of $\phi$, and if we can show $\ker \phi=I$, we are done. We have $f\equiv g \mod I$ where $g$ has monomials indivisible by any of $wz, wy,$ and $xz$ (a vector space basis for $K[w,x,y,z]/I$ consists of monomials indivisible by those three). A monomial $w^i x^j y^k z^l$ is indivisible by each of $wz, wy, xz$ if and only if its not the case that $i\ge 1$ and $k\ge 1$ or $j\ge 1$ and $l\ge 1$ or $i\ge 1$ and $l\ge 1$ if and only if $i=0$ and $j=0$ or $i=0$ and $l=0$ or $k=0$ and $l=0$, i.e. the set of monomials of the form $y^kz^l$ or $x^jy^k$ or $w^ix^j$. Then $g$ is a $K$-linear combination of $\{y^kz^l, x^jy^k, w^ix^j \mid i,j,k,l\in \N\}$. We will now show that $\phi$ is injective on these monomials, thus implying $g=0$ and then that $f\in I$. We compute that $\phi(y^k z^l)=a^k b^{2k+3l}$, $\phi(x^j y^k)= a^{2j+k}b^{j+2k}$, and $\phi(w^i x^j)=a^{3i+2j}b^j$. 
        
        It's clear $\phi(y^k z^l)=\phi(y^{k'} z^{l'})$ implies $k=k'$ and $l=l'$, and similarly $\phi(w^i x^j)=\phi(w^{i'} x^{j'})$ implies $i=i'$ and $j=j'$.

        \vspace{0.1in}
        $a^{2j+k} b^{j+2k}=a^{2j'+k'} b^{j'+2k'}$ if and only if $2j+k=2j'+k'$ and $j+2k=j'+2k'$. These equations imply $j'=j+2k-2k'$ which implies $2j+k=2j+4k-4k'+k'$, which is true if and only if $k=k'$. Now that $k=k'$, we see $j=j'$ as well.

        \vspace{0.1in}
        $a^k b^{2k+3l}= a^{2j'+k'} b^{j'+2k'}$ if and only if $k=2j'+k'$ and $2k+3l=j'+2k'$. These equations imply that $4j'+2k'+3l=j'+2k'$, or equivalently $3(j'+l)=0$, thus $j'=l=0$ and $k=k'$.
        
        \vspace{0.1in}
        $a^k b^{2k+3l}=a^{3i'+2j'} b^{j'}$ if and only if $k=3i'+2j'$ and $2k+3l=j'$. These equations imply that $6i'+4j'+3l=j'$ or equivalently $2i'+j'+l=0$, hence $i'=j'=l=0$ and thus $k=0$ as well.

        \vspace{0.1in}
        Lastly, $a^{2j+k}b^{j+2k}=a^{3i'+2j'} b^{j'}$ if and only if $2j+k=3i'+2j'$ and $j+2k=j'$. These equations imply that $2j+k=3i'+2j+4k$, or equivalently $3(i'+k)=0$ so $i'=k=0$. Thus $j=j'$ as well.

        \vspace{0.1in}
        We have now shown that if the images under $\phi$ of the monomials $g$ is written in are linearly dependent, they must have had $0$ coefficients to begin with. But they all must cancel by hypothesis that $g\in \ker \phi$, so $g=0$ as desired.
        \item The matrix 
        \[
        \begin{pmatrix}
            x_0 & \dots & x_{n-1}\\
            x_1 & \dots & x_{n}
        \end{pmatrix}
        \]
        has rank at most one if and only if any two columns are scalar multiples of each other, which is true if and only if the determinant of $\begin{pmatrix}
            x_i & x_{j}\\
            x_{i+1} & x_{j+1}
        \end{pmatrix}$
        is zero for every $0\le i<j \le n-1$, i.e. $h_{ij} \coloneqq x_i x_{j+1}- x_j x_{i+1}=0$. We claim the ideal $I$ generated by all $h_{ij}$'s is prime. We will show that $K[x_0,\dots, x_n]/I$ injects into a subring of a domain, thus implying the source is a domain as desired. We define a morphism of $K$-algebras $\phi:K[x_0,\dots,x_{n}]\to K[a,b]$ defined by $x_i \mapsto a^{n-i} b^{i}$ for each $0\le i\le n$. Indeed, each $h_{ij}\in \ker \phi$ because \begin{align*}
            &\phi(h_{ij})=\phi(x_i) \phi(x_{j+1}) - \phi(x_j) \phi(x_{i+1}) = a^{n-i} b^i a^{n-j-1} b^{j+1} - a^{n-j} b^j a^{n-i-1} b^{i+1}\\
            &=a^{2n-i-j-1} b^{i+j+1}- a^{2n-i-j-1} b^{i+j+1}=0.
        \end{align*}
        We now claim that $\ker \phi \subset I$, which would then prove $I$ is prime. Fix $f\in \ker \phi$, and so there exists some $g$ such that $f\equiv g \mod I$ and that each monomial in $g$ is indivisible by each $x_ix_{j+1}$ for $0\le i,j\le n-1$. This is allowed because $K[x_0,\dots, x_n]/I$ is spanned as a $K$-vector space by the monomials indivisible by each $x_ix_{j+1}$. Our goal is now to show that $g=0$, using the fact that $g\in \ker \phi$ since $I\subset \ker \phi$ and $f\in \ker \phi$. Notice that a monomial $x_0^{k_0} \dots x_n ^{k_n}$ is indivisible by $x_i x_{j+1}$ for each $0\le i < j \le n-1$ only if it's not true that there are such indices $i,j$ with $k_i \ge 1$ and $k_{j+1}\ge 1$, or equivalently for every $0\le i<j\le n-1$, $k_i = 0$ or $k_{j+1}=0$. Then for every index $0\le i \le n-1$ if $k_i\ne 0$, we observe that every $k_j=0$ for $j\ge i+2$. Similarly, for every index $1\le i \le n$, if $k_i \ne 0$ then $k_j = 0$ for every $j\le i-2$. Therefore $g$ is a $K$-linear combination of monomials of the form $x_{i-1}^ \alpha x_i ^\beta x_{i+1}^\gamma$. Moreover, it cannot be that $\alpha, \gamma \ge 1$ otherwise the monomial is divisible by $x_{i-1} x_{i+1}$. Thus $g$ is a $K$-linear combination of monomials of the form $x_i^j x_{i+1}^k$ with $0\le i \le n-1$ and $j,k \in \N$. We will now show that $\phi$ preserves linear independence of this set, which would then imply $g=0$ as $g\in \ker \phi$. Because $\phi$ takes monomials to monomials (which are always linearly independent), it suffices to show $\phi$ is injective on the monomials $g$ is in. We will achieve this by looking at two cases, one where the index $i$ is the same, and one where it is different.

        \vspace{0.1in}
        Suppose \[
        a^{(n-i)(j+k)-k}b^{i(j+k)+k}=\phi(x_i^j x_{i+1}^k)= \phi(x_i^{j'} x_{i+1}^{k'}) = a^{(n-i)(j'+k')-k'}b^{i(j'+k')+k'},
        \]
        or equivalently $(n-i)(j+k)-k = (n-i)(j'+k')-k'$ and $i(j+k)+k = i(j'+k')+k'$. Then $k'=i(j+k-j'-k')+k$, so $(n-i)(j+k)-k=(n-i)(j'+k')-(j+k-j'-k')-k$, or equivalently $n(j+k)=n(j'+k')$ so $j+k=j'+k'$. Substituting back, we see $k=k'$, which then implies $j=j'$.

        \vspace{0.1in}
        Suppose \[
        a^{(n-i)(j+k)-k}b^{i(j+k)+k}=\phi(x_i^j x_{i+1}^k)= \phi(x_{i'}^{j'} x_{i'+1}^{k'}) = a^{(n-i')(j'+k')-k'}b^{i'(j'+k')+k'},
        \]
        or equivalently $(n-i)(j+k)-k = (n-i')(j'+k')-k'$ and $i(j+k)+k = i'(j'+k')+k'$ with $i'>i$. Then $k'=i(j+k)-i'(j'+k')+k$, so $(n-i)(j+k)-k=(n-i')(j'+k')-i(j+k)+i'(j'+k')-k$, or equivalently $$n(j+k)=n(j'+k').$$ This implies $j+k=j'+k'$, so substituting back, we see $$k=(i'-i)(j+k)+k'.$$ As $i'-i\ge 1$ and $j+k\ge k$, we see $(i'-i)(j+k)\ge k$ with equality if and only if $i'-i=1$ and $j=0$. As $k'\ge 0$, we get $i'=i+1$ and $j=k'=0$. Then our original monomials are $x_{i+1}^k$ and $x_{i+1}^{j'}$. By our previous work, we get $k=j'$ as desired.

        \vspace{0.1in}
        As mentioned before, this shows $g=0$ so $f\in I$.
    \end{enumerate}
\end{proof}
\subsubsection{G}\label{3.6.G}
\begin{proof}
    \begin{enumerate}[(a)]
        \item Suppose $\{U_i\}_{i\in I}$ is an open cover of $\Spec A$. By Exercise \ref{3.5.A}A, for each $i\in I$, we may write $U_i = \bigcup_{j\in J_i} D(f_{ij})$. Then as $\{D(f_{ij})\}_{i\in I, j\in J_i}$ is an open cover of $\Spec A$ by distinguished open sets, by Exercise \ref{3.5.C}C, there is a finite subset $I'\subset I$ such that for each $i\in I'$, there is a finite subset $J'_i \subset J_i$, such that $\{ D(f_{ij})\}_{i\in I', j\in J'_i}$ is an open cover of $\Spec A$. We claim that $\{U_i\}_{i\in I'}$ covers $\Spec A$. \iffalse For any $i\in I'$, we have
        \[
        U_i = U_i \cap \Spec A = \bigcup_{j\in J_i} D(f_{ij}) \cap \bigcup_{k\in I'} \bigcup_{j\in J'_k} D(f_{kj}).
        \]
        Thus
        \[
        \bigcup_{i\in I'} \bigcup_{j\in J_i} D(f_{ij}) \cap \bigcup_{k\in I'} \bigcup_{l\in J_{k}'} D(f_{kl}).
        \]
        \fi
        If we fix $x\in \Spec A$, then $x\in D(f_{ij})$ for some $i\in I'$ and $j\in J'_i \subset J_i$. As $ \bigcup_{j\in J_i} D(f_{ij})=U_i$, we get $x\in U_i$, and thus indeed $\{U_i\}_{i\in I'}$ covers $\Spec A$.
        \item Let $A=k[x_1,x_2,\dots]$ for a field $k$, and let $\frkm=(x_1,x_2,\dots)$ be the irrelevant ideal. We also let $\frkp_n=(x_1,x_2,\dots, x_n)$ for each positive integer $n$. We claim the $\Spec A \setminus V(\frkp_n)$'s cover $\Spec A \setminus V(\frkm)$, all of which are open, i.e. $\Spec A \setminus \bigcap_{n=1}^\infty V(\frkp_n)=\bigcup_{n=1}^\infty \Spec A \setminus V(\frkp_n) = \Spec A \setminus V(\frkm)$, which is equivalent to the claim that $\bigcap_{n=1}^\infty V(\frkp_n)=V(\frkm)$. This is simply because $\frkm= \bigcup_{n=1}^\infty \frkp_n$.

        However, we will also show that if $J\subset \N$ is a finite subset, then $\bigcup_{j\in J} \Spec A \setminus V(\frkp_j) = \Spec A \setminus \bigcap_{j\in J} V(\frkp_j) \ne \Spec A \setminus V(\frkm)$, or equivalently $\bigcap_{j\in J} V(\frkp_j) \ne V(\frkm)$. Notice that for each $i>j$, $V(\frkp_i)\subset V(\frkp_j)$ because $\frkp_i\supset \frkp_j$. Therefore if $m=\max J$, we get $\bigcap_{j\in J} V(\frkp_j)=V(\frkp_m)$. However, $\frkp_m\in V(\frkp_m)$ and $\frkp_m \subsetneq \frkm$ implies $\frkp_m \notin V(\frkm)$. Therefore $\Spec A \setminus V(\frkm)$ has an open cover not admitting a finite subcover.
    \end{enumerate}
\end{proof}
\subsubsection{H}\label{3.6.H}
\begin{proof}
    \begin{enumerate}[(a)]
        \item Suppose $X=\bigcup_{i=1}^n X_i$, where each $X_i$ is quasicompact, and we have an open cover $\{U_i\}_{i\in I} U_i$. Then for each $j=1,\dots, n$,
        \[
        X_j= X\cap X_j = \bigcup_{i\in I} X_i \cap X_j,
        \]
        so there is a finite $I_j\subset I$ such that $X_j=\bigcup_{i\in I_j} U_i \cap X_i$. To show the set of all $U_i$'s for $i \in \bigcup_{j=1}^n I_j$ covers $X$ (and the number of such $i$'s are finite because each $I_j$ is also finite), if we pick any $x\in X$, then $x\in X_j$ for some $j=1,\dots,n$, and then $x\in U_i\cap X_j$ for some $i\in I_j$.
        \item If $Z\subset X$ is closed and $X$ is quasicompact, then let $\{Z\cap U_i\}_{i\in I}$ be an open cover of $Z$ (with the subspace topology). Then $\{U_i\}_{i\in I} \cup \{X\setminus Z\}$ is an open cover of $X$, so there is some finite $J\subset I$ such that $\{U_j\}_{j\in J} \cup \{X\setminus Z\}$ covers $X$. Then $\{U_j\}_{j\in J}$ covers $Z$, hence $\{Z\cap U_j\}_{j\in J}$ is a finite subcover of $Z$.
    \end{enumerate}
\end{proof}
\subsubsection{I}\label{3.6.I}
\begin{proof}
    On one hand, suppose $\frkp\in \Spec A$ is a closed point, so there is an ideal $I$ such that $\{\frkp\} = V(I)$. Because there is a maximal ideal $\frkm$ containing $I$, we see $\frkm\in V(I)$, and thus $\frkp=\frkm$ so $\frkp$ is maximal.

    Conversely, if $\frkm\in \Spec A$ is maximal, then $\{\frkm\}=V(\frkm)$ because no prime can contain $\frkm$ other than itself.
\end{proof}
\subsubsection{J}\label{3.6.J}
\begin{proof}
    \begin{enumerate}[(a)]
        \item As suggested, we will show that for any $f\in A\setminus \frkN(A)$, $D(f)$ contains a maximal ideal. We notice that $A_f$ is a finitely generated $k$-algebra as well by the map $k[x_1,\dots,x_{n+1}]\twoheadrightarrow A_f$ sending $x_i$ to $\phi(x_i)$ (where $\phi:k[x_1,\dots,x_n]\twoheadrightarrow A$ by hypothesis) for each $1\le i \le n$, and $x_{n+1}\mapsto \frac{1}{f}$. In addition, $A_f\ne 0$ else $0=1$ in $A_f$, which would imply that $f^m=0$, or equivalently $f\in \frkN(A)=\bigcap_{\frkp \in \Spec A} \frkp$ by Exercise \ref{3.2.S}S, i.e. $D(f)=\emptyset$. Then there exists a maximal $\frkm \in \Spec A_f \cong D(f)$ by Exercise \ref{3.2.N}N. We will show that $\frkm \cap A \in D(f)$ is maximal, which would prove the desired result.
        
        Notice that if $A\hookrightarrow B \hookrightarrow C$ is a chain of subrings and $A\hookrightarrow C$ is a module-finite extension, then $B\hookrightarrow C$ is also a module-finite extension. Then as we have the chain of inclusions $k\hookrightarrow A/(\frkm \cap A) \hookrightarrow A_f/\frkm$ and $A_f/\frkm$ is a finite field extension of $k$ by the Nullstellensatz, it follows that $A_f/\frkm$ is a finite $A/(\frkm\cap A)$-module, or equivalently $A/(\frkm \cap A) \hookrightarrow A_f/\frkm$ is an integral extension. By Theorem 5.7 of \cite{Atiyah-Macdonald}, stating that if $A\hookrightarrow B$ is an integral extension of rings, then $A$ is a field if and only if $B$ is. We then get that $A/(\frkm \cap A)$ is a field, i.e. $\frkm \cap A$ is maximal as needed.
        \item We will show the $k$-algebra $k[x]_{(x)}$ does not have its closed points dense. By Exercise \ref{3.4.K}K, we have $\Spec k[x]_{(x)}=\{0, (x)\}$. Then $D(x)=\{0\}$, and $0$ is not a closed point by Exercise \ref{3.6.I}I since $0$ is not maximal. Then $0\in \Spec k[x]_{(x)}$ has a neighborhood $D(x)$ with no closed point.
    \end{enumerate}
\end{proof}
\subsubsection{K}\label{3.6.K}
\begin{proof}
    If $f \ne g$ in $A$, then $f-g\ne 0$, and as $\frkN(A)=0$, we have $D(f-g)\ne \emptyset$ (a distinguished open subset is empty if and only if the element is nilpotent by Exercise \ref{3.2.S}S). By Exercise \ref{3.6.J}J(a), there is a maximal ideal $\frkm \in D(f-g)$. Then $f-g\not \equiv 0 \mod \frkm$, so $f\not \equiv g \mod \frkm$, so $f$ and $g$ differ at a closed point. Note there was no need for the algebraically closed assumption.
\end{proof}
\subsubsection{L}\label{3.6.L}
\begin{proof}
    For one direction, assuming $\frkq$ is a specialization of $\frkp$ if and only if $\frkq \in \bigcap_{V(I) \ni \frkp} V(I)=\overline{\{\frkp\}}$, then $\frkq \in V(\frkp)$, i.e. $\frkq \supset \frkp$.

    Conversely if $\frkq \supset \frkp$, then for any $V(I)$ containing $\frkp$, we would then see $\frkq \supset \frkp \supset I$, hence $\frkq \in V(I)$ as well. Then $\frkq \in \bigcap_{V(I) \ni \frkp} V(I) = \overline{\{\frkp\}}$.

    Then $\frkq \in V(\frkp)$ if and only if $\frkq \supset \frkp$ if and only if $\frkq \in \overline{\{\frkp\}}$, hence $V(\frkp)=\overline{\{\frkp\}}$.
    
\end{proof}
\subsubsection{M}\label{3.6.M}
\begin{proof}
    By Exercise \ref{3.6.L}L, it suffices to show $(y-x^2)$ is prime. But $\C[x,y]/(y-x^2)\cong \C[x]$ is a domain is equivalent to $(y-x^2)$ being prime.
\end{proof}
\subsubsection{N}\label{3.6.N}
\begin{proof}
    Letting $q\in K$ be arbitrary, we have $K=\overline{\{p\}}=\{p\} \cup \{p\}'$ where here $\{p\}'$ denotes the set of limit points of $\{p\}$, i.e. the set of all elements of $X\setminus \{p\}$ whose neighborhoods all contain $p$. Then either $q=p$ or every neighborhood of $q$ contains $p$, and in either event the claim holds.

    Now for any $q\in X\setminus K$, as $K$ is closed, $X\setminus K$ is a neighborhood of $q$ not containing $p$.
\end{proof}
\subsubsection{O}\label{3.6.O}
\begin{proof}
    Fix $p\in X$, and let $I$ be the set of irreducible subsets of $X$ containing $p$, partially ordered by inclusion. If $Z_1\subset Z_2 \subset \dots$ is a chain in $I$, there is an upper bound in $I$, namely $Z=\bigcup_i Z_i$. This is irreducible because if we have some closed $U,V\subset X$ where $U\cap Z\subsetneq Z$ and $V\cap Z\subsetneq Z$, then for large indices $i$, $U\cap Z_i \subsetneq Z_i$ and $V\cap Z_i \subsetneq Z_i$ because $U\cap Z = U\cap \bigcup_i Z_i = \bigcup_i U\cap Z_i$ (the same is true replacing $U$ by $V$). 
    
    Then we cannot write $Z=(U\cap Z) \cup (V\cap Z)$, else we would get
    \[
    Z_i=Z_i \cap Z = Z_i \cap ((U\cup V)\cap Z)=(U\cap Z_i)\cup (V\cap Z_i)
    \]
    for all $i$, a contradiction to the irreducibility of $Z_i$ for large $i$.

    Then Zorn's Lemma gives an irreducible set $Z$ containing $p$, maximal in $I$. Then if $Z'\supset Z$ and $Z'$ is irreducible, then $p\in Z\subset Z'$ implies $Z'\in I$ so by maximality $Z'=Z$. Thus $Z$ is a maximal irreducible subset that also contains $p$, i.e. an irreducible component containing $p$.
\end{proof}
\subsubsection{P}\label{3.6.P}
\begin{proof}
    By the Hilbert Basis theorem 3.6.17, $\C[x,y]$ is a Noetherian ring. Then by Exercise \ref{3.6.T}T
    we get that $\A^2_\C$ is a Noetherian topological space.

    However, $\C^2$ with the classical topology is not Noetherian because for each $n\in \N$, $S_n=\{ (z,0)\in \C^2\mid z\in \N_{\ge n}\}$ is closed since for any $(z_1,z_2)\in \C^2\setminus S_n$, if $z_2\ne 0$ we take $B_{|z_2|}(z_1,z_2)$ which does not even intersect $\C\times \{0\}$, and if $z_2=0$, then $z_1\notin \N_{\ge n}$, in which case we may find the integer $m$ closest to $z_1$, and then $B_{|z_1-m|}(z_1,0)$ does not even intersect $\Z \times \{0\}$. Then we have $$S_1\supsetneq S_2\supsetneq S_3 \supsetneq \dots,$$showing $\C^2$ is not Noetherian.
\end{proof}
\subsubsection{Q}\label{3.6.Q}
\begin{proof}
    \begin{enumerate}[(i)]
        \item To show that every connected component of a topological space $X$ is the union of irreducible components of $X$, we first recall Remark 3.6.13, which says that connected components are closed. Thus closed subsets of a connected component $C$ of $X$ are just closed subsets of $X$ contained in $C$. Now by Exercise \ref{3.6.O}O, we can write $C=\bigcup_i Z_i$ where each $Z_i \subset C$ is an irreducible component of $C$. We will now show that each $Z_i$ is actually an irreducible component of $X$. 
        
        For any fixed index $i$, suppose $Z_i = U \cup V$ where $U$ and $V$ are closed subsets of $X$. It follows that $U$ and $V$ are closed subsets of $C$, and thus by irreduciblility of $Z_i$ in $C$, we get $U=Z_i$ or $V=Z_i$, so $Z_i$ is an irreducible closed subset of $X$. Now suppose we have some irreducible component $Z$ of $X$ containing $Z_i$. Supposing for a contradiction that $Z_i\subsetneq Z$, then we see $Z\nsubset C$, else we would contradict maximality of $Z_i$ in $C$. Then $Z\cup C$ must be disconnected because $C$ is a connected component, so $Z\cup C = (U \cap (Z\cup C)) \sqcup (V \cap (Z\cup C))$ for some open $U,V$ in $X$ with $U\cap (Z\cup C) \ne \emptyset$ and $V\cap (Z\cup C) \ne \emptyset$. In other words, $Z\cup C\subset U\cup V$ and $U\cap V \subset X\setminus (Z\cup C)$. In particular, $Z\subset U\cup V$ and $U\cap V \subset X\setminus Z$. Hence $U\cap Z \ne \emptyset$ because otherwise we would have $Z\cup C = (U\cap C)\sqcup (V\cap(Z\cup C))$, so by intersecting each side with $C$, we have $C=(U\cap C) \sqcup (V\cap C)$ and $U\cap C\ne \emptyset$ implies that $V\cap C = \emptyset$ by connectedness of $C$. However, having $U$ and $V$ cover $Z\cup C$ and being disjoint on $Z\cup C$ is impossible because $\emptyset \ne Z_i \subset Z\cap C$, and so for an element $x\in Z_i$, we get $x\in U$ or $x\in V$, but then as $x\in Z\cap C$, we get $U\cap Z \ne \emptyset$ or $V\cap C \ne \emptyset$, a contradiction. Similarly, it must be that $V\cap Z \ne \emptyset$. But then we have $Z=(U\cap Z)\sqcup (V\cap Z)$ with each side nonempty, which contradicts Exercise \ref{3.6.D}D.

        We have now proven that $Z_i=Z$, so $Z_i$ is indeed an irreducible component of $X$, which gives the result since the index $i$ was arbitrary.

        \item Now suppose $U$ is simultaneously closed and open in $X$. For each $p\in X$, there is a connected component $Z_p$ of $X$ containing $p$. Then $U\subset \bigcup_{p\in U} Z_p$. For fixed $p\in U$, $Z_p\cap U$ is an open subset of $Z_p$. In addition, $Z_p \cap (X\setminus U)$ is an open subset of $Z_p$ (because $p \notin X\setminus U$ but $p\in Z_p$). But now we see that
        \[
        (Z_p \cap U) \cup (Z_p \cap (X\setminus U)) = Z_p
        \]
        and
        \[
        (Z_p \cap U) \cap (Z_p \cap (X\setminus U)) = \emptyset,
        \]
        so as 
        \[
        Z_p = (Z_p \cap U) \sqcup (Z_p \cap X\setminus U),
        \]
        either $Z_p \cap U = \emptyset$ or $Z_p \cap X\setminus U= \emptyset$ by connectedness of $Z_p$. But $p\in Z_p\cap U$ implies that the latter intersection is empty, or equivalently $Z_p\subset U$. As $p\in U$ was arbitrary, we get $U=\bigcup_{p\in U} Z_p$.
        \item Now suppose $X$ is a Noetherian topological space. Each connected component of $X$ can be written uniquely as a finite union of irreducible subsets of $X$ contained in the connected component by Proposition 3.6.15 and (i), and as $X$ has only finitely many irreducible components by the same proposition, it follows that $X$ only has finitely many connected components because distinct connected components are disjoint. Then $X=\coprod_{i=1}^n Z_i$ where each $Z_i$ is a connected component of $X$ (and hence closed). Thus any union of connected components is both open and closed (a finite union of closed subsets whose complement is also a finite union of closed subsets).
    \end{enumerate}
\end{proof}
\subsubsection{R}\label{3.6.R}
\begin{proof}
    Immediate by Exercise \ref{3.6.S}S.
\end{proof}
\subsubsection{S}\label{3.6.S}
\begin{proof}
    First suppose the ascending chain condition fails, so there is an infinite ascending chain $I_1 \subsetneq I_2 \subsetneq \dots$ of ideals in $A$. Then $I=\bigcup_{n=1}^\infty I_n$ cannot be finitely generated; otherwise $I = (f_1, \dots, f_k)$ for some $f_1,\dots, f_k\in A$. Moreover, there exists some $m\in \N$ such that each $f_j \in I_m$ because each $f_i$ is in $I=\bigcup_{n=1}^\infty I_n$. Then for each $n\ge m$, we have $$I_n\supset I_m \supset (f_1,\dots,f_k)=\bigcup_{l=1}^\infty I_l \supset I_n$$ so $I_n=I_m$, and thus the chain becomes stationary past $m$. This is a contradiction, so $I$ is not finitely generated.

    

    Conversely, if there is an ideal $I$ of $A$ that is not finitely generated, we inductively define $f_1=0$, and for $n\in \N$, letting $I_n=(f_1,\dots, f_n)$, pick $f_{n+1} \in I\setminus I_n$ (such an element must always exist otherwise we get a finite generating set for $I$, which is impossible by assumption). Then we have constructed an infinite ascending chain $$I_1 \subsetneq I_2 \subsetneq \dots,$$ demonstrating the ascending chain condition on ideals fails.
\end{proof}
\subsubsection{T}\label{3.6.T}
\begin{proof}
    By Exercise \ref{3.4.B}B, we may take
    \[
    V(I_1)\supset V(I_2)\supset \dots
    \]
    to be an arbitrary descending chain of closed subsets in $\Spec A$ where each $I_n$ is an ideal of $A$. For arbitrary ideal $I,J$ of $A$, Exercise \ref{3.4.F}F tells us that $\sqrt{I}=\bigcap_{\frkp \supset I} \frkp$, so we see $V(I)\subset V(J)$ if and only if $\sqrt{I}\supset \sqrt{J}$. The forward direction is clear since the set of primes being intersected for $\sqrt{I}$ is contained in the set of primes being intersected for $\sqrt{J}$. For the backward direction, assume we have some $\frkp\in V(I)$ and that $\sqrt{I}\supset \sqrt{J}$. Then
    \[
    \frkp \supset \sqrt{I}\supset \sqrt{J} \supset J
    \]
    so $\frkp \in V(J)$ as well. Then we have an infinite ascending chain 
    \[
    \sqrt{I_1}\subset \sqrt{I_2}\subset \dots
    \]
    of ideals in $A$, which by hypothesis stabilizes at some $m\in \N$. Then for every $k\ge m$, $\sqrt{I_m}=\sqrt{I_k}$ implies that $V(I_m)=V(I_k)$, so the chain of closed sets stabilizes at $m$.

    For a ring $A$ with $\Spec A$ not a Noetherian space, we let $A=k[x_1,x_2,\dots]$ for $k$ a field. Then $\Spec A$ contains the descending chain
    \[
    V(x_1)\supsetneq V(x_1,x_2)\supsetneq V(x_1,x_2,x_3)\supsetneq \dots
    \]
    where $V(x_1,\dots,x_n)\supsetneq V(x_1,\dots, x_{n+1})$ because both $(x_1,\dots, x_n)$ and $(x_1,\dots, x_{n+1})$ are prime (and hence primary), we see $(x_1,\dots, x_n)\subsetneq  (x_1, \dots, x_{n+1})$ implies $V(x_1, \dots, x_n) \supsetneq V(x_1, \dots, x_{n+1})$.
\end{proof}
\subsubsection{U}\label{3.6.U}
\begin{proof}
    Suppose $X$ is a topological space and $A\subset X$ is any subspace. We will show that if $A$ is not Noetherian, then neither is $X$. By assumption, there exists an infinite descending chain $A \cap Z_1 \supsetneq A\cap Z_2 \supsetneq \dots$ where each $Z_i$ is closed in $X$. %Then for each $n\in \N$, $A\cap Z_n \supsetneq A\cap Z_{n+1}$ implies there exists some $x\in X$ such that for every $i$
    Then for each $n\in \N$,
    \[
    \bigcap_{i=1}^{n+1} Z_i\subsetneq \bigcap_{i=1}^n Z_i,
    \]
    where containment is clear, and the containment must be proper else we would see that
    \[
    A\cap Z_{n+1}= \bigcap_{i=1}^{n+1} A \cap Z_i = A \cap \bigcap_{i=1}^{n+1} Z_i = A \cap \bigcap_{i=1}^n Z_i = \bigcap_{i=1}^n A\cap Z_i = A\cap Z_n,
    \]
    contradicting our assumptions. Then we have an infinite descending chain
    \[
    Z_1 \supsetneq Z_1 \cap Z_2 \supsetneq Z_1 \cap Z_2 \cap Z_3 \supsetneq \dots
    \]
    of closed sets in $X$.
\end{proof}
\subsubsection{V}\label{3.6.V}
\begin{proof}
    The equivalence of the ascending chain condition on submodules and every submodule being finitely generated is a direct generalization of Exercise \ref{3.6.S}S by replacing ``ideal" by ``submodule" and the elements $f_i \in A$ instead by elements in the $A$-module $M$.
\end{proof}
\subsubsection{W}\label{3.6.W}
\begin{proof}
    Suppose 
    \[
    0\to M' \to M \to M'' \to 0
    \]
    is an exact sequence of $A$-modules (and we will take $M'\subset M$ and $M''=M/M'$ by the first isomorphism theorem). Given an ascending chain of submodules $M_1\subset M_2 \subset \dots$ of $M$, we get two more chains
    \[
    M_1\cap M' \subset M_2 \cap M' \subset \dots
    \]
    and
    \[
    M_1+M' \subset M_2+M'\subset \dots
    \]
    of submodules of $M'$ and $M''$ respectively. Then assuming $M'$ and $M''$ are both Noetherian $A$-modules, there is some $m\in \N$ such that both chains have stabilized at $m$. In addition, we have a short exact sequence in $\Com_A:$
    \begin{center}
        \begin{tikzcd}
            & \vdots\ar{d}& \vdots \ar{d}& \vdots \ar{d} &  \\
            0 \ar{r}& M_i \cap M' \ar{r} \ar[d]& M_i \ar{r}\ar[d]& M_i+M' \ar{r}\ar[d]&0\\
            0 \ar{r} & M_{i+1} \cap M' \ar[d]\ar{r}& M_{i+1} \ar{r}\ar[d]& M_{i+1}+M' \ar{r}\ar[d]&0\\
             &\vdots&\vdots&\vdots&
        \end{tikzcd}
    \end{center}
    where commutativity of the left square is because each map is simply an inclusion, and each path of the right square sends an element $m\in M_i$ to $m+M'$. Then for $n \ge m$, the below diagram commutes and is exact on the horizontals:
    \begin{center}
        \begin{tikzcd}
            0 \ar{r}& M_m \cap M' \ar{r} \ar[d, "\id"]& M_m \ar{r}\ar[d]& M_m+M' \ar{r}\ar[d, "\id"]&0\\
            0 \ar{r} & M_{n} \cap M'\ar{r}& M_{n} \ar{r}& M_{n}+M' \ar{r}&0.
        \end{tikzcd}
    \end{center}
    By the five lemma, we see that $M_m\to M_n$ is also an isomorphism, and being an inclusion, it is the identity. Thus the original chain $M_1\subset M_2 \subset \dots $ stabilizes at $m$.

    Conversely, since submodules of $M'$ are submodules of $M$ and submodules of $M''$ correspond to submodules of $M$ containing $M'$ by the lattice isomorphism theorem, it's clear that if $M$ is Noetherian than so too are $M'$ and $M''$.
\end{proof}
\subsubsection{X}\label{3.6.X}
\begin{proof}
    We will show that if $M$ and $N$ are Noetherian $A$-modules, than $M\oplus N$ is also a Noetherian $A$-module. We get a short exact sequence
    \[
    0\to M\oplus 0\to M\oplus N \to 0\oplus N \to 0,
    \]
    and immediately notice that $M\oplus 0$ and $0\oplus N$ are both Noetherian, being isomorphic to $M$ and $N$ respectively. Then by Exercise \ref{3.6.W}W, we get that $M\oplus N$ is also Noetherian.

    By induction, any finite direct sum of Noetherian modules is Noetherian, and because a ring $A$ is a Noetherian $A$-module if and only if $A$ is a Noetherian ring, we immediately get that $A^{\oplus n}$ is Noetherian.
\end{proof}
\subsubsection{Y}\label{3.6.Y}
\begin{proof}
    Suppose $A$ is a Noetherian ring and $M$ is finitely generated by $f_1,\dots, f_n$ as an $A$-module. Given an ascending chain $M_1\subset M_2 \subset \dots$ of submodules of $M$, for each index $k$ we let $I_k = \{a_1\oplus \dots \oplus a_n \in A^{\oplus n} \mid a_1f_1 + \dots + a_n f_n \in M_k \}$. Each $I_k$ is a submodule of $A^{\oplus n}$, and we also notice that for any $m\ge k$, $I_m \supset I_k$. Thus the ascending chain of ideals
    \[
    I_1 \subset I_2 \subset \dots
    \]
    stabilizes at some $m\in \N$ because $A^{\oplus n}$ is a Noetherian $A$-module by Exercise \ref{3.6.X}X. Thus for indices $k\ge m$, we take $\sum_{i=1}^n a_i f_i \in M_k$ to be an arbitrary element because every element of $M$ (and thus any of the $M_i$'s) can be written as an $A$-linear combination of the $f_i$'s. Then $a_1\oplus \dots \oplus a_n \in I_k = I_m$, so by definition of $I_m$ we get $\sum_{i=1}^n a_i f_i \in M_m$ as well, thus showing $M_k= M_m$ so the chain has stabilized at $m$.
\end{proof}
\subsection{}
\subsubsection{A}\label{3.7.A}
\begin{proof}
    We claim $I(S)=(y)\cap (x,y-1)=(xy, y^2-y)$. It's clear that $\supset$ holds, so our job is to show $\subset$. Thinking of elements of $k[x,y]$ as elements of $k[x] [y]$, we take an arbitrary element
    \[
    \sum_{i=0}^m P_i(x)y^i +(y-1) \sum_{j=0}^n Q_j(x) y^j = \sum_{i=0}^m \left(P_i y^i \right) + Q_n y^{n+1}+ \sum_{j=0}^{n-1} \left( (Q_j - Q_{j+1})y^{j+1} \right) - Q_0
    \]
    of $(x,y-1)$ (so each $P_i$ is divisible by $x$), and furthermore assume that this element is divisible by $y$, i.e. that $P_0=Q_0$ so there are no monomials appearing without $y$. As $x\mid P_0$, we see $x\mid Q_0$ as well. We may now rewrite our element as
    \[
    \sum_{i=1}^m (P_i y^i) +Q_n y^{n+1}+\sum_{j=0}^{n-1}(Q_j-Q_{j+1})y^{j+1}=\sum_{i=1}^m (P_i y^i) +Q_n y^{n+1}+\sum_{j=1}^{n-1}\left((Q_j-Q_{j+1})y^{j+1}\right)-Q_1y+Q_0y.
    \]
    We notice that $xy \mid \sum_{i=1}^m P_i y^i$, and $xy\mid Q_0 y$ as well. Lastly,
    \[
    Q_n y^{n+1}+\sum_{j=1}^{n-1}\left((Q_j-Q_{j+1})y^{j+1}\right)-Q_1y =(y^2-y) \sum_{j=0}^{n-1} Q_{j+1} y^j,
    \]
    showing our arbitrary element is in $(xy, y^2-y)$.
\end{proof}
\subsubsection{B}\label{3.7.B}
\begin{proof}
    We claim $I(S)=(x,y)\cap (x,z)\cap (y,z) = (xy, xz, yz)$, where $\supset$ is clear. We take
    \[
    \sum_{l=0}^n \sum_{i+j+k=l} a_{ijk} x^i y^j z^k
    \]
    to be an element of $(x,y)\cap (x,z)\cap (y,z)$. It must then be that
    \[
    \sum_{i=0}^n a_{i00} x^i =0
    \]
    i.e. each $a_{i00} = 0$ by considering our element mod $(y,z)$. Similarly each $a_{0j0}=0$ and each $a_{00k}=0$ by considering our element mod $(x,z)$ and $(x,y)$ respectively. For each $0\le l \le n$, let $\varphi_l$ denote the set of all nonnegative integers $i,j,k$ with $i+j+k=l$, not $j=k=0$ and not $i=k=0$ and not $i=j=0$. Then we can rewrite our element as
    \[
    \sum_{l=0}^n \sum_{\varphi_l} a_{ijk} x^i y^j z^k.
    \]
    For any $l$ and any $i,j,k\in \varphi_l$, we notice that if $i\ne 0$, then also $j\ne 0$ or $k\ne 0$, so $x^i y^j z^k$ is divisible by either $xy$ or $xz$. If $i=0$, then $j\ne 0$ and $k\ne 0$, so $yz \mid x^iy^jz^k$. Then as each term of our element is in the ideal $(xy, xz, yz)$, our entire element is in the ideal.
\end{proof}
\subsubsection{C}\label{3.7.C}
\begin{proof}
    For a subset $S\subset \Spec A$, we want to show $V(I(S))=\bar S = S\cup S'=\bigcap_{V(I)\supset S} V(I)$, where $S'$ is the set of limit points of $S$ in $\Spec A$. If $\frkp\notin V(I(S))$, i.e. $\frkp \not \supset \bigcap_{\frkq \in S} \frkq$, then clearly $\frkp \notin S$, and there exists some $f\in \bigcap_{\frkq \in S} \frkq \setminus \frkp$. We then see $D(f)$ does not intersect $S$, but simultaneously $\frkp \in D(f)$, so $\frkp$ is not a limit point for $S$ either. Thus $\frkp \notin \bar S$.

    Conversely, if $\frkp \notin \bar S$, then there is some $V(I) \supset S$ with $\frkp \notin V(I)$. Then for each $\frkq \in S$, we have $\frkq \supset I$ implies that $I(S) \supset I$ as well. Because $V(\cdot)$ is inclusion reversing, we then have
    \[
    V(I(S))\subset V(I),
    \]
    and as $\frkp \notin V(I)$, we get $\frkp \notin V(I(S))$.
\end{proof}
\subsubsection{D}\label{3.7.D}
\begin{proof}
    Exercise \ref{3.4.J}J tells us that $f\in \sqrt{J}$ if and only if $f\in \bigcap_{\frkp \supset J} \frkp$, or equivalently $f\in I(V(J))$.
\end{proof}
\subsubsection{E}\label{3.7.E}
\begin{proof}
    Notice that $J=(x^2+y^2-1,y-1)=(x^2, y-1)$ since $y^2-1=(y+1)(y-1)$. Thus $x\notin J$ (else $J$ would be the maximal ideal $(x,y-1)$, but $k[x,y]/(x^2,y-1) \cong k[x]/(x^2)$ is not even a domain), but $x\in I(V(J))$ because $I(V(J))=\sqrt{J}$ by Exercise \ref{3.7.D}D, and $x^2 \in J$ means $x\in \sqrt{J}$.
\end{proof}
\subsubsection{F}\label{3.7.F}
\begin{proof}
    Exercises \ref{3.7.C}C and \ref{3.7.D}D tell us that $V(I(S))=\bar S$ and $I(V(J))=\sqrt{J}$, we know prime ideals are radical, and Theorem 3.7.1 tells us that $V(\cdot)$ and $I(\cdot)$ are inclusion reversing bijections between closed subsets of $\Spec A$ and radical ideals of $A$. Thus it suffices to show that $V(\cdot)$ takes prime ideals of $A$ to irreducible closed subsets of $\Spec A$, and that $I(\cdot)$ takes irreducible closed subsets of $\Spec A$ to prime ideals of $A$.

    Let $S\subset \Spec A$ be any subspace. If $I(S)$ is not prime, there are primes $\frkp, \frkq \in S$ and some $f\notin \frkp$ and $g\notin \frkq$ with $fg\in I(S)$. But then we have nonempty open subsets $D(f)\cap S$ and $D(g)\cap S$ with $D(f)\cap D(g) \cap S=D(fg)\cap S=\emptyset$ (the second equality is by Exercise \ref{3.5.D}D). Having two nonempty open subsets that do not intersect means $S$ is reducible by 3.6.4. Thus $I(\cdot)$ takes irreducible closed subsets of $\Spec A$ to prime ideals.

    Now suppose $V(J)$ is reducible, so there exist $f,g\in A$ with $D(f)\cap V(J)$ and $D(g)\cap V(J)$ both nonempty, and $D(fg)\cap V(J)=\emptyset$. The last condition is equivalent to the statement that $fg\in \sqrt{J}$ by Exercise \ref{3.4.F}F. Then $J$ cannot be prime, else $J=\sqrt{J}$, and then $fg\in J$ means $f\in J$ or $g\in J$ by primeness, which contradicts that $D(f)\cap V(J)$ and $D(g)\cap V(J)$ are nonempty. Thus $V(\cdot)$ takes prime ideals to irreducible sets, and it's clear $V(\cdot)$ takes prime ideals to closed subsets.

    Because prime ideals are by definition the points of $\Spec A$, we get a bijection between points of $\Spec A$ and irreducible closed subsets of $\Spec A$. For any point $\frkp \in \Spec A$, we have $I(\{\frkp \})= \frkp$, and thus $V(\frkp)=\overline{\{\frkp\}}$ is the described bijection.

    
\end{proof}
\subsubsection{G}\label{3.7.G}
\begin{proof}
    Given an irreducible component $S\subset \Spec A$, then $I(S)$ must be a minimal prime. To see this, if $\frkq \subset I(S)$ ($\frkq \in \Spec A$), then $S=\bar S = V(I(S))\subset V(\frkq)$ by Exercise \ref{3.7.C}C, and $V(\frkq)$ is an irreducible closed subset by Exercise \ref{3.7.F}F. Then by maximality of $S$ amongst the irreducible subsets, we see that $S=V(\frkq)$. By applying the inverse $I(\cdot)$ to both sides, we get $I(S)=\frkq$, so indeed $I(S)$ is a minimal prime.

    Conversely if $\frkq \in \Spec A$ is a minimal prime, then $V(\frkp)$ is an irreducible closed subset. To see this, if $S$ is an irreducible subset of $\Spec A$ containing $V(\frkp)$, then $\bar S$ is also irreducible by Exercise \ref{3.6.B}B, and then by the bijection described in Exercise \ref{3.7.F}F, we get
    \[
    I(\bar S) \subset I(S)  \subset I(V(\frkp))=\frkp
    \]
    implies by minimality of $\frkp$ that $I(\bar S)=\frkp$. Then applying the inverse $V(\cdot)$, we get $\bar S = V(\frkp)$, so
    \[
    V(\frkp)\subset S \subset \bar S = V(\frkp)
    \]
    shows $V(\frkp)=S$, so indeed $V(\frkp)$ is maximal amongst irreducible subsets, and is thus an irreducible component.
\end{proof}
\subsubsection{H}\label{3.7.H}
\begin{proof}
    By Exercise \ref{3.7.G}G, we equivalently need to show that the minimal primes of $A=k[x_1,\dots,x_n]/(f)$ are the irreducible factors of $f$. Letting $f_1,\dots, f_m$ be the distinct irreducible factors of $f$. In a UFD (such as $k[x_1,\dots,x_n]$, irreducible elements are the same thing as prime elements. Because $\Spec A/f \cong V(f) \subset \A^n_k$, any prime $\frkp \in \Spec A/f$ must contain at least one $f_i$ (because $f=\prod_i f_i\in \frkp$, and $\frkp$ is prime). Now if we have some $\frkp \in V(f)$ contained in some $(f_i)$, i.e. we have the chain $(f)\subset \frkp \subset (f_i)$ in $k[x_1,\dots, x_n]$. If $f_i \notin \frkp$, then $f_j\in \frkp$ for some $j \ne i$, so then we get the chain
    \[
    (f)\subset (f_j) \subset \frkp \subset (f_i)
    \]
    thus implying $f_i \mid f_j$, contradicting that $f_i$ and $f_j$ are distinct irreducible factors. Thus indeed $f_i \in \frkp$, so $\frkp =(f_i)$, proving each $(f_i)$ is a minimal prime.

    An if $\frkp \in V(f)$ is a minimal prime, as we noticed earlier, there is some $f_i \in \frkp$, so $(f_i)\subset \frkp$ implies by minimality that $\frkp = (f_i)$.

    Thus we have show the minimal primes of $A/f$ are exactly the irreducible factors of $f$, and remark that the only important feature of $k[x_1,\dots, x_n]$ is that it is a UFD.
\end{proof}
\subsubsection{I}\label{3.7.I}
\begin{proof}
    By the proof of Exercise \ref{3.7.H}H, the minimal primes of $k[x,y]/(xy)$ are the irreducible factors of $xy$, being $(x)$ and $(y)$.
\end{proof}

\section{}
\subsection{}
\subsubsection{A}\label{4.1.A}
\begin{proof}
    By Exercise \ref{3.5.E}E,
    we have that $D(f)\subset D(g)$ if and only if $f\in \sqrt{(g)}$ if and only if $g$ is a unit in $A_f$. Then for the map $A_f \to \fO(D(f))$, we let $S$ be the set of elements of $A$ that are units in $A_f$, which is also the same as the set of all elements $g$ such that $D(f)\subset D(g)$ by the exercise. Then by definition, we have $\fO(D(f))=S^{-1}A$, so we wish to show $A_f \cong S^{-1}A$. We have a natural candidate, $\frac{a}{f^n}\mapsto \frac{a}{f^n}$. This map is injective because $\frac{a}{f^n}$ is $0$ in $S^{-1}A$ if and only if there is some unit $g\in A_f$ that annihilates $a$. Then either $A_f = 0$ (or equivalently $f$ is nilpotent, so $D(f)=\emptyset)$ in which case $\fO(\emptyset)=0$, so we have an isomorphism, or necessarily $a=0$ in $A_f$. This shows injectivity.

    For surjectivity, fix $\frac{a}{g}$ with $g^{-1}\in A_f$. Then ${ag^{-1}} = \frac{a}{g}$ in $S^{-1}A$, so $ag^{-1}\mapsto \frac{a}{g}$ as needed.
\end{proof}
\subsubsection{B}\label{4.1.B}
\begin{proof}
    Suppose $\Spec A_f \cong D(f)=\bigcup_i D(f_i)$, or equivalently by Exercise \ref{3.5.B}B,
    there is a finite subset $f_1, \dots, f_n$ of these $f_i$'s that generate $A_f$, or equivalently $\bigcup_{i=1}^n D(f_i)=\Spec A_f\cong  D(f)$. Suppose we are given $\frac{s}{f^n}\in A_f=\fO(D(f))$ that vanishes upon restriction to each $A_{f_i}=\fO(D(f_i))$. To show $\frac{s}{f^n}=0$, we notice that there is some large $m\in \N$ with $f_i^m s =0$ for each $i=1, \dots, n$. In addition, $f_1^m, \dots, f_n^m$ generate $A_f$ since $\Spec A_f = \bigcup_{i=1}^n D(f_i) = \bigcup_{i=1}^n D(f_i^m)$, so we apply Exercise \ref{3.5.B}B, 
    again. Thus there exists $r_1, \dots, r_n \in A_f$ with $\sum_{i=1}^n r_i f_i^m =1$. But then
    \[
    s= \left( \sum_{i=1}^n r_i f_i^m\right) s=\sum_{i=1}^n r_i (f_i^m s)=0.
    \]
\end{proof}
\subsubsection{C}\label{4.1.C}
\begin{proof}
    Suppose $\bigcup_i D(f_i) = D(f)\cong \Spec A_f$, and suppose further that we are given elements in each $A_{f_i}$ that agree on the overlaps $A_{f_i f_j}$. Assume first that the index set is finite, say $\{1, \dots, n\}$. Then we have elements $\frac{a_i}{g_i}\in A_{f_i}$ where $g_i=f_i^{l_i}$, agreeing on overlaps $A_{f_if_j}$, and we may consider each $\frac{a_i}{g_i}$ as an element of $A_{g_i}$. The assumption that $\frac{a_i}{g_i}$ and $\frac{a_j}{g_j}$ agree on $A_{g_ig_j}$ means that for some $m_{ij}\in \N$, $$(g_ig_j)^{m_{ij}}(g_j a_i - g_i a_j)=0$$ in $A$. By letting $m$ be the maximum of the $m_{ij}$'s (allowed because the index set is assumed to be finite), we have
    \[
    (g_i g_j)^m (g_j a_i - g_i a_j)=0
    \]
    for each $i,j$. We now let $b_i = a_i g_i^m$ and $h_i = g_i ^{m+1}$ (so $D(h_i)=D(g_i)$). Then on each $D(h_i)$, we have a function $\frac{b_i}{h_i}$, and the overlap condition now is that $h_j b_i = h_i b_j$. Because $\bigcup_i D(h_i)=\bigcup_i D(f_i)=\Spec A_f$, by Exercise \ref{3.5.B}B,
    there are some $r_i$'s in $A_f$ such that $1=\sum_{i=1}^n r_i h_i$. Now the overlap condition $h_jb_i=h_i b_j$ gives that if we define $r=\sum_{i=1}^n r_i b_i$, then
    \[
    rh_j = \sum_{i=1}^n r_i b_i h_j = \sum_{i=1}^n r_i h_i b_j = b_j,
    \]
    so indeed $r$ restricts to $\frac{b_j}{h_j}$ for each $j=1, \dots, n$.

    For the case where the index set is infinite, we are able to choose a finite generating set $f_1, \dots, f_n$ for $A_f$ by quasi-compactness of $\Spec A_f$, and again let $r=\sum_{i=1}^n r_i b_i$ as before. Then for any index $z$ not in $\{1, \dots, n\}$, we claim that $r$ restricts to $\frac{a_z}{f_z^{l_z}}$ in $A_{f_z}$. Then because $\{1, \dots, n , z\}$ is again finite, we do the same process and obtain an $r'\in A_f$ which restricts to $\frac{a_i}{f_i^{l_i}}$ for each $i=1, \dots, n, z$. By identity (proven in Exercise \ref{4.1.B}B), we see $r=r'$, and the claim follows.
\end{proof}
\subsubsection{D}\label{4.1.D}
\begin{proof}
    Suppose $D(f)=\bigcup_{i\in I} D(f_i)$, so there exists a finite $\{1, \dots, n\}\subset I$ such that $f_1, \dots, f_n$ generate $A_f$ (by quasi-compactness and again Exercise \ref{3.5.B}B
    ).

    For identity, suppose we are given $\frac{s}{f^n}\in M_f = \widetilde M(D(f))$ such that $\frac{s}{f^n}\vert_{D(f_i)}=0$ for each $i\in \{1, \dots, n\}$. To show $\frac{s}{f^n}=0$, we notice that we have a large $m\in \N$ with $f_i^m s =0$ for each such $i$. Now because $\Spec A_f = \bigcup_{i=1}^n D(f_i) = \bigcup_{i=1}^n D(f_i^m)$, we see that also $f_1^m, \dots, f_n^m$ generate $A_f$, so there exists some $r_i$'s in $A_f$ with $\sum_{i=1}^n r_i f_i^m =1$. Then
    \[
    s=(\sum_{i=1}^n r_i f_i^m)s = \sum_{i=1}^n r_i(f_i^m s) = 0.
    \]

    For gluability, suppose we are given elements in each $M_{f_i}$ that agree on $M_{f_i f_j}$. First, we suppose that $I=\{1, \dots, n\}$ is finite, and so we have elements $\frac{m_i}{g_i}\in M_{f_i}$ where $g_i=f_i^{l_i}$, agreeing on overlaps $M_{f_if_j}$. We now consider $\frac{m_i}{g_i}$ as an element of $M_{g_i}$. Then $\frac{m_i}{g_i}$ and $\frac{m_j}{g_j}$ agree on $M_{g_ig_j}$ means that for some $m_{ij}\in \N$,
    \[
    (g_ig_j)^{m_{ij}}(g_jm_i-g_im_j)=0
    \]
    in $M$. Letting $m$ be the maximum of these $m_{ij}$'s (allowed because $I$ is finite), we have
    \[
    (g_i g_j)^m (g_j m_i-g_i m_j)=0.
    \]
    Letting $b_i=g_i^m m_i$ and $h_i=g_i^{m+1}$, we notice $D(h_i)=D(g_i)$. Then on each $D(h_i)$, we have a section $\frac{b_i}{h_i}$, and the overlap condition is now $h_j b_i = h_i b_j$. Now $\bigcup_i D(h_i) = A_f$ implies that there are some $r_i$'s in $A_f$ with
    \[
    1=\sum_{i=1}^n r_i h_i.
    \]
    Defining $r=\sum_{i=1}^n r_i b_i$, we notice
    \[
    rh_j=\sum_{i=1}^n r_i h_j b_i = \sum_{i=1}^n r_i h_i b_j= b_j
    \]
    by the overlap condition, so $r$ restricts to $\frac{b_j}{h_j}$ for each $j\in I$.

    For the case where $I$ is infinite, we again let $(f_1, \dots, f_n)$ generate $A_f$ with $\{1, \dots, n\} \subset I$ by quasi-compactness, and define $r=\sum_{i=1}^n r_i b_i$ as before. Then for any index $z\in I\setminus \{1, \dots, n\}$, we want to show that $r\vert_{D(f_z)}=\frac{m_z}{f_z^{l_z}}$. Because $\{1, \dots, n, z\}$ is also finite, we obtain some $r'$ which has the desired property. By identity, we get that $r=r'$, which gives the result.

    Now that $\widetilde M$ is a sheaf on the distinguished base of $\Spec A$, we want to show that it is also a $\fO_{\Spec A}$-module. It suffices to show this on the distinguished base, for the sheaves on $\Spec A$ are defined in the natural way by the action on compatible germs. We can see this because Exercise \ref{4.1.E}E gives that $\widetilde M_\frkp \cong M_\frkp$. Therefore
    \begin{align*}
        \widetilde M (U) = \{(m_\frkp \in M_\frkp) \mid \forall \frkp \in U, \exists f \in A \text{ with } \frkp \in D(f)\subset U \text{ and } \exists s \in M_f \text{ such that } s_\frkq = f_\frkq \forall \frkq \in D(f)\},
    \end{align*}
    i.e. just the compatible germs. Then $(f_\frkp)_{\frkp \in U} \cdot (m_\frkp)_{\frkp \in U} = (f_\frkp m_\frkp)_{\frkp \in U}$ is the action, and indeed the below diagram commutes
    \begin{center}
        \begin{tikzcd}
            (f_\frkp)_{\frkp \in U} \times (m_\frkp)_{\frkp \in U} \ar[mapsto]{r} \ar[mapsto]{d}& (f_\frkp m_\frkp)_{\frkp \in U}\ar[mapsto]{d}\\
            (f_\frkp)_{\frkp \in V} \times (m_\frkp)_{\frkp \in V} \ar[mapsto]{r}& (f_\frkp m_\frkp)_{\frkp \in V}
        \end{tikzcd}
    \end{center}
    as needed. That the action is $\fO(U)$-linear is easy to see.
\end{proof}

\subsubsection{E}\label{4.1.E}
\begin{proof}
    To show $\widetilde M_\frkp \cong M_\frkp$, we will show $M_\frkp$ satisfies the universal property of $\widetilde M_\frkp$. Notice that $\widetilde M_\frkp = \colim_{U \ni \frkp} \widetilde M(U) = \colim_{D(f) \ni \frkp} = \colim_{f\notin \frkp} M_f = M_\frkp$. The last equality comes from the fact that 
    \begin{center}
        \begin{tikzcd}
            &M_\frkp \\
            M_f \ar{ur} \ar{rr} && M_g \ar{ul}
        \end{tikzcd}
    \end{center}
    commutes, and if 
    \begin{center}
        \begin{tikzcd}
             &N \\
            M_f \ar{ur}{\phi_f} \ar{rr} && M_g \ar{ul}[swap]{\phi_g}
        \end{tikzcd}
    \end{center}
    commutes, i.e. $N$ satisfies the same commutative diagram that defines $\colim_{f\notin \frkp} M_f$, then we define $\varphi: M_\frkp \to N$ by $\frac{m}{a}\mapsto \phi_a(\frac{m}{a})$. Indeed, this is required by the condition that the below diagram must commute 
    \begin{center}
        \begin{tikzcd}
            N\\
            &M_\frkp \ar{ul}[swap]{\varphi}\\
            M_f \ar{uu}{\phi_f} \ar{ur}
        \end{tikzcd}
    \end{center}
    which proves uniqueness. To show this map is a module morphism, we check explicitly 
    \begin{align*}
        &\varphi(\frac{m_1}{f}-\frac{m_2}{g}) = \varphi(\frac{gm_1-fm_2}{fg}) = \phi_{fg}(\frac{gm_1-fm_2}{fg}) = \phi_{fg}(\frac{gm_1}{fg})-\phi_{fg}(\frac{fm_2}{fg})\\
        &=\phi_f(\frac{m_1}{f})-\phi_g(\frac{m_2}{g}) = \varphi(\frac{m_1}{f})-\varphi(\frac{m_2}{g})
    \end{align*}
    and also that
    \[
    \varphi(a \frac{m}{f})=\phi_f(\frac{am}{f})=a\phi_f(\frac{m}{f})=a\varphi(\frac{m}{f}).
    \]
\end{proof}
\subsubsection{F}\label{4.1.F}
\begin{proof}
    \noindent(a) Let $m\in M$ be an arbitrary nonzero element. We want to show that there is some $\frkp \in \Spec A$ such that $m\ne 0$ in $M_\frkp$, i.e. for all $x\notin \frkp$, $xm \ne 0$. Notice that $\Ann(m)$ is a proper ideal (since $1\cdot m=m \ne 0)$, so there is some maximal $\frkm \in \Spec A$ with $\Ann(m)\subset \frkm$, i.e. $\emptyset = \frkm^c \cap \Ann(m)$. This gives the result, for then there is at least one component of $\prod_{\frkp \in \Spec A} M_\frkp$ with the image of $m$ not zero, so the kernel is trivial.

    \noindent (b) Exercise \ref{2.4.A}A
    says for a sheaf $\fF$, $\fF(U) \hookrightarrow \prod_{p \in U} \fF_p$. Then by Exercise \ref{4.1.E}E, we have $\widetilde M_\frkp \cong M_\frkp$, so 
    \[
    \widetilde M(\Spec A) = M \hookrightarrow \prod_{\frkp \in \Spec A} \widetilde M_\frkp = \prod_{\frkp \in \Spec A} M_\frkp.
    \]
\end{proof}
\begin{lemma}\label{lem:equivalences preserve fullness}
        If $\fA$ is a full subcategory of $\fB$ and $\fB$ is equivalent to $\fC$ (i.e. there exists a fully faithful and essentially surjective functor $F:\fB\to \fC$), then $F(\fA)$ is a full subcategory of $\fC$, equivalent to $A$.
    \end{lemma}
    \begin{proof}
       It's clear that $F(\fA) \simeq A$ since $F$ is assumed to be fully faithful so its restrictions retain that property, and is surjective by construction. We use the fact that an equivalence of categories is the same as the existence of a fully faithful and essentially surjective functor (a surjective functor is essentially surjective). Then if $\phi: F(X)\to F(Y)$ is a morphism in $\fC$ and where $X,Y\in \fB$, we get a unique morphism $\varphi:X\to Y$ in $\fB$ such that $F(\varphi)=\phi$. Because $\fA$ is a full subcategory of $\fB$, $\phi$ is also a morphism of $\fA$. Thus $F(\phi):F(X)\to F(Y)$ is equal to $\phi$, and shows $\phi$ is a morphism in $F(\fA)$.
    \end{proof}
\subsubsection{G}
\begin{proof}
    By Remark 2.5.3, the category of sheaves on a base is equivalent to the category of sheaves on the whole space. Therefore it suffices by Lemma \ref{lem:equivalences preserve fullness} to work over the category of sheaves on the distinguished base. On one hand, if we're given a map $\varphi:M\to N$, for any $f\in A$, we get a map $\widetilde \varphi (D(f)):M_f \to N_f$ given by the localization functor so the following diagram commutes:
    \begin{center}
        \begin{tikzcd}
            M \ar{r}{\varphi} \ar{d} & N \ar{d}\\
            M_f \ar{r}{\widetilde \varphi(D(f))}& N_f
        \end{tikzcd}
    \end{center}
    Moreover, if $D(f)\subset D(g)$ (i.e. $g \in A_f^\times$ by Exercise \ref{3.5.E}E
    ), the below diagram commutes:
    \begin{center}
        \begin{tikzcd}
            M_g \ar{r}{\widetilde \varphi (D(g)} \ar{d} & N_g \ar{d}\\
            M_f \ar{r}{\widetilde \varphi(D(f))}& N_f.
        \end{tikzcd}
    \end{center}
    To see this commutativity explicitly, $\widetilde \varphi(D(f))(\frac{m}{f^n})=\frac{\varphi(m)}{f^n}$, and by $g\in A_f^\times$ we have that $\frac{1}{g}=\frac{a}{f^n}$ for some $n\in \Z_+$ and $a\in A$, so that under the vertical maps anything of the form $\frac{m}{g^k}$ is sent to $\frac{a^km}{f^{nk}}$. Now commutativity is easy by $A$-linearity of $\varphi$ and by our constructions. Therefore we get a map $\Hom(M,N)\to \Hom (\widetilde M, \widetilde N)$ given by $\varphi \mapsto \widetilde \varphi$. Since $M(\Spec A)=M(D(1))=M_1=M$, any map $\psi:\widetilde M\to \widetilde N$ already encodes the data of a map $\psi(\Spec A):M\to N$, which gives a map $\Hom(\tilde M, \tilde N) \to \Hom(M,N).$ We will now show these maps are inverses to each other. For any $\psi:\tilde M \to \tilde N$, we want to show $\widetilde{\psi(\Spec A)} = \psi$. We check
    \[
    \widetilde{\psi(\Spec A)}(D(f))(\frac{m}{f^n})=\frac{\psi(\Spec A)(m)}{f^n} = \frac{\psi(D(f))(m)}{f^n} = \psi(D(f))(\frac{m}{f^n})
    \]
    where the last equality is by $A_f$-linearity of $\psi(D(f))$ and the middle equality is because $\psi$ is a map of sheaves, and thus commutes with the restriction from $\Spec A$ to $D(f)$.

    On the other hand, if $\varphi:M\to N$ is a morphism, we want to show $\widetilde{\varphi}(\Spec A) = \varphi$. This is easy to see by our construction of $\widetilde{\varphi}$ and that $\Spec A = D(1)$.
\end{proof}
\subsection{}
There are no exercises in this section.
\printbibliography
\end{document}
